\documentclass[letterpaper,oneside]{memoir}
\hyphenation{arche-type}
\usepackage[oldstyle]{kepler}
\usepackage{color}
\usepackage{latexsym}
\usepackage{stmaryrd}
\settrimmedsize{11in}{8.5in}{*}
\setlength{\trimtop}{0pt}
\setlength{\trimedge}{\stockwidth}
\addtolength{\trimedge}{-\paperwidth}
\setulmargins{4cm}{*}{*}
\setlrmargins{2in}{*}{*}
\raggedbottom
\settypeblocksize{507pt}{26pc}{*}
\checkandfixthelayout
\pagestyle{ruled}
%For lettrines and epigraphs
%\renewcommand{\LettrineFontHook}{\fontfamily{pwt}\fontseries{mn}}
\setlength{\epigraphwidth}{.4\textwidth}
\usepackage{lettrine}
\usepackage{epigraph}

\title{Kwantlen University College \\Creative Writing 3230 and 4130:\\A Reader and Resource Guide}
\author{Ross A. Laird}
\date{Spring, 2008}
%Text formatting
\usepackage{microtype} %must be loaded after all font corrections
\usepackage{everysel}
\begin{document}
\begin{titlingpage}
\aliaspagestyle{titlingpage}{plain}
\setlength{\droptitle}{30pt}
\maketitle
\end{titlingpage}
\newpage
\begin{titlingpage}
\aliaspagestyle{titlingpage}{plain}
\setlength{\droptitle}{30pt}
\maketitle
\end{titlingpage}
\frontmatter
\vspace*{5cm}
\begin{framed}
  \begin{description}
  \item[Note]For complete course details, please visit the websites
    for Creative Writing 3230 and 4130. 
    This guide is an adjunct to the core course materials.
  \end{description}
\end{framed}
\newpage
\tableofcontents*
\mainmatter
\chapter{Preliminary Thoughts on Creativity}
\begin{epigraphs}
\qitem{What goes first on four legs, then on two legs, then on three?}
{\textsc{The Ancients}}\\
\end{epigraphs}
\section{First on Four Legs\ldots}
The oldest artifacts of human endeavor --- cave paintings at Lascaux and Altamira, tools in the Blombos caves, Venus figurines so fantastically old we hardly recognize ourselves ---  are works of art. Creativity is the imprint of humanity, from the outline of a hand painted with ochre on a cave wall, to the mandalas and sacred paintings of the medieval traditions, to the films and music and poetry of today. Throughout all of human history, creativity has been the means by which we understand the inner and the outer worlds, the crucible in which we store our collected wisdom and our fears. The function of all creative traditions --- the arts and the sciences, religion and philosophy, politics and war --- is to explore the extent to which we can know ourselves.

In that exploration, which is fraught with conflicts and dead-ends and  transformations, we're always coming back to one question: it's an old one, old even to the Greeks from whom Freud borrowed it, already old when the Egyptians fashioned a great monument to symbolize it. It is perhaps the oldest riddle of humanity, certainly the favorite riddle of psychology, and it is this: ``What goes first on four legs, then on two legs, then on three?'' The simple answer, widely known, is human beings (first we crawl on four legs, then we walk on two, then we use a cane --- three legs --- in old age). But this riddle, asked by a magical animal that is part hunting lion, part thinking woman, is actually about the contradictions of human experience. The sphinx asks about the essence hidden within our diversities; it's question is about the soul. And creativity, more than any other endeavor, is the study of the soul.

These days we have new ways of asking the riddle of the sphinx, and in our drive to be more scientific and medical, our responses to the riddle have become more complex; but they are no more precise. The answer, after all, is unfathomable. We're still putting our hands up to the cave wall and inscribing around them with ochre. Only now, our tools are slightly different: computers, research modalities, theories. Our approaches to the riddle are possibly more robust than those of our ancestors, but they are probably more fragile as well. What has not changed, in a hundred thousand years of human psychology, is the fundamental creative impulse to understand ourselves and our world.

This is why we possess myths and stories, why we cherish works of art, why many people return (often grudgingly, as though creative approaches diminish us) to the holistic and creative modalities that have proved reliable for millennia. We like the new, we like to demonstrate that we're making progress, that we've discovered fresh and important truths. And possibly we have. But one of the truths we continue to rediscover is that human nature is consistent, for good and for ill, and that it sometimes requires the simple symbols and stories embedded within its history. The creativity of those stories connects us to something inside ourselves that is strong and strange and elemental; a kind of empowerment, to use a term from modern psychology. This empowerment, which extends beyond the individual, is the reason for the Taliban banning music. It's the reason that the cellist Vedran Smilovic played in the street during the war in Sarajevo, while snipers fired upon him. It's the reason that many of us return, again and again, to the stories and images that comprise our identity.  When nothing else is left, when the world's ruthlessness has stripped us of our carefully constructed modernism or intellectualism or rationalism, it's often to creativity that we turn for our deepest solace.

Perhaps the ancients were right: the arts and sciences beckon the gods, beseech them to intervene in the struggle between humanity and its own nature. Now, as always, that struggle preoccupies us: in Iraq, in Liberia, in Israel; and in our jobs, along the streets of our neighborhoods, in our minds as we lie awake late into the night. We used to tell stories in the night, around fires and under the stars that travelers at sea looked to for direction. We used to find order there, and constancy, and a sense of harmony beyond the world's duress. We no longer tell those kinds of stories, and we can't return to a belief in them. Now we find order in the genome, constancy in biochemistry. These are new stories, crafted carefully.

But sometimes the new stories are insufficient to the task of making sense of a world that is routinely nonsensical. Our genome and our biochemistry do not explain why we continue to war against each other, or why we struggle with poverty and environmental degradation. It's easy, as a citizen of such a world, to become cynical, or to feel helpless, or desperate. Most of us can no longer believe in the old gods, and the new ones seem indifferent.

Creativity is perhaps the only means of resolving this conundrum. Creativity forges a path where none exists. Right now, because people sense this, we're seeing a tremendous resurgence in the creative spirit. This is the natural response to trauma and insecurity. We make new stories, we challenge the way things are, by means of visions of the way things could be. This is neither denial nor wishful thinking. It is the soul's affirmation that our world is what we make it; that even as we limp along on three legs, tired and haggard and threadbare, something or someone comes to our aid. All the old tales speak of this.

There are many stories of the crossroads. Monsters and unexpected guides tend to show up there, as do talismans and visions and magic. The gates of the city where Oedipus faced the sphinx are at the crossroads. And the challenge of the crossroads is ever the same: to answer the sphinx's riddle \textit{who are we}?

The sphinx's riddle is the essence of creative work. We stand at the crossroads, looking the beast in the eye, trying not to be the first one to blink.

A great many people are at the crossroads these days, looking out at the horizon to see which way the road goes. Our society is at the crossroads, too: one of our paths, the one we're now on, leads away from the creativity and toward determinism. On this road we study the mind but not the soul. The other road, which is indistinct, hazy, which seems not to run straight but instead wanders across the plain, is the soul's road. It embodies our heritage, as storytellers and witnesses, and the promise of our continuance. Some of us are choosing to sit around the fire of stories, to speak and listen and dream, so that the night sounds from the forest don't seem so baleful and lonely. It's an archaic, simple, creative urge, easy to dismiss amid the complexities of today. But it's how we answer the riddle.

\chapter{Suggested Readings}
\begin{epigraphs}
\qitem{First of all, becoming a writer is mainly a matter of cultivating a writer's temperament.}
{\textsc{Dorothea Brande}}\\
\end{epigraphs}
\section{\textit{A Writer's Temperament}\ldots}
Writers learn their art in large part by reading the works of others; absorbing it, distilling it, rendering it down until it coalesces into a new and personal style. A writer writes, yes, but first a writer reads.

The readings for these courses are excerpted from the following books and essays. As you make your way through the reader, choose one (or more, if you like) of the excerpted creative non-fiction books and read the entire volume. (Each book is in the Kwantlen library.) Or, choose several essays (about five) that are listed but not already in the reader. If you are particularly enthusiastic, One strategy might be to choose three books: one within the genre of your own work; one in an unfamiliar genre or subject; and one for enjoyment. In other words, choose for resonance, craft, and fun (though, only one book --- in addition to the reader --- is required).

\subparagraph{Note}If you have taken Creative Writing 3230 and are now enrolled in 4130, please be aware the reading lists for these two courses are shared: as of 2007, the reader you used for 3230 is this document. If you completed 3230 before 2007, this document will be new to you. In either case, use the guidelines outlined in the previous paragraph. If you have completed 3230, choose another book and/or selection of essays.
\newpage
\section{Creative Non-fiction Books}
\begin{description}
\item [Broadkey, Harold.] \textit{This Wild Darkness: The Story of My Death}. \\Owl Books, 1997. \textsc{ISBN: 0805055118}.

\item [Butala, Sharon.] \textit{Wild Stone Heart}. \\HarperFestival,
  2000. \textsc{ISBN: 000255397X}.

\item [Calvino, Italo.] \textit{Six Memos for the Next Millennium}.
  \\Vintage Classics, 2000.  \textsc{ISBN: 9780099730514}.

\item [Calvo, C\'esar.] \textit{The Three Halves of Ino Moxo.} \\Translated by Kenneth Symington. \\Inner Traditions, 1995. \textsc{ISBN: 0892815191}.\\(Those fluent in Spanish may prefer to read \textit{Las tres mitades de Ino Moxo}, Iquitos, Peru. Proceso Editores, 1981.)

\item [Hedges, Chris.] \textit{War Is a Force that Gives Us Meaning}. \\Anchor, 2003. \textsc{ISBN: 1400034639}.

\item [Hyde, Lewis.] \textit{Trickster Makes This World: Mischief, Myth, \& Art}. \\North Point Press, 1999. \textsc{ISBN: 0865475369}.

\item [Jerome, John.] \textit{Stone Work: Reflections on Serious Play and Other Aspects of Country Life}. \\UP of New England. \textsc{ISBN: 0874517621}.

\item [Krakauer, John.] \textit{Into Thin Air: A Personal Account of the Mt. Everest Disaster}. \\Anchor, 1999. \textsc{ISBN: 0385494785}.

\item [Kingston, Maxine Hong.] \textit{The Woman Warrior: Memoirs of a
    Girlhood Among Ghosts}. \\Vintage, 1989. \textsc{ISBN:
    0072435194}.

\item [Kundera, Milan.] \textit{The Art of the Novel}. \\Perennial,
  1988. \textsc{ISBN: 0060093749.}

\item [Kwan, Michael David.] \textit{Things that Must Not be Forgotten: A Childhood in Wartime China.} \\Soho Press \textsc{ISBN: 1569472823}

\item [Langewiesche, William.] \textit{American Ground: Unbuilding \\the World Trade Center}. \\North Point Press, 2002. \textsc{ISBN: 0865475822}. (Also see \textit{Inside the Sky}.)

\item [Lopate, Phillip.] \textit{The Art of the Personal Essay: An Anthology from the Classical Era to the Present}. \\Anchor, 1997. \textsc{ISBN: 038542339X}.

\item [Macfarlane, David.] \textit{The Danger Tree: Memory, War and the Search for a Family's Past}. \\Walker, 2001. \textsc{ISBN: 0802776167}.

\item [Merwin, W.S.] \textit{The Mays of Ventadorn}. \\National Geographic Directions, 2002. \textsc{ISBN: 0792265386}.

\item [Ondaatje, Michael.] \textit{Running in the Family}. \\Vintage, 1993. \textsc{ISBN: 0679746692}.

\item [Pirsig, Robert.] \textit{Zen and the Art of Motorcycle
    Maintenance}. \\HarperTorch, 2006 (reprint). \textsc{ISBN:
    0060589469}.

\item [Rushdie, Salman.] \textit{Imaginary Homelands: Essays and
    Criticism 1981 --- 1991}. \\Penguin, 1991. \textsc{ISBN: 0670839523}.

\item [Saint-Exup\'ery, A.] \textit{Wind, Sand and Stars}. \\Harvest, 2002. \textsc{ISBN: 0156027496}.

\item [Sanders, Scott Russell.] \textit{Writing from the Center}. \\Indiana UP, 1997. \textsc{ISBN: 0253211433}.

\item [Terpstra, John.] \textit{The Boys: Or, Waiting for the Electrician's Daughter}. \\Gaspereau Press, 2005. \textsc{ISBN: 1554470110}.
\end{description}

\newpage
\section{Essays}
\begin{description}
\item [William Langewiesche] \textit{Fear and Lodging in Baghdad}
\item [Carlos Fuentes] \textit{How I Started to Write}
\item [Wendell Berry] \textit{An Entrance to the Woods}
\item [G.K. Chesterton] \textit{A Piece of Chalk}
\item [E.B. White] \textit{Once More to the Lake}
\item [Virginia Woolf] \textit{Street Haunting}
\item [Lu Hsun] \textit{This Too is Life}
\item [Walter Benjamin] \textit{Unpacking my Library}
\item [Jorge Luis Borges] \textit{Blindness}
\item [Roland Barthes] \textit{Leaving the Movie Theater}
\item [Natalia Ginzburg] \textit{He and I}
\item [Wole Soyinka] \textit{Why do I Fast?}
\item [Henry David Thoreau] \textit{Walking}
\item [Robert Benchley] \textit{My Face}
\item [Adrienne Rich] \textit{Split at the Root}
\item [Joan Didion] \textit{In Bed}
\item [Annie Dillard] \textit{Seeing}
\item [Richard Selzer] \textit{The Knife}
\item [Scott Russell Sanders] \textit{Under the Influence}
\item [Richard Rodriguez] \textit{Late Victorians}
\item [Stanton Michaels] \textit{How to Write a Personal Essay}
\item [Ralph Ellison] \textit{On Being the Target of Discrimination}
\item [Philip Lopate] \textit{The Dead Father: A Remembrance of Donald Barthelme}
\item [Vivian Gornick] \textit{At the University: Little Murders of the Soul}
\item [Jane Shapiro] \textit{This is What You Need for a Happy Life}
\item [Joy Williams] \textit{Save the Whales, Screw the Shrimp}
\end{description}
\newpage{}
\chapter{Resources}
\begin{epigraphs}
\qitem{Universal history is the history of a handful of metaphors.}
{\textit{Pascal's Sphere} \\
\textsc{Jorge Luis Borges}}
\end{epigraphs}
\section{Books on Creativity and Associated Philosophies}
\begin{description}
\item [Achebe, Chinua] \textit{Hopes and Impediments}. New York: Doubleday, 1989.
\item [Barron, F., ed] \textit{Creators on Creating: Awakening and Cultivating the Imaginative Mind}. New York: Putnam, 1997.
\item [Benjamin, Walter] \textit{Theses on the Philosophy of History}.
\item [Borges, Jorge Luis.] \textit{Collected Fictions}. \\Penguin, 1999. \textsl{ISBN: 0140286802}.
\item [Bohm, David] \textit{Wholeness and the Implicate Order}. London: Ark, 1980.
\item [Bohm, David] \textit{Unfolding Meaning}. New York: Routledge, 1985.
\item [Bohm, David] \textit{On Creativity}. New York: Routledge, 1998.
\item [Bronowski, Jacob] \textit{Science and Human Values}. New York: Harper, 1956.
\item [---------] \textit{The Face of Violence}. London: Turnstile Press, 1964.
\item [---------] \textit{A Sense of the Future: Essays in Natural Philosophy}. Cambridge, MIT Press, 1977.
\item [Degler, Teri] \textit{The Fiery Muse: Creativity and the Spiritual Quest}. Toronto: Random House, 1996.
\item [Demos, John.] \textit{The Unredeemed Captive: A Family Story from Early America}. \\Vintage, 1995. \textsc{ISBN: 0679759611}.
\item [Flack, Audrey] \textit{Art and Soul: Notes on Creating}. New York: Penguin, 1986.
\item [Franklin, Ursula] \textit{The Real World of Technology}. Toronto: Anansi, 1999.
\item [Fulford, Robert] \textit{The Triumph of Narrative: Storytelling in an Age of Mass Culture}. Toronto: Anansi, 1999.
\item [Goldberg, Natalie] \textit{Writing Down the Bones}. Boston: Shambhala, 1986.
\item [Herrigel, Eugen] \textit{Zen in the Art of Archery}. New York: Random House, 1977.
\item [Hildegard of Bingen] \textit{Secrets of God: Writings of Hildegard of Bingen}.\\ Boston: Shambhala, 1996.
\item [Hyde, Lewis] \textit{The Gift: Imagination and the Erotic Life of Property}. New York: Vintage, 1983.
\item [---------] \textit{Trickster Makes This World: Mischief, Myth, and Art}. New York: North Point Press, 1998.
\item [Jim\'enez, Juan Ramon] \textit{The Complete Perfectionist: A Poetics of Work}. Edited and translated by Christopher Maurer. New York: Doubleday, 1997.
\item [Jung, C.G] \textit{The Spirit in Man, Art and Literature}. Translated by R.F.C. Hull. Princeton: Princeton University Press, 1998.
\item [London, Peter] \textit{No More Secondhand Art}. Boston: Shambhala, 1989.
\item [Lorca, Federico] \textit{In Search of Duende}. Translated by Christopher Maurer. New York: New Directions, 1998.
\item [Lyndon, Susan] \textit{The Knitting Sutra: Craft as a Spiritual Practice}. San Francisco: Harper, 1997.
\item [Needleman, Carla] \textit{The Work of Craft: An Inquiry Into the Nature of Crafts and Craftsmanship}. New York: Kodansha, 1979.
\item [Pye, David] \textit{The Nature and Art of Workmanship}. Cambridge: Cambridge UP, 1968.
\item [Richards, Mary] \textit{Centering in Pottery, Poetry and the Person}. Middletwon, CT: Wesleyan UP.
\item [Sarton, May] \textit{Journal of a Solitude}. New York: Norton, 1973.
\item [Sennett, Richard] \textit{The Corrosion of Character: The Personal Consequences of Work in the New Capitalism}. New York: Norton, 1998.
\item [Thoreau, Henry David] \textit{Walden}. New York: Norton, 1985.
\item [Wilson, Frank] \textit{The Hand: How Its Use Shapes the Brain, Language and Human Culture}. New York: Vintage, 1998.
\end{description}
\newpage
\section{Interesting Fiction}
\begin{description}
\item [Joseph Conrad] \textit{Lord Jim, Heart of Darkness}
\item [Thomas Wharton] \textit{Salamander}
\item [Madeleine Thien] \textit{Simple Recipes}
\item [Milan Kundera] \textit{Life is Elsewhere}
\item [Carlos Fuentes] \textit{The Orange Tree}
\item [Gabriel Garcia Marquez] \textit{One Hundred Years of Solitude}
\item [Jorge Luis Borges] \textit{Labyrinths}
\item [Vikram Seth] \textit{An Equal Music, A Suitable Boy}
\item [Alberto Manguel (editor)] \textit{Black Water: The Book of Fantastic Literature}
\item [Don DeLillo] \textit{Underworld}
\item [Somerset Maugham] \textit{The Razor's Edge}
\item [Philip K. Dick] \textit{The Man in the High Castle}
\item [Keri Hulme] \textit{The Bone People}
\item [Salman Rushdie] \textit{Midnight's Children}
\item [John Fowles] \textit{The Magus}
\item [Stephen King] \textit{The Stand}
\item [Philip Roth] \textit{Operation Shylock}
\item [Walter Miller] \textit{A Canticle for Leibowitz}
\end{description}
\newpage
\section{Other Resources}
\subsection{Magazines}
\textit{The Atlantic Monthly\\
Event
The Sun\\
The Utne Reader}
\subsection{Movies and Television}
\textit{Wonder Boys\\
Finding Forrester\\
Pollock\\
The Piano\\
My Brilliant Career\\
The Red Violin\\
Pi\\
Barton Fink\\
Memento\\
Almost Famous\\
Magnolia\\
Stand by Me\\
O Brother, Where Art Thou?\\
Clerks\\
Black Hawk Down\\
2001: A Space Odyssey\\
The Life Aquatic\\
Primer\\
Intacto\\
A Mighty Wind\\
Crash\\
Six Feet Under (TV)\\
Arrested Development (TV)}\\
\newpage
\section{Internet Resources}
\subsection{Online Essays by Your Instructor}
  \begin{description}
  \item [Beyond Word Processors] rosslaird.info/node/315
  \item [Literary Traditions and the Modern Writer]  rosslaird.info/node/233
\item [More Thoughts on Minimalism] rosslaird.info/node/221
\item [The Business of Journalism] rosslaird.info/node/207 
\item [Homework for Writers] rosslaird.info/node/205
\item [Niche Markets and the Lonely Artist] rosslaird.info/node/189
\item [Book Reviewer Lingo] rosslaird.info/node/188
\item [Fiction, Reality, and Philip K. Dick] rosslaird.info/node/186
\item [Publishing Dreams and Woes] rosslaird.info/node/176
\item [Data Backup for Writers] rosslaird.info/node/164
  \end{description}
\subsection{Other Internet Resources}
A full listing of writing links and web resources for writing (and creativity, and psychology, and a raft of other subjects), may be found at:\\
http://del.icio.us/rosslaird/writing
\newpage
\chapter{Practices}
\begin{epigraphs}
\qitem{Improvement makes strait roads,\\ 
but the crooked roads \\
without Improvement,\\ 
are the roads of Genius.}
{\textit{The Marriage of Heaven and Hell}\\
\textsc{William Blake}}
\end{epigraphs}
\section{Who Writes (or Wrote) on What?}
\subsection{By Hand}
\begin{description}
\item [John Irving] In small notebooks, then on a typewriter.
\item [Stephen King] On a Mac, until the accident; now uses a Waterman   fountain pen.
\item [Annie Dillard] Pen and paper.
\item [Toni Morrison] Number 2 pencil, yellow legal pad
\item [Rudyard Kipling] Black ink, blue paper.
\item [Thomas Wolfe] Pen and paper, using the top of his fridge as a desk.
\item [Mark Twain] The first writer to submit a typewritten manuscript.
\item [Jack Kerouac] By candlelight, after prayer.
\item [Arthur Miller] Notebooks, then typewriter.
\item [Elizabeth Ann Paulin] Number 2 pencil, yellow legal pad.
\item [Robert Pirsig] Index cards.
\item [John Steinbeck] (Pencil, round shaft only.
\item [Arthur Hailey] Legal pad, with '600 words' written at the   top.
\item [Joan Didion] Yellow, blue, and white paper.
\item [Robert Lowell] Printing.
\item [Ann Tyler] Fountain pen.
\item [John Barth] Fountain pen.
\item [Sophy Burnham] By hand only when having trouble.
\end{description}

\subsubsection{Other Means}
  \begin{description}
  \item [Typewriter] Don Delillo, Joyce Carol Oates, William Carlos Williams, Agatha Christie (using three fingers).
\item [Windows PC] Terry Brooks, Michael Crichton, J.K. Rowling
\item [Mac] Tom Clancy, Salman Rushdie, Kurt Vonnegut
\item [Dictation] John Milton (blindness), Jorge Luis Borges (blindness), Jean Little (blindness; dictates to computer), Osip Mandelstam.
  \end{description}
\newpage
\section{An Argument for Minimalist Writing Tools}
In a recent workshop on the materials and tools of writing I asked the group to indicate which method they used to input text on a computer. Almost everyone used \textit{Microsoft Word} --- with the exception of a sole advocate for the \textit{OpenOffice} word processor (which is far better than \textit{Word}, and which uses the open source \textsc{XML} file format).

But the trend --- even among seasoned writers --- seemed decidedly weighted toward mainstream word processors. This homogenization of word processing, whether on the Mac or PC, inevitably delivers a consistent --- and therefore conformist --- experience to the act of typing on screen. The method of input unquestionably influences the output. By way of subtle cues and imagery (icons, menus, procedures), word processors inculcate a particular type of consciousness. Works of writing from different authors but produced on the same word processor will be more similar than those produced using separate tools. The differences will be subtle but not inconsequential.

Moreover, all word processors introduce a level of aesthetic abstraction that is perhaps not useful to the writing process. Word processors encourage us to fiddle with fonts and spacing, with countless page layout options, with the visual aspect of work that in its initial stages should be primarily visceral.

And word processors are ergonomically inefficient. The mouse, which requires the full use of one arm, is a primary tool in word processors, as are menus and keystrokes assigned for mnemonic rather than ergonomic functions (control-S to save, for example, requires the removal of the left hand from the home row of the keyboard). Years ago, as I began to understand that my persistent arthritic aches were essentially caused by mouse and keyboard use, ergonomics became a core consideration.

Another liability of word processors involves their proprietary file formats, which impede the export and transfer of information to open source formats, such as are found on the World Wide Web. If you want to publish on the Web, word processors make this extraordinarily difficult. Moreover, consider this: if you are writing in \textit{Word}, Microsoft owns your material. After all, your text is in their file format, and if you want to view or edit that text you need to buy their software. Does this seem like an equitable or safe arrangement in terms of safeguarding your intellectual property?

Now, to alternatives and solutions. Go online and search for Bram Moolenaar's \textit{Seven Habits of Effective Text Editing} (Bram's examples use the \textit{Vim} editor, but are germane to writing in general). Learn to reduce the number and increase the efficiency of the keystrokes you make. Wean yourself from the mouse. And by all means look into \textit{Vim}, which is a most robust and efficient text editor (Bram is \textit{Vim}'s main creator). \textit{Vim} was originally designed for the Unix operating system, and its current version is used mostly by Linux users (like me). However, \textit{Cream} is a vim equivalent for Windows.

\textit{Vim} is cryptic, it has a steep learning curve, and some of its functions are improbably arcane. \textit{Emacs}, another editor that originally evolved on Unix (and which I used to write this reader) is similarly cryptic and powerful. In this sense, these editors mirror the writing process. Unlike word processors, which blanket that process, covering it with neat type, text editors lay bare your words and force you to face the page.

Although \textit{Vim}, \textit{Emacs} and other minimalist tools (such as \textit{BBEdit}, for the Mac) are used mostly by programmers, writers are increasingly recognizing the advantages of such tools. They are immeasurably more powerful and efficient for writing of all kinds.

In its beginnings and its evolution, the act of writing is about the bare bones, the essential, the elemental. Use a tool that delivers, rather than distracts, from that wonderful trajectory.
\newpage
\section{What Writers Do (or Did) to Prepare}
\begin{description}
\item [Ernest Hemingway] Sharpen pencils.
\item [Willa Cather] Read the Bible.
\item [Thomas Wolfe] Walk.
\item [Wallace Stevens]	Walk (to work).
\item [James Horan] Commute (on the ferry).
\item [Agatha Christie]	Wash dishes.
\item [Friedrich Schiller] Draw the curtains (red).
\item [Edgar Allen Poe]	Position cat on shoulder.
\item [Toni Morrison] Rent hotel room, remove pictures.
\end{description}
\subsection{The Writer's Posture}
\begin{description}
\item [Virginia Woolf]	Standing
\item [Ernest Hemingway] Standing
\item [Philip Roth] Standing (in his `studio')
\item [Nathaniel Hawthorne] Standing
\item [Lewis Carroll] Standing
\item [Thomas Wolfe] Standing (leaning over the top of the fridge)
\item [Mark Twain] In bed
\item [Truman Capote] In bed
\item [Eudora Welty] In bed
\item [Paul Bowles] In bed
\item [Henry David Thoreau] In bed (in the dark)
\end{description}
\subsection{When?}
\begin{description}
\item [Anthony Trollope] Early morning (5am)
\item [Paul Valery] Early morning (5am)
\item [Daniel Boorstin]	Early morning (5am)
\item [Toni Morrison] Early morning (5am)
\item [Katherine Anne Porter] Early morning (5am)
\item [Eudora Welty] Early morning (5am)
\item [Aldous Huxley] Morning
\item [Henry Miller] Morning
\item [Thomas Mann] Morning
\item [Pablo Neruda] Morning
\item [P.G. Wodehouse] Afternoon/early evening
\end{description}
\newpage
\section{Blinking Cursor, Blank Page}
  Late in \textit{Heart of Darkness}, after Marlow has meandered deep   into the jungle but before he meets Kurtz, who utters his now-famous   judgement upon human nature, ``The horror! The horror!'' --- before   this, the most famous scene in twentieth century literature, Marlow   finds himself making necessary repairs to the ship. He ruminates on   these activities as distractions from the shadows around him, from   the haunting underbelly of his own nature that he sees in the   wilderness around him, in the passionate abandon of the local   tribes-people.  Here's the full passage:
\begin{quotation}
  The earth seemed unearthly. We are accustomed to look upon the   shackled form of a conquered monster, but there --- there you could   look at a thing monstrous and free. It was unearthly, and the men   were --- No, they were not inhuman. Well, you know, that was the   worst of it --- this suspicion of their not being inhuman. It would   come slowly to one. They howled and leaped, and spun, and made   horrid faces; but what thrilled you was just the thought of their   humanity --- like yours --- the thought of your remote kinship with   this wild and passionate uproar. Ugly. Yes, it was ugly enough; but   if you were man enough you would admit to yourself that there was in   you just the faintest trace of a response to the terrible frankness   of that noise, a dim suspicion of there being a meaning in it which   you --- you so remote from the night of first ages --- could   comprehend. And why not?  The mind of man is capable of anything ---   because everything is in it, all the past as well as all the future. What was there after all?  Joy, fear, sorrow, devotion, valour, rage --- who can tell? --- but truth --- truth stripped of its cloak of time. Let the fool gape and shudder --- the man knows, and can look on without a wink. But he must at least be as much of a man as these on the shore. He must meet that truth with his own true stuff --- with his own inborn strength. Principles won't do. Acquisitions, clothes, pretty rags --- rags that would fly off at the first good shake. No; you want a deliberate belief. An appeal to me in this fiendish row --- is there? Very well; I hear; I admit, but I have a voice, too, and for good or evil mine is the speech that cannot be silenced. Of course, a fool, what with sheer fright and fine sentiments, is always safe. Who's that grunting? You wonder I didn't go ashore for a howl and a dance? Well, no --- I didn't. Fine sentiments, you say? Fine sentiments, be hanged! I had no time. I had to mess about with white-lead and strips of woolen blanket helping to put bandages on those leaky steam-pipes --- I tell you. I had to watch the steering, and circumvent those snags, and get the tin-pot along by hook or by crook. There was surface-truth enough in these things to save a wiser man.
\end{quotation}

Marlow employs seamanship as a kind of shield against the chaos, against the frightening shapes of his inner life. After all, he is a civilized man, an Englishman, for whom the shadow must be contained. Marlow is a sailor, one who traverses the waters but remains above them. Writers, conversely, are involved in plumbing those depths, in encountering their lights and shadows, in struggling with the full breadth of human propensity.

But we distract ourselves too, and the most common method of doing this is to allow the electronic world to continually divert us from the blank page and blinking cursor. Email alerts, news feeds, blogs: there is enough distraction in these things to doom the wisest writer. We must go into the darkness, into the bardo, to discover our treasures. That we often have difficulty doing so is, in part, due to the persistent emphasis of our environment upon the facile and the transient and the ephemeral. Always an update, a flashing notice, a clamoring icon which seems to confirm our importance --- but in fact belies our addiction to the inconsequential.

Writers have not been well-served by technology since about 1990, when the last console versions of \textit{WordPerfect} showed us a black screen upon which a small, blinking cursor waited patiently for us to dive into the waters. Since the advent of graphical user interfaces, the blackness has been hidden, has been replaced by smilies and floral wallpaper and pastel icons. I'm not against the \textsc{GUI}; but for writing, it's a serious impediment, the modern equivalent of Marlow's leaky steam pipes.

The confrontation with what lies behind, or beneath, or hidden, is the essence of all good writing. And whether one perceives that hidden-ness as darkness, as Conrad did, or as a terrifying whiteness, as did Melville, the mystery is the same. Here's Melville describing the peculiar terror evoked by the whiteness of the whale:
\begin{quotation}
  Is it that by its indefiniteness it shadows forth the heartless   voids and immensities of the universe, and thus stabs us from behind   with the thought of annihilation, when beholding the white depths of   the milky way? Or is it, that as in essence whiteness is not so much   a colour as the visible absence of colour; and at the same time the   concrete of all colours; is it for these reasons that there is such   a dumb blankness, full of meaning, in a wide landscape of snows ---   a colourless, all-colour of atheism from which we shrink? And when   we consider that other theory of the natural philosophers, that all   other earthly hues --- every stately or lovely emblazoning --- the   sweet tinges of sunset skies and woods; yea, and the gilded velvets   of butterflies, and the butterfly cheeks of young girls; all these   are but subtile deceits, not actually inherent in substances, but   only laid on from without; so that all deified Nature absolutely   paints like the harlot, whose allurements cover nothing but the   charnel-house within; and when we proceed further, and consider that   the mystical cosmetic which produces every one of her hues, the   great principle of light, for ever remains white or colourless in   itself, and if operating without medium upon matter, would touch all   objects, even tulips and roses, with its own blank tinge ---   pondering all this, the palsied universe lies before us a leper; and   like wilful travellers in Lapland, who refuse to wear coloured and   colouring glasses upon their eyes, so the wretched infidel gazes   himself blind at the monumental white shroud that wraps all the   prospect around him. And of all these things the Albino whale was   the symbol. Wonder ye then at the fiery hunt?
\end{quotation}

The writer approaches mystery by way of the white page or the black screen. That is our task, nothing more. Not to explain the mystery, or to resolve it, or to erase it; but to encounter it, struggle with it, allow it to enter and change us. This will not happen, cannot happen, if we are checking our emails and news feeds all day long. We must sit in silence, waiting for the mystery to descend.

Now, from the mystic to the concrete:

\begin{itemize}
\item Turn off all email notification.
\item Turn off all news notification.
\item Turn off software update notification.
\item Configure all desktop panels to ``auto-hide.''
\item Remove all icons from the desktop (organize yourself!).
\item Choose a wallpaper that will not scream at you.
\item Close all browsers.
\item Do not use \textit{Microsoft Word} (of which, much discourse elsewhere and in the course).
\item Use a text-based editor such as \textit{cream}, \textit{vim}, or   \textit{emacs}.
\item Learn to use the editor with keystrokes only (this takes time).
\item Choose a color scheme for the editor that has high contrast and, ideally, a dark background.
\end{itemize}
Write from the blinking cursor, with no other distractions. Just you and the waiting blackness (or whiteness, which amounts to the same thing). Reclaim the mystery.
\newpage
\section{Suggestions for Success}
\begin{description}
\item [Dorothea Brande] First of all, becoming a writer is mainly a matter of cultivating a writer's temperament.
\item [Joyce Carol Oates] One must be pitiless about this matter of `mood.' In a sense, the writing will create the mood\ldots I have forced myself to begin writing when I've been utterly exhausted, when I've felt my soul as thin as a playing card, when nothing seemed worth enduring for another five minutes\ldots and somehow the activity of writing changes everything.
\item [Jacques Barzun] It is wise to have not simply a set time for writing --- it need not be daily and yet be regular --- but also a set `stint' for the day, based on a true, not vainglorious estimate of your powers. Then, when you come to a natural stop somewhere near the set amount, you can knock off with a clear conscience.
\item [Terrence McNally] It's very simple, really. You have to go to the typewriter, that's all you have to do. I have a word processor now, and you turn it on and something happens after a while.
\item [Bernard Malamud] You write by sitting down and writing. There's no particular time or place --- you suit yourself, your nature. How one works, assuming he's disciplined, doesn't matter.
\item [Stuart Cohen] Writing is a marathon of the spirit. Don't give up.
\end{description}

\newpage
\section{Sentence Composition Checklist}
\begin{description}
\item [\boxempty] The sentence contains no extra words.

\item [\boxempty] The sentence is written in the present tense.

\item [\boxempty] The sentence is written in active voice, using I if suitable.

\item [\boxempty] The order of items in the sentence suits the relevance of those items.
	(The most important item is either at the beginning or the end.)

\item [\boxempty] The sentence contains adverbs (-ly words) only where necessary.

\item [\boxempty] The sentence avoids gerunds (-ing words) wherever possible.
	(``A dog runs'' is better than ``a dog is running''.)

\item [\boxempty] The words within the sentence are strong and descriptive.

\item [\boxempty] The imagery of the sentence is concrete and specific.

\item [\boxempty] The sentence avoids awkward constructions (such as ``there is...'' and ``would...'').

\item [\boxempty] The sentence is clear, and communicates precisely what I wish to say.

\item [\boxempty] The sentence hints at larger themes, perhaps universal themes, 
	but is not preachy, pedantic, or pretentious. (Show, don't tell).

\item [\boxempty] When I read the sentence aloud, the rhythm is appealing and poetic.
	(If I separate the phrases of the sentence into separate lines, 
	the sentence becomes a non-rhyming poem.)



\newpage
\chapter{Exercises}
\begin{epigraphs}
\qitem{Now tonight,\\I am a burning bush,\\
my bones a grill of fire,\\
I burn these words in praise\\
of our meeting, our friendship.}
{\textsc{Jimmy Santiago Baca}}
\end{epigraphs}
\newpage
\section{Stages in the Creative Process}
\subsection{A Mythological Journey}
In which first there is 
\begin{description}
\item[The Call\ldots] a beginning, an initiating force (or event) behind all creative and personal development. The Call is an unexpected event, a trauma, an intrusion into the sedate and comfortable lives we craft so carefully. In creative work, the Call is the moment of vision. In turn, it is a stage requiring of us a disruption in routines, an openness, an encouragement of the mystery. In myths and stories, the Call takes the symbol of the unexpected letter, or the sudden injury, or the surprising twist away from the ordinary. The Call is the gateway, and is followed in turn by

\item[Refusal of the Call\ldots] in which we assert for business as usual, for the way things were, for the re-establishment of our ordinary world. The task of the artist (and the writer) is to refuse to refuse. We must slow down, and listen, and open the eye of seeing. Universally, the opening of that eye is assisted by

\item[The stranger\ldots] who we meet on the road: the wise one, the elder, the mentor. The stranger offers compassionate assistance, evokes our openness and our patience. Without the stranger, neither the work of creativity nor of healing is possible. With assistance from the stranger (who is an archetype, and may therefore also be a trusted friend), we cross the threshold, take a deep breath, and enter our own wilderness. Clarity is required here, and intent, and a willingness to open the gate. Wind lies on the other side, and the unknown. Our path lies that way, toward

\item[The labyrinth\ldots] in which we become confused and disoriented. We seek but do not find shelter. We become lost, and fall into ourselves. Trusting the process is the task here: dealing with the dark, the cliff, the shadow. Discomfort, fear, and inertia become companions. We hear the monster which haunts all labyrinths, and which is our own inner life projected outward. But the labyrinth has one path only: toward a confrontation with the monster. We must keep going. All tales confirm this. And if we do keep going --- simply, with trust, with purpose --- we

\item[Face the monster\ldots] and find the beast to be our own wisdom in disguise. The monster is a teacher, a guide, an enemy who becomes an ally. From the monster we learn

\item[Clarity\ldots] and we move onward to discover the still point at the center of the labyrinth. Healing and spirituality and creativity reside at this center. Peace is made with the past there. We gather up the scattered threads of our inner knowing. We recognize the illumination to be found at the centre, and in so doing we begin to shape the tale of our journey. Above all other junctures in creative work, the still center is of the core and essence. It is here that all parts of ourselves align, and for a moment we glimpse ourselves all the way down into the Soul. When the still point arrives, creative work is almost done; healing is almost done. But first we must make our way back into the world, by way of

\item[The shallows\ldots] and the bridge, which will deliver us back to the world we departed so long ago. That land now seems foreign, and strange, and we find ourselves uncertain about how to find our place within it. Creativity, after all, is a journey of the inner life, and is only peripherally about what we craft. Creativity is the personal path inward, toward our own discoveries. The shallows and the bridge are ways forward, and outward, to

\item[Return\ldots] to the world, to the bright day of sharing our discoveries with the community. And yet, because the inner journey is so rich, and intense, and powerful, often we

\item[Refusal to return\ldots] and instead we become addicts of the creative process. We want to move to a mountain hut, we wish to leave the world by way of the imagination. Creative work becomes its own hurdle on the path. We dream of becoming the eternal traveler on that wondrous path. But, as the old stories tell, there appears again

\item[The stranger\ldots] who calls us back (and who need not be the same stranger); the one who invites and demands that we share our work with the world --- so that they too might see, and know, and be healed. They set watch fires for us, and they wait, and we embark upon a mysterious journey back. We cross

\item[The return threshold\ldots] and enter the world again. We bear gifts of wisdom and of healing. We have been burned by the light of illumination and are healed. We share our gifts with the community; and in this celebration there is a pausing, a 

\item[Conscious integration\ldots] of what we have undertaken and learned, a recognition of wholeness and completion and healing. We become the stranger for others. We have crossed the wide sea and know its ways. We rest, for a moment. And in this space of quiet, while we are not paying attention,

\item[The cycle begins again.]
\end{description}
\newpage
\section{Twenty Themes}
\begin{enumerate}
\item Write something about yourself that no one else yet knows.
\item Describe a childhood memory in which you felt particularly empowered, or loved.
\item Write about one of your dreams.
\item Describe an experience you have had that you would define as spiritual.
\item Describe an experience in which you faced and overcame suffering.
\item Describe a humorous event in your life.
\item Write about something with which you are currently struggling.
\item Write about your own nature and talents, what are you good at? Where do you shine?
\item Write about one of your fears.
\item Write about the smell, or touch, or taste, or sound, that you love the most.
\item Write about something you know deeply.
\item If you could change one thing about the world, what would it be?
\item If you could change one thing about yourself, what would it be?
\item What does your shadow look like, or sound like? What does it say? How does it write?
\item Write about something you believe.
\item When you are feeling weak and vulnerable, what must you remember?
\item What is your favorite book, or story, or film, or poem, or painting? Why?
\item Describe one thing you feel is necessary for you to do before you die.
\item Who do you love? Why? Write about this.
\item Write about something else.
\end{enumerate}
\newpage
\section{A Theme Example: What Must I Remember?}
\subsection{The Jasper Queen}
The indomitable spirit cannot be diminished --- by negligence, by war, by time spun farther than the grasp of memory. This occurs to me on September ninth, in the Egyptian gallery of New York's Metropolitan Museum of Art, as I stand before the only remaining fragment of an ancient sculpture. The body has vanished, and most of the head is gone. What remains is a small artefact, about six inches high: an elegant mouth --- smiling, in repose --- and the beginning curve of a face, carved from yellow jasper. Between ragged fractures where the stone is sheared off --- one just above the top lip, the other below the chin --- the mouth has been sculpted with astonishing precision by the craft of a culture now strewn across the debris field of history. This statue, all that's left of the queen of a remote age, was fashioned in devotion and shattered by war, almost twenty-five centuries ago. And still, she smiles.

I remain in the gallery for a long while, absorbing the details of this remarkable object: bright and smooth, polished to a high sheen. Yellow jasper, symbol of the imperishable, the rain-bringer, a stone reputed to drive away evil spirits, has long been associated with healing. Perhaps this mouth, so fragile, the instrument of a forgotten voice, has been preserved by virtue of the jasper's protection. This relic endures, even as the Taliban destroy stone Buddhas in Afghanistan. In countless guises, the instinct for beauty prevails.
Two days later, back home in British Columbia, as I prepare to work a stone I found on the mountain north of where I live, terrorists fly hijacked airplanes into the World Trade Center, into the Pentagon, into the ground. Like their ancient allies, they tear down the standing stones, endeavor to destroy all that is foreign and strange. The old fires have not stopped burning.

I am drawn away from the shop and into my grief for many days. I sit with my wife in the quiet sanctuary she has made of our yard. The first ochre leaves appear, and we wonder how to make sense of such unfathomable events. My eight-year-old daughter writes a poem about the end of summer, in which birds fly to nice, warm places. Safe passage. As the season turns, I pray that I find the wisdom to weigh, in my own small and quotidian life, the will to heal against the wish to harm.

When I can no longer abide images from the television, when the rawness within me must be assuaged, I return to my workbench. My affliction is softened as I cradle my tools and guide them across the stone, restoring a shattered visage. The dust gathers into great storm clouds as I work, falls like ash onto every surface of my shop. The facade of the stone cracks, gathers itself into the contours of a resolute chin, a strong mouth and a cheek rising toward a restful eye.

Rage and tears and a strange dread, lurking and tenebrous, find their way into the rhythm of my work. Bits of loose stone fall onto the floor, abrade my skin with their sharp edges, scrape the benchtop I so carefully protect from harm. I persist, straining to reclaim, in the grain of dark stone, the soft faces of those now lost to our sight. I mourn the death, too, of the isolated innocence of my culture. And I try to answer the questions of my four-year-old son, who cannot understand why the hijackers would hurt anyone. He devises surprisingly elaborate plans for talking to them, for asking them to stop.

He watches me work, brings me tools, draws close in this time of elemental fear. My hands, searching for the stone's redemption, trace their way across the emerging contours of a jaw, and the rough edge where the forehead will be. I imagine the craftsmen of the jasper queen, and I wonder, as I inspect my work during a bright and warm afternoon, if it's her voice I hear, humming among the trees out back. I discover, once again, that the simple work of hands is a guide in my own healing. I am shaped by the work of creativity as a stone is by tools. And I am sustained, finally, by the hope that my one stone might stand with the destroyed and colossal Buddhas, with the scattered and the fallen, with those on their way back home.

Creativity can be a deep sustenance --- whether in stone or wood or soil. And though my carving is crude, fails utterly to match the surpassing skill of those ancient craftsmen, I persevere; for the work of creation calls not only to the practiced hand. Slowly, easing into the surface, I peel back the many layers that hide the finished face. The air is thick with transformations.

I wash dust from the stone. The bright surface beneath, smoothed by countless tool strokes, appears alive. Dark striations weave their way across the rudimentary cheek, and flecks of white --- feldspar --- scatter like snowflakes along the brow. There's more work, much more: the nose, the eyes, the left side of the jaw. But I've begun. And as I gaze upon the face before me, collected from the ashes of mountains and the visions of my own troubled days, I glimpse a woman both serene and fair. She looks upon our fractured world with an indomitable spirit. And she smiles.\\
\newpage
\section{Partner Exercises}
\begin{enumerate}
\item Ask your partner for three words. Incorporating these three words, write one paragraph of a myth. Share it with your partner. Discuss the process.
\item Ask your partner to make or describe three movements. Based on the three movements, write or draw the story they tell. Feel free to embellish. Show your partner what you created. Discuss the process.
\item Ask your partner to make or describe three sounds. Adapting the three sounds, and other sounds and words (if you like) of your own, write a one-verse song. Share the song with your partner (sing it if you like!). Discuss the process.
\item Ask your partner to describe three images. Using the three images, compose a story about how the world began. Share the story with your partner. Discuss the process.
\item Ask your partner to describe two smells and two tastes. Using the smells and tastes (and others of your own choosing), write a non-rhyming poem of about 12-20 lines. Share the poem with your partner. Discuss the process.
\item Ask your partner to describe one memory. Using the memory as your own, write your own version. Share this version with your partner. Discuss the process.
\end{enumerate}
\newpage
\section{Parable of the Warrior Princess
\\(Adapted from Tibetan Buddhism)}
A young warrior princess completed her training under a renowned teacher and was accorded the title \textit{Princess of Five Weapons}. Armed appropriately, and embodying her forty-two virtues, she set out on the road leading to the eternal city.

The road led the princess west, across the wide desert and into a forest. At twilight she reached the first trees, where she found other travelers who warned her to turn back. They spoke in fearful tones about an ogre, an eater of hearts, who lurked along the most shadowed paths, killing all those who happened by. But the princess was confident of her training. Fearless, she pressed on.

At a dark place, where branches overhung a stagnant stream, the ogre emerged from the underbrush. It was a phantom, a wraith, a brute with crushing hands. The princess deployed her five weapons, but the ogre was strong (and crafty) --- one by one, the weapons of the princess were defeated. But she did not relent. After each weapon was spent and lay broken on the ground, the princess resumed the battle, challenging the ogre again and again.
 
Finally, the ogre paused, and asked her, ``Youth, why are you not afraid?''

``Ogre,'' replied the princess. ``Why should I be afraid? For in life, death is absolutely certain. What's more,''\ldots
\newpage
\section{The Golden Key 
\\(Last Tale of the Brothers Grimm)}
Once in the wintertime when the snow was very deep, a poor boy had to go out and fetch wood on a sled. After he had gathered it together and loaded it, he did not want to go straight home, because he was so frozen, but instead decided to make a fire and warm himself a little first. So he scraped the snow away, and while he was clearing the ground he found a small golden key. Now he believed that where there was a key, there must also be a lock, so he dug in the ground and found a little iron chest. ``If only the key fits!'' he thought. ``Certainly there are valuable things in the chest.'' He looked, but there was no keyhole. Finally he found one, but so small that it could scarcely be seen. He tried the key, and fortunately it fitted. Then he turned the lock once, the lid popped open, and in the chest the boy saw\ldots
\newpage
\section[A Parable of Envy]{On Reading W.S. Merwin in \textit{The New Yorker}}
I was reading a poem by W. S. Merwin in the New Yorker the other day\\
and as usual I was feeling pretty intimidated by all those long lines\\
he always writes and of course all the stuff that was going to be\\
in there about how things look in New England especially in the fall\\
and maybe even trout fishing and what it all means to the human soul\\ on a
universal level because Fall in New England is always dynamic\\
and everywhere else is parochial but I decided to read his poem\\
anyway because I thought maybe I could just stand it\\
and he started in by talking about a barn door and some \\
stones on a hillside and an old man hoeing the dirt which seemed\\
allright to me even though it was as usual Fall in New England\\
because I really liked the imagery he made which is something\\
I always like because it puts pictures in my head even if I\\
am parochial and never even seen New England in the Fall when he\\
started in to saying as to how all this imagery really felt to him\\
which also meant how his personal feelings were all\\
about what the universal condition of man is and I got to \\
thinking about how glad I was I wasn't in some English\\
class again because those last five or six lines about universal\\
New England consciousness are always the ones your\\
freshman English instructor wants you to write a six page\\
double-spaced paper on and I hate it when that happens\\
\\James DeFord
\chapter{Language}
\begin{epigraphs}
\qitem{Then came Thoth, master of spells and words of power,\\ voice of truth.}{\textit{The Book of Illumination}\\
\textsc{3000 BCE}}\\
\end{epigraphs}
\section{The Writer as World Maker}
\subsection{An Alternate View of the Writer's Mind}
The relationship between medium and message, which we exalt as a quintessentially modern discovery, was exhaustively explored five thousand years ago. Richard Wilkinson, in his \textit{Symbol and Magic in Egyptian Art} [London: Thames and Hudson, 1994], contends that Egyptian hieroglyphs ``transcended the boundaries of most written scripts in successfully blending symbolic representation and the written word to a degree that no other system of writing has surpassed'' (161). The integration of written language and symbol displays a level of virtuosity so profound that we can hardly grasp it. The interconnections of the Internet, for example, in which objects are embedded within objects in almost infinite, nested sequences, would have seemed, to the ancient Egyptians, rudimentary indeed. The Internet is, for all its complexity, a non-symbolic system: each of its elements (files, web pages, computer keyboards and monitors, cables, and so on) possesses a discrete and single-level meaning. A keyboard can be represented by an onscreen icon , but it cannot mean anything other than a keyboard --- unless you give it further levels of meaning yourself (in which case other people may not understand you).

There are, in English, no shared multi-level meanings of a keyboard. But imagine what a hieroglyphic keyboard would have been like. Each key is shaped into a distinctive symbol, and the letter displayed upon it is an expression, in another form, of the same symbol. But not exactly the same: the letter is another flavor of the motif of the key. The placement of the keys is related to the fingers that touch them as you type, and the symbols of the keys relate to the symbolic functions of each finger. Moreover, the key placements are designed to evoke specific movements of the fingers as you type, the order of their action as well as the shapes they make. You animate animals and ideas and gods as you move your fingers. And you have a choice about what you animate, because every expression offers hundreds of ways to type it. You can write forwards or backwards, or vertically. You can change the order of letters, insert symbols that abstractly or concretely express the sense of your words. Your sentences can interlock like crossword puzzles. You can specify that your writing contain anywhere from two to twelve levels of meaning (but not one level, as in English).

Every letter, every phrase, every passage you write is a hologram of your entire communication. From the simplest rebus to the most elaborate phonogram, every fragment is a map of the whole, and interconnects with every other fragment by way of a web of meanings so colossal no one can grasp all of it. Besides, the movements of your fingers evoke the gods in unfathomable ways, and they insert further meanings which you do not intend and do understand. The meaning of the text, finally --- any text, from a household budget to a philosophical treatise --- extends beyond the world, reaches toward the eternal. Your keyboard has five thousand keys.

The symbology of Egyptian hieroglyphic texts employs the dialect of dreams: nuanced, multi-faceted, holographic. The glyphs are woven with interlocking symbols and strands of meaning through which reading becomes a devotional act. Word or phrase reversals --- palindromes --- hide in every text, leading the reader forward and back along a spiraling track of understanding. Every textual path unfolds in at least two directions: two truths, two selves. Palindromes are hinges upon which the attention of the reader turns. \textit{Ben} is the primordial stone; \textit{neb} is gold, the symbol of conscious awareness (and the root of our modern word nebula). \textit{Ais} is the brain, \textit{sia} consciousness. \textit{Ab} is the heart, \textit{ba} the soul. These reflected and doubled pairs form a concealed language beneath the pragmatic. They turn awareness back on itself, back to the source.

Passages are strung from left to right, as in English, or in reverse, or even vertically (top down, though never bottom up). The intended direction for reading is denoted by the orientation of the ideograms: an animal or a god looks toward the beginning of the text. A palindromic glyph passage can be written forward in a section of reversed text, or backward in a section of forward-facing text. Such suppleness of linguistic structure has made decipherment a difficult task.

The thorniest aspect of the language involves the absence of written vowels. For the ancient Egyptians, vowels were the keys to magic and power; to speak them in the glyphs was to be in possession of the means to create a world. Scribes were sufficiently convinced of the inherent power of the vocalized glyphs that they were careful to limit that power: many of the glyphs inscribed on the stone walls in the pyramid of Unas were modified to prevent them from coming to life unbidden. Ideographic glyphs of birds were carved with their feet intentionally defaced so the birds could not launch themselves from the stone. Glyphs depicting humans were often carved with missing arms or legs. Sometimes substitute glyphs were used in place of those thought more likely to animate themselves. For the Egyptians, the hieroglyphs were, literally, living documents.

Nowhere in the thousands of columns of the early Egyptian sacred texts is there hint of a single vowel. Our modern pronunciation of the lost language is a reconstruction, a best guess at where the vowels belong, but it's not the language itself. The authentic voice of the words is gone. We are left to gather up the fragments of a library strewn across a debris field wide as history.
\newpage
\section{Words}
\subsection{Scholarship and the Politics of Language}

``Our words are similar to wells,'' says the poet C\'esar Calvo, ``and those wells can accommodate the most diverse waters: cataracts, drizzles of other times, oceans that were and will be of ashes, of human beings, and of tears as well. Our words are like people, and sometimes much more, not simple carriers of only one meaning.''

Words have power, and presence, and a history of which we are sometimes unaware. It is prudent, as a writer, to use language consciously, to be as intentional as possible about tones and moods and the colors of the page.

The following list is cautionary: yes, feel free to use the words on this list, and perhaps builds tropes (a hifalutin' word) around them; but be aware of the impact such words may have, of their sharpness or fuzziness, of the surprising ways in which readers might respond.

\begin{description}

\item \textit{{Slippery Words\ldots}} for which all definitions are provisional: creativity, multicultural, objectivity Self/self, universal, subjectivity, objectivity, consciousness, Mind/mind, culture, art, mind-body, bodymind, minority, cognition, fulfilment, dominant, soul, mainstream, gender, spirit (and spirituality), transformation, truth, internal, external, healing, enlightenment, growth

\item \textit{{Flag-Draping and Eyebrow-Raising Words\ldots}} which telegraph particular political perspectives: corporate, colonial, anything-centric, mindset, postcolonial, deep, ecology, liberal, conservative, radical, ahistorical, postmodern, therapeutic

\item \textit{{Hifalutin' Words\ldots}} which are often used improperly in service of erudition: Cartesian, Newtonian, aesthetic, duality, modality, schema, construct, notion, praxis, hegemony, structural (con/de/post), pedagogy, liminal, archetype, paradigm (/shift), positivism, hermeneutic, teleology

\item \textit{{Hand Grenade Words\ldots}} that tend to provoke strong reactions in readers: oppression, prejudice, marginalized, race, conspiracy,  agenda, supposed, aggression, trauma, wound, academia, terrorism, tyranny, shame-based (hand grenade words have fuses of roughly fifty pages)

\end{description}
\newpage{}
\section{Sentences}
\begin{description}
\item [\textit{Chris Hedges}] In wartime the state seeks to destroy its own culture.
\item [\textit{Jorge Luis Borges}]
It may be that universal history is the history of a handful of metaphors.

\item [\textit{Stephen King}] The man in black fled across the desert, and the gunslinger followed.

\item [\textit{Keri Hulme}] He walks down the street.

\item [\textit{Joseph Conrad}] \textit{The Nellie}, a cruising yawl, swung to her anchor without a flutter of the sails, and was at rest.

\item [\textit{John Fowles}] I was born in 1927, the only child of middle-class parents, both English, and themselves born in the grotesquely elongated shadow, which they never rose sufficiently above history to leave, of that monstrous dwarf Queen Victoria.

\item [\textit{Sharon Butala}] Our house was haunted.

\item [\textit{Phil Jenkins}] Leave where you are and come stand beside me.

\item [\textit{Carlos Fuentes}] All this I saw.

\item [\textit{W.S Merwin}] In the light between rains on a morning late in spring, the wooded hillsides, the squat stone farmhouses, the barnyards, and the tall, shell-gray, isolated ruins on the ridge appeared to be standing in a single shadow.

\item [\textit{Scott Sanders}] I began life by supposing, as all children do, that my home ground was the world.

\item [\textit{David Macfarlane}] These people come in from out there.

\item [\textit{Lewis Hyde}] The first story I have to tell is not exactly true, but it isn't exactly false, either.

\item [\textit{John Terpstra}] There is the story in a nutshell.

\item [\textit{David Michael Kwan}] Life was new then.

\item [\textit{Harold Broadkey}] I have \textsc{AIDS}.
\item [\textit{Salman Rushdie}] I was born in the city of Bombay\ldots once upon a time.

\item [\textit{Amy Tan}] The old woman remembered a swan she had bought many years ago in Shanghai for a foolish sum.

\item [\textit{Philip Roth}] I learned about the other Philip Roth in January 1988, a few days after the New Year, when my cousin Apter telephoned me in New York to say that Israeli radio had reported that I was in Jerusalem attending the trial of John Demjanjuk, the man alleged to be Ivan the Terrible of Treblinka.

\item [\textit{Maxine Hong Kingston}] ``You must not tell anyone,'' my mother said, ``what I am about to tell you.''

\item [\textit{Philip K. Dick}] In 1932 in April a small boy and his mother and father waited on an Oakland, California pier for the San Francisco ferry.

\item [\textit{Antoine de Saint-Exup\'ery}] Once again I had found myself in the presence of a great truth and had failed to recognize it.

\item [\textit{David Macfarlane}] I remember the night my grandfather died.

\end{description}
\chapter{Ethical Considerations}
\begin{epigraphs}
\qitem{To light a candle is to \\cast a shadow.}{\textit{A Wizard of Earthsea} \\\textsc{Ursula Le Guin}}
\end{epigraphs}
\section{Ethical Dilemmas}For the sake of simplicity and clarity, the following ethical dilemmas have been stripped of identifying facts; but they have not been disguised in any way. Each of the following situations is real. You may easily be able to identify some of the authors involved.
\subsection{Ethical Dilemma 1}
You are a board member of an organization that supports rights and freedoms worldwide. Part of the organization's mandate is to provide assistance, including funding, to writers and artists who find themselves at odds with legal systems and governments that do not recognize freedom of expression. Past funding recipients have included imprisoned writers, exiled writers, and those at odds with various political movements. Funding takes the form of grants that are decided upon by a committee that evaluates nominations each year. A typical grant is between one thousand and ten thousand dollars. You are a member of the evaluation committee.

One of the applications for a 2004 grant involves a Canadian who is writing about a famous criminal case in which the details were disturbing enough that a ban upon publication was issued. The writer in question is alleged to have broken this publication ban and now faces 97 criminal charges laid by the Canadian government. The writer has published two books about the criminal case, each of which contains unique details derived from privileged court documents. The writer feels he is being unfairly persecuted, and is seeking legal funding assistance for his upcoming court battle.

Should he get it?

\subsection{Ethical Dilemma 2}
As a struggling writer, you frequently do other kinds of work to support yourself: ditch-digging, laundry service, waiting tables. Sometimes you do editing for other writers. You take on an editing project for a writer who could generously be called eccentric. He has good ideas, but his his writing is far from good. You work with him, and you craft a book together. But the vast majority of the writing is actually done by you. All but the last twenty pages is entirely your work.

The book is published, but with no mention of your involvement. You receive no credit whatsoever. After its release, the book is met with critical acclaim and wins a prestigious literary prize. The ‘author’ (in fact the author is essentially you, but the author in name is the other person) refrains from mentioning your involvement whenever he is interviewed about the book and the process of writing it.

You become frustrated, and decide to come forward. You ask the `author' why he has not credited you. In response, he writes you a letter (in which he proves his poor writing skill), and says, among other things, that you are ``dull, colourless, vulgar, and a complete failure.'' He speaks of legal action, saying ``rest assured my campaign has just began [sic]. I have set aside fifty thousand of my hard-earned money to ruin you.''

What do you do?

\subsection{Ethical Dilemma 3}
You are working on a contemporary novel of interconnected stories based on your cultural and religious background. The religion of your background, like every religion, has strict adherents, moderates, non-practicing members, and a few extremists. The novel involves a revisioning of some of the religious stories of your background, and you are aware that some of what you write may be considered blasphemous to strict adherents (many of whom live in your community) and certainly to extremists. Moderates, on the other hand, will welcome your work as an invitation to conversation about the ways in which religions must be updated and adapted to the modern world.

What do you do?

\subsection{Ethical Dilemma 4}
You write a non-fiction book that explores alternative historical ideas. You complete exhaustive, original research. You present, for the first time in print, a compelling alternative view of one of the basic threads of accepted religious history. Your book does well.
Twenty0 years pass. A new book, a novel that clearly builds upon your ideas, is published. The ideas and themes in the new book are clearly derived from your work. It's as if the author simply took your ideas, adapted them slightly into different language, and added a novelistic plot. The new book is, in fact, a detective story.

In interviews, the author of the novel acts as though he did the research, he discovered these interesting threads of alternative history, he is the scholar and innovator. From your point of view, he has stolen your intellectual property.

What do you do?

\subsection{Ethical Dilemma 5}
You are a member of a culture with a contemporary history of trauma and appropriation (let us say, broadly speaking, that yours is an aboriginal culture). A poet and storyteller (someone not from your culture) shows great interest in the myths and stories of your people. He makes friends with you, your family, and your extended community. You recognize that he is ideally suited for the tasks of transcribing and interpreting the archaic myths of your culture --- which, if something is not done, will soon be lost. Such myths, after all, tend to fade away during times of great cultural change.

The poet writes several books of and about your cultural myths. These books do well: they are professionally produced and do not appear to have been done solely in service of the poet's own career.
Additionally, the books provide a contextualization of your cultural myths in terms of archetypal or universal themes in other literatures. In other words, the poet succeeds in securing for your myths a unique place in world literature. No one in your culture possesses the literary background to accomplish this important task.

However. After the books come out, you hear grumblings from various people in your community about the things the poet got wrong, how he told incomplete or improper tales, how he failed to consult the proper people. Slowly, animosity grows in your community, and the poet is no longer entirely welcome there.

Where do you stand?

(And, turn it around: imagine you are the poet.)
\newpage
\section{Typical Ethical Domains in Creative Non-fiction}
\begin{center}
\begin{tabular}{rl}
Writing Domain   & Ethical Domain\\[12pt]
Family           & Privacy (e.g. Allen Ginsberg)\\
Adaptation       & Truthfulness (e.g. James Frey)\\
Culture          & Appropriation (e.g. Robert Bringhurst)\\
Contracts        & Commerce vs.\ creativity\\
Blogging         & Copyright, liability\\
Copyright        & Rights vs.\ access\\
Inspiration      & Attribution (e.g. Dan Brown)\\
\end{tabular}
\end{center}
\section{Ethical Guidelines of Ross A. Laird, author}
I remind myself to honor the creative spirit in my own life and in the lives of others. I acknowledge the fragility and resilience of that spirit, and I seek to increase its presence in everything I do. I recognize that my creative work exists within an interconnected web of relationships. My own creative process is one aspect of a much larger
cultural, social, familial and interpersonal network in which numerous divergent values must be honored. The quality and impact of my writing depends on my ability to navigate this network with honesty and clarity.

A central obligation of my creativity involves my willingness to explore and refine the ethical dimensions of my work, to challenge my assumptions and biases, to expand my awareness so that it includes many other features of my internal and external world. The essence of my responsibility, therefore, is to be aware.

I recognize that all rights are accompanied by matching obligations. In the context of creative writing, this means that I amplify my ethical awareness as I increase my expressiveness. When I seek readership, I make an equivalent challenge to enhancing my own integrity. As I achieve greater exposure, my ethical responsibility grows.

I try to balance obligations to myself, to the work, and to society. I accept the responsibility to be aware of the consequences of my work and to consider its impact. Whenever possible, I honor my own creative spirit while minimizing harm to others. When obligations to myself and to the work conflict with my obligation to minimize harm to others, I seek appropriate feedback from colleagues and peers.

I recognize that I am accountable for my written words in the same way that I am accountable for my speech. Therefore, I seek to be fair, honest, trustworthy and sensitive to the ethical dimensions in my writing. I recognize that writing has power, for good and for ill, and I choose to exert that power toward the common good --- as I understand it through conscious engagement with myself and others.

I respect the dignity and rights of others: characters, readers, subjects, and other interested parties. Whenever possible, I acquire informed consent from those who may be affected by my work.

I recognize that creative writing, like every profession or vocation, is founded on core competencies. Therefore, I continue my professional development and consistently strive to improve the quality of my work. I pay particular attention to the feedback I receive from others, and I balance this with my own perspectives and values. I explore new legal and ethical considerations as they arise, and I incorporate them into my ethical framework.

I believe that the foundation of creativity is a joyful engagement with mystery. My ethical obligations include a sensitivity to that mystery, a nurturing of it, a resistance to its ultimate resolution. I remember that the ethics of a creative artist are the means by which mystery is ushered into the world.
\chapter{Essays}
\begin{quotation}
Any work of art makes one very simple demand on anyone who genuinely wants to get in touch with it. And that is to stop. You've got to stop what you're doing, what you're thinking, and what you're expecting and just be there for however long it takes. 
\end{quotation}
{\textit{The Fine Art of Cabinetmaking} \\
  \textsc{James Krenov}}
\newpage
\vspace*{5cm}
\section{Annie Dillard, \textit{Seeing}}
\subsection{A meditation on nature, meaning, and memory}
\newpage
\addtocounter{page}{+9}
\vspace*{5cm}
\section{Jorge Luis Borges \textit{Blindness}}
\subsection{A reverie of literature, metaphor, and perception}
\newpage
\addtocounter{page}{+7}
\vspace*{5cm}
\section{Chris Hedges, \textit{War Is a Force That Gives Us Meaning}}
\subsection{Reflections on death, love, loss and war}
\newpage
\addtocounter{page}{+16}
\vspace*{5cm}
\section{Scott Russell Sanders, \textit{Under the Influence}}
\subsection{A memoir of addiction, family, and meaning}
\newpage
\addtocounter{page}{+8}
\vspace*{5cm}
\section{William Langewiesche, \textit{Inside the Sky}}
\subsection{A meditation on flight}
\newpage
\addtocounter{page}{+14}
\vspace*{5cm}
\section{E.B. White, \textit{Once More to the Lake}}
\subsection{On memory, meaning, aging and death}
\newpage
\addtocounter{page}{+4}
\vspace*{5cm}
\section{Wendell Berry, \textit{An Entrance to the Woods}}
\subsection{On nature, life, and purpose}
\addtocounter{page}{+6}
\newpage
\section{Ross A. Laird, \textit{Myths of the Primordial Waters}}
\subsection{Ancient Mariners, Human Migration, and the Sea}

\lettrine[lhang=0.4, nindent=-2pt]{\textcolor[gray]{0.1}{P}}{lato
wrote that the past } is like the wake behind a boat; it spreads,
and diminishes behind us, and merges with the surrounding sea. The
past rolls under and is gone.

We stand upon the foredeck of Plato's boat, gazing forward, cleaving
our path toward the future. Along the track of our traveling many
things are lost -- because we are always searching ahead, because the
wake is jostling and turbulent, because our craft is small and the
ocean is vast.

It is by means of this manner of journeying into the future that our
knowledge of ancient peoples is vanishingly small. We know a fair
amount about the last thousand years of our history, we surmise a
sketch of the thousand years before that -- and of the remote ages
before that, we know very little. Snatches, really, vignettes gathered
from scattered documents and fragmentary tales. For the great majority
of the history of modern humans --- a hundred thousand years, two
hundred thousand, no one knows --- we understand almost nothing. Along
our own coasts, which once were at lower altitude than they are now,
ancient villages lie hidden beneath the wake of passing boats above.

And yet, old stories have been handed down from that long, invisible
stretch of years: fables, epics, mythologies of archaic and unknown
origin. Among those ancient tales is a set of related motifs, from
many cultures, that tell of seafarers who found their way to distant
shores. In China, Polynesia, Japan, Egypt, Africa, Scandinavia --- in
most places bordered by the sea -- we find fantastic tales of oceanic
travel. On our own coasts --- in Haida Gwaii, and along the sheltered
eastern shore of Vancouver Island, and inland all the way to the
Kootenays -- similar stories are told of those who came long ago, and
lived upon the land, and vanished.

For at least a century, since archaeology and anthropology became
sciences based on hard evidence, such cultural tales have been
dismissed as folklore and wishful thinking. The evidence simply did
not support the stories. The timelines claimed by various cultures
seemed inconsistent with what was surmised about technologies and
methods from various historical and pre-historical periods. The ruins
of ancient sites could not be found (near Atlin, for example, or near
Telkwa, both sites where aboriginal tales describe cities of utmost
antiquity). The longevity of known sites could not be established from
existing data (the Nanaimo petroglyphs, for example). Eventually, the
scientific consensus was that the claims of myth were just that:
imagined tales, with no actual basis.

But within about the last decade, a wealth of new evidence challenges,
and will likely soon overturn, traditional scientific views concerning
human migration in the ancient world. The emerging data comes from
various fields: genetics, archaeology, anthropology, linguistics, and
the developing field of archaeo-astronomy. Working sometimes in
concert and other times in conflict, these fields are leading us
through a fundamental paradigm shift in our perspective of the past.

The history of science consistently confirms something we easily
forget: that most of our certainties will turn out to be wrong. What's
turning out to be wrong at the moment is our conception of the
peopling of the Americas. The standard theory --- the Bering land
bridge, ice-free corridors, southward migration --- has begun to give
way to a more nuanced and complex view involving multiple waves of
ancient immigrants arriving at different times and by disparate means.

Debates and developments within the scientific community typically
take place in closed meetings at universities and at conferences not
attended by the general public. But the conversation about ancient
migration has become very public since the 1996 discovery of a
skeleton known as Kennewick Man. He was found on the banks of the
Columbia River, in Washington State, by a pair of spectators watching
hydroplane races. Initially, local aboriginal groups claimed him as
one of their own; an ancestor, perhaps a fallen warrior from long ago.

But archaeologists who studied Kennewick Man found a curious thing: he
is not aboriginal. His remains are old --- approximately 9,300 years
old --- but he is not an ancestor of any current aboriginal population.
In fact, he's Asian. He may be an ancestor of modern Pacific
Islanders, or of the Ainu people of northeast Asia. In either case, he
traveled here more than ten thousand years ago, likely with a small
population of others like him who made their careful way inland and
across the Pacific Northwest.

Kennewick Man is not the only oddity of the ancient human landscape.
Many so-called anomalous remains and sites have been found in both
North and South America: Monte Verde, for example, in Chile, and the
entire collection of colossal stone remains in Mexico known as the
Olmec culture. Along British Columbia's Inside Passage, near the
Yuculta rapids, stone sculptors carved somber faces into twenty-six
granite boulders on the shore, more than at any other site on the
Pacific coast. The carvers are long gone, vanished but for these stone
traces of mystery.

As the number of anomalies has accumulated, the trajectory of the
scientific conversation has changed too: from dismissal, to caution,
to contention, and finally to a new consensus. That final, new
consensus has not yet fully emerged, but its basic elements are
already in place: many groups of migrating people came to North and
South America --- ten, twenty, perhaps as much as thirty thousand years
ago --- in separate and commingling waves of odyssey, exile, and
accident.

And how did they come? By boat.

Imagine those ancient mariners, navigating by the stars, uncertain of
their destination, traveling in what might have been open canoes or
out-rigged rafts or makeshift kayaks. No compass, no map, no
protection against the sea's indifference. Nothing but sheer guts and
necessity.

They came at different times and, no doubt, by varying means: from
Japan, Russia, Southeast Asia, Polynesia (likely from Europe, as
well). They established settlements here, lived upon the land for some
stretch of time, then disappeared. Perhaps they were subsumed into
existing or descendant groups. Perhaps most of them were wiped out by
an asteroid impact 13,000 years ago (as one recent theory suggests).
But no one knows. The descendants of the original, pre-migration
peoples still exist in Japan, Russia, and Polynesia. They are the
Ainu, the Jomon, the Polynesians; and they are still here, thousands
of years after small clusters of their people sailed across the sea.

The puzzle of the most archaic groups is deepened by the fact that sea
levels are now as much as 30 metres higher than they were 10,000 or
more years ago. Villages that once lay at the seaside are now long
immersed, swept by the amnesia of the waters, erased beneath Plato's
persistent wake.

However, anomalous underwater stone sites have been found in Japan,
Cuba, Malta, Egypt, and elsewhere. After the 2004 Indian Ocean
earthquake, stone artifacts from an ancient and fabled submerged city,
once dismissed by archaeologists as mythological, were washed up on
the beach by the force of the tsunami. These artifacts include
six-foot high statues of the head and shoulders of an elephant, a
horse in flight, and a reclining lion.

In Haida Gwaii, traditional myths tell of the ancient rise of the sea,
of ice floes moving across the land, of sudden and drastic upheavals
that transformed the islands. And those Haida myths also speak of an
earlier people, now gone, who inhabited that mystic place long ago,
and of whom nothing is now left but ghosts.

Those ghosts take many contemporary forms: the sea-wolf petroglyph
south of Nanaimo, the unique Christina Lake petroglyph, the funereal
mound at Keremeos, the persistent tales of the fabled city of
Dimlahamid in northern British Columbia, between the Bulkley and
Skeena rivers. And Kennewick Man, of course, who may have known, when
he was alive, the meaning of the stone sculptures at Yuculta, or might
himself have carved images into stones scattered across a river delta.
His people were here, after all --- in what is now Vancouver, and
Victoria, and inland by way of the rivers --- and the settlements of
our people today are laid over those of his people by thousands of
years of rainfall, wind, and memory.

And yet the ancient evidence swells, and spreads, and cannot be laid
to final rest: scattered human remains, colossal in their age;
Polynesian chicken bones found in Peru; genetic anomalies among
various cultural groups (the Scots, for example, may be descended from
ancient seafaring Egyptians).

The old boats are gone, of course, long undone by the alchemy of salt
water on wood. But the tales remain, and have not surrendered their
claims of authenticity. And now, finally, science is coming forward to
meet the mythological narrative. The new and shared story, woven
together by the threads of both science and cultural memory, is this:

No single people came first to the Americas, but instead many came, in
small sorties and great armadas, during a period of human history
about which we are profoundly ignorant. Before the Ice Age and after
it they arrived, and made homes for themselves, and left only the tiny
traces typical of the human story. Their cultures appeared and
vanished again (as our cultures will also).

These disparate groups were united by the sea, the great trackless
track that challenged and delivered them. The mariners of today are
the descendants, in spirit, of those early nomads who first harnessed
the wind. We pass over their graves, somewhere between the shore and
the deep water. Watch for that place --- 30 metres of depth --- and
recognize, as you pass over that line, the legacy you inherit: love of
the wide waters, the quest for adventure, the longing for what lies
over the horizon. These are the gifts of the vanished peoples, whom we
will never know except by the ways in which we are stirred, even now,
by their ancient dreams.


\chapter{Book Excerpts}
\begin{verse}
First, the word itself: an island.\\
Next, the blissful joining, as in love, of two words.\\
Later, finally, the whole period, like a world both closed and open,\\
containing (within itself, within itself alone)\\
the infinite.
\end{verse}
{\textit{The Complete Perfectionist}\\
\textsc{Juan Ram\'on Jim\'enez}}\\
\newpage
\vspace*{5cm}
\section{Harold Broadkey, \textit{This Wild Darkness}}
\subsection{On life and death}
\newpage
\addtocounter{page}{+5}
\vspace*{5cm}
\section{C\'esar Calvo, \textit{The Three Halves of Ino Moxo}}
\subsection{A rumination on language, consciousness, drugs, and meaning}
\newpage
\addtocounter{page}{+6}
\vspace*{5cm}
\section{John Terpstra, \textit{The Boys}}
\subsection{A meditation on health, illness, and endurance}
\newpage
\addtocounter{page}{+13}
\vspace*{5cm}
\section{Lewis Hyde, \textit{Trickster Makes This World}}
\subsection{On the nature of creativity}
\newpage
\addtocounter{page}{+8}
\vspace*{5cm}
\section{John Jerome, \textit{Stone Work}}
\subsection{On serious play, nature, and ritual}
\newpage
\addtocounter{page}{+7}
\vspace*{5cm}
\section{Maxine Hong Kingston, \textit{The Woman Warrior}}
\subsection{A memoir of culture, loss and redemption}
\newpage
\addtocounter{page}{+9}
\vspace*{5cm}
\section{Michael David Kwan, \textit{Things that Must Not be Forgotten}}
\subsection{On childhood, memory, and meaning}
\newpage
\addtocounter{page}{+9}
\vspace*{5cm}
\section{W.S. Merwin, \textit{The Mays of Ventadorn}}
\subsection{A travel memoir and poetic inquiry}
\newpage
\addtocounter{page}{+16}
\vspace*{5cm}
\section{Carlos Fuentes, \textit{The Orange Tree}}
\subsection{An imaginal recreation}
\newpage
\addtocounter{page}{+10}
\vspace*{5cm}
\section{Sharon Butala, \textit{Wild Stone Heart}}
\subsection{On nature, history, and the present}
\newpage
\addtocounter{page}{+10}
\vspace*{5cm}
\section{Scott Russell Sanders, \textit{Writing from the Center}}
\subsection{On creativity, writing, and the profession of the writer}
\newpage
\addtocounter{page}{+19}
\vspace*{5cm}
\section{Salman Rushdie, \textit{Imaginary Homelands}}
\subsection{On belonging, exile, and the meaning of literature}
\newpage
\addtocounter{page}{+3}
\vspace*{5cm}
\section{Milan Kundera, \textit{The Art of the Novel}}
\subsection{On craft, creativity, and composition}

\newpage
\addtocounter{page}{+13}
\vspace*{5cm}
\section{Italo Calvino, \textit{Six Memos for the Next Millennium}}
\subsection{On visibility and the imagination}

\newpage
\pagestyle{empty}
\vspace*{5cm}
\begin{center}
  \textsc{Colophon}\\
\vspace*{.5cm}
This reader was created by Ross A. Laird for the Kwantlen College
creative writing program. The main textual material (i.e.\ the
material not photocopied from books) was composed in the Emacs text
editor and typeset using \LaTeX with the \textit{memoir} class. The
body text is set in Kepler, \\designed by Robert Slimbach of Adobe.
\\
Named after the German Renaissance astronomer, Kepler is a
contemporary type family in the tradition of classic eighteenth
century typefaces. Traditionally, modern typefaces are known for their
cool intellectual quality, but Kepler captures the modern
style in a humanistic manner. It is elegant and refined with a hint of
oldstyle proportion and calligraphic detailing that lend it warmth
and energy.


\end{center}
\end{document}

%%% Local Variables:  %%% mode: latex %%% TeX-master: t %%% End: 