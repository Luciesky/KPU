\message{ !name(psych_app_sample_lecture.tex)}   \TeXXeTstate=1
\documentclass[12pt,DIV9,oneside,headsepline,footsepline]{scrreprt}
   % packages to support xelatex:
   \usepackage{fontspec,xunicode}
   % install these TrueType/OpenType fonts in
   %   ~/.fonts          (Linux)
   % then select them with:
   \defaultfontfeatures{Numbers=OldStyle,Scale=MatchLowercase,Mapping=tex-text}
   \setmainfont{Sabon LT Std}
   \setromanfont[Mapping=tex-text]{Sabon LT Std}
   \setsansfont[Scale=MatchUppercase,Letters=SmallCaps]{Syntax LT Std}
   \usepackage{hyperref}
   \usepackage{xltxtra}

\newfontfamily\headerfont[Scale=MatchUppercase,Letters=SmallCaps]{Syntax
LT Std}

\typearea[current]{last}
\usepackage{epigraph}
\usepackage{paralist}
\usepackage[sf]{titlesec}
\usepackage{url}
\usepackage{scrpage2}
\pagestyle{scrheadings}
\setkomafont{pagehead}{\headerfont}
\chead{Mentorship}
\ohead{\pagemark}
\cfoot{Ross A. Laird, PhD}
\usepackage{titling}
\usepackage{color}
\setcounter{secnumdepth}{-1}
\hyphenation{sl-ow think-ing wheth-er every-thing quick-ly
ex-per-ien-ce seek-er quan-dary schoo-ls thr-ead-ed ren-der-ed
tr-au-ma org-ani-zat-io-nal}

%For epigraphs
\setlength{\epigraphwidth}{.4\textwidth}
%Text formatting
\usepackage{everysel}
\clubpenalty=6000
\widowpenalty=6000

%Document details
\title{Developmental Psychology:\\Adolescent Development}
\author{Ross A. Laird, PhD}
\begin{document}

\message{ !name(psych_app_sample_lecture.tex) !offset(-3) }

\pagestyle{empty}
\pagenumbering{roman}
\begin{center}
\maketitle
\end{center}
\tableofcontents
\pagestyle{scrheadings}
\chapter{Cultural Identity and the Body}
\pagenumbering{arabic}
\begin{flushleft}
Ross A. Laird, PhD\\
\url{http://www.rosslaird.info/}
\end{flushleft}

\section{Developmental Imprinting and Incomplete Integration: A Quick
Review}

Early in their lives, from about birth to age twelve, children pass
through roughly seven stages of development (the number of stages
varies depending on the developmental model). These stages have to do
with themes such as belonging, trust, safety, empowerment,
self-expression, and so on. Typically, some of these stages go well
for the child whereas others are more difficult. If a given stage is
difficult, the child may not fully learn the psychological tasks of
that stage. For example, a child who experiences significant illness
in the first year of life is more likely to feel anxiety about need
fulfillment than another child who does not have the same experience
(This is because need fulfillment is the theme of roughly the first
year, and problems during that year tend to impact that particular
theme.)


Everyone is shaped by these developmental stages. In fact, these
stages are the single most important factor in determining a person's
character. This is the essential basis of modern psychology, and it's
an idea supported by enough research -- a mountain of research -- as
to be beyond dispute. Essentially, our basic character is formed by
the time we are four years old. But our childhood development never
unfolds perfectly. Everyone undergoes developmental themes that are
less than ideal. When this happens, the child gets through the stage
and moves on to the next one. The stage is left unfinished and the
theme is incomplete. Children cannot afford to get stuck in one stage
too long, so they leave unfinished themes behind and try to catch up
with them later. 

Adolescence (which now spans from about age 9 to about age 32) is the
developmental phase of catching up with and resolving unfinished
themes. Starting around age nine -- with a process known as brain
pruning -- children begin to revisit the unfinished themes of their
earlier development. (They do this unconsciously, but it manifests as
rapid mood cycling.) 

\subsection{Research Details and Links}

\begin{itemize} \item [Association for Psychological Science] (2007,  
March 5). Genes And Stressed-out Parents Lead To Shy Kids. Retrieved
May 1, 2008, from \\ 
\url{www.sciencedaily.com/releases/2007/03/070302111100.htm}

\item [JAMA and Archives Journals] (2007, January 3). Child Abuse And
Neglect Associated With Increased Risk Of Depression Among Young
Adults. Retrieved May 1, 2008, from \\
\url{www.sciencedaily.com/releases/2007/01/070102092229.htm}

\item [Oregon Health and Science University] (2005, November 16).
Research Reveals Likely Connection Between Early-life Stress And
Teenage Mental Health Problems. Retrieved May 1, 2008, from \\ 
\url{www.sciencedaily.com/releases/2005/11/051116174754.htm}

\item [University of New South Wales] (2006, March 2). Nature, Nurture
And The Risk Of Depression. Retrieved May 1, 2008, from \\ 
\url{www.sciencedaily.com/releases/2006/03/060301091020.htm}

\item [Society for Research in Child Development] (2006, March 22).
Effects Of Preterm Birth And Early Environmental Risks Continue Into
Adolescence. Retrieved May 1, 2008, from \\
\url{www.sciencedaily.com/releases/2006/03/060322141815.htm}

\item [UK Institute of Psychiatry] (2006, February 27). Prematurity
Affects Personality. Retrieved May 1, 2008 from \\
\url{http://news.bbc.co.uk/2/hi/health/4747694.stm}

\end{itemize}


\section{Adolescent Development and the Nervous System}

The nervous system possess habits of consciousness and action.
These are developmental, and might be (somewhat arbitrarily) grouped
under four themes:

\begin{itemize}
\item Flight Response
\item Freeze Response
\item Orient Response
\item Fight Response
\end{itemize}

These imprints are learned in the first few years of life. They
control a great deal of our behavior, emotion, and consciousness. The
developmental features associated with the nervous system are among
the most exhaustively researched aspects of psychology (especially
with regard to trauma). 

During the first four phases of childhood development, the four states
of the nervous system are imprinted and tuned. This happens by way of
parenting, immersion in the environment, genetic predisposition, and
various other factors (some of which are still unknown).

\subsection{Belonging}

Flight response is correlated with belonging (roughly from birth to
one month). If an individual does not experience a sense of belonging,
he or she will withdraw (psychologically and physically), and will
seek ways of adapting through imagination and inner resourcing. (Cold
hands and feet are one symptom of this withdrawal, as is adolescent
cutting.) In adolescence, this adaptive mechanism makes such
individuals prone to hallucinogen addiction, addiction to the
imagination, and addiction to the technologies of fantasy.

\subsection{Need Fulfilment}

Freeze response is correlated with need fulfilment (roughly one month
to eight months). If an individual is abused or neglected during this
period (any period, really), he or she will adapt by surrendering
needs or fixating on specific needs (such as food). Surrender and
fixation are two aspects of nervous system freezing. Surrender and
fixation are two aspects of opiate addictions, which are
developmentally predisposed during this period of development. If
individuals with lingering vulnerabilities from this stage go on to
develop technology addictions in adolescence, those addictions will be
focused toward online shopping, text messaging, image viewing (e.g.
pornography) and television watching.

\subsection{Autonomy}

The developmental stage of negotiating the relationship between self
and other (which occupies the period roughly from 8 months to 1.5
years) involves significant milestones of movement, exploration,
personal challenge, and orienting. If an individual does not receive
balanced imprinting at this stage, he or she will tend to become
hyper-vigilant and hyperactive (not all hyperactivity is derived from
this stage, however). One symptom of this adaptation is a craving for
excitement and newness. If such individuals go on to develop
addictions in adolescence, those addictions are more likely to involve
stimulants. If the addictions involve technology, the individual will
likely be drawn to stimulating video games, online gambling, and
extreme immersive environments.

\subsection{Will and Power}

Between two and four years of age, individuals negotiate their
relationship to their own power. It has been well-established that
domestic violence and corporal punishment at this age are highly
correlated with developmental and lifespan difficulties. Such
difficulties are not only psychological: the risk of adolescent and
adult obesity is increased (by fifty per cent) by the experience of
childhood neglect. The fight response is developed and tuned at this
stage. For those who will develop addictions in adolescence, the
experience of neglect and abuse of power in childhood creates the
predisposition toward alcoholism. This is why the rates of alcoholism
are so high in war-torn countries and in cultures where cultural power
has been destroyed. In terms of technology addictions, such
predispositions are likely to involve addictions to video games
involving fighting.

\subsection{The Link to Adolescent Addictions}

Addiction involves uncompleted impulses and fractured imprinting
typically derived from childhood experience (this is not universally
the case, but is almost universally the case). The nature of the
addiction involves the way in which the addiction completes,
temporarily, the unfinished imprinting:

\begin{itemize}

\item Flight response addictions allow one to fly away
\item Freeze response addictions enable stillness and solace 
\item Orienting response addictions stimulate action and exploration 
\item Fight response addictions enable the illusion of empowerment
\end{itemize}

The more childhood difficulty an individual experiences, the more
likely the individual is to seek multiple addictions in adolescence.

\subsection{A Note on Addictions Predisposition}

Adolescence begins with the brain pruning stage at roughly age eleven
and continues until the end of the twenties (for the youth of today).
This long period of development involves the integration of previous
developmental stages. Incomplete or fragmented childhood imprinting
re-emerges as adolescent psychological difficulty. Addiction is one
method of easing the stress of such unfinished imprinting~-- by
completing it temporarily. 

\subsection{Research Details and Links}

\begin{itemize}

\item [Ross A. Laird] (2008). Labyrinth: Addictions and the Search for
Healing. Vancouver.

\item [Peter A. Levine] (1997). Waking the Tiger: Healing Trauma. San
Francisco, North Atlantic Books.

\item [Yale University] (2003, June 19). Adolescents Are
Neurologically More Vulnerable To Addictions. Retrieved May 1, 2008,
from \\
\url{www.sciencedaily.com/releases/2003/06/030619075547.htm}

\item [SAGE Publications] (2005, December 30). Bullying In Middle
School May Lead To Increased Substance Abuse In High School. Retrieved
May 1, 2008, from \\
\url{www.sciencedaily.com/releases/2005/12/051230085006.htm}

\item [National Institute of Mental Health] (2008, January 8).
Scientists Can Predict Psychotic Illness In Up To 80 Percent Of
High-risk Youth. Retrieved May 1, 2008, from \\
\url{www.sciencedaily.com/releases/2008/01/080107181615.htm}

\item [National Institute on Alcohol Abuse and Alcoholism] (2006, July
4). Early Drinking Linked To Higher Lifetime Alcoholism Risk.
Retrieved May 1, 2008, from \\
\url{www.sciencedaily.com/releases/2006/07/060703211131.htm}

\item [JAMA and Archives Journals] (2006, September 5). Drinking
During Pregnancy Linked To Offspring's Risk Of Alcohol Disorders In
Early Adulthood. Retrieved May 1, 2008, from \\
\url{www.sciencedaily.com/releases/2006/09/060904160530.htm}

\item [The Bodynamic Institute]. \\
\url{http://www.bodynamicusa.com/}

\end{itemize}

\section{Mentorship and Belonging}

Adolescence is typically the most pivotal phase of a person's life. We
decide, often without recognizing it, our trajectory into the world.
And how we enter is how we go on. Adolescence is the first tentative
step forward, the juncture at which we establish our speed and
direction and even our purpose. The character of our movement is
defined. And that character is shaped by mentorship more than by any
other force. The mentor might be a parent, or grandparent, or friend,
or coach~-- it doesn't matter much. But it must be someone whose
temperament coaxes from us our better nature.

Without mentorship a child becomes a wanderer in a strange country.

At an indistinct age~-- fourteen, fifteen, perhaps as late as
seventeen~-- most kids seek mentors and guides who are not parental.
The horizon of adolescents opens, and they enter a wider world.
Historically, grandparents have been the ushers and guides of kids at
this delicate stage. But in the modern age grandparents are often
absent, or disconnected from the child's reality. In the wake of such
absence, and without alternate mentoring provided by school teachers
or coaches or spiritual leaders in the community, teens turn to one
another. Sometimes they form a youth gang and choose the most vicious
among them to be their mentor and guide. And the first thing such
mentors wish to do is get high.

The mentor, perhaps more than any other social role, is in a unique
position to influence, in fundamental and lasting ways, the entire
lifespan of a developing child. This is a sacred trust, a gift of
engagement offered by the generous spirit of childhood.

The mentor's task is to witness, to trust in the spirit of healing, to
offer honesty and compassion. And to offer it to the defiant, the
truculent, the dismissive, the unready and the unsteady in equal
measure. Nothing less.

In the oldest Egyptian tombs and temples that have been unearthed, in
rooms festooned with hieroglyphics, in texts that lay undeciphered for
five thousand years, one may read of an ancient god who is the bringer
of knowledge and of illumination. He is the mythological ancestor of
Merlin, of Gandalf, and of the many guides and mentors who populate
the old tales of every culture. He is the original storyteller, the
inventor of writing, the trickster and wayfinder. His name is Thoth.
The Greeks called him Hermes. He illuminates the labyrinths, the lost
and switchbacking tunnels, and he is keeper of the great and hidden
library.

Mentors today assume the storied mantle of the wayfinder. 


\section{Mentorship and Identity Development}

The only way for an adolescent to develop integration, containment,
and identity is through mentorship. The impulse of kids to form groups
is healthy. In evolutionary terms, groups of young people seek
leadership from adult mentors. In the absence of healthy adult
mentors, adolescents form a youth gang, which comes to be led by the
adolescent among them who is most aggressive, gregarious, or
risk-prone. The absence of mentorship for adolescents is the most
serious problem in our society today. Absence of mentorship is a
primary cause of the addictions problem among both youth and adults,
the suicide problem among youth, the homelessness problem in youth and
adults, and the depression and anxiety problem of many people.

\subsection{Mentorship Tasks}

A mentor is someone who can assist a child to complete their
unfinished childhood themes and to further develop their character.
After parenting, it is the most important role a human being can
undertake (despite the low status it earns). A good mentor encourages
an adolescent (or child) to feel safe, to take appropriate risks, to
express whatever remains unexpressed. Mentorship does not have to be a
long-term intervention. An adolescent can undergo a transformative
experience in a single meeting with a good mentor. One outstanding
experience is enough to complete the learning for an entire unfinished
developmental stage. (This is a possible but not common experience.)

Mentoring requires immense sensitivity and interpersonal skill. Just
as a good mentor can profoundly influence a child or adolescent, so
can a poor one. An inappropriate mentorship experience can severely
damage the psychological development of a child. Mentorship is a
trust, a role that is profound and powerful. It is a gift offered to
us by children. Usually, parents cannot fulfill the mentorship role,
which requires a balance of deep caring and emotional neutrality.
Parents possess deep caring, but they cannot be neutral about the
choices their children make.

\subsection{Mentorship for the Body-Mind}

We live within a scientific context that is almost completely
brain-centered. In many ways, our hyper-focus on the brain allows us
to forget that the brain is only part of the larger nervous system,
which in turn is part of the body-mind. Body and mind, as research
consistently affirms, cannot be separated. And healthy development, of
course, involves the entire body-mind.

One of the ways to simplify the immense complexities of the body-mind
system is to use terminologies of the nervous system. These in turn
can be grouped into mentorship roles:

\begin{itemize}
\item Flight response mentorship encourages trust, safety, and
  belonging
\item Freeze response mentorship encourages need fulfilment and solace
\item Orienting response mentorship encourages healthy action and
  exploration 
\item Fight response mentorship encourages healthy empowerment
\end{itemize}

Mentorship involves both physical and psychological work. The nervous
system of the developing adolescent must be addressed on a physical
level, through activity, as well as on an interpersonal level.

The essential goal of adolescent mentorship is twofold: To assist
youth in completing the incomplete or fragmented nervous system
imprinting from childhood, and to assist youth in expanding their
range of choice of action through recognizing and broadening nervous
system habits (for example, many fighters need to learn how to freeze
or flee, many freezers need to fight or flee, and many fleers need to
freeze or fight.)

\subsection{Research Details and Links}

\begin{itemize}

\item [Ross A. Laird] (2008). Labyrinth: Addictions and the Search for
Healing. Vancouver.

\item [Ross A. Laird] (2008). Adolescent Mentorship Resource Guide. \\
\url{http://www.rosslaird.info/node/347}

\item [Ross a. Laird] (2008). Technology Addictions, Mentorship, and
the Link to Substance Abuse. \\
\url{http://rosslaird.com/node/361}

\item [Iowa State University] (2007, November 14). How Violent Video
Games Are Exemplary Aggression Teachers. Retrieved May 1, 2008, from
\\
\url{www.sciencedaily.com/releases/2007/11/071113160359.htm}

\item [University of Pittsburgh Medical Center] (2006, February 6).
Course Of Bipolar Disorder In Youths Described For The First Time.
Retrieved May 1, 2008, from \\
\url{www.sciencedaily.com­ /releases/2006/02/060206230600.htm}

\item [Potts, Jonathan] (2005, October 20). Humans are Governed by
Emotions -- Literally. Carnegie Mellon University. \\
\url{www.eurekalert.org/pub_releases/2005-10/cmu-hag102005.php}

\end{itemize}

\section{Cultural Inclusion, Adolescent Tribalisms, and Myth}

The recursive developmental themes of adolescence tend to provoke
young people toward themes of inclusion and belonging (because these
are the basis of identity and are imprinted in the first age of
childhood development). The foundational topics are belonging, trust,
safety, and community.

Adolescence is the period of mythological identity formation and
inclusion. Myth, in this sense, refers to the larger social narratives
into which the developing child is ushered. Cultural myths and symbols
are never fashioned anew, out of whole cloth, by parents and other
storytellers. Myths are adapted, modified, edited, recast into forms
sufficiently specific to a given culture as to permit the claim of
authenticity. But the thread goes back, always back, to ancestral
bards, claimed or disowned, who themselves borrowed the tales from a
well of souls that has no bottom.

Myths come in layers, in nested boxes.

The narratives of cultural myth claim the storied child. \emph{Look,
he has his father’s eyes. Her mother’s hands. One of us. We will make
a place for you, who have come to the river, the manger, the cave of
wonders. We will claim you as our own, we will defend your place among
us against those who would challenge it. We will believe the tale that
you were born within the circle of our relations. We will forget that
we have seen your brothers dancing in the fields of our enemies.}

The claiming -- the tribalism, the mentorship -- of adolescents by
peers and by the communities of their choosing leads to the motif of
the badge. Cultural and ancestral norms in human society have
traditionally involved ritual scarification -- the badging, or
branding, even -- of adolescents as a means of group identity and
identification. Such marks indicate where one belongs.

Adolescents seek badges as essential aspects of their identity
formation. From about age twelve onward (after puberty has begun,
usually), adolescents will seek to join groups that are not
necessarily affiliated with parental groups. Such groups are new
cultures of inclusion, and provide the ground in which the adolescent
will explore and develop themes of identity.

In traditional societies, badging and branding tend to follow
prescribed routes of gender and cultural tradition. In those
societies, tattooing and scarification are ways of infusing ancestral
magic into the body and spirit of a young person. Although modern,
urban societies use different language, the impulse is the same. 

\subsection{Research Details and Links}

\begin{itemize}
\item [Ross A. Laird] (2002). A Stone's Throw: The Enduring Nature of
Myth. Toronto: McClelland and Stewart.
\item [Ross A. Laird] (2007). The Evolution of Addictions. \\
\url{http://rosslaird.com/node/180}

\item [Morris, Desmond] (1999). Bodyguards: Protective Amulets and
Charms. Boston: Element Books.

\item [Levi-Strauss, Claude.] (1988). The Way of the Masks. Vancouver:
UBC Press.

\end{itemize}

\section{Branding, Inclusion, and Pop Culture}

Various modern cultures utilize tattooing, piercing, and other types
of decoration and scarification to broadcast group inclusion in
adolescence. (The older view that modern scarification is associated
with mental health challenges has been widely discredited.) A few
cultures and groups of note are listed below:

\begin{itemize}
\item [Goth culture].
\url{http://en.wikipedia.org/wiki/Goth_subculture}
\item [Japanese Irezumi]. 
\url{http://en.wikipedia.org/wiki/Irezumi}
\item [Low Back tattooing].
\url{http://en.wikipedia.org/wiki/Lower_back_tattoo}
\item [Piercing.] \url{http://en.wikipedia.org/wiki/Piercing}
\item [Punk.] \url{http://en.wikipedia.org/wiki/Punk_subculture}

\end{itemize}

An argument can be made that clothing logos and similar types of
accessory branding are aspects of adolescent (and adult) virtual
scarification. 

Athlete tattoos are a common trend, as are Canadian flag tattoos. Both
are (minor) scarification rituals. See, for example:\\

\url{http://sports.aol.com/photos/tattoos}\\



Slow, careful, and non-intrusive development of trust and safety.
Nurturing of a sense of belonging in a community of caring and
support. Emphasis on integrating imagination into actual work.

As a result of complexities in the modern world, the achievement of
adulthood has shifted from age 19 to age 35 since the Second World
War. The central task of this stage is to integrate one's life
experience, including the unresolved childhood themes, and to develop
a sense of the path one will choose in life. Broadly speaking, this is
consistent with what psychologists call the adult ego, or adult
observing ego. This stage is the beginning of one’s ``life wisdom."

These are the stages of adult development:

\begin{description}
\item[12 to 19:] First integration of childhood themes.
\item[19 to 28:] Transition to adult ego. (Myelin development
  continues until about age 25---30.)
\item[28 to 32:] Choosing of life path.
\item[32 to 35:] Final choices toward adulthood.
\item[35:] Adulthood!
\end{description}

These are complex developmental stages during which mentors are
required. The role of the mentor in the life of the developing adult
is to be supportive, to guide without coercion, to invite a sense of
openness and possibility. But the mentor also must assist the
developing person to grapple with difficult questions. Here are a few:

\begin{list}{}{}
\item What remains unfinished from your childhood development, and how
does this make you vulnerable to certain kinds of moods or behaviors?
\item In what ways do you get stuck?
\item What are your deepest values and beliefs? How are you going to
manifest them?
\item What is your experience of other people? How do you approach
relationships with them?
\item What is the one thing you must remember?
\item What is the one thing you must un-learn, or re-learn?
\item What are you good at? What are you so good at that it works
against you?
\item Where are you going?
\item Who are you?
\end{list}

\section{A Suggested Student Activity}

\begin{enumerate}
\item Sit quietly. Turn down your thoughts. Breathe.
\item Allow your consciousness to settle down. If you drift toward
thinking (thoughts such as \textit{this is really stupid}, for
example), focus on your breathing.
\item On a sheet of paper, write down the phrase \textit{This is what
I know}. \\
Beneath this phrase, write down several things that you know about
yourself and your life: what kind of person you are, what important
truths you have learned, what you believe (about anything). Keep it
positive.
\item Write down \textit{Who I am}. \\
Beneath this, write a few things about yourself: your culture, or
background, or interests, or career direction, or family role -- or
whatever you like. Imagine that you are describing yourself in a
nutshell.
\item Write down \textit{I am very resourceful and skilled at}\ldots
\\
Beneath this, write down a few areas in which you excel: sports, or
certain types of situations, or specific areas of knowledge, or ways
of thinking (or whatever). Come up with at least three.
\item Write down \textit{Sometimes I get stuck when}\ldots \\
Beneath this, write down two areas in your life where you have
difficulty. This might be a psychological thing, such as anxiety or
cynicism or impatience; it might be a specific type of situation, such
as interpersonal conflicts in your family, or thinking about your
career vision; it might be an odd little thing like a pet peeve. Write
down whatever works for you. No one is going to see this, unless you
choose to show it to them.
\item Write down \textit{I must remember}\ldots \\
Beneath this, try to articulate what it is that you must always
remember.
\item Write down \textit{My personal development depends on}\ldots \\
Beneath this, write down whatever you thoughts you have about your own
direction and development.
\item Find a partner. Taking one item at a time, share with one
another what you have written down. Share as much or as little as you
like. The partner's job is to listen, to be curious, to be
non-judgemental.
\item Take your paper home, put it in a safe place, and review it in
five years.
\end{enumerate}

\newpage
\section{Proposed Field Trip}

Visit to The Dutchman \\
\url{http://www.dutchman-tattoos.com/index.htm}

\end{document}


\message{ !name(psych_app_sample_lecture.tex) !offset(-630) }
