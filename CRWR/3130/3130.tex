\documentclass[letterpaper,10pt,headsepline]{scrreprt}
\usepackage[latin1]{inputenc}
\usepackage[ngerman]{babel}
\usepackage[T1]{fontenc}
\usepackage[dvips]{graphicx}
\usepackage{mathpazo}
\usepackage{microtype}
\usepackage{scrpage2}
\clubpenalty=6000
\widowpenalty=6000
\author{Ross Laird}
\title{Creative Writing 3130}
\date{10/01/07}
\ohead{Creative Writing 3130}
\pagestyle{scrheadings}
\setcounter{secnumdepth}{-1}
\begin{document}
\section{Creative Writing 3130}
Instructor: Ross Laird, Ph.D.\\ 
Telephone: 604-916-1675\\
Email: ross@rosslaird.info\\
Website: www.rosslaird.info\\
Location: Kwantlen University College\\
Fall 2008\\

\subsection{Basic Philosophy of the Course}
Creative writing is a powerful, ancient, and yet delicate practice. We
write~--- quietly, often in isolation, in tentative and mercurial
moods. (Quick tip: an em-dash is often more effective than a colon
when used before a short list, as in the sentence above.) We revise,
and turn back upon our own narratives, and wonder about the reception
our work might meet in the world. Sometimes we hide manuscripts in
drawers, or take deliberate action~--- as did Franz Kafka and Mahatma
Gandhi~--- to prevent our words from making their way to an audience.
Kafka and Gandhi were both unsuccessful in preventing their writings
from being destroyed; but their impulse to do so, to keep hooded the
hawk of their creativity, is common among writers of all stripes.

We're not sure that we have, really, anything to say; or we are afraid
that if our words are not well met we might ourselves be wounded. Or
we believe, as did the ancient Egyptians, that words have their own
life, for good or for ill, and that writing is a means of seizing the
power of the gods. This course attempts to explore this conversation
~--- between the writer and the wider world~--- and to find ways of
bringing our writing safely out of hiding.

We will be exploring craft, and method, and the strategic practices
every writer must learn in wrestling with narrative. Each of us will
examine our strengths~--- the ways in which the natural mood and
flavour of our writing makes itself known~--- and our vulnerabilities
as well: how we get stuck, or lazy, how we lost confidence and gain
doubt. How we learn to shut down and hope the whole thing will go
away. (Quick craft tip: a sentence does not always require a
traditional subject-verb structure. Sometimes, as in the sentence
above, setting off a dependent clause as its own sentence is an
effective method of punctuating the thought.)

This course is about writing, and reading, and making a claim for our
fundamental right to use words on paper. Within that context, we will
explore the ethics of writing (particularly about one's own family or
culture of history), the hurdles of writing (as they involve craft and
precision and clarity) and the great gifts we might receive from
others of our creative kin (that is to say, the long tradition of
writers of creative non-fiction).

The threshold between fact and fiction (which is not the same as that
between truth and lie) is the territory of creative non-fiction. In
this course we stake out that territory, inspecting the geology of its
forms and ideals, finding our own individual places to homestead.
Creative non-fiction involves the search for truth, and fidelity to
fact, yet also an awareness that truth and fact are often provisional,
and mythological; they are shapeshifters on the wide-open plain of
creativity. We will explore what this means, and what to do about it.
And, finally, the goal of the course (from my point of view, at
least), is to have fun: to preserve and nurture the creative and
imaginative spirit that is the foundation of all the arts and
sciences.

Art, craft, power, purpose: writing is all these things, and many
others too. In this course we will make connections between the
various threads of creative writing, forge those connections in our
own work, and share that work with one another. (Quick craft tip:
``each other'' is best used when two people or items are involved; ``one
another'' is best used when there are more than two.)

\subsection{Learning Goals}
\begin{itemize}
\item Learn to recognize, define, and evaluate literary creative
  non-fiction techniques such as narrative arc, tone, style, point of
  view, imagery and thematic intention.
\item Learn to use specified techniques in the composition of creative
  writing projects.
\item Learn to exercise critical judgment in revising creative work
  and the work of peers.
\item Write with increased awareness and sensitivity to language.
\item Write a memoir, a literary travelogue and a personal essay.
\end{itemize}
\clearpage

\subsection{Learning Experiences}
The course will include a variety of learning experiences contingent
upon regular attendance and dedicated participation. Because
creativity is an interactive process, much of the class time will be
devoted to group experiential exercises, individual reflective tasks,
collaborative endeavors, composition, and practical assignments.

We will create a collaborative environment in this class. We are not
going to cobble together the type of group one often hears about in
the arts: competitive, cut-throat, critical. Repeat: we are not
creating such a group. Instead, we will direct our efforts toward
building upon the individual strengths of each participant, finding
ways for each of us to be self-reflective in terms of assessing our
creative work, discovering a means of protecting the quality and
integrity of our writing. The creative spirit is remarkably
persistent, yet it is also fragile, especially at its inception, and
we must be conscious of this fragility. Think about it: did you not
experience, as a child, the strangulation of your creativity in
school, by way of a culture of insensitive peers or teachers? Why do
you think hardly anyone feels comfortable singing in public, or
dancing, or drawing, or reading their written work to others? We have,
most of us, been the victims of inappropriate feedback and judgment.
We have to be careful about this, in our course, so that we do not
harm one another.

\section{Readings}
\subsection{Required Course Text}
The required course text is reader is Bill Roorbach's \\
\textit{Contemporary Creative Nonfiction: The Art of Truth}.

\subsection{Suggested Books}
\begin{description}
\item [Broadkey, Harold.] \textit{This Wild Darkness: The Story of My
    Death}. \\Owl Books, 1997. \textsc{ISBN 0805055118}.
\item [Butala, Sharon.] \textit{Wild Stone Heart}. \\HarperFestival,
  2000. \textsc{ISBN 000255397X}.
\item [Calvo, C\'esar.] \textit{The Three Halves of Ino Moxo}.
  \\Translated by Kenneth Symington. \\Inner Traditions, 1995.
  \textsc{ISBN 0892815191}.
\item [Hedges, Chris.] \textit{War Is a Force that Gives Us Meaning}.
  \\Anchor, 2003. \textsc{ISBN 1400034639}.
\item [Hyde, Lewis.] \textit{Trickster Makes This World: Mischief,
    Myth, \& Art}. \\North Point Press, 1999. \textsc{ISBN
    0865475369}.
\item [Jerome, John.] \textit{Stone Work: Reflections on Serious Play
    and Other Aspects of Country Life}. \\UP of New England.
  \textsc{ISBN 0874517621}.
\item [Krakauer, John.] \textit{Into Thin Air: A Personal Account of
    the Mt. Everest Disaster}. \\Anchor, 1999. \textsc{ISBN
    0385494785}.
\item [Kingston, Maxine Hong.] \textit{The Woman Warrior: Memoirs of a
    Girlhood Among Ghosts}. \\Vintage, 1989. \textsc{ISBN
    0072435194}.
\item [Kwan, Michael David.] \textit{Things that Must Not be
    Forgotten: A Childhood in Wartime China}. \\Soho Press
  \textsc{ISBN 1569472823}
\item [Langewiesche, William.] \textit{American Ground: Unbuilding
    \\the World Trade Center}. \\North Point Press, 2002.
  \textsc{ISBN 0865475822}. (Also see \textit{Inside the Sky}.)
\item [Lopate, Phillip.] \textit{The Art of the Personal Essay: An
    Anthology from the Classical Era to the Present}. \\Anchor, 1997.
  \textsc{ISBN 038542339X}.
\item [Macfarlane, David.] \textit{The Danger Tree: Memory, War and
    the Search for a Family's Past}. \\Walker, 2001. \textsc{ISBN
    0802776167}.
\item [Merwin, W.S.] \textit{The Mays of Ventadorn}. \\National
  Geographic Directions, 2002. \textsc{ISBN 0792265386}.
\item [Ondaatje, Michael.] \textit{Running in the Family}. \\Vintage,
  1993. \textsc{ISBN 0679746692}.
\item [Pirsig, Robert.] \textit{Zen and the Art of Motorcycle
    Maintenance}. \\HarperTorch, 2006 (reprint). \textsc{ISBN
    0060589469}.
\item [Saint-Exup\'ery, A.] \textit{Wind, Sand and Stars}. \\Harvest,
  2002. \textsc{ISBN 0156027496}.
\item [Sanders, Scott Russell.] \textit{Writing from the Center}.
  \\Indiana UP, 1997. \textsc{ISBN 0253211433}.
\item [Terpstra, John.] \textit{The Boys: Or, Waiting for the
    Electrician's Daughter}. \\Gaspereau Press, 2005. \textsc{ISBN
    1554470110}.
\end{description}

\subsection{Books on Creativity and Associated Philosophies}
\begin{description}
\item [Achebe, Chinua] \textit{Hopes and Impediments}. New York:
  Doubleday, 1989.
\item [Barron, F., ed] \textit{Creators on Creating: Awakening and
    Cultivating the Imaginative Mind}. New York: Putnam, 1997.
\item [Benjamin, Walter] \textit{Theses on the Philosophy of History}.
\item [Borges, Jorge Luis.] \textit{Collected Fictions}. \\Penguin,
  1999. \textsc{ISBN 0140286802}.
\item [Bohm, David] \textit{Wholeness and the Implicate Order}.
  London: Ark, 1980.
\item [Bohm, David] \textit{Unfolding Meaning}. New York: Routledge,
  1985.
\item [Bohm, David] \textit{On Creativity}. New York: Routledge, 1998.
\item [Bronowski, Jacob] \textit{Science and Human Values}. New York:
  Harper, 1956.
\item [---------] \textit{The Face of Violence}. London: Turnstile
  Press, 1964.
\item [---------] \textit{A Sense of the Future: Essays in Natural
    Philosophy}. Cambridge, MIT Press, 1977.
\item [Degler, Teri] \textit{The Fiery Muse: Creativity and the
    Spiritual Quest}. Toronto: Random House, 1996.
\item [Demos, John.] \textit{The Unredeemed Captive: A Family Story
    from Early America}. \\Vintage, 1995. \textsc{ISBN 0679759611}.
\item [Flack, Audrey] \textit{Art and Soul: Notes on Creating}. New
  York: Penguin, 1986.
\item [Franklin, Ursula] \textit{The Real World of Technology}.
  Toronto: Anansi, 1999.
\item [Fulford, Robert] \textit{The Triumph of Narrative: Storytelling
    in an Age of Mass Culture}. Toronto: Anansi, 1999.
\item [Goldberg, Natalie] \textit{Writing Down the Bones}. Boston:
  Shambhala, 1986.
\item [Herrigel, Eugen] \textit{Zen in the Art of Archery}. New York:
  Random House, 1977.
\item [Hildegard of Bingen] \textit{Secrets of God: Writings of
    Hildegard of Bingen}.\\ Boston: Shambhala, 1996.
\item [Hyde, Lewis] \textit{The Gift: Imagination and the Erotic Life
    of Property}. New York: Vintage, 1983.
\item [---------] \textit{Trickster Makes This World: Mischief, Myth,
    and Art}. New York: North Point Press, 1998.
\item [Jim\'enez, Juan Ramon] \textit{The Complete Perfectionist: A
    Poetics of Work}. Edited and translated by Christopher Maurer. New
  York: Doubleday, 1997.
\item [Jung, C.G] \textit{The Spirit in Man, Art and Literature}.
  Translated by R.F.C. Hull. Princeton: Princeton University Press,
  1998.
\item [London, Peter] \textit{No More Secondhand Art}. Boston:
  Shambhala, 1989.
\item [Lorca, Federico] \textit{In Search of Duende}. Translated by
  Christopher Maurer. New York: New Directions, 1998.
\item [Lyndon, Susan] \textit{The Knitting Sutra: Craft as a Spiritual
    Practice}. San Francisco: Harper, 1997.
\item [Needleman, Carla] \textit{The Work of Craft: An Inquiry Into
    the Nature of Crafts and Craftsmanship}. New York: Kodansha, 1979.
\item [Pye, David] \textit{The Nature and Art of Workmanship}.
  Cambridge: Cambridge UP, 1968.
\item [Richards, Mary] \textit{Centering in Pottery, Poetry and the
    Person}. Middletwon, CT: Wesleyan UP.
\item [Sarton, May] \textit{Journal of a Solitude}. New York: Norton,
  1973.
\item [Sennett, Richard] \textit{The Corrosion of Character: The
    Personal Consequences of Work in the New Capitalism}. New York:
  Norton, 1998.
\item [Thoreau, Henry David] \textit{Walden}. New York: Norton, 1985.
\item [Wilson, Frank] \textit{The Hand: How Its Use Shapes the Brain,
    Language and Human Culture}. New York: Vintage, 1998.
\end{description}
\newpage
\subsection{Interesting Fiction}
\begin{description}
\item [Joseph Conrad] \textit{Lord Jim, Heart of Darkness\/}
\item [Thomas Wharton] \textit{Salamander\/}
\item [Madeleine Thien] \textit{Simple Recipes\/}
\item [Milan Kundera] \textit{Life is Elsewhere\/}
\item [Carlos Fuentes] \textit{The Orange Tree\/}
\item [Gabriel Garcia Marquez] \textit{One Hundred Years of Solitude\/}
\item [Jorge Luis Borges] \textit{Labyrinths\/}
\item [Vikram Seth] \textit{An Equal Music, A Suitable Boy\/}
\item [Alberto Manguel (editor)] \textit{Black Water: The Book of
    Fantastic Literature\/}
\item [Don DeLillo] \textit{Underworld\/}
\item [Somerset Maugham] \textit{The Razor's Edge\/}
\item [Philip K. Dick] \textit{The Man in the High Castle\/}
\item [Keri Hulme] \textit{The Bone People\/}
\item [Salman Rushdie] \textit{Midnight's Children\/}
\item [John Fowles] \textit{The Magus\/}
\item [Stephen King] \textit{The Stand\/}
\item [Philip Roth] \textit{Operation Shylock\/}
\item [Walter Miller] \textit{A Canticle for Leibowitz\/}
\end{description}

\subsection{Magazines}
\textit{The Atlantic Monthly\\
The Sun\\
The Utne Reader\/}
\subsection{Movies and Television}
\textit{Wonder Boys\\
Finding Forrester\\
Pollock\\
The Piano\\
My Brilliant Career\\
The Red Violin\\
Pi\\
Barton Fink\\
Memento\\
Almost Famous\\
Magnolia\\
Stand by Me\\
O Brother, Where Art Thou?\\
Clerks\\
Black Hawk Down\\
2001: A Space Odyssey\\
The Life Aquatic\\
Primer\\
Intacto\\
A Mighty Wind\\
Crash\\
Six Feet Under (TV)\\
Arrested Development (TV)\/}\\

\subsection{Online Essays by Your Instructor}
  \begin{description}
  \item [Beyond Word Processors] rosslaird.info/node/315
  \item [Literary Traditions and the Modern Writer]  rosslaird.info/node/233
\item [More Thoughts on Minimalism] rosslaird.info/node/221
\item [The Business of Journalism] rosslaird.info/node/207 
\item [Homework for Writers] rosslaird.info/node/205
\item [Niche Markets and the Lonely Artist] rosslaird.info/node/189
\item [Book Reviewer Lingo] rosslaird.info/node/188
\item [Fiction, Reality, and Philip K. Dick] rosslaird.info/node/186
\item [Publishing Dreams and Woes] rosslaird.info/node/176
\item [Data Backup for Writers] rosslaird.info/node/164
  \end{description}

\subsection{Other Internet Resources}
A full listing of writing links and web resources for writing (and creativity, and psychology, and a raft of other subjects), may be found at:\\
http://del.icio.us/rosslaird/writing


\clearpage

\section{Demonstration of Learning}

\subsection{Assignments}
Three writing assignments are required for this course: a memoir, a
literary travelogue, and a personal essay. These genres overlap, so it
may be simpler to think of the assignments as three projects of
creative non-fiction on subjects of your choosing. The assignments are
relatively brief in terms of length, and are intended to
provoke your creativity and your thought, to encourage you to start
thinking in terms of writing as a craft as opposed to simply a means
of ejecting your thoughts onto paper. (This last comment is a jibe
against a mode of writing much in vogue today: spontaneous writing, in
which we are encouraged to write without thought, splashing words onto
the page, in haste, in full emotional flight, saving editing and
precision for a later, more sober frame on mind. As we will discuss in
the class, I am not a fan of this style of composition. Among other
liabilities, it wastes a great deal of effort and time.)

For philosophical reasons, I do not prescribe a particular length for
the projects: a great essay can be a few pages long (as we'll see).
Yet it is difficult to craft a non-fiction essay in less than a few
thousand words. So, if you prefer a guideline for the length of the
projects, I offer two recommendations: make them as long as they
need to be; make them somewhere between two and five thousand words.
There is no upper limit on the length of the projects.

I'm not interested in how much you can write but rather in the quality
of your writing. Perhaps you write like Hemingway, perhaps like
Melville or Tolstoy. I don't know, and maybe you don't know either.
But I can tell you this: writing a shorter piece of great precision is
more difficult than writing a longer, more relaxed and wandering work.
In the context of smaller projects every word is on display and under
scrutiny, whereas in longer works the sheer bulk of the material tends
to hide various flaws. Melville, in fact, is a good example of this.

You may write short narratives in this course, but please do not write
short form as a means of avoiding work. You will know, I will notice,
and neither of us will be happy. Instead, make your work as long as it
needs to be. If you compose a lovely, resonant, short piece, you will
receive an excellent evaluation. But as I said, writing shorter pieces
is actually more difficult.

With reference to good, short pieces, I suggest (strongly) looking at
the poetry of W.S. Merwin, his lovely un-punctuated poetry in which
the bare words embody some strange and familiar light. Here's an
example of what I mean:
\clearpage
\begin{verse}
Why did he promise me\\
that we would build ourselves\\
an ark all by ourselves\\
out in back of the house\\
on New York Avenue\\
in Union City New Jersey\\
to the singing of the streetcars\\
after the story\\
of Noah whom nobody\\
believed about the waters\\
that would rise over everything\\
when I told my father\\
I wanted us to build\\
an ark of our own there\\
in the back yard under\\
the kitchen could we do that\\
he told me that we could\\
I want to I said and will we\\
he promised me that we would\\
why did he promise that\\
I wanted us to start then\\
nobody will believe us\\
I said that we are building\\
an ark because the rains\\
are coming and that was true\\
nobody ever believed\\
we would build an ark there\\
nobody would believe\\
that the waters were coming\\
\end{verse}

\subsubsection{Evaluation of Assignments}
My primary focus, as an instructor, is to assist you in developing
your creativity. Grades are quite far down on the list of priorities
for me. I am focused on your engagement with the process, your
commitment to your own work, the extent to which you show up,
metaphorically, to be as present as you can be. These are evaluation
criteria for me.

My own criteria will be blended with the evaluation guidelines from
the creative writing program, which are directed more toward craft and
articulated below:

\begin{description}

\item[$80-100$] Excellent: the use of language is pleasing and
  vigorous. The writing invites reading; the work is well-crafted and
  grammatically flawless. The author is perceptive.

\item[$76-79$] Very good: the use of language is generally correct.
  There may be a need for further editing. This could be in the
  presentation of the work (style, voice, characterisation, plot,
  point of view), or in the language (diction, grammar, usage,
  spelling, punctuation)~--- but the writing is involving.

\item[$72-75$] Good: the author has created a manuscript with
  substantial content and without any serious errors in tone or
  narration. Problems with creative shaping and delivery may occur,
  and there may be a further need for learning the mechanics of
  language use, but generally, the problems do not interfere with the
  reader's appreciation of the work.

\item[$68-71$] Manuscripts with repeated errors in grammar, usage or
  punctuation will result in a grade of no higher than 71 regardless
  of the proficiency and imagination demonstrated in the creative
  aspects of the work. On the other hand, manuscripts with no problems
  in grammar, usage or punctuation may not receive a grade higher than
  71 if they fail to demonstrate an understanding of the challenges
  (of style or voice, for example) involved in writing in the genre.


\item[$64-67$] Satisfactory: this writing shows constrained use of language
  (either in the creative shaping and delivery of content or in
  repeated errors in grammar punctuation, diction and usage), and the
  treatment of the material has not resulted in sufficient depth. The
  writing is potentially interesting, and a revision may improve the
  manuscript.

\item[$60-63$] This meets the minimum criteria of the assignment
  without in any way exceeding it. There are repeated errors such as
  spelling mistakes, sloppiness or a lack of depth to the writing.


\item[$56-59$] Below average: the writing is difficult to read because
  of inappropriate delivery or repeated grammatical errors or both.
  Furthermore, the idea may not be appropriate for the form. This
  grade does not permit students to pursue another course for which
  the graded course was a pre-requisite.

\item[$50-55$] Fail: the author fails to understand the nature of
  creative writing or has not tried.

\end{description}


We will discuss the three assignments at length in class. Each is
worth 25 per cent of your course grade.

\subsection{Attendance, Participation, and Peer Feedback}
The expectation is that you will attend all sessions and involve
yourself in the class process. Your willingness to engage creatively
with the learning process, to take appropriate personal risks, and to
participate in group activities are all central to your involvement in
this class. Because developing a style of creative writing is very
much a process of blending your own personal awareness with skills and
practical techniques, your own emotional involvement in the class is
as important as your academic knowledge of the material.

\subsubsection{Assessment Criteria for Attendance, Participation, and
  Peer Feedback}

\begin{itemize}
\item Demonstration of commitment to the development of self-awareness.
\item Openness to interpersonal process.
\item Ability to participate in appropriate self-disclosure.
\item Consideration of and responsiveness to others.
\item Willingness to take appropriate risks and to challenge oneself.
\item Commitment to enhancing the interpersonal experience of everyone in the class.
\item Ability to take personal responsibility for learning.
\item Willingness to deal with conflicts appropriately if and when they arise.
\item Ability to be open and responsive to appropriate feedback.
\item Willingness to speak up, to join conversations, and to contribute.
\end{itemize}

The above criteria represent 15 per cent of your course grade.
Attendance is worth an additional 10 per cent. The combined total for
participation and attendance is 25 per cent.

\clearpage

\subsection{Grade Inflation}
Almost every semester there are students who do well on the
assignments, complete all the associated learning goals of the course,
participate well, and wonder why they do not receive a grade of one
hundred percent (or 98, anyway). Here is the reason: almost every
semester there are students who demonstrates a level of commitment
that goes beyond the course requirement. Such students complete extra
work, or hand in exemplary assignments, or undertake a significant
amount of personal development in addition to the course expectations.
Such students typically receive the highest grades.

If you do reasonably well in the course you will receive a reasonable
grade. Very high grades are intended for extra or exemplary work.
Unfortunately, over the past thirty years the post-secondary
educational system in North America has participated in a process of
grade inflation. Since the 1980's, the average grade for typical
course work has been increasing by about 25 per cent each decade.
Elevated assessments do not accurately reflect the work of most
students. Even worse, grade inflation has caused many students to
expect high grades for average work. I am not a particularly stringent
assessor; but I will not inflate grades artificially.

The grades for the course will be distributed along a curve, with a
small number of students (likely) receiving high grades, most students
receiving grades in the middle range, and a few students struggling
with lower grades. If you are uncertain about your assessment for a
given assignment, or if you wish to know where, roughly, you are along
the distribution curve of the class, or if you would like suggestions
for how to improve your grade, please ask me for clarification.
\clearpage
\section{Class Schedule}
The class structure involves 27 sessions (50 hours, roughly). These
sessions will be balanced between experiential exercises, academic
material, peer feedback, and composition. The content for each session
will evolve as the semester progresses. We will cover the following
themes (though, perhaps not in the order listed below):

\begin{description}
\item[Getting Started] Introductions and student goals.\\
  Course objectives and assignments.\\
  What is creative nonfiction?\\
  Historic examples and the development of the genre.
\item[The Big Picture] Elements of creative nonfiction.\\
  Structural and compositional details.\\
  Great words, sentences, and paragraphs.
\item[Narrative Arc] Structures of narrative in literature, film,
  mythology and storytelling.\\
  How to find and follow the narrative arc.
\item[Integrating Personal Development with Creativity] Why and how
  creativity is a map of your inner life.\\
  Writing and the evolution of the self.\\
  Finding the themes of your creative life.
\item[Themes] Why mythology lasts, and what its themes have to say.\\
  How to recognize and use the archetypal.\\
  Modern adaptations of ancient themes.
\item[Composition] How to write good.\\
  The evilness of gerunds and adverbs.\\
  The virtues of concision and clarity.\\
  How to work a sentence through to clarity.\\
  How to build paragraphs from sentences.
\item[Point of View, Character, and Integration] Where, why, and how
  to position yourself within the narrative.\\
  Narrative voice: use and misuse.
\item[Imagery] The virtues of the concrete.\\
Strategies for avoiding abstraction.
\item[Tightening] Why every sentence is a poem.\\
Rhythm, pacing, and clarity.
\item[Style, Tone, and Voice] Structural elements of communication.\\
  Why a creative composition is not an academic essay.\\
  Twists and adaptations within the modern genres.
\item[Continuation] Where to from here?\\
Creativity and the writing life.\\
Finding a creative practice.
\end{description}
\newpage
\section{An Argument for Minimalist Writing Tools}
In a recent workshop on the materials and tools of writing I asked the
group to indicate which method they used to input text on a computer.
Almost everyone used \textit{Microsoft Word\/}~--- with the exception of
a sole advocate for the \textit{OpenOffice\/} word processor (which is
far better than \textit{Word}, and which uses the open source
\textsc{XML} file format).

But the trend --- even among seasoned writers~--- seemed decidedly
weighted toward mainstream word processors. This homogenization of
word processing, whether on the Mac or PC, inevitably delivers a
consistent~--- and therefore conformist~--- experience to the act of
typing on screen. The method of input unquestionably influences the
output. By way of subtle cues and imagery (icons, menus, procedures),
word processors inculcate a particular type of consciousness. Works of
writing from different authors but produced on the same word processor
will be more similar than those produced using separate tools. The
differences will be subtle but not inconsequential.

Moreover, all word processors introduce a level of aesthetic
abstraction that is perhaps not useful to the writing process. Word
processors encourage us to fiddle with fonts and spacing, with
countless page layout options, with the visual aspect of work that in
its initial stages should be primarily visceral.

And word processors are ergonomically inefficient. The mouse, which
requires the full use of one arm, is a primary tool in word
processors, as are menus and keystrokes assigned for mnemonic rather
than ergonomic functions (control-S to save, for example, requires the
removal of the left hand from the home row of the keyboard). Years
ago, as I began to understand that my persistent arthritic aches were
essentially caused by mouse and keyboard use, ergonomics became a core
consideration.

Another liability of word processors involves their proprietary file
formats, which impede the export and transfer of information to open
source formats, such as are found on the World Wide Web. If you want
to publish on the Web, word processors make this extraordinarily
difficult. Moreover, consider this: if you are writing in
\textit{Word}, Microsoft owns your material. After all, your text is
in their file format, and if you want to view or edit that text you
need to buy their software. Does this seem like an equitable or safe
arrangement in terms of safeguarding your intellectual property?

Now, to alternatives and solutions. Go online and search for Bram
Moolenaar's \textit{Seven Habits of Effective Text Editing\/} (Bram's
examples use the \textit{Vim\/} editor, but are germane to writing in
general). Learn to reduce the number and increase the efficiency of
the keystrokes you make. Wean yourself from the mouse. And by all
means look into \textit{Vim}, which is a most robust and efficient
text editor (Bram is \textit{Vim\/}'s main creator). \textit{Vim\/} was
originally designed for the Unix operating system, and its current
version is used mostly by Linux users (like me). However,
\textit{Cream\/} is a vim equivalent for Windows.

\textit{Vim\/} is cryptic, it has a steep learning curve, and some of
its functions are improbably arcane. \textit{Emacs}, another editor
that originally evolved on Unix (and which I used to write this
document) is similarly cryptic and powerful. In this sense, these
editors mirror the writing process. Unlike word processors, which
blanket that process, covering it with neat type, text editors lay
bare your words and force you to face the page.

Although \textit{Vim}, \textit{Emacs\/} and other minimalist tools (such
as \textit{BBEdit}, for the Mac) are used mostly by programmers,
writers are increasingly recognizing the advantages of such tools.
They are immeasurably more powerful and efficient for writing of all
kinds.

In its beginnings and its evolution, the act of writing is about the
bare bones, the essential, the elemental. Use a tool that delivers,
rather than distracts, from that wonderful trajectory. 

\newpage
\section{Blinking Cursor, Blank Page}
Late in \textit{Heart of Darkness}, after Marlow has meandered deep
into the jungle but before he meets Kurtz, who utters his now-famous
judgement upon human nature, ``The horror! The horror!'' --- before
this, the most famous scene in twentieth century literature, Marlow
finds himself making necessary repairs to the ship. He ruminates on
these activities as distractions from the shadows around him, from the
haunting underbelly of his own nature that he sees in the wilderness
around him, in the passionate abandon of the local tribes-people.
Here's the full passage:
\begin{quotation}
  The earth seemed unearthly. We are accustomed to look upon the
  shackled form of a conquered monster, but there --- there you could
  look at a thing monstrous and free. It was unearthly, and the men
  were --- No, they were not inhuman. Well, you know, that was the
  worst of it --- this suspicion of their not being inhuman. It would
  come slowly to one. They howled and leaped, and spun, and made
  horrid faces; but what thrilled you was just the thought of their
  humanity --- like yours --- the thought of your remote kinship with
  this wild and passionate uproar. Ugly. Yes, it was ugly enough; but
  if you were man enough you would admit to yourself that there was in
  you just the faintest trace of a response to the terrible frankness
  of that noise, a dim suspicion of there being a meaning in it which
  you --- you so remote from the night of first ages --- could
  comprehend. And why not? The mind of man is capable of anything ---
  because everything is in it, all the past as well as all the future.
  What was there after all? Joy, fear, sorrow, devotion, valour, rage
  --- who can tell? --- but truth --- truth stripped of its cloak of
  time. Let the fool gape and shudder --- the man knows, and can look
  on without a wink. But he must at least be as much of a man as these
  on the shore. He must meet that truth with his own true stuff ---
  with his own inborn strength. Principles won't do. Acquisitions,
  clothes, pretty rags --- rags that would fly off at the first good
  shake. No; you want a deliberate belief. An appeal to me in this
  fiendish row --- is there? Very well; I hear; I admit, but I have a
  voice, too, and for good or evil mine is the speech that cannot be
  silenced. Of course, a fool, what with sheer fright and fine
  sentiments, is always safe. Who's that grunting? You wonder I didn't
  go ashore for a howl and a dance? Well, no --- I didn't. Fine
  sentiments, you say? Fine sentiments, be hanged! I had no time. I
  had to mess about with white-lead and strips of woolen blanket
  helping to put bandages on those leaky steam-pipes --- I tell you. I
  had to watch the steering, and circumvent those snags, and get the
  tin-pot along by hook or by crook. There was surface-truth enough in
  these things to save a wiser man.
\end{quotation}

Marlow employs seamanship as a kind of shield against the chaos,
against the frightening shapes of his inner life. After all, he is a
civilized man, an Englishman, for whom the shadow must be contained.
Marlow is a sailor, one who traverses the waters but remains above
them. Writers, conversely, are involved in plumbing those depths, in
encountering their lights and shadows, in struggling with the full
breadth of human propensity.

But we distract ourselves too, and the most common method of doing
this is to allow the electronic world to continually divert us from
the blank page and blinking cursor. Email alerts, news feeds, blogs:
there is enough distraction in these things to doom the wisest writer.
We must go into the darkness, into the bardo, to discover our
treasures. That we often have difficulty doing so is, in part, due to
the persistent emphasis of our environment upon the facile and the
transient and the ephemeral. Always an update, a flashing notice, a
clamoring icon which seems to confirm our importance --- but in fact
belies our addiction to the inconsequential.

Writers have not been well-served by technology since about 1990, when
the last console versions of \textit{WordPerfect\/} showed us a black
screen upon which a small, blinking cursor waited patiently for us to
dive into the waters. Since the advent of graphical user interfaces,
the blackness has been hidden, has been replaced by smilies and floral
wallpaper and pastel icons. I'm not against the \textsc{GUI}; but for
writing, it's a serious impediment, the modern equivalent of Marlow's
leaky steam pipes.

The confrontation with what lies behind, or beneath, or hidden, is the
essence of all good writing. And whether one perceives that
hidden-ness as darkness, as Conrad did, or as a terrifying whiteness,
as did Melville, the mystery is the same. Here's Melville describing
the peculiar terror evoked by the whiteness of the whale:
\begin{quotation}
  Is it that by its indefiniteness it shadows forth the heartless
  voids and immensities of the universe, and thus stabs us from behind
  with the thought of annihilation, when beholding the white depths of
  the milky way? Or is it, that as in essence whiteness is not so much
  a colour as the visible absence of colour; and at the same time the
  concrete of all colours; is it for these reasons that there is such
  a dumb blankness, full of meaning, in a wide landscape of snows ---
  a colourless, all-colour of atheism from which we shrink? And when
  we consider that other theory of the natural philosophers, that all
  other earthly hues --- every stately or lovely emblazoning --- the
  sweet tinges of sunset skies and woods; yea, and the gilded velvets
  of butterflies, and the butterfly cheeks of young girls; all these
  are but subtile deceits, not actually inherent in substances, but
  only laid on from without; so that all deified Nature absolutely
  paints like the harlot, whose allurements cover nothing but the
  charnel-house within; and when we proceed further, and consider that
  the mystical cosmetic which produces every one of her hues, the
  great principle of light, for ever remains white or colourless in
  itself, and if operating without medium upon matter, would touch all
  objects, even tulips and roses, with its own blank tinge ---
  pondering all this, the palsied universe lies before us a leper; and
  like wilful travellers in Lapland, who refuse to wear coloured and
  colouring glasses upon their eyes, so the wretched infidel gazes
  himself blind at the monumental white shroud that wraps all the
  prospect around him. And of all these things the Albino whale was
  the symbol. Wonder ye then at the fiery hunt?
\end{quotation}

The writer approaches mystery by way of the white page or the black
screen. That is our task, nothing more. Not to explain the mystery, or
to resolve it, or to erase it; but to encounter it, struggle with it,
allow it to enter and change us. This will not happen, cannot happen,
if we are checking our emails and news feeds all day long. We must sit
in silence, waiting for the mystery to descend.

\end{document}