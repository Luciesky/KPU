%!TEX TS-program = xelatex
    %!TEX encoding = UTF-8 Unicode
%    \documentclass[10pt, letterpaper]{article}
\documentclass[letterpaper,10pt,headsepline]{scrreprt}
    \usepackage{fontspec} 
    \usepackage{placeins}
    \usepackage{multibbl}
    \usepackage{graphicx}
    \usepackage{hieroglf}
    \usepackage{txfonts}
    \usepackage{url}
    \usepackage{listings}
    \lstset{language=HTML}
    \usepackage{titling}
    \usepackage{geometry} 
    \geometry{letterpaper, textwidth=5.5in, textheight=8.5in, marginparsep=7pt, marginparwidth=.6in}
    %\setlength\parindent{0in}
    \defaultfontfeatures{Mapping=tex-text}
    \setromanfont [Ligatures={Common}, SmallCapsFont={Sabon LT Std}, BoldFont={Sabon LT Std Bold}, ItalicFont={Sabon LT Std Italic}]{Sabon LT Std}
    \setmonofont[Scale=0.8]{Lucida Sans Typewriter Std} 
    \setsansfont [Ligatures={Common}, SmallCapsFont={Myriad Pro Regular}, BoldFont={Myriad Pro Bold}, ItalicFont={Myriad Pro Italic}]{Myriad Pro Italic} 
\usepackage[ngerman,english]{babel}
\usepackage{scrpage2}
\usepackage{paralist}
\clubpenalty=6000
\widowpenalty=6000
\author{Ross A. Laird, PhD}
\title{Creative Writing 2140}
\date{\today}
\ohead{Creative Writing 2140}
\chead{Notes and Responses}
\pagestyle{scrheadings}
\setcounter{secnumdepth}{-1}
%\contentsname{Contents}
\begin{document}
\begin{titlingpage}
\begin{center}
\maketitle
\end{center}
\end{titlingpage}
\tableofcontents

\subsection{Notes and Responses}

Prepared for the HSCC meeting,\\
March 16, 2012

\section{The Context of Course Development}

For ease of reference we have included documentation relevant to the course development and approval process. The Course Outline Manual is the primary source of this documentation, and it describes the curriculum development context as follows:

\begin{quote}
Senate has standardized the procedures and documents of the curriculum development process. It has also undertaken to review curriculum for adherence to institutional policies and standards. \textit{The actual course development, however, arises from the expertise of faculty members and takes place within disciplines} and faculties. Departmental and Faculty curriculum committees direct curriculum development and ensure that \textit{the standards and goals appropriate to each area are met}, and adhere to policy L.11 Program and Curriculum Development, effective February 2008. [emphases added.]
\end{quote}

Accordingly, faculty members within departments are the primary developers of curriculum. Their expertise arises from familiarity with their discipline and is the most reliable guide in developing curriculum.

The Course Outline Manual also describes the steps for approval of a course, which include:

\begin{itemize}
\item An instructor (sometimes a team) is designated as course developer with the task of developing or revising a course. This course developer, in consultation with colleagues, designs a new course (or revises an existing one) \textit{consistent with program goals}, and writes the course outline.

\item The department/departmental curriculum committee reviews the outline and ensures that it fits with \textit{departmental plans and Kwantlen standards}. On approval, the outline is submitted to the divisional curriculum committee for approval.

\item The course may be previewed by the member of the Secretariat responsible for SCC, who will provide feedback.

\item The divisional curriculum committee reviews the outline and ensures that it fits with \textit{departmental plans and Kwantlen standards}. On approval, all new outlines (and those revised outlines that require Senate approval) will be submitted to the Senate Subcommittee on Course Curriculum (SCC) for approval. [emphasis added.]
\end{itemize}

This documentation is clear in assigning the roles of departmental faculty members and divisional curriculum committees. Departmental members possess the academic expertise and are accorded the sole discretion and autonomy to develop curriculum consistent with their departmental plans and Kwantlen standards.

Divisional curriculum committees ensure that course outlines are consistent with departmental plans and Kwantlen standards. Those standards are not articulated at any divisional or institutional level; rather, they are left as matters for departments to decide. Policy L11 describes five principles for developing departmental plans and standards, which should:

\begin{itemize}
\item be consistent with the institution's mission, goals, educational priorities and plans;
\item meet the University's standards of excellence; 
\item are based on community needs;
\item support student success; and
\item promote good use of University resources.
\end{itemize}

These principles are intended to inform curriculum development at all levels, and are the basis of the curriculum development process described in the Course Outline Manual.

In turn, divisional curriculum committees use the Course Outline Manual to assess course outlines in light of the principles outlined above. As a means of focusing that process, the Course Outline Manual specifically defines the mandate and the purview of divisional curriculum committees in terms of two criteria:

\begin{itemize}
\item That course outlines are consistent with departmental plans;
\item That course outlines are consistent with Kwantlen standards (which, again, are developed by departments).
\end{itemize}

A number of the questions we received about Creative Writing 2140 clearly go beyond these criteria. Indeed, some elements of the questions challenge the qualifications and professionalism of the department of Creative Writing. Such challenges are inappropriate to the context of HSCC.

However, as a collegial gesture, we have provided answers to all aspects of the questions. The promotion of a collegial environment is important to us.

\section{Question 1: Online Presence}

\subsection{Part One}

\begin{quote}
\textit{In the Learning Objectives/Outcomes, there is no clear indication of a comparative/critical approach to the question of how writers/artists currently establish and/or actively subvert the production of their ``online presence.''}
\end{quote}

We are interested in helping students find voice and place on the Web. Our goals focus on awareness, confidence, creativity, and community. This course is not intended to explore questions such as ``how writers/artists... actively subvert the production of their online presence." On the other hand, those themes -- as well as many others, such as privacy, digital citizenship, online identity and the risks and benefits of online presence -- are explored in the prerequisite course, CRWR 1240. 

In Creative Writing, we do not exert a particular bias that students should, or should not, subvert or promote their online presence. We do not determine what students should think or believe. Rather, our courses help students explore the landscape of their intended careers.

This approach is consistent with other departments and courses at Kwantlen. For example, the Fine Arts department offers a course entitled ``Digital Media: Interactive Art on the Web," the description of which is as follows:

\begin{quote}
\textit{Students will expand their knowledge of various digital software such as Photoshop, Dreamweaver, Director, and Flash. They will apply these programs to create artwork that will encompass digital stills, online work, installation, video and sound. Students will also learn to locate contemporary digital practices within the broader history of digital production.}
\end{quote}

This description, as with the description for CRWR 2140, emphasizes student learning, and is neutral on the question of whether students should, or should not, be actively promoting, subverting, or deconstructing their online presences and digital activities. These are matters for students to navigate in the context of learning environments which provide both context and opportunity.

\subsection{Part Two}

\begin{quote}
\textit{Much of the proposal deals with notions of ``promotion," ``selling," ``leveraging," and ``designing for emotion." Is there a plan to complicate or introduce comparative approaches to what ``design" on the Internet entails, or is this course primarily teaching students soft marketing skills?}
\end{quote}

Design on the Web is a rapidly-shifting landscape with many approaches and considerations (responsive design, mobile design, customer-focused design, user experience design, platform design, open vs. closed design, social media design, etc.). In this course we cover elements of Web design that are relevant to writers (particularly the experience layer and designing for publication).

We are not biased against promotion, selling, leveraging, or designing for emotion. On the contrary, these activities and approaches will form the basis of the future careers of our students. Students need now -- more than ever -- to explore issues such as marketing and promotion (which is one reason why Kwantlen offers degrees in Marketing). Students need the ability to be reflective and purposeful about these themes, and to come to their own conclusions (beyond the biases of instructors or of the University) about how they wish to proceed in a professional world awash in marketing, promotion, and leveraged design.

In Creative Writing, we do not treat marketing and promotion as inherently good or bad, professional or academic, useful or corrosive. Rather we help students understand how to understand these themes. We use ``comparative approaches" as well as many other modalities in exploring these themes.

Our approach is consistent with Kwantlen's mission and mandate, which indicates that our programs of study ``provide applied learning and broad-based university education." Additionally, the mission document indicates that ``we support multiple approaches to research and innovation to address community, industry, and market needs. We encourage faculty and learners to participate in many forms of knowledge generation and research, including those focused on discovery, creativity, application, and teaching."

Discovery, creativity, and application are essential aspects of this course.


\section{Question 2: Uniqueness}

\begin{quote}
\textit{What specifically does this course offer students that other similar courses already available at Kwantlen do not?}
\end{quote}

This course is for writers.

Many courses at Kwantlen offer similar content viewed through a distinctive lens (several other departments offer curriculum that is essentially identical to what is offered in the Creative Writing department).

There is no requirement that a given course at Kwantlen must be entirely distinct from any other course. Policy L11, the definitive guide to curriculum development at Kwantlen, specifies five considerations for course development:

\begin{itemize}
\item consistency with Kwantlen's mission, goals, educational priorities and plans;
\item fulfilment of standards of excellence (set by departments); 
\item consideration of community needs;
\item support for student success; and
\item good use of University resources.
\end{itemize}

Uniqueness of curriculum is not a criteria for course development. However, we would welcome greater collaboration and integration of courses, and would be happy to explore how curriculum from various departments might be integrated or co-created. Such an approach would be a ``good use of University resources.'' But, at the moment, there is no history of such collaboration, nor any practical means of doing it, nor any evident openness to a more collegial and supportive environment in which such co-creations might take place. We would like to see this situation change.

\section{Question 3: Scope}

\subsection{Part One}

\begin{quote}
\textit{The course, as presented, covers a significant and diverse range of material.}
\end{quote}

Our course outlines provide sufficient information such that a visiting instructor would be able to create the proper learning environment, learning experiences, and learning materials for a course section by consulting the course outline only. All Creative Writing courses are developed in this manner (which is consistent with the Course Outline Manual).

\subsection{Part Two}

\begin{quote}
\textit{How can a 12-week course meaningfully instruct students in web design, writing skills for the Internet, writing for promotion, publishing, and portfolio development?}
\end{quote}

We don't cover all of these themes in exhaustive detail (each one could be a degree in itself). We cover them all in some detail, with the general aim of offering students a means of developing their digital literacy and promoting their digital creativity (with writing, but with design as well).

Our experience has been somewhat ironic: the more details one includes in a course outline, the more questions people ask and exceptions people tend to raise. We prefer robust outlines with substantive descriptions; but other departments take different approaches. The Creative Writing department prefers the inclusion of robust details in outlines as a means of helping our cohort of instructors navigate the curriculum and the teaching process.

\section{Question 4: Code}

\begin{quote}
\textit{What does ``a gentle introduction to code" mean?}
\end{quote}

Students will learn how to read HTML5 and CSS (and, possibly, some Javascript). They will learn to identify and understand the code behind compliant and responsive design. They will discuss digital document formats and applications (such as Git, ReStructured Text and \LaTeX) and will learn about Web development frameworks such as Ruby on Rails, JQuery, LAMP, and Django. They will not learn to debug stack traces. For example, they will learn how to read this:

\vspace{10 pt}
\texttt{
\noindent 
<div itemscope itemtype=''http://schema.org/Book''>
<section class=''entry group">\\
<h2>The Transformation of Creativity in Publishing</h2>\\
<p><span itemprop=''author''>Djehuty Ibis</span>
<ul class="entry-meta">\\
<li><h4>November 18, 2011</h4></li>\\
<li>Posted at 14:36 <abbr>AM</abbr></li>\\
<li><a href="/">Read the Summary</a></li>\\
</ul>\\
</section><!-- /section>\\
</div><!-- /itemscope>
}

\vspace{10 pt}
They will not learn how to read this:
\vspace{10 pt}

\texttt{
if not pat in cache\\
res = translate(pat)\\
if len(cache) >= MAXCACHE:\\
cache.clear()\\
cache[pat] = re.compile(res)\\
return cache[pat].match(name) is not None\\
}

We believe that language which promotes optimism and encourages student confidence (language such as the word ``gentle") is useful in guiding a curriculum that includes material (coding) that students often find very challenging.

\section{Question 5: Course Title}

\begin{quote}
\textit{The title of the course is ``Writing and Creativity on the Web." Might this instead be ``Writing and Creativity on the Internet"? \\
Comment: The ``web'' is merely a system of interlinked hypertext documents while the ``Internet" is a broader global system of interconnected computer networks that includes interfaces like IM, chat, FTP, email etc...  presumably aspects that a course like this is likely to address.}
\end{quote}

The Wikipedia definition of the Internet as a ``global system of interconnected computer networks" is accurate in basic terms; but, more technically, the Internet is the set of communications protocols (TCP and IP, among others) that form a network of many protocols and layers. One of these protocols (IP) now includes everything from televisions to cars to (some) refrigerators and light bulbs.

We do not focus on TCP, IP, and the many products and services enabled by these protocols. Instead, this course focuses on one protocol --  Hypertext Transfer Protocol (HTTP, commonly known as the Web). We do not explore (in this course) IM, FTP, SSH, IMAP (email), GPS, and the like. We focus on the Web, which is specific to Hypertext Transfer Protocol (HTTP) documents and applications.

Because the course focuses on the Web, the course title reflects this. Our choice of the word Web (as opposed to the word Internet) is also consistent with what other departments have chosen (the Fine Arts department, for example) in titling similar courses.

\section{Question 6: HTML5}

\begin{quote}
\textit{Why is a book introducing HTML5, a hypertext markup language (which remains in an experimental phase of development) on the syllabus? It seems well-beyond the understanding of beginner students outside computer science.}
\end{quote}

HTML5 is not in an experimental phase. Since May 25, 2011, HTML5 has been advanced to Last Call status as a Working Draft (\url{http://www.w3.org/TR/html5/}). The target date for Recommendation status is 2014. Moreover, much new Web development is now done in HTML5 (\url{http://www.alistapart.com/topics/code/html5/}), and most high-traffic sites are already built with HTML5 (Google, Twitter, YouTube, Yahoo, Flickr, Pinterest, Facebook, Bing, Pandora, Linkedin, Typekit, and so on).

HTML5 is not new, nor nascent. Developers have been using it since before 2009. HTML5 simply includes (most of) and extends the previous version of HTML (4.01), and is the current standard for Web development. In our view, all discussions of Web development should include HTML5 as the foundation.

Basic literacy in HTML5 will not be ``well-beyond the understanding" of students of CRWR 2140, who will already have been exposed to HTML5 in the prerequisite for this course, CRWR 1240.

Writers today need to have the basic ability to read HTML5, to alter/adapt code as necessary, and to create valid, semantically-correct, standards-compliant, responsive websites. CRWR 2140 will help them develop this skill.

\section{Question 7: Learning Resources}

\subsection{Part One}

\begin{quote}
\textit{Please say more about the selection of learning resources in relation to course learning objectives and the placement of the course at the second year level.\\
Comment: The learning resources selected for this course do not seem to reflect a scholarly or rigorous approach to the subject matter. The statement that the library has sufficient resources in this area is simply
incorrect. Examining the list, the Anderson text is a popular business book that mostly discusses the online market place as a potential for niche markets and, along with the Howe, Marcotte, and Walter books, foregrounds
an uncritical and less than scholarly approach to the very real stakes involved with establishing a web presence in the creative economy. This is especially problematic since this course targets writers and artists -- a group that has historically possessed its own special concerns and interests in how they are to be ``marketed'' to the public.}
\end{quote}

The word \textit{rigor} (which has its roots in Medieval medicinal descriptions involving illness and numbness) is used frequently at Kwantlen in discussions about curriculum. The argument is often made that rigor (and, by extension, scholarship) is a matter of what books students read, or what content instructors provide in the classroom. Our view is that rigor is not primarily a matter of content delivery but rather of student engagement. Rigor is what students do, not what instructors or textbooks say.

Furthermore, the previous question (and the third question) assert that the materials and learning resources for CRWR 2140 (especially the inclusion of a book on HTML5) are ``well beyond the understanding'' of students in the course. This implies that the course is \textit{too} rigorous. Now, in this subsequent question, the claim is made that the resources are not rigorous enough. These contradictory assertions illuminate the fundamental problem of the questions: they are challenges to Creative Writing from outside of our discipline.

Creative Writing is a discipline built upon engagement, experience, and application. We help students learn the diverse skills required of professional writers. These skills are complex, multifaceted, and require a blend of many activities. For us, engagement and excellence consist of:

\begin{itemize}

\item How much purposeful and relevant work students undertake in a given course.

\item How much depth, commitment and engagement students demonstrate in the context of their learning.

\item How much interpersonal interaction students participate in and encourage in their peers (in writing workshops, for example).

\item How much development, in the skills of writing, students build through their involvement in course work.

\end{itemize}

Kwantlen's Mission and Mandate document indicates that ``we provide and promote a learning environment in which learners examine and develop their values, goals, and character through the integration of personal, academic, and professional inquiry." This is the essence of rigor and engagement, and is about what happens in the classroom.

Furthermore, standards of rigor and excellence have not been defined at Kwantlen. The Course Outline Manual contains no information about rigor, no discussion of rigor, and no criteria for excellence or rigor. Policy L11 (which defines our curriculum development process) contains no information about rigor, no discussion of rigor, and no criteria for rigor. Policy L11 does mention ``standards of excellence," which are not defined nor discussed in that policy (or in any policy) -- which, as noted previously, leaves individual departments with the autonomy to develop standards consistent with their disciplines and their expertise.

In the Creative Writing department we trust that other departments know what they are doing. We expect the same collegiality and trust in return -- as should all departments. Without such trust and collegiality there can be no community of scholarship.

Our development process in Creative Writing is consistent with Kwantlen's mission, which calls for an approach that is real-world, applicable, community-focused, and broad-based, ``to prepare our learners for a complex world." Accordingly, students in CRWR 2140 -- who are preparing for careers in the world of new publishing -- need to read what Web publishers and Web developers read. And they need to participate in the kinds of creative communities that flourish (through crowdsourcing) on the Web. Our courses provide opportunities for them to accomplish these aims.

At the moment, Web development firms and new publishers hire few university graduates. They prefer applicants who have real-world experience with Web design and development. These firms and publishers often note that the digital knowledge of university graduates (their literacy with HTML5, for example, or their knowledge of online collaboration tools) is several years out-of-date. Conversely, self-trained, entrepreneurial applicants who have spent less time in libraries and more time in communities on the Web are much better prepared for careers in new and social media.

This is one of the reasons why the course outline for CRWR 2140 does not focus on library resources; we don't use them much. Instead, we use emerging technologies and practices, such as digital collaboratories, social media, and crowdsoursing. These are the tools and practices of the Web and of new publishers. Our students seek to work in the field of digital creativity, and they need to learn these skills as fundamental aspects of their degrees.

\subsection{Part Two}

``The statement that the library has sufficient resources in this area'' is correct for the purposes of this course. We simply do not use the library (instead, we build our readings online, as almost all new publishers and new media professionals do).

\subsection{Part Three}

\begin{quote}
\textit{The Anderson text is a popular business book that mostly discusses the online market place as a potential for niche markets...}
\end{quote}

Knowledge of niche markets, the long tail, varieties of free, and Extremistan economic environments is essential for any writer moving into the world of digital publishing. Chris Anderson's books are unquestionably the most influential texts in this area, and students should know about them. As for literary quality: Anderson has won the US National Magazine Award three times. His publication history includes the journals \textit{Science} and \textit{Nature} and the magazines \textit{The Economist} and \textit{Wired}, where he is now editor-in-chief. He is number 74 on \textit{Time} Magazine's list of the 100 most influential people in the world. He is not a lightweight. The choice to include material from Chris Anderson in this course derives from his importance to the Web universe.

The same is true of the other book resources for this course. Ethan Marcotte, for example (\url{http://unstoppablerobotninja.com}), is perhaps the most well-regarded developer on the Web. Jeff Howe is an editor at \textit{Wired} magazine, the world's premier technology publication. Bruce Lawson is a founding member of the Web Standards Project (\url{http://webstandards.org/}) and a renowned developer for Opera (\url{http://www.opera.com}). Remy Sharp is a user experience designer and founder of jQuery for Designers (\url{http://jqueryfordesigners.com}) a major destination for Web developers. He is also a founding partner of html5Doctor (\url{http://html5doctor.com}) and the creator of HTML5shiv (\url{http://code.google.com/p/html5shiv/}), one of the most influential code solutions for adaptability on the Web. Aarron Walter is the lead user experience designer for the Rocket Science Group, makers of MailChimp (\url{http://mailchimp.com}, one of the most successful marketing sites on the Web). He is also a member of the Web Standards Project and lead developer for the InterACT Curriculum Project (\url{http://interact.webstandards.org/}).

These names are as familiar to anyone working seriously in the field of Web development as Lawrence, Faulkner, and Proust are to students of English literature. (To extend this analogy: Tim Berners-Lee would be Shakespeare, Jeffrey Zeldman would be T.S. Eliot, and Dan Cederholm would be Christopher Hitchens).

\subsection{Part Four}

\begin{quote}
\textit{...an uncritical and less than scholarly approach...}
\end{quote}

The tone of this statement -- dismissive and disdainful -- highlights the challenges of faculty collegiality that are common at Kwantlen. We haven't yet found ways of working together that emphasize values of respect, inclusion, cooperation, and trust. As we move forward as an institution, as we increasingly integrate practices and services in an environment of resource scarcity, we will need to learn to work together in the spirit of authentic collegiality. The tone of these questions makes this difficult, and presents everyone involved (not just the Creative Writing department) with yet another experience of interpersonal corrosion.

Looking ahead, beyond this exact moment, it's not clear to us how the damage to our collegiality will be repaired. We have done what we can, by responding to these questions in good faith, and we will continue to emphasize the importance of positive working relationships. Whatever subsequent steps are taken by others will largely determine whether this process helps us move forward or provokes us to remain in conflict.

As for the question itself: a course outline does not (and cannot, under the current Kwantlen course development system) reflect the ways in which critical approaches and scholarship are integrated into the learning environments of classrooms. Critical thinking and critical approaches are practices and activities that students do; they are not  determined by book titles or descriptions in a course outline. Critical thinking is a practice, a methodology. Its role in a given course is not determined by the outline but rather in how the instructor engages students.

We use a rhizomatic approach in these courses. In rhizomatic learning -- probably the fastest-growing field of educational innovation in Canada, with several thousand Canadian academics currently involved in a Massively Open Online Course based on rhizomatic methods (\url{http://change.mooc.ca}) -- the community is the curriculum.

\begin{quote}
In the rhizomatic model of learning, curriculum is not driven by predefined inputs from experts; it is constructed and negotiated in real time by the contributions of those engaged in the learning process. This community acts as the curriculum, spontaneously shaping, constructing, and reconstructing itself and the subject of its learning... learners come from different contexts, they need different things, and presuming you know what those things are is like believing in magic... Organizing a conversation, a course, a meeting or anything else to be rhizomatic involves creating a context, maybe some boundaries, within which a conversation can grow... (\url{http://davecormier.com/})
\end{quote}

In rhizomatic learning -- which draws upon research into complex adaptive systems theory, cognitive science, heuristics, anthropology, and hermeneutics,

\begin{quote}
 ``The knowledge lives in the community. You engage with it by probing into the community, sensing the response and then adjusting... It is a learning approach that is full of uncertainty, not least for the educator. But it's one that allows for the development of the literacies that will allow us to sharpen our ability to participate in complex decision making. Dealing with the uncertainty is what the learning is all about"  (\url{http://davecormier.com/}).
\end{quote}

Kwantlen's mission articulates the role of our scholarship in preparing ``our learners for a complex world. We value scholarship as a socially relevant obligation and opportunity. We support multiple approaches to research and innovation to address community, industry, and market needs. Community engagement is the manner in which we demonstrate our commitment to the social value of the University."

Creative Writing 2140 focuses on relevant curriculum (socially and professionally), community and industry needs, and market conditions. The course emphasizes engagement in the real world of the Web. We read what Web developers and new publishers read; we use what they use; we do what they do. We also question, and think critically, and approach these studies with a blend of scepticism and curiosity. And we look forward to a Kwantlen in which real-world learning -- which, after all, is the basis of our charter -- is not subject to disparagement and instinctive opposition.

\subsection{Part Five}

\begin{quote}
\textit{A cursory look at this field’s rich scholarly research reveals many possibilities for a far more university-appropriate level of critical engagement.}
\end{quote}

Beneath this statement (which again reflects a troubling tone) lie several assumptions:

\begin{itemize}

\item Critical engagement is contingent upon books.

\item ``University-appropriate" means books authored by university professors (all of the books cited in the question were authored by university professors, mostly English professors).

\item ``Scholarly research" means university professors using methods endorsed by specific, traditional fields within the university system.

\end{itemize}

We question these assumptions and offer alternate viewpoints for each.

\subsubsection{Books}

Books are the slowest mainstream medium. The workflow cycle of traditional, print-based publishing is currently about 18 months to two years. This does not include the length of time required to author the text; typically, anywhere from one to three years. One consequence of the publishing cycle is that the contents of books usually reflect a time period more than three years in the past. Academic books are more likely to reflect time frames that are at least four or five years in the past.

This is unacceptable for a course based on the Web, which evolves rapidly (Moore's Law determines fundamental revolutions roughly every 18 months: \url{http://bit.ly/w3M9EZ}). Most content on the Web is out of date within a month. Accordingly, we use digital technologies: collaboratories, social media, blogs, Web platforms. Examples include The Humanities, Arts, Science, and Technology Advanced Collaboratory (\url{http://hastac.org}), O'Reilly Tools of Change (\url{http://oreilly.com/toc/}), the InterACT Curriculum Project (\url{http://interact.webstandards.org/}), the World Wide Web Consortium (\url{http://www.w3.org/}), A List Apart (\url{http://www.alistapart.com/}), and The Massive Open Online Course in Theory and in Practice (\url{http://mooc.ca}). These resources, among many others, provide current information by leading scholars (and not only scholars in the university system). We will build and rebuild these resources, in keeping with our rhizomatic pedagogies, as the course evolves each semester.

\subsubsection{Critical Engagement}

As noted previously, critical engagement is not solely a function of course readings nor of syllabus topics. Critical engagement involves instructors engaging students in the classroom. The traditional academic notion that texts are a sufficient means of ensuring critical engagement and rigor has been thoroughly challenged and discredited in many studies by many scholars. A community of such scholars may be found in the 21st Century Literacies section of HASTAC: \url{http://hastac.org/topics/academia/21st-century-literacies}.

Critical engagement is a function of student output, not instructor input. This is a foundational principle of rhizomatic learning and is utilized in every new media course in Creative Writing. The texts we use are current in the field, are relevant to the future careers of students, and engage students in a manner that helps then successfully navigate post-secondary education. We find and use texts that combine high levels of professionalism with practical, real-world applications. We build our engagement through texts (both digital and printed), interactions, activities, and a host of other modalities that promote authentic critical engagement.

We use texts that students will use in their careers, and we build classroom experiences that prepare them for those careers. We engage students at the level of their life direction. Naturally, students find this approach to be relevant, motivating, and highly engaging.

Critical engagement of students through facilitation is essential at Kwantlen, where 82 percent of first-year students achieved either less than 75 percent in grade 12 (27 percent) or lack an academic GPA altogether (they graduated high school but did not pass one or more of the 4 academic courses, including English, required for an academic GPA). More than half of all Kwantlen students (55 percent) lack an academic GPA.

\vspace{10 pt} 
Let's be clear: more than half of all students entering Kwantlen did not pass high school.
\vspace{10 pt} 

Moreover, in the past year, Kwantlen dropped minimum entrance requirements for English to a classroom mark of C or better in either English or Communications. Kwantlen is the only post-secondary institution in the Lower Mainland that accepts Communications with a C as a minimum entrance requirement. 

Our challenges are unique and pressing:

\begin{itemize}

\item More than half of our students enter our learning environment without adequate preparation for post-secondary studies. We have not yet taken any institution-wide steps to help these students -- who are vulnerable in many ways -- make the transition to university.

\item More Kwantlen students have jobs than at any other university in Canada. We have not taken any institution-wide steps to integrate the academic lives of students with their employment activities or goals, nor have we worked with students to achieve the best balance between work and school.

\item Almost half of Kwantlen's students drop out before the end of their first year. Most do not return to post-secondary studies. We have not yet taken any institution-wide steps to correct this problem, which is unquestionably the largest challenge we face as a community.

\end{itemize}

We do not have a typical or traditional culture of university students, and we lose most of them.

Why?

The short answer is a lack of engagement. As an institution, we have not yet found sufficient ways of engaging students, of encouraging them to speak up and speak out, of helping them to bootstrap their conversations into university-level discourse, of helping them to integrate work and school. In rising to this challenge, the Creative Writing department focuses on purposeful and engaging approaches. Here are a few examples of student responses to our methodology in new media classes:

\begin{quotation}
``Every week when I came to class I felt very encouraged and inspired by all of the different creative ideas... I am walking away from this semester after putting in a great effort, with new skills and experiences that I will be able to reflect on forever."
\end{quotation}

\begin{quotation}
``I have observed students, including myself, come out of their shells, contribute with purpose and meaning to class discussion and activities, and have transformative learning experiences. This course helped in development of my academic, social, professional and personal life."
\end{quotation}

\begin{quotation}
``This class has fuelled my passion for learning and has been a tremendous influence in my pursuit of a career in creativity."
\end{quotation}

\begin{quotation}
``I have never left a class without something written down to punch into Google -- extra-curricular study stuff. This wasn't a requirement, this was the course sparking genuine interest in me, which spurred me to continue learning in my own spare time."
\end{quotation}

Critical engagement, purposefulness, mentorship, and creativity are all foundational to our approaches to learning. Creative Writing 2140 will follow and build upon these foundations.

\subsubsection{Scholarly Research}

Every faculty member within Creative Writing is a professional writer. We don't just write about writing, we actually do it. As a group we have published more (per capita) than any other department in the university. Writing is what we do; it's integral to our professional practice. Writing is our research: heuristic, hermeneutic, rhizomatic. The scholarship of professional writing and publishing is not constrained (nor should it be) by notions about scholarly research that derive from fields contingent upon academic theory and abstraction from practice. 

In Creative Writing, theory and practice are integrated. Research and scholarship are fused with the intensely demanding task of composing literary works. We utilize heuristic research methodologies to create those works, and we employ those same methodologies to create and sustain communities of creativity. We do not simply write about the field; we are the field.

Our scholarly research is unique, and the materials we use to support that research are also unique. We use the tools of our trade, and of our peers, and of the emerging communities of creativity which we inhabit. 

\subsection{Part Six}

This course is positioned at the second-year level because it follows, and builds upon, the first-year course CRWR 1240, which is a prerequisite for CRWR 2140. These two courses provide foundational skills that students will use as they move forward into upper-level courses.

\section{Final Thoughts}

Kwantlen has now defined itself as \textit{a leader in innovative and interdisciplinary education} (Mission and Mandate document: \url{http://www.kwantlen.ca/mission/mission-mandate.html}). We have defined our approaches as \textit{applied, relevant, engaging, inclusive}, and \textit{facilitative}. We have defined our methods of teaching as \textit{based on mentorship, discovery, purposefulness, and learner autonomy}. In turn, we have defined out teaching environment as \textit{collaborative, innovative, creative, and respectful}. Our new identity supports \textit{ emerging and experimental teaching methods} in a culture of learning that is \textit{learner-focused, academically rigorous, innovative, interdisciplinary, and socially responsible}.

In the Creative Writing department we embrace these values and principles. We see the many ways in which the cultures of education are moving toward innovation and engaged learning. We recognize the importance of providing students with educational experiences that help them \textit{integrate personal, academic, and professional inquiry}. We encourage learners to create and pursue \textit{workplace experiences} in which they can explore \textit{applied, relevant}, and meaningful experiences. Our scholarship \textit{involves learners and the broader community}. 

We support Kwantlen's new mission and we develop curriculum consistent with that mission. We also recognize that many other faculty members and departments do the same, and that many of us are working toward an environment of scholarship and collegiality in which the principles we support become integral to our community.

\end{document}
