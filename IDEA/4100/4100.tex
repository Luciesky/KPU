%!TEX TS-program = xelatex
    %!TEX encoding = UTF-8 Unicode
%    \documentclass[10pt, letterpaper]{article}
\documentclass[letterpaper,10pt,headsepline]{scrreprt}
    \usepackage{fontspec} 
    \usepackage{placeins}
    \usepackage{multibbl}
    \usepackage{graphicx}
    \usepackage{hieroglf}
    \usepackage{txfonts}
    \usepackage{url}
    \usepackage{titling}
    \usepackage{geometry} 
    \geometry{letterpaper, textwidth=5.5in, textheight=8.5in, marginparsep=7pt, marginparwidth=.6in}
    %\setlength\parindent{0in}
    \defaultfontfeatures{Mapping=tex-text}
    \setromanfont [Ligatures={Common}, SmallCapsFont={ITC Officina Serif Std}, BoldFont={ITC Officina Serif Std Bold}, ItalicFont={ITC Officina Serif Std Book Italic}]{ITC Officina Serif Std Book}
    \setmonofont[Scale=0.8]{Lucida Sans Typewriter Std} 
    \setsansfont [Ligatures={Common}, SmallCapsFont={ITC Officina Sans Std}, BoldFont={ITC Officina Sans Std Bold}, ItalicFont={ITC Officina Sans Std Book Italic}]{ITC Officina Sans Std} 
\usepackage[ngerman,english]{babel}
\usepackage{scrpage2}
\usepackage{paralist}
\clubpenalty=6000
\widowpenalty=6000
\author{Ross A. Laird, PhD}
\title{Interdisciplinary Expressive Arts 4100}
\date{\today}
\ohead{Ross Laird}
\chead{IDEA 4100}
\pagestyle{scrheadings}
\setcounter{secnumdepth}{-1}
%\contentsname{Contents}
\begin{document}
\begin{titlingpage}
\begin{center}
\maketitle
\end{center}
\end{titlingpage}
\tableofcontents
\subsection{Instructor}
Instructor: Ross Laird, Ph.D.\\ 
Office: D308, Surrey campus (by appointment)\\
Telephone: 604-916-1675\\
Direct Email: \url{ross@rosslaird.com}\\
Support Website: \url{help.rosslaird.com}\\

\section{Basic Philosophy of the Course}

This course continues the foundational work begun in Interdisciplinary Expressive Arts 3100 or Mythological Narratives 3301, in which students have explored the many dimensions of interdisciplinary inquiry. The course is, for the most part, led by the initiatives and interests of students working individually and in groups. The course content will evolve as the course develops, but will include a diverse number of current interdisciplinary approaches within the expressive arts. These include creative writing, music, movement, fine arts, theatre, expressive arts therapies, storytelling, mythopoetics, photography, film making, etc.. Additionally, students will explore potential academic and career directions within the cultural and social services sectors.

As a reminder, and for new students (those who have not enrolled in IDEA 3100), Interdisciplinary Expressive Arts refers to a specific set of educational activities, goals and strategies. Based on innovative pedagogy and integrative approaches to learning, interdisciplinary
studies involve the synthesis and synergy of various disciplines
toward a cohesive, unified educational experience. Interdisciplinarity
is much more than enrollment in courses from more than a single
discipline. Authentic interdisciplinarity emphasizes the linkages
between disciplines by focusing on contrasting and complementary
aspects of diverse educational domains.

Interdisciplinary studies encourage students to develop broader
intellectual skills, greater facility for critical thinking, and
greater awareness of the social relevance of their education.
Interdisciplinary students have the opportunity to develop exemplary
skills in problem solving, insight, team-building, lateral thinking,
and multi-modal learning styles. Interdisciplinary strategies involve
approaching an issue or problem from various perspectives. This
typically entails intellectual inquiries that range beyond the borders
of any single discipline or domain. While still respecting the
function of the boundaries between domains, interdisciplinary
approaches recognize that those boundaries are essentially arbitrary
and do not always serve the goals of learning. Global warming and the
AIDS pandemic are two examples of contemporary issues that require
interdisciplinary approaches.

This course is about creativity, about making a claim for the
fundamental right of intentional creative action. Within that context,
we will explore the ancient and modern practices of creative endeavor
(particularly as regards family and culture), the hurdles of
creativity (as they involve craft and precision and clarity) and the
great gifts we might receive from others of our creative kin (that is
to say, the long tradition of writers, poets, sculptors, dancers,
craftspeople of all stripes, musicians, myth-makers, and so on).
Throughout this process, our guiding archetype will be that of the
trickster.

In this course we stake out the territory of the creative, inspecting
the geology of its forms and ideals, finding our own individual places
to homestead. Creativity involves the search for truth, yet also an
awareness that truth and fact are often provisional, and mythological;
they are shapeshifters on the wide-open plain of creativity. We will
explore what this means, and what to do about it.

And, finally, the goal of the course (from my point of view, at
least), is to have fun: to preserve and nurture the creative and
imaginative spirit that is the foundation of all the arts and
sciences.

\section{Learning Goals}
\begin{itemize}

\item Read selected interdisciplinary texts and discuss their origin, development, and contemporary relevance
\item Interpret interdisciplinary traditions within the context of contemporary academic and professional inquiry
\item Articulate (verbally, in writing, and through creative endeavour) knowledge of interdisciplinarity as both an ancient and a current mode of inquiry
\item Describe the ways in which interdisciplinary expressive arts are more than a simple aggregation of various disciplines
\item Evaluate diverse interdisciplinary perspectives and approaches using heuristic modalities
\item Complete a series of linked interdisciplinary projects and presentations

\end{itemize}
\clearpage

\section{Learning Experiences}
We use a rhizomatic approach in this course (and in all IDEA courses). In rhizomatic learning, the community is the curriculum. Dave Cormier, the leader in rhizomatic learning in Canada, describes the process as follows:

\begin{quote}
In the rhizomatic model of learning, curriculum is not driven by predefined inputs from experts; it is constructed and negotiated in real time by the contributions of those engaged in the learning process. This community acts as the curriculum, spontaneously shaping, constructing, and reconstructing itself and the subject of its learning... learners come from different contexts, they need different things, and presuming you know what those things are is like believing in magic... Organizing a conversation, a course, a meeting or anything else to be rhizomatic involves creating a context, maybe some boundaries, within which a conversation can grow... (\url{http://davecormier.com/})
\end{quote}

In rhizomatic learning -- which draws upon research into complex adaptive systems theory, cognitive science, anthropology, and hermeneutics, ``the knowledge lives in the community. You engage with it by probing into the community, sensing the response and then adjusting... It is a learning approach that is full of uncertainty, not least for the educator. But it's one that allows for the development of the literacies that will allow us to sharpen our ability to participate in complex decision making. Dealing with the uncertainty is what the learning is all about"  (\url{http://davecormier.com/}).

The course will include a variety of learning experiences contingent upon regular attendance and dedicated participation. Because creativity is an interactive process, much of the class time will be devoted to group experiential exercises, individual reflective tasks, collaborative endeavors, and practical assignments.

We will create a collaborative environment in this class. We are not going to cobble together the type of group one often hears about in the arts: competitive, cut-throat, critical. Repeat: we are not creating such a group. Instead, we will direct our efforts toward building upon the individual strengths of each participant, finding ways for each of us to be self-reflective in terms of assessing our creative work, discovering a means of protecting the quality and integrity of our writing. The creative spirit is remarkably persistent, yet it is also fragile, especially at its inception, and
we must be conscious of this fragility. Think about it: did you not experience, as a child, the strangulation of your creativity in school, by way of a culture of insensitive peers or teachers? Why do you think hardly anyone feels comfortable singing in public, or dancing, or drawing, or reading their written work to others? We have, most of us, been the victims of inappropriate feedback and judgment. We have to be careful about this, in our course, so that we do not harm one another.

\section{Readings}
\subsection{Required Course Texts}

\begin{description}
\item [Barron, F., Montouri, A., and Barron, A., eds.] \textit{Creators on Creating: Awakening and Cultivating the Imaginative Mind.} 
\\New York: Putnam, 1997.
\item[Wilson, Frank.] \textit{The Hand: How Its Use Shapes the Brain, Language and Human Culture.}
\\New York: Vintage, 1998.
\item[Plus] {one text chosen by the student.}
\end{description}

\subsection{Suggested Books}
\begin{description}
\item [Allen, Pat.] \textit{Art is a Way of Knowing.} \\Shambhala, 1995.
\item [Hyde, Lewis.] \textit{The Gift: Imagination and the Erotic Life of Property.} 
\\New York: Vintage, 1983.
\item [Butala, Sharon.] \textit{Wild Stone Heart}. \\HarperFestival,
  2000. \textsc{ISBN 000255397X}.
\item [Calvo, C\'esar.] \textit{The Three Halves of Ino Moxo}.
  \\Translated by Kenneth Symington. \\Inner Traditions, 1995.
  \textsc{ISBN 0892815191}.
\item [Campbell, Joseph.] \textit{The Mythic Image}.
  \\Princeton UP, 1974.
\item [Ellis, Normandi.] \textit{Dreams of Isis: A Woman's Spiritual
    Sojourn}.
  \\Quest, 1995.
\item [Hancock, Graham.] \textit{Heaven's Mirror: Quest for the Lost
    Civilization.}.
  \\Crown, 1998.
\item [Hedges, Chris.] \textit{War Is a Force that Gives Us Meaning}.
  \\Anchor, 2003. \textsc{ISBN 1400034639}.
\item [Kingston, Maxine Hong.] \textit{The Woman Warrior: Memoirs of a
    Girlhood Among Ghosts}. \\Vintage, 1989. \textsc{ISBN
    0072435194}.
\item [Kwan, Michael David.] \textit{Things that Must Not be
    Forgotten: A Childhood in Wartime China}. \\Soho Press
  \textsc{ISBN 1569472823}
\item [Laird, Ross A.] \textit{Grain of Truth: The Ancient Lessons of Craft}. \\MWR, 2000.
\item [Langewiesche, William.] \textit{American Ground: Unbuilding
    \\the World Trade Center}. \\North Point Press, 2002.
  \textsc{ISBN 0865475822}. (Also see \textit{Inside the Sky}.)
\item [London, Peter.] \textit{No More Secondhand Art.} 
\\Boston: Shambhala, 1989.
\item [Lopate, Phillip.] \textit{The Art of the Personal Essay: An
    Anthology from the Classical Era to the Present}. \\Anchor, 1997.
  \textsc{ISBN 038542339X}.
\item [Macfarlane, David.] \textit{The Danger Tree: Memory, War and
    the Search for a Family's Past}. \\Walker, 2001. \textsc{ISBN
    0802776167}.
\item[McNiff, S.] \textit{Art Heals: How Creativity Cures the Soul.} \\Boston: Shambhala Publications, 2004.
\item [Merwin, W.S.] \textit{The Mays of Ventadorn}. \\National
  Geographic Directions, 2002. \textsc{ISBN 0792265386}.
\item [Ondaatje, Michael.] \textit{Running in the Family}. \\Vintage,
  1993. \textsc{ISBN 0679746692}.
\item [Pirsig, Robert.] \textit{Zen and the Art of Motorcycle
    Maintenance}. \\HarperTorch, 2006 (reprint). \textsc{ISBN
    0060589469}.
\item [Saint-Exup\'ery, A.] \textit{Wind, Sand and Stars}. \\Harvest,
  2002. \textsc{ISBN 0156027496}.
\item [Sanders, Scott Russell.] \textit{Writing from the Center}.
  \\Indiana UP, 1997. \textsc{ISBN 0253211433}.
\item [Sullivan, William.] \textit{The Secret of the Incas: Myth,
    Astronomy, and the War Against Time.} \\Crown, 1996.

\end{description}

\subsection{Books on Creativity and Associated Philosophies}
\begin{description}
\item [Achebe, Chinua] \textit{Hopes and Impediments}. New York:
  Doubleday, 1989.
\item [Barron, F., ed] \textit{Creators on Creating: Awakening and
    Cultivating the Imaginative Mind}. New York: Putnam, 1997.
\item [Benjamin, Walter] \textit{Theses on the Philosophy of History}.
\item [Borges, Jorge Luis.] \textit{Collected Fictions}. \\Penguin,
  1999. \textsc{ISBN 0140286802}.
\item [Bohm, David] \textit{Wholeness and the Implicate Order}.
  London: Ark, 1980.
\item [Bohm, David] \textit{Unfolding Meaning}. New York: Routledge,
  1985.
\item [Bohm, David] \textit{On Creativity}. New York: Routledge, 1998.
\item [Bronowski, Jacob] \textit{Science and Human Values}. New York:
  Harper, 1956.
\item [---------] \textit{The Face of Violence}. London: Turnstile
  Press, 1964.
\item [---------] \textit{A Sense of the Future: Essays in Natural
    Philosophy}. Cambridge, MIT Press, 1977.
\item [Degler, Teri] \textit{The Fiery Muse: Creativity and the
    Spiritual Quest}. Toronto: Random House, 1996.
\item [Demos, John.] \textit{The Unredeemed Captive: A Family Story
    from Early America}. \\Vintage, 1995. \textsc{ISBN 0679759611}.
\item [Flack, Audrey] \textit{Art and Soul: Notes on Creating}. New
  York: Penguin, 1986.
\item [Franklin, Ursula] \textit{The Real World of Technology}.
  Toronto: Anansi, 1999.
\item [Fulford, Robert] \textit{The Triumph of Narrative: Storytelling
    in an Age of Mass Culture}. Toronto: Anansi, 1999.
\item [Goldberg, Natalie] \textit{Writing Down the Bones}. Boston:
  Shambhala, 1986.
\item [Herrigel, Eugen] \textit{Zen in the Art of Archery}. New York:
  Random House, 1977.
\item [Hildegard of Bingen] \textit{Secrets of God: Writings of
    Hildegard of Bingen}.\\ Boston: Shambhala, 1996.
\item [Hyde, Lewis] \textit{The Gift: Imagination and the Erotic Life
    of Property}. New York: Vintage, 1983.
\item [---------] \textit{Trickster Makes This World: Mischief, Myth,
    and Art}. New York: North Point Press, 1998.
\item [Jim\'enez, Juan Ramon] \textit{The Complete Perfectionist: A
    Poetics of Work}. Edited and translated by Christopher Maurer. New
  York: Doubleday, 1997.
\item [Jung, C.G] \textit{The Spirit in Man, Art and Literature}.
  Translated by R.F.C. Hull. Princeton: Princeton University Press,
  1998.
\item [London, Peter] \textit{No More Secondhand Art}. Boston:
  Shambhala, 1989.
\item [Pye, David] \textit{The Nature and Art of Workmanship}.
  Cambridge: Cambridge UP, 1968.
\item [Richards, Mary] \textit{Centering in Pottery, Poetry and the
    Person}. Middletwon, CT: Wesleyan UP.
\item [Sarton, May] \textit{Journal of a Solitude}. New York: Norton,
  1973.
\item [Sennett, Richard] \textit{The Corrosion of Character: The
    Personal Consequences of Work in the New Capitalism}. New York:
  Norton, 1998.
\item [Thoreau, Henry David] \textit{Walden}. New York: Norton, 1985.
\end{description}
\newpage

\subsection{Further Suggestions}

\begin{description}

\item [Arendt, Hannah] \textit{Illuminations}. London: Cape.

\item [Chodorow, J.]  \textit{Dance Therapy and Depth Psychology: The Moving Imagination}. Routledge, 1991.

\item [Dewey, John] \textit{Art As Experience}. New York, Perigee, 1931.
\item [Diamonstein, Barbara] \textit{Handmade in America: Conversations with Fourteen Craftmasters}. New York: Abrams, 1983.
\item [Gadamer, H.G.] \textit{Philosophical Hermeneutics}. Trans. D.E. Linge. Berkeley: University of California Press, 1976.
\item [Gardner, Harold] \textit{The Unschooled Mind: How Children are Taught and How Teachers Should Teach}. New York, Basic, 1993.
\item [Greene, Brian] \textit{The Elegant Universe: Superstrings, Hidden Dimensions, and the Quest for the Ultimate Theory}. New York: Norton, 1999.
\item [Hammarskjold, Dag] \textit{Markings}.
\item [Knill, P., Levine, E. and Levine, S.] \textit{Principles and Practice of Expressive Arts Therapy}. New York: Jessica Kingsley, 2004.
\item [Levine, S. K. and Levine, E. G.] \textit{Foundations of Expressive Arts Therapy}. London: Jessica Kingsley Publishers, 1999.
\item [Levine, E.] \textit{Tending the Fire}. Ontario: EGS Press, 2003.
\item [Levine, Peter] \textit{Waking the Tiger: Healing Trauma}. North Atlantic Books.
\item [Lorca, Federico] \textit{In Search of Duende}. Translated by Christopher Maurer. New York: New Directions, 1998.
\item [Lyndon, Susan] \textit{The Knitting Sutra: Craft as a Spiritual Practice}. San Francisco: Harper, 1997.
\item [McNiff, Shaun] \textit{Art Heals: How Creativity Cures the Soul}. Boston: Shambhala Publications, 2004.
\item ---------. \textit{Creating with Others: The Practice of Imagination in Art, Life, and the Workplace}. Boston: Shambhala Publications, 2003.
\item ---------. \textit{Art As Medicine}. Boston: Shambala, 1992.
\item ---------. \textit{Fundamentals of Art Therapy}. Springfield, IL: Charles C. Thomas, 1988.
\item ---------. \textit{Art-Based Research}. London: Jessica Kingsley, 1988.
\item [Mazza, Nick] \textit{Poetry Therapy: Interface of the Arts and Psychology}. CRC Press, 1999.
\item [Minnich, E. K.] \textit{Transforming Knowledge}. Temple University Press., 1990.
\item [Moustakas, C.] \textit{Heuristic Research: Design, Methodology, and Applications}. Newbury Park: Sage, 1990.
\item [Needleman, Carla] \textit{The Work of Craft: An Inquiry Into the Nature of Crafts and Craftsmanship}. New York: Kodansha, 1979.

\end{description}

\clearpage

\section{Demonstration of Learning}

\subsection{Assignments}
Three linked projects and several presentations (see below) are required for this course. These assignments may be comprised of any type of art expression (writing, music, imagery, dance, movement, photography, etc.). The central idea is for you to choose a specific theme or thread and explore it in some depth. We will discuss these projects at length in class. They are opportunities for you to discover and explore creativity in your own life.

For philosophical reasons, I do not prescribe a particular length or structure for the projects. There is no upper limit on the length or complexity of the projects.

The three linked projects can be completed as separate projects or as one large project with three segments or stages. Either way, each project (or each segment) will include a written self-evaluation of at least 1,000 words. The self-evaluation will include answers to the following questions (answers need not be itemized): 

 \begin{itemize}
\item What research did you do to prepare for this project? Research might include readings, investigative interviews, online searches, self-reflections, ruminations, and many other modalities. 

\item What learning resources did you use? These might include books, articles, online resources, and so on.

\item Of the research and readings you undertook, what impressed you as being most interesting or relevant?

\item What kinds of experiments did you undertake with this project? What did you build, write, craft, or try? How did you spend your time, and how did it go? (Look at the criteria on the next page.)

\item What went well, where did you struggle, and how do you feel about the process you undertook during this project?

\item What were the best and worst moments of this project? What did you learn from these moments?

\item Are you proud of this project? Is it your best work? What grade would you give yourself?

\item How might you have improved this project, of your experience of it?

\item What did you learn about interdisciplinarity while working on this project?

\item What did you learn about yourself while working on this project?

\item What will you remember about this project in five years?

\item How does what you learned apply to your studies at Kwantlen and to your sense of your future direction?

\item What advice would you give to others who might be undertaking a similar project?

\item What did this project mean to you? What might it mean for others?

\item Do you plan to continue this project further, or to work on similar projects in the future?

\item You learned something crucial in this project which you won't discover for a while. Make a guess now about what that might be.

\end{itemize}

The three individual projects are worth 25 percent each.

\subsubsection{Assessment Criteria for Creative Projects}

Projects for this course are focused on creativity. Accordingly, the following criteria -- which are based on the philosophy of creativity  --  are used to evaluate engagement and commitment to the projects:

\begin{itemize}
\item Willingness to take appropriate risks and to challenge oneself creatively.
\item Willingness to try new things, especially when doing so provokes creative discomfort.
\item Openness to personal and interpersonal process.
\item Willingness to collaborate with others.
\item Consideration of and responsiveness to others.
\item Willingness to examine personal values, beliefs, and judgments.
\item Ability to take personal responsibility and initiative for learning.
\item Willingness to approach creativity as a skill with discrete steps and standards.
\item Commitment to improvement in writing and other creative projects.
\item Ability to be open and responsive to appropriate feedback.
\end{itemize}


\subsection{Group Presentations}

Each student will be a member of several different peer groups; each
peer group will present at least one mini-presentation (roughly thirty minutes each) on various topics. Each class session after the second will involve presentations, with one presentation from each group. Class time will be given for preparing the presentations. The structure and content of the presentations will be discussed in class.

\subsubsection{Presentation Methods and Goals}

The central idea of the presentations for this course is to give you
 opportunities to practice interdisciplinary thinking and expression.
 As such, the presentation should be interdisciplinary. Essentially,
 this means that you should try to use multiple presentation
 strategies and modalities. These might include (but are certainly not
 limited to) any of the following:

 \begin{itemize}
 \item Storytelling
 \item Poetry
 \item Music (playing)
 \item Drumming
 \item Singing
 \item Dance
 \item Movement
 \item Sport
 \item Ritual
 \item Film (showing)
 \item Film making
 \item Photography
 \item Web content
 \item Craft work
 \item Art making
 \item Individual reflection
 \item Meditation
 \item Health practices
 \item Creative process (any type)
 \item Group communication
 \item Cultural practices
 \item Nature experiences
 \end{itemize}

 Whenever possible (and workable), try to mix together multiple
 modalities into a single presentation. For example, you might ask the
 group to do some individual reflection using the modality of poetry,
 then create a series of movements based on the poetry, then work in
 small groups to talk about and share the process. Many configurations
 are possible. The trick is to choose an activity that you enjoy, then
 find a way to apply it to the content (suggested presentation topics
 are listed below). Please do not create your presentations using only
 written and/or spoken materials. In other words, don't just stand up
 at the front of the class and talk about the presentation topic.
 Utilize the energy of the group. Remember that in interdisciplinary
 work divergences are valued as unique opportunities. So, feel free to
 experiment with activities and modalities that may not seem, on the
 surface, to be related to the topic at hand but which might, upon
 experiment, yield surprising connections and results. Be playful.
 Allow yourself to laugh at yourself, to be embarrassed, to engage
 with the process in novel and interesting ways.

In interdisciplinary work, riddles and puzzles are highly prized.
Accordingly, the presentations should (ideally) not be complete
explanations or presentations of material. Feel free to play with
challenging exercises, with impossible scenarios, and other conundra.
One way to think about this is to consider insoluble riddles, such as
the one in \textit{Alice in Wonderland}: Why is a raven like a
writing desk?

\begin{quote}
  ``Have you guessed the riddle yet?'' the Hatter said, turning to Alice again.\\
``No, I give it up,'' Alice replied. ``What's the answer?''\\
``I haven't the slightest idea,'' said the Hatter.\\
``Nor I,'' said the March Hare.\\
Alice sighed wearily. ``I think you might do something better with the time,'' she said, ``than wasting it in asking riddles that have no answers.''
\end{quote}

The best interdisciplinary topics offer more questions than answers.
They, are essentially, gateways into the mysterious--which, as
Einstein will tell you, is an important place to be:

\begin{quote}
  The most beautiful thing we can experience is the mysterious. It is
  the source of all true art and science.
\end{quote}

\subsubsection{Suggested Topics for Interdisciplinary Presentations}

\begin{itemize}
\item Akhenaten and the invention of monotheism
\item Albrecht Durer and alchemy
\item Aristotle's book of comedy
\item Bill Evans and the Peace Piece
\item Buckminster Fuller and the geodesic
\item Chenrizi and the politics of China
\item Chuang Tzu and the butterfly
\item Coleridge and the person from Porlock
\item Csikszentmihalyi and the flow experience
\item David Bohm's Implicate Order
\item Darwin, the bassoon, and the sundew
\item Eugen Herrigel and the practice of archery
\item Francis Yates and the \textit{Art of Memory}
\item Freud, Jung, and the ``bosh'' incident
\item Fulcanelli and \textit{Mysteries of the Cathedrals}
\item Giordano Bruno and the Hermetic tradition
\item Godel's uncertainty principle
\item Hanna Arendt at Nuremberg
\item Henri Rousseau in the jungle
\item Howard Carter and ``wonderful things''
\item Jacob Bronowski at Auschwitz
\item Jacob Bronowski, Nagasaki, and \textit{Science and Human Values}
\item Jan Tschichold and the Nazis
\item John Cage on the subway with the \textit{\textit{I Ching}}
\item Kepler's \textit{Somnium}
\item Mary Shelley and the genesis of \textit{Frankenstein}
\item Newton's \textit{Principia}
\item Nikola Tesla and universal energy
\item Philip K. Dick, VALIS, and 2-3-74
\item Picasso, Guernica, and Expo 1937
\item R.D. Laing and madness as reality
\item Ramanujan's notebooks
\item Richard Feynman and the invention of quantum mechanics
\item Schwaller de Lubicz at Karnak
\item Simone Weil and leading from desire
\item St. Exupery flying into the desert
\item The Reimann Hypothesis
\item The Voynich Manuscript
\item The visions of Hildegard of Bingen
\item Thoth's legacy
\item Walter Benajmin and the Angel of History
\item Wendell Berry going \textit{Into the Woods}
\item Wilhelm Reich's Cloudbuster
\item William Blake's \textit{Marriage of Heaven and Hell}

\end{itemize}


\subsubsection{Assessment Criteria for Group Presentations and Overall Engagement}

This course utilizes experiential learning approaches, which depend upon student involvement and active participation. Accordingly, the following criteria are used to evaluate overall participation and engagement in the group presentations and the class:

\begin{itemize}
\item Willingness to take appropriate risks and to challenge oneself.
\item Willingness to speak up and to lead.
\item Openness to interpersonal process.
\item Willingness to collaborate with others.
\item Consideration of and responsiveness to others.
\item Commitment to enhancing the interpersonal experience of everyone in the group.
\item Willingness to examine personal values, beliefs, and judgments.
\item Ability to take personal responsibility for learning.
\item Willingness to deal with conflicts appropriately if and when they arise.
\item Ability to be open and responsive to appropriate feedback.
 
\end{itemize}


The group presentations and overall course engagement are worth a total of 25 per cent of your grade.

\subsection{Attendance and Participation}
The expectation is that you will attend all sessions and involve yourself in the class process. Your willingness to engage creatively with the learning process, to take appropriate personal risks, and to participate in group activities are all central to your involvement in this class. Because developing a style of creativity is very much a process of blending your own personal awareness with skills and practical techniques, your own emotional involvement in the class is as important as your academic knowledge of the material.

Creativity is a unique process. Unlike many other fields, in which competence and skill may be measured objectively, using replicable and consistent means (tests of factual knowledge, for example), authentic creativity depends greatly on the interpersonal skills of the practitioner. Computer programmers can be assessed by their ability to write code; chiropractors can be evaluated based on their skill at manipulating the human skeleton; race car drivers can be clocked around a track. But for writers and artists there are no such fixed measures. The interpersonal skills upon which creativity so much depends are subtle, difficult to quantify, and complex beyond any measurement scheme.

And yet we can identify those who possess exemplary personal and creative skills. They are relaxed, open, responsive, kind. Often they exhibit skills that we tend to assign to the social sphere: personal warmth, consideration of others, hesitancy to judge, sensitivity to emotions. In our class we focus on these interpersonal factors as a foundation for our experiences with one another. And we itemize them as features along the continuum of self-awareness:

\begin{itemize}
\item Commitment to the development of self-awareness.
\item Openness to interpersonal process.
\item Ability to participate in appropriate self-disclosure.
\item Consideration of and responsiveness to others.
\item Commitment to enhancing the interpersonal experience of everyone in the class.
\item Willingness to examine personal values, beliefs, and judgments.
\item Ability to take personal responsibility for learning.
\item Willingness to deal with conflicts appropriately if and when they arise.
\item Ability to be open and responsive to appropriate feedback.
\end{itemize}

Each item on the above list is an aspect of the first item: self-awareness. The most foundational skill in creativity is self-awareness. Those who develop this skill consistently query their own responses, thoughts, and feelings. They ask themselves:

\begin{itemize}
\item What am I feeling right now?
\item What am I thinking right now?
\item Why am I reacting in this particular way?
\item What do my thoughts, feelings, and reactions tell me about myself?
\item Is there anything about my current behavior that suggests unresolved themes in my life?
\item Is my perception of myself consistent with what other people tell me about the kind of person I am?
\item When and how do I get stuck, and what am I doing to work on this?
\item In what ways do I get overwhelmed, or shut down, or avoid?
 
\end{itemize}

These questions, and many others, require the capacity for self-reflection and self-awareness. As we continue in the course, you may wish to consider these questions as they apply to you. At the very least, you might wish to consider what you are currently working on in your life, in which direction your attention is drawn, into which of the innumerable themes of human nature you are now called to delve.

In my role as your instructor, I will be paying attention to how thoughtful you are in examining and responding to questions like those in the lists above. I will not be analyzing you, but rather noticing what kinds of things you do, what your reactions are to various situations. My goal in observing your behaviors and interacting with you is to assist you in developing greater self-awareness (and, by extension, greater creativity).

I will use the lists above, as well as the assessment criteria listed for each assignment, to assess your overall participation in the course. I will not be evaluating your level of self-awareness but rather your openness to the process of developing your self-awareness.

\subsection{Grade Inflation}
Almost every semester there are students who do well on the assignments, complete all the associated learning goals of the course,
participate well, and wonder why they do not receive a grade of one hundred percent (or 98, anyway). Here is the reason: almost every
semester there are students who demonstrates a level of commitment that goes beyond the course requirement. Such students complete extra
work, or hand in exemplary assignments, or undertake a significant amount of personal development in addition to the course expectations.
Such students typically receive the highest grades.

If you do reasonably well in the course you will receive a reasonable grade. Very high grades are intended for extra or exemplary work.
Unfortunately, over the past thirty years the post-secondary educational system in North America has participated in a process of
grade inflation. Since the 1980's, the average grade for typical course work has been increasing by about 25 per cent each decade.
Elevated assessments do not accurately reflect the work of most students. Even worse, grade inflation has caused many students to
expect high grades for average work. I am not a overly stringent assessor; but I will not inflate grades artificially.

In this course a small number of students will (likely) receive high grades; most students will receive grades in the middle range; and a few students will struggle with lower grades. If you are uncertain about your assessment for a given assignment, or if you wish to know where, roughly, you are along the distribution curve of the class, or if you would like suggestions for how to improve your grade, please ask me for clarification.

If you wish to achieve a good grade, please do the following:

\begin{itemize}
\item Show up for class -- every class. This course depends on student engagement. (This becomes especially important during the final weeks of the semester.)
\item Be attentive and mindful to the various criteria listed for each of the projects and the course overall.
\item Take the initiative to plan and develop your projects and presentations. This course is (very likely) more fluid and spontaneous than you are used to. Your ability to manage your time, commitment, and energy is crucial.
\item Speak up in every class (review the criteria for group engagement and presentations).
\item Don't look for the right answer to a question or challenge. Instead, find the answer that is meaningful to you.
\item Ask for help if you need it.
\item Commit to your projects in a substantial way. Good projects take time. Rushed projects are obviously rushed.
\end{itemize}

Finally, please be attentive to the Kwantlen policies on academic honesty and plagiarism, which can be found at the following URLs:

\noindent
Academic Honesty: \url{http://www.kwantlen.ca/__shared/assets/Honesty1432.pdf}\\
Plagiarism and Cheating: \url{http://www.kwantlen.ca/policies/C-LearnerSupport/c08.pdf}

\section{Due Dates}

Group presentation dates will be assigned in class \\(and will fall between weeks three and thirteen).
\noindent\\
Project one (or segment one) is due at the end of week four \\(by midnight on Sunday of that week).
\noindent\\
Project two (or segment two) is due at the end of week eight \\(by midnight on Sunday of that week).
\noindent\\
Project three (or segment three) is due at the end of week twelve \\(by midnight on Sunday of that week).

\clearpage

\section{Thematic Schedule}
The class sessions will be balanced between presentations (by the instructor and students) academic material, group collaboration, and composition. The content for each session will evolve as the semester progresses. We will cover the following themes (though, perhaps not in the order listed below):
\\
\begin{compactdesc}

\item[The Nature of Interdisciplinarity]
Definitions and principles.
Clarifications of common misunderstandings about interdisciplinary approaches and practices.
An introduction to interdisciplinarity as a mode of inquiry in the arts and sciences.
\\
\item[Traditions and Practices]
Consideration of interdisciplinary practices as effective vehicles for the transmission of  sacred, social, political, artistic, and scientific information.
Examination of interdisciplinarity as a fundamental and necessary function of human nature and inquiry.
\\
\item[Developments and Milestones]
Consideration of the development of interdisciplinarity and its role in the contemporary world.
Examination of the relationship between interdisciplinary practices and traditional, faculty-based inquiry in the academic environment.
\\
\item[Themes and Philosophies]
Introduction to the historical background of interdisciplinary philosophy and practice in science and literature.
Explication of ancient world views, with particular emphasis on spirituality, science, mythological concepts, and approaches to the imagination.
\\
\item[Hermetic Threads]
Reading of excerpts from core Egyptian, alchemical, and early scientific texts, with particular emphasis on foundational interdisciplinary ideas.
Examination of the ways in which the sciences and the arts were entwined in the practices and perspectives of all peoples until the twentieth century.
Exploration of the transmission of interdisciplinary ideas into the contemporary world.
\\
\item[Interdisciplinarity in the Expressive Arts]
Exploration of the transmission of interdisciplinary philosophies and practices by way of the expressive arts.
Examination of practices and structures within the interdisciplinary expressive arts (writing, storytelling, dance, movement, ritual, religious practice, music, philosophy, etc.), with emphasis on the traditions of psychology and mythology.
Reading of selected texts within the interdisciplinary expressive arts.
Examination of the ways in which texts within the interdisciplinary expressive arts have influenced the development of the arts and sciences.
\\
\item[Multicultural Principles]
Introduction to the interdisciplinary contributions made by Asian mythology, literature, and science.
Explication of ancient Asian world views, with particular emphasis on interdisciplinary spirituality, mythological concepts, and approaches to knowledge.
Reading of excerpts from Asian texts, with particular emphasis on the synthesis of interdisciplinary expressive arts within those traditions.
\\
\item[Other Stuff (that we'll think up as we go along)]
Much of the course content will be generated by students.
\end{compactdesc}
\section{Sample Project}

\subsection{Deep Water}

Today the work is tense, rushed, possessed of a sharpness I feel inside, as though I might crack open from the cold. The heater is cranked up, I’ve opened the door to the adjoining furnace room, and I’m wearing a thick fleece jacket. But these measures barely soften the hard edge of the cold in the shop. It’s December, and the sun shows itself so infrequently there are days I wonder if it’s come up at all. Elizabeth tells me I should come in, my shop is not comfortable enough to use during the winter, I should leave it all until spring when I can open the garage door and let the glorious morning light sweep through. She’s right, it is too cold. My hands are red and puffy; the tools seem less forgiving, the wood brittle.

Yet I persevere. This is supposed to be fun, and I ­won’t allow a bit of physical discomfort to get in the way of my forced enjoyment. After all, I’m excited about this project, the marimba that I’m hoping will make the wood sing with a new voice. Marimbas are musical instruments of African and Latin American origin, similar to xylophones but with wooden keys that produce a sonorous ululation of notes. I’ve never made a project such as this, and the novelty of it, the challenge of enticing the wood to call out with its own rhythm, has me working every night, battling the cold, my little shop lights defiantly brightening my small space against the wide, indifferent darkness.

Sounds of the world travel far on the chill air: the crack of brittle bark in the forest, the steady drip of water falling from the eaves onto the tarp-covered fir beside the house, the slam of a car door shutting out the wind. The lights and colors of fall have given themselves over entirely to the clear sounds of winter, the physical austerity of it, and my body responds haltingly. I watch the wood less and feel it more. My hands are immersed in the sensations of the shop, steel tools and a concrete floor and boards that wake me with their intensity as I touch them.

The work I do in winter is like a journey on a wide, dark sea. I slow, wrapped in expectation, as if waiting for the great bell of the sky to be struck. Winter lays bare what is fundamental, strips the world of adornment, beckons me to wait and listen for the voice of the invisible. It was once thought that a great silence would allow one to hear the music of the spheres; winter encourages that silence, leads me down to an elemental realm of dark sea, dark stone, dark shore. And I listen, trying to understand the unspoken language.

My kids have a small marimba, nothing fancy, yet the music it produces is surprisingly resonant. And I thought, I could make one of these, could probably even improve on it. The joints could be tighter, the keys more finely cut, the finish refined. Soon enough I was out in the shop, making plans, sawing wood, embarking on another grand expedition of craft. But I’m out of my depth: that’s soon apparent. Sure, I bought a book on instrument design and looked at a few marimbas, even took some measurements to get my bearings on how to proceed. But a map is not the territory, and I ­don’t get far before I start to wonder why this seemed like a good idea.

It begins with the keys: selecting a number of woods, making a key from each, testing for the richest sound. There’s no guideline about which woods work best for marimba keys. Each has a distinctive character and makes music that reflects its nature, its experience of growing in the forest. A key may make a gentle sound that sustains for four or five seconds, like a weathered trunk yielding to the wind, or it may possess a more distinctive ring, sharper, clearer, but more quickly fading, like a woodpecker hunting deep in the tree.

I plan to make roughly half a dozen preliminary keys to test for the optimum combination of wood species and key shape. Because I’ve read that exotic hardwoods can work well, I select several pieces from the small inventory of cocobolo, purpleheart, and padauk that I keep in my shop, being careful to waste as little as I can, knowing I might use only one of the test keys in the final instrument. The others will be put aside, odd-­shaped remnants difficult to use in other projects. I keep such offcuts in a bin that still contains discarded pieces from my first project, the cedar bench, for which I twice miscut the curved rails. Occasionally, as I rummage through the bin looking for a piece that’s just right for a jig or a clamping aid (a caul, to be precise), I come across those cast-­off rails and wonder how, in the depth of my ignorance of the craft at that time, I made anything at all. Yet the bench is still there, weathering the seasons in our backyard, joints still tight, lines still true.

At first glance, making a marimba key seems like a straightforward procedure with a couple of basic re­quirements. A key that is too large, say anything over two inches wide and eighteen inches long, vibrates at a frequency too low for making music. A key that is too small, about eight inches long or less, results in a tintinnabulation too high to be pleasing. I make my test keys in the middle range, about an inch and a half wide, twelve inches long, and an inch thick. Then I add a wide, curved cut halfway through the underside of each key. This lowers the basic tone of the key (the fundamental) and enhances its resonance. As I work, I become aware of instrument-­making as an activity distinct from other kinds of woodworking. Most woodcraft involves searching for a visual aesthetic, laboring with the body to shape that which pleases the eye. In instrument-­making, conversely, one searches for the most pleasing sound, favoring auditory clarity over visual beauty. It is a realm not only of aesthetics but audiosthetics.

This is where I run into trouble. I can make an attractive key, one with clean lines, a smooth, curved underside, and grain that flows well with the shape and size of the key. But I discover that I have no way of assessing the sound the key makes. I line up four or five on a set of foam pads I’ve rigged up for testing purposes and tap them in sequence: bing, bong, bung. Are those good sounds? Are they musical? Do they make notes like the notes on a marimba, or on a piano? And although I am aware that the sounds from each key are different – I can hear the distinctiveness – I cannot tell which tones are higher or lower. I tap a key, then another, going back to the first again and trying to listen for the higher tone. The first key sounds higher, then the second. I tap a third for contrast, and they all sound the same.

Since the keys are made from different woods, perhaps the variances in tone are due to their different properties. To find out, I make another four keys, all of purpleheart. I shape the largest to be eighteen inches; the smallest is ten. When I test them I can clearly discern the lower notes from the higher, which is a definite improvement. Now I need to know the specific note each key produces so I can create a musical scale. I take my kids’ marimba into the shop and compare the sounds of its keys with my test keys, looking for similar notes. Once again the sounds blend together, making it impossible to distinguish the tonal differences. I ­don’t get a particular sense of the pitch rising or falling; at least I ­don’t perceive it as a vertical scale. I simply hear the sounds: pleasant, sonorous shapes in the air, brushing past me and dwindling into the winter day.

I enjoy the sounds, and I play around just tapping keys, not looking for a particular tune but listening to the results of an amazing discovery: wood sings. Every woodworker knows that wood speaks, but this singing – it is like discovering your arms can also be wings.

I ask around and discover that there’s no such thing as tone deafness, just a lack of ear training. My inability to hear the scale of the notes is simply perceptual innocence. Armed with this revelation, I rush down to the music store on a rainy night and purchase two slender steel tuning forks as reference notes. Back in my car I strike the forks one at a time against the steering wheel and hold them to my ear. They produce clear, resonant notes, an A and a C, which sustain for close to ten seconds. Lovely. But when I hold them up at the same time, one against each ear, I ­can’t tell which is higher. I have to check the tiny letters inscribed in them to be sure.

I return to the shop and try to compare the sounds of the tuning forks with those of my marimba test keys. They seem so different, one set produced by metal and the other by wood, that I ­can’t discern which keys most closely match the forks. Everything makes its own sound: distinct and resonant yet secluded. There is no harmonious choir of notes. Which is well and good if all I want is a bunch of separate keys on my workbench. But I am trying to make an instrument, something that will produce an even scale of notes, an object to sing with harmony. So far I have quite a few keys made from expensive, endangered woods, a couple of nice-­sounding tuning forks, and no results. And I ­don’t even know how to get on track. I have no map, and though I’m used to exploring in this craft, watching and discovering as I go, it’s not my eyes that I need but my ears, and they seem to wander all over, finding many delights but no clear channel to follow. I’m starting to feel a little lost.

In Taoist philosophy the energy of deep water begins with disorientation, with a sense of being drawn onward in ways that are uncomfortably challenging. In creative terms, deep water lies at the heart of authentic work. There is a fashionable notion that involvement in creative work, in the spirituality of creativity, will lead not into shadow but toward a great light, toward happiness or self-­improvement or refinement of one’s energy and purpose. Yet the creative journey also leads to a shadowland of doubt, fear, frustration, and depression. It passes directly, unavoidably, across ominous stretches of water where destructive, threatening forces lie in wait. In my truest creative work I must be taken apart by those forces, relieved of my carefully constructed masks, and laid bare to see how restless and fugitive I really am.

I hang around in the shop for a few days, halfhearted at best, trying to make sense of this whole procedure. I’m steadily going through my supply of good wood, it’s getting colder every morning, and the kids, virtually housebound from the rain, never give me more than ten minutes of actual working time. Both Rowan and Avery are quite capable of playing happily for at least an hour; they do so regularly when I am washing dishes or making supper or cleaning up toys. For them, the main thing is that I be in sight, immediately accessible. As long as I stay in the same room with them, I can find two or three hours in a given day to accomplish household tasks. Usually this means puttering around the kitchen while they play in the adjoining family room. But as soon as I go through the door into the shop – either announcing my exit or sneaking off quietly, it ­doesn’t matter – the sounds of bedlam come echoing down the hall: wailing and crying, screaming matches over the ownership of toys, plaintive moans for food or juice as though the fridge has been bare for weeks.

Sometimes I encourage them to come into the shop with me. They both have small kits of plastic tools – saws, hammers, screwdrivers – and I help them work with offcuts. Last year Rowan made a couple of hanging mobiles with plane shavings and bits of string. Originally I thought that bringing the kids along would give me more time to do hand work (I ­don’t use power tools when they’re around), but the regret ­wasn’t long in coming. The kids have the idea that sawdust is the most entertaining thing in the shop. They make little piles with it, spread it around the floor, drop it into bins already filled with equipment, throw it at each other by the fistful, dump it on each other’s heads. I fill a thirty-­gallon plastic garbage container with sawdust twice a month and gather it up only when it becomes a serious hazard on the floor, so they have plenty of ammunition. What do I expect, letting loose a two-­year-­old and his five-­year-­old sister in a giant sandbox of shavings?

It’s a good thing a marimba key takes only about ten minutes to make, because that’s all the dedicated time I have during these long, dark days. Elizabeth helps when she can, diverting the kids from the shop and sometimes taking them out for a couple of hours when I’m desperate for working time. But it’s a slow, frustrating haul. Deprived by the cold of backyard diversions, the kids turn to Elizabeth and me as virtually their sole entertainment. Sometimes, when they are jumping up and down on my last nerve, I envy the isolated artists, the ones whose solitary lives offer endless stretches of time and freedom in which to work. Craft so often demands selfishness, a resolute focus on one’s own ideas and plans. Attending to the needs of others falls distantly behind. Parents, conversely, are required constantly to surrender their own needs in favor of their children’s. Negotiating the balance between sacrifice and self-­interest in relation to one’s children is the most creatively demanding task available to a human being. It is far more difficult, by any measure, than the personal odyssey of the artist.
Which, I suppose, raises the question of why we often look to artists as teachers of society and to art as a repository of wisdom. ­Shouldn’t we look to parents instead, and to children? Working as an artist or craftsman can be a substitute proof that one is really in the world. But being a true parent requires that one be firmly, inescapably, rooted in the soil of life.

All this loose talk gets me no closer to making a marimba. The keys are still there, indecipherable. I’m wasting time. The workshop is cold and damp. There’s mess everywhere, sawdust and discarded tools and offcuts drifting into little piles at the foot of the bandsaw. Yet I ­can’t get motivated to clean up, ­don’t seem to be going anywhere, there’s not enough time to get anything done. It really is cold. And now I’m complaining, as though this craft forces unmanageable demands on my otherwise pristine life.

I decide to give up for a while, let myself drift until something nudges me back on track or convinces me to abandon the project altogether. I am adrift, far out in the sea of winter without any means of movement. If I remain this way too long, waiting for the puzzle of the keys to resolve itself, I will likely be drawn into the depths where depression, disillusionment, and their many companion shadows lie. I’d rather move forward, yet I’m aware that in my creative work there are times when the momentum departs, energy dwindles, the safe passage vanishes, and all comes to a shuddering halt. Sometimes what I need, the particular flavor of truth I seek, can be found only in the deep water.
I stay out of the shop for close to a month as the flurry of Christmas gives way to a new year. A muted snowfall in early January has me out clearing the driveway, desultorily, the blade of the shovel ringing in the cold air as it scrapes the pavement. Its wooden handle is remote and unyielding as I plow away mechanically, distantly aware that my relationship with tools, with wood, has been stalled by my impasse with the marimba. Not enough knowledge, no discernible track to follow. I begin to sink.

\vspace{1em}
\noindent
One minute she was there on the boat deck, shouting and crying with a look of mingled rage and terror, desperate; the next, she had jumped over the side into the dark water. It happened quickly, and was so unexpected that we all rose up in a chorus of panic. My younger brother began to cry, the confusion and madness of it swelling inside him. He was only four – what was he to make of his mother plunging into the frigid waters? My father handed the wheel to my elder brother and jumped in the dinghy. My brother stood there, small and frightened, his cold hands gripping the bright steel. He brought the boat around as my father pulled her up from the water and into the dinghy, her jacket sodden, her hair wild. I ­didn’t want to look at her, not knowing how to respond to such violent upheaval. It would have been an intrusion into her own private world of despair. And I hated her for living in that world. My father turned the boat back to the dock and helped us unload our bags. My mother ran up the ramp and disappeared. My father drove us home before going to look for her. No one spoke. I was nine.

My mother never came up from that dive, though it took her twenty years to complete her long sweep into oblivion. The water laid claim to her, never letting go; it kept drawing her down until there was nothing left but to surrender to that cold embrace. We followed her down, of course, into the dark labyrinth.

The seascape of constant stress and upheaval that I walked through with my mother was not graceful, yet it was among the most powerful journeys of my life. Through her I learned deeply about what is inevitable in true creativity: descent to the abyss. The truth most often lies hidden there, and holding out long enough to glimpse it – without drowning – is a serious race indeed.

It takes just over a minute for a drowning person to take a deep breath of water. No matter how desperately the breath is held, waiting, clutching for the surface, there comes a moment – the racing heart of panic brings it more quickly – when the danger of imminent unconsciousness from oxygen deprivation causes instinct to override volition. It’s the body’s last gamble: holding the breath much longer will result in certain disaster, and a quick breath might coincide with the availability of oxygen. One final bet, no matter how ill-­advised, is usually a good idea when there’s nothing left to lose. As the closure of the aperture of consciousness approaches, the lungs heave, the mouth opens, and water makes its ultimate trespass. This entry of the catastrophic is called the break point.

Near the end of January it becomes clear that I must dive into the work again or risk losing my motivation altogether. The break point has arrived. Cutting my losses and moving on to a new project feels like a failure of courage – I need to know how to decipher the resonant language of the wood. I need to understand the song. No more holding, no more waiting.

I spend the last week of January in San Francisco, where I get my hands on some California redwood. I’ve heard it works wonderfully for marimba keys. Perhaps the keys possess a deep song, what the poet Lorca called cante jondo, because of the particular qualities of the wood. After all, redwoods are among the largest and oldest trees on the planet; some were alive well before the time of Christ and along with the other conifer species possess the most complex dna on the planet.3 More complex than our own, for reasons that are obscure only until the wood and the hand begin to speak.

I purchase an eight-­foot board, enough for twenty keys, and have it cut into eighteen-­inch lengths. I cram the pieces into my luggage; they ­don’t quite fit. Several ends stick out the top of my carry-­on bag, and as I make my way home through airports and in shuttle buses, I notice the sorrel, almost pink hue of the wood. It reminds me of the first tendrils of sunrise, sweeping the eastern sky with light.
I abandon my efforts at ear training, tuck the tuning forks away at the back of my workbench, and buy an electronic tuner. I hold the magic little machine beside a key, strike the key, and wait for the led readout to show me the exact pitch of the sound along with its assigned note. So far so good. It’s a lazy work-­around for the problem of my ill-­trained ear, but it provides the jumpstart I need.

The redwood shapes nicely into two test keys. I place them alongside the keys of padauk and purpleheart and tap them all with the mallet, listening, watching the readout on the tuner. The hardwoods possess a clearer tone but the redwood is more sonorous, sounds more forgiving, sustains a bit longer. Of all the woods I’ve tried so far it seems the best choice. With more than a little chagrin I place all the other test keys aside, hoping that I can use them in other projects but knowing their odd shapes and relatively small size will likely mean a long residency in the offcut bin. I keep wondering if I could have found a way to be less wasteful.

I fashion fifteen keys between eight and sixteen inches long. The instrument book informs me of the length each key should be to produce a sound one note higher than its companion.4 Once shaped, each key requires fine-­tuning with sandpaper on its underside or edges to bring out its exact note. Things are starting to improve, the winter seems less indifferent. The feeling of being dragged down is starting to dissipate.

I finish tuning six keys and playfully run the mallet up and down the even-tempered scale, thinking how easy this is becoming. I’m almost back in known territory. Then I take the completed keys inside to show Elizabeth and leave them on the kitchen counter while we eat supper. By the time I get around to showing off my prowess as a musical instrument maker, the keys have warmed up to room temperature, the wood has taken on different tonal properties, and I realize, with growing despair about this entire project, that the keys are no longer in tune. The scale sounds rough and unbalanced, the notes a haphazard mix of noise.

It is difficult to get back on track and then be broadsided by unforeseen problems. I ­don’t seem any further ahead. How am I supposed to tune the keys, and hold them in tune, if every time I move the damn things to another room all my efforts are wiped out? I ­can’t remember when a project has been so frustrating. And I’m still near the beginning; I have the frame to make as well, and the suspension system for the keys. And finishing. What a mess this is all turning out to be.

The \textit{I Ching} offers many strategies for finding safe passage through the deep waters. Several are collected under the heading “Mastering Pitfalls,” which is represented by the image of two deep-­water trigrams; diffi­culty upon difficulty. The commentary emphasizes truthfulness and perseverance: accepting the reality of descent, preparing to meet the hidden faces of the deep. Thus it helps to be honest with myself and acknowledge my growing hatred of this project, this quaint little instrument that began as an exploration and is turning into a disaster. I must also believe in the danger of it, the way it threatens to stall the good feelings I hold for my craft. And despite these I must keep going, “without hypocrisy or deception, practicing truly,” as the Taoist commentators urge.

Perhaps I am sinking into the deepest waters and my burst of enthusiasm was nothing more than the involuntary breath of a drowning person. Perhaps there’s nothing I can do to prevent this project from taking me down. If prior experience is any indication, I should just go with it.

After the break point, when water invades the body, the bloodstream is deprived of whatever remaining oxygen might be in the lungs. The brain begins to shut down, awareness becomes irretrievably distant, muscles cease their spasmodic flailing. The surface dwindles, sinking accelerates.

Some of the deepest creative work derives from affliction, from the racking pain inflicted by life’s indifference. I see clearly, with the final look of my childhood innocence, that my mother will truly drown, clawing at the bright bonds of my family as she goes, dragging us down, grasping us with such furious desperation that we will eventually be fractured, worn almost to nothing. I glimpse the later moments of horror and terror, the long road of many years of secrets, violence, and confusion. And at the end, when our last chances for making peace as a family come and go, we are left with only the grief of our inevitable disappointment.

Creative energy lies buried in the heart of the wound. And the creative path requires that I redeem myself through that wound, take my deepest wisdom from it. That’s the way it is. There’s no fullness without first that rough, implacable opening as the world pries into you with its darkest shapes.

I keep working, making more keys, leaving the issue of final tuning for later. I try not to get carried away with the philosophical or spiritual resonance of the word surrender but keep focusing on the actual work, what it presents each moment physically and emotionally: the challenge to keep going, even when the results seem distant, the recognition that I might waste not only the test keys but also this lovely redwood, the persistent doubt and anxiety, the memory of those childhood days of deep water this project has revived. The keys come alive with strange music in my work, blending their harmonies with the sounds of my feeling.

Other, more disturbing concerns arise: that I might become stranded in the deep water, trapped like my mother in an endless whirlpool of descent; that in the oppressive darkness I might lose my way and let go of this craft that has so nurtured and challenged me; that my creative work, bereft of light, adrift in the shadow of this incomplete and persistently difficult endeavor, might slide into purposeless apathy; that I might forget myself, and in forgetting abandon whatever creative contribution I could have made. I might just forget the whole thing and live another, quieter life, haunted by what I should have done.

The curious thing about wood is that it does not allow me to give up. It cajoles and prods, it demands that I keep my end of the bargain and bring out its hidden forms. At times it even taunts me, this ancient material possessed of its own craftiness. I’m not sure exactly how the wood makes its demands – saying that it speaks is only a shorthand for something more elusive and mysterious. Seeking to define it is only a game of words.

From some musician friends I discover that the tuning of many instruments fluctuates; the secret is to keep them in an environment of stable humidity and temperature. I need to keep the shop warmer, at room temperature, but around the time I learn this the heater breaks down. The only warmth now comes from the halogen lights and the fleece jacket I wear. I pretend it’s warm enough to work. Another snowstorm hits and I am forced to evacuate the shop. The cold is just too much. I catch myself hoping for a lightning strike, something to jolt me into action, to burn away this suffocating paralysis. 

But lightning is not one of the gifts of the deep water.

February passes. I finish all the keys, working in fits and starts between spells of intense cold. The work feels rudderless, even though the solution to my tuning problem seems straightforward. I buy a new heater and warm up the shop, but I hardly go in at all. I’m sinking, somewhere between the surface and the ocean floor. I wonder about the craft I’ve chosen, try to remember why it’s so important. I wish the sun would come out, or another storm would break the monotony. But nothing. Rain and gray and the feeling I’ve lost the thread.

Then, in mid-­March, one of my drowning eyes pops open. I’ve been working exclusively on the marimba keys, trying to complete them before designing the frame. A dream of a Japanese bridge stays with me through a damp morning and suddenly, just swimming out of the black water, comes an idea for the frame: the delicate curve of a bridge, arching upward to support keys that can be a walkway across the deep water. I make a simple sketch, which is about as far as I ever get with design, and an inkling of excitement thrums through me. The depths have offered me clear passage, a gift from the dark waters. I’m reminded of Rua, the Tahitian god of craftsmen, whose name means ``abyss."

There are spirits in the deep water, phantasms that appear only to the drowning and to submariners of surpassing skill. Down deep, where the water is colder than ice but under too much pressure to freeze, where the darkness is absolute and hydrothermal vents expel poisonous toxins in black, acidic spumes – far down, in an environment most remote from us – there is exotic, pristine, awesome life, the oldest life of this world. The abyss is home to the most primordial forms, and they are guardians of a great secret: dark, still waters are the sanctuary of the soul.

Apparitions both angelic and bestial inhabit this seascape of elemental dreams. The long-­nosed chimera, also known as the ghost shark, has a dorsal spine so venomous a single touch can kill. The male of the triplewart sea devil, with its bioluminescent tendrils, chews through the skin of the female and fastens himself on, merging their bloodstreams in a bizarre sexual dance.

This pantheon of monstrosities is large indeed. The viperfish boasts fangs so long they extend beyond its mouth, arcing upward toward its deep eye sockets. Its abyssal companions include the fangtooth fish, also known as the ogrefish, and the group of gulper fish that can swallow prey larger than their own bodies. One of their number, the black swallower, draws its curved, needle-­like teeth slowly over the entire length of its trapped victim. And there are creatures of mythological heritage: Vampyroteuthis infernalis (literally “vampire squid from hell”) has the largest eyes of any animal, and the basket starfish, belonging to the family Gorgonocephalidae, is named after the snake-­haired Gorgons of Greek mythology.

But just as the winged horse Pegasus emerged from the bloody remains of the Gorgon, the deep is a place not only of nightmares. Elegant swordfish travel the wide range between deep water and inviting surface. Among the many varieties of lantern fish, which give off their own shimmering and refulgent light from within, are those that journey to within a few feet of the surface at night and return to the abyss at daybreak. Their bodies are richly colored in glistening blue and iridescent silver. Vast arrays of them have appeared beneath shipwrecked sailors drifting in the north Pacific, hovering just below the surface, watching or waiting or wondering – no one knows – and vanishing as the rescue boat draws near.

The deep is home to ten million species of life, far more than are known to exist on land. Only the drowned and the recklessly brave have the opportunity to enter that hidden realm where no light shines. Yet, like the lantern fish, one can undertake that far journey and return.

Above my workbench, hanging in front of my first-­aid kit on the upper shelf, is an old black fishing float, a long wooden ovoid with a large central hole through which a net rope once passed. I found it on the beach, near my grandmother’s summer home, around the time my mother jumped from the boat. My father helped me fashion on it the shape of a rudimentary face: the curve of a shell glued on to make a mouth, eyes animated with gold and silver hobby paints, a segment of frayed rope through the hole making a shock of wild hair. I made that float into a charm, a deep-­water talisman that stayed with me through the turmoil of those years. And still it watches, its black, elemental form like a creature of the abyss come to make a companion of the light. Who knows what forces come to our aid in the deep water? We never quite see their elusive faces but rather sense the accompaniment of a guiding gentleness that ushers us onward.

I decide to use pine for the marimba frame. I have a generous supply in my shop and if I run into further trouble it’s inexpensive to replace. The only drawback to pine is its softness, though it has not always been a very soft wood. At one time, a hundred years ago and more, the forests of North America were tightly packed with trees that grew more slowly as a result of diminished light and space. Growth rings on those old trees were closer together and the wood, therefore, was much stronger. In those days one could use pine for the hull of a ship. Today, forests have been thinned dramatically and pine is among the softest woods, easy to saw but difficult to shape with precision.

In creative terms, the only safe way out of the depths is to let the darkness slowly release its grip, to glide upward in a slow, easy arc. Creative crisis or creative illness usually happens when one tries to surface too quickly or cannot surface at all. Sometimes one is caught between the terrible inertia of the deep and the reckless rapture of surfacing. This is what psychologists used to call manic-­depressive illness. It is the cycle of the gifted artist: to descend fully into the depths where ancient secrets lie, partake of the original dreams of the world, and then arrow upward to burst gloriously from the water, flanks sparkling in the sun like a great swordfish, hanging for a moment like a dazzling fire poised on the palm of the sea. But only for a moment.

Surfacing without incident requires evading predators at every depth. The tablesaw, ravenous shark that it is, reminds me of this as soon as I begin to shape the parts for the frame. Halfway through my first cut the saw seizes the workpiece in its jaws and flings it back at me. A whistling blur rushes past my ribs, arrows across the shop, and punches through the wall twenty feet away. A small cloud of wallboard dust settles to the ground. I can see the exposed wiring of the house through the hole. Careful.

The mishaps continue. Gluing up the frame panels goes poorly: the grooves on the curved frame legs are difficult to cut and require completion by hand; an end panel fails to seat correctly, resulting in a flurry of panic to correct the problem as the glue is drying. But I’ve come to expect such hurdles with this project. I begin to take them in stride, as though the deep water must extract a toll for allowing a glimpse of its astonishing shapes. The Japanese bridge I saw is among the most elegant forms I’ve attempted to reproduce. It is a gift of the abyss.

And I am surprised to discover, upon measuring where the curved legs meet the uprights of the frame, that the angle is seventy-­two degrees, the “ruling number” of ancient sacred geometry. It is the number that unlocks the esoteric religion of astronomy to which the ancient Egyptian culture was devoted, and here it is, like a secret code in my marimba, come back like a stowaway from the black depths.

The frame goes together in stages through March and April as winter slowly retreats. The air feels more damp than cold, the street at night more slick than hard. Scents of spring ride the air, but like the music of the marimba keys, their identification eludes my untrained nose. The shop can now be kept at room temperature, and I tune all the keys to within a hair of what the machine points to as the perfect note. The suspension system – thin colored rope threaded through steel tubing, O-­rings at the rubbing points to dampen vibration – is an odyssey of experimentation. I become a metalworker for a few days, grinding and drilling material that glows red-­hot under duress, stains my hands the color of dark stone, and, more than once, spits out a tongue of flame.

Prior to mounting the keys I apply a coat of rubbing varnish to the frame – just one coat so as not to impede the amplifying function of the panels. I leave the keys themselves unfinished after discovering on one of the test pieces that a single coat of oil affects the tuning by as much as half a note. I polish the keys instead, using fine abrasive paper, until the grain shimmers, languid.
I thread the keys through the arrangement of cord and steel tubing, adjust the tension – and then it’s done. I find the small mallet I’ve been using all along, the one I lifted from my kids’ marimba, the one that’s been buried beneath flakes of ground steel and pine shavings since I began the frame. It feels cool in my hand, expectant. 

I tap the keys, poised now alongside their companions as a choir might arrange itself. The sounds are like bells shaped by hands of water, low and soft as the whispering voice of the sea at twilight.



\end{document}

