%!TEX TS-program = xelatex
    %!TEX encoding = UTF-8 Unicode
%    \documentclass[10pt, letterpaper]{article}
\documentclass[letterpaper,10pt,headsepline]{scrreprt}
    \usepackage{fontspec} 
    \usepackage{placeins}
    \usepackage{multibbl}
    \usepackage{graphicx}
    \usepackage{hieroglf}
    \usepackage{txfonts}
    \usepackage{url}
    \usepackage{titling}
    \usepackage{geometry} 
    \geometry{letterpaper, textwidth=5.5in, textheight=8.5in, marginparsep=7pt, marginparwidth=.6in}
    %\setlength\parindent{0in}
    \defaultfontfeatures{Mapping=tex-text}
    \setromanfont [Ligatures={Common}, SmallCapsFont={ITC Officina Serif Std}, BoldFont={ITC Officina Serif Std Bold}, ItalicFont={ITC Officina Serif Std Book Italic}]{ITC Officina Serif Std Book}
    \setmonofont[Scale=0.8]{Lucida Sans Typewriter Std} 
    \setsansfont [Ligatures={Common}, SmallCapsFont={ITC Officina Sans Std}, BoldFont={ITC Officina Sans Std Bold}, ItalicFont={ITC Officina Sans Std Book Italic}]{ITC Officina Sans Std} 
\usepackage[ngerman,english]{babel}
\usepackage{scrpage2}
\usepackage{paralist}
\clubpenalty=6000
\widowpenalty=6000
\author{Ross A. Laird, PhD}
\title{Interdisciplinary Expressive Arts 1100}
\date{\today}
\ohead{Ross Laird}
\chead{IDEA 3301}
\pagestyle{scrheadings}
\setcounter{secnumdepth}{-1}
%\contentsname{Contents}
\begin{document}
\begin{titlingpage}
\begin{center}
\maketitle
\end{center}
\end{titlingpage}
\tableofcontents
\subsection{Instructor}
Instructor: Ross Laird, Ph.D.\\ 
Office: D308, Surrey campus (by appointment)\\
Telephone: 604-916-1675\\
Direct Email: \url{ross@rosslaird.com}\\
Support Website: \url{help.rosslaird.com}\\

\section{Basic Philosophy of the Course}

This course explores interdisciplinary inquiry and creativity. We examine the uses of interdisciplinary academic approaches and the applications of interdisciplinarity in culture, education, and professional domains. The course is, for the most part, led by the initiatives and interests of students working individually and in groups. The course content will evolve as the course develops, but will include a diverse number of current interdisciplinary approaches to help students explore potential academic and career directions.

Interdisciplinary approaches embody a specific set of educational activities, goals and strategies. Based on innovative pedagogy and integrative approaches to learning, interdisciplinary
studies involve the synthesis and synergy of various disciplines toward a cohesive, unified educational experience. Interdisciplinarity is much more than enrollment in courses from more than a single discipline. Authentic interdisciplinarity emphasizes the linkages between disciplines by focusing on contrasting and complementary aspects of diverse educational domains.

Interdisciplinary studies encourage students to develop broader intellectual skills, greater facility for critical thinking, and greater awareness of the social relevance of their education. Interdisciplinary students have the opportunity to develop exemplary skills in problem solving, insight, team-building, lateral thinking, and multi-modal learning styles. Interdisciplinary strategies involve approaching an issue or problem from various perspectives. This typically entails intellectual inquiries that range beyond the borders of any single discipline or domain. While still respecting the function of the boundaries between domains, interdisciplinary approaches recognize that those boundaries are essentially arbitrary and do not always serve the goals of learning. Global warming and the AIDS pandemic are two examples of contemporary issues that require interdisciplinary approaches.

In this course we stake out the territory of creative inquiry, inspecting the geology of its forms and ideals, finding our own individual places to homestead. Authentic inquiry involves the search for truth, yet also an awareness that truth and fact are often provisional, and mythological; they are shapeshifters on the wide-open plain of human awareness. We will explore what this means, and what to do about it.

And, finally, the goal of the course (from my point of view, at least), is to have fun: to preserve and nurture the creative and imaginative spirit that is the foundation of all the arts and sciences.

\section{Learning Goals}
\begin{itemize}

\item Read selected interdisciplinary texts and discuss their origin, development, and contemporary relevance.
\item Interpret interdisciplinary traditions within the context of contemporary academic and professional inquiry.
\item Articulate (verbally, in writing, and through creative endeavour) knowledge of interdisciplinarity as both an ancient and a current mode of inquiry.
\item Describe the ways in which interdisciplinary modalities are more than a simple aggregation of various disciplines.
\item Use interdisciplinary approaches in compositions, creative projects, and online collaborations.
\item Apply interdisciplinary approaches to the contexts of personal and professional development.
\item Evaluate diverse interdisciplinary perspectives and approaches using heuristic modalities.
\item Complete a series of linked interdisciplinary projects and presentations.
\item Contribute to an interdisciplinary community.
\item Facilitate interdisciplinary collaborations, discussions and projects.
\item Create an interdisciplinary digital portfolio.

\end{itemize}

\section{Learning Experiences}
The course will include a variety of learning experiences contingent upon regular attendance and dedicated participation. Because authentic interdisciplinary inquiry is an interactive process, much of the class time will be devoted to group experiential exercises, individual reflective tasks, collaborative endeavors, and practical assignments.

We will create a collaborative environment in this class. We are not going to cobble together the type of group one often hears about in the arts and education: competitive, cut-throat, critical. Repeat: we are not creating such a group. Instead, we will direct our efforts toward building upon the individual strengths of each participant, finding ways for each of us to be self-reflective in terms of assessing our work together, discovering a means of protecting the quality and integrity of our contributions. The creative spirit is remarkably persistent, yet it is also fragile, especially at its inception, and we must be conscious of this fragility. Think about it: did you not experience, as a child, the strangulation of your creativity in school, by way of a culture of insensitive peers or teachers? Why do you think hardly anyone feels comfortable singing in public, or dancing, or drawing, or reading their written work to others? We have, most of us, been the victims of inappropriate feedback and judgment. We have to be careful about this, in our course, so that we do not harm one another.

\section{Readings}
\subsection{Required Course Text}

\begin{description}

\item [Pirsig, Robert] \textit{Zen and the Art of Motorcycle Maintenance: An Inquiry into Values}. Any edition.
\end{description}

\subsection{Suggested Books}
\begin{description}
\item [Barron, F., Montouri, A., and Barron, A., eds.] \textit{Creators on Creating: Awakening and Cultivating the Imaginative Mind.} 
\\New York: Putnam, 1997.
\item[Wilson, Frank.] \textit{The Hand: How Its Use Shapes the Brain, Language and Human Culture.}
\\New York: Vintage, 1998.
\item [Allen, Pat.] \textit{Art is a Way of Knowing.} \\Shambhala, 1995.
\item [Hyde, Lewis.] \textit{The Gift: Imagination and the Erotic Life of Property.} 
\\New York: Vintage, 1983.
\item [Butala, Sharon.] \textit{Wild Stone Heart}. \\HarperFestival,
  2000. \textsc{ISBN 000255397X}.
\item [Calvo, C\'esar.] \textit{The Three Halves of Ino Moxo}.
  \\Translated by Kenneth Symington. \\Inner Traditions, 1995.
  \textsc{ISBN 0892815191}.
\item [Campbell, Joseph.] \textit{The Mythic Image}.
  \\Princeton UP, 1974.
\item [Ellis, Normandi.] \textit{Dreams of Isis: A Woman's Spiritual
    Sojourn}.
  \\Quest, 1995.
\item [Hancock, Graham.] \textit{Heaven's Mirror: Quest for the Lost
    Civilization.}.
  \\Crown, 1998.
\item [Hedges, Chris.] \textit{War Is a Force that Gives Us Meaning}.
  \\Anchor, 2003. \textsc{ISBN 1400034639}.
\item [Kingston, Maxine Hong.] \textit{The Woman Warrior: Memoirs of a
    Girlhood Among Ghosts}. \\Vintage, 1989. \textsc{ISBN
    0072435194}.
\item [Kwan, Michael David.] \textit{Things that Must Not be
    Forgotten: A Childhood in Wartime China}. \\Soho Press
  \textsc{ISBN 1569472823}
\item [Laird, Ross A.] \textit{Grain of Truth: The Ancient Lessons of Craft}. \\MWR, 2000.
\item [Langewiesche, William.] \textit{American Ground: Unbuilding
    \\the World Trade Center}. \\North Point Press, 2002.
  \textsc{ISBN 0865475822}. (Also see \textit{Inside the Sky}.)
\item [London, Peter.] \textit{No More Secondhand Art.} 
\\Boston: Shambhala, 1989.
\item [Lopate, Phillip.] \textit{The Art of the Personal Essay: An
    Anthology from the Classical Era to the Present}. \\Anchor, 1997.
  \textsc{ISBN 038542339X}.
\item [Macfarlane, David.] \textit{The Danger Tree: Memory, War and
    the Search for a Family's Past}. \\Walker, 2001. \textsc{ISBN
    0802776167}.
\item[McNiff, S.] \textit{Art Heals: How Creativity Cures the Soul.} \\Boston: Shambhala Publications, 2004.
\item [Merwin, W.S.] \textit{The Mays of Ventadorn}. \\National
  Geographic Directions, 2002. \textsc{ISBN 0792265386}.
\item [Ondaatje, Michael.] \textit{Running in the Family}. \\Vintage,
  1993. \textsc{ISBN 0679746692}.
\item [Pirsig, Robert.] \textit{Zen and the Art of Motorcycle
    Maintenance}. \\HarperTorch, 2006 (reprint). \textsc{ISBN
    0060589469}.
\item [Saint-Exup\'ery, A.] \textit{Wind, Sand and Stars}. \\Harvest,
  2002. \textsc{ISBN 0156027496}.
\item [Sanders, Scott Russell.] \textit{Writing from the Center}.
  \\Indiana UP, 1997. \textsc{ISBN 0253211433}.
\item [Sullivan, William.] \textit{The Secret of the Incas: Myth,
    Astronomy, and the War Against Time.} \\Crown, 1996.

\end{description}

\subsection{Books on Interdisciplinary Creativity}
\begin{description}
\item [Achebe, Chinua] \textit{Hopes and Impediments}. New York:
  Doubleday, 1989.
\item [Barron, F., ed] \textit{Creators on Creating: Awakening and
    Cultivating the Imaginative Mind}. New York: Putnam, 1997.
\item [Benjamin, Walter] \textit{Theses on the Philosophy of History}.
\item [Borges, Jorge Luis.] \textit{Collected Fictions}. \\Penguin,
  1999. \textsc{ISBN 0140286802}.
\item [Bohm, David] \textit{Wholeness and the Implicate Order}.
  London: Ark, 1980.
\item [Bohm, David] \textit{Unfolding Meaning}. New York: Routledge,
  1985.
\item [Bohm, David] \textit{On Creativity}. New York: Routledge, 1998.
\item [Bronowski, Jacob] \textit{Science and Human Values}. New York:
  Harper, 1956.
\item [---------] \textit{The Face of Violence}. London: Turnstile
  Press, 1964.
\item [---------] \textit{A Sense of the Future: Essays in Natural
    Philosophy}. Cambridge, MIT Press, 1977.
\item [Degler, Teri] \textit{The Fiery Muse: Creativity and the
    Spiritual Quest}. Toronto: Random House, 1996.
\item [Demos, John.] \textit{The Unredeemed Captive: A Family Story
    from Early America}. \\Vintage, 1995. \textsc{ISBN 0679759611}.
\item [Flack, Audrey] \textit{Art and Soul: Notes on Creating}. New
  York: Penguin, 1986.
\item [Franklin, Ursula] \textit{The Real World of Technology}.
  Toronto: Anansi, 1999.
\item [Fulford, Robert] \textit{The Triumph of Narrative: Storytelling
    in an Age of Mass Culture}. Toronto: Anansi, 1999.
\item [Goldberg, Natalie] \textit{Writing Down the Bones}. Boston:
  Shambhala, 1986.
\item [Herrigel, Eugen] \textit{Zen in the Art of Archery}. New York:
  Random House, 1977.
\item [Hildegard of Bingen] \textit{Secrets of God: Writings of
    Hildegard of Bingen}.\\ Boston: Shambhala, 1996.
\item [Hyde, Lewis] \textit{The Gift: Imagination and the Erotic Life
    of Property}. New York: Vintage, 1983.
\item [---------] \textit{Trickster Makes This World: Mischief, Myth,
    and Art}. New York: North Point Press, 1998.
\item [Jim\'enez, Juan Ramon] \textit{The Complete Perfectionist: A
    Poetics of Work}. Edited and translated by Christopher Maurer. New
  York: Doubleday, 1997.
\item [Jung, C.G] \textit{The Spirit in Man, Art and Literature}.
  Translated by R.F.C. Hull. Princeton: Princeton University Press,
  1998.
\item [London, Peter] \textit{No More Secondhand Art}. Boston:
  Shambhala, 1989.
\item [Pye, David] \textit{The Nature and Art of Workmanship}.
  Cambridge: Cambridge UP, 1968.
\item [Richards, Mary] \textit{Centering in Pottery, Poetry and the
    Person}. Middletwon, CT: Wesleyan UP.
\item [Sarton, May] \textit{Journal of a Solitude}. New York: Norton,
  1973.
\item [Sennett, Richard] \textit{The Corrosion of Character: The
    Personal Consequences of Work in the New Capitalism}. New York:
  Norton, 1998.
\item [Thoreau, Henry David] \textit{Walden}. New York: Norton, 1985.
\end{description}
\newpage

\subsection{Further Suggestions}

\begin{description}

\item [Arendt, Hannah] \textit{Illuminations}. London: Cape.

\item [Chodorow, J.]  \textit{Dance Therapy and Depth Psychology: The Moving Imagination}. Routledge, 1991.

\item [Dewey, John] \textit{Art As Experience}. New York, Perigee, 1931.
\item [Diamonstein, Barbara] \textit{Handmade in America: Conversations with Fourteen Craftmasters}. New York: Abrams, 1983.
\item [Gadamer, H.G.] \textit{Philosophical Hermeneutics}. Trans. D.E. Linge. Berkeley: University of California Press, 1976.
\item [Gardner, Harold] \textit{The Unschooled Mind: How Children are Taught and How Teachers Should Teach}. New York, Basic, 1993.
\item [Greene, Brian] \textit{The Elegant Universe: Superstrings, Hidden Dimensions, and the Quest for the Ultimate Theory}. New York: Norton, 1999.
\item [Hammarskjold, Dag] \textit{Markings}.
\item [Knill, P., Levine, E. and Levine, S.] \textit{Principles and Practice of Expressive Arts Therapy}. New York: Jessica Kingsley, 2004.
\item [Levine, S. K. and Levine, E. G.] \textit{Foundations of Expressive Arts Therapy}. London: Jessica Kingsley Publishers, 1999.
\item [Levine, E.] \textit{Tending the Fire}. Ontario: EGS Press, 2003.
\item [Levine, Peter] \textit{Waking the Tiger: Healing Trauma}. North Atlantic Books.
\item [Lorca, Federico] \textit{In Search of Duende}. Translated by Christopher Maurer. New York: New Directions, 1998.
\item [Lyndon, Susan] \textit{The Knitting Sutra: Craft as a Spiritual Practice}. San Francisco: Harper, 1997.
\item [McNiff, Shaun] \textit{Art Heals: How Creativity Cures the Soul}. Boston: Shambhala Publications, 2004.
\item ---------. \textit{Creating with Others: The Practice of Imagination in Art, Life, and the Workplace}. Boston: Shambhala Publications, 2003.
\item ---------. \textit{Art As Medicine}. Boston: Shambala, 1992.
\item ---------. \textit{Fundamentals of Art Therapy}. Springfield, IL: Charles C. Thomas, 1988.
\item ---------. \textit{Art-Based Research}. London: Jessica Kingsley, 1988.
\item [Mazza, Nick] \textit{Poetry Therapy: Interface of the Arts and Psychology}. CRC Press, 1999.
\item [Minnich, E. K.] \textit{Transforming Knowledge}. Temple University Press., 1990.
\item [Moustakas, C.] \textit{Heuristic Research: Design, Methodology, and Applications}. Newbury Park: Sage, 1990.
\item [Needleman, Carla] \textit{The Work of Craft: An Inquiry Into the Nature of Crafts and Craftsmanship}. New York: Kodansha, 1979.

\end{description}

\clearpage

\section{Demonstration of Learning}

\subsection{Assignments}
Three individual projects and several presentations (see below) are required for this course. These assignments may be comprised of any type of expression (writing, music, imagery, dance, movement, photography, etc.). The central idea of each project is for you to choose a specific theme or thread and explore it in some depth. We will discuss these projects at length in class. They are opportunities for you to discover and explore your own life.


\subsubsection{Individual Project}

The individual project is an opportunity for you to explore a given domain, or set of domains, in your own way. For example, you might wish to complete an art project, or a mathematics challenge, or a community service initiative. It's up to you; there are no specific restrictions on the type of inquiry you undertake. However, your inquiry must be applicable to the university context and must include research into a specific university context. (So, to use the art project example from above, your project would include some research into arts programs at Kwantlen or elsewhere.)

The individual project will include a written self-assessment of at least 500 words. The self-assessment will include answers to the following questions (answers need not be itemized): 

 \begin{itemize}
\item What research did you do to prepare for this project? Research might include readings, investigative interviews, online searches, self-reflections, ruminations, and many other modalities. 

\item What learning resources did you use? These might include books, articles, online resources, and so on.

\item Of the research and readings you undertook, what impressed you as being most interesting or relevant?

\item What kinds of experiments did you undertake with this project? What did you build, write, craft, or try? How did you spend your time, and how did it go? (Look at the criteria on the next page.)

\item What went well, where did you struggle, and how do you feel about the process you undertook during this project?

\item What were the best and worst moments of this project? What did you learn from these moments?

\item Are you proud of this project? Is it your best work? What grade would you give yourself?

\item How might you have improved this project, of your experience of it?

\item What did you learn about interdisciplinarity while working on this project?

\item What did you learn about yourself while working on this project?

\item What will you remember about this project in five years?

\item How does what you learned apply to your studies at Kwantlen and to your sense of your future direction?

\item What advice would you give to others who might be undertaking a similar project?

\item What did this project mean to you? What might it mean for others?

\item Do you plan to continue this project further, or to work on similar projects in the future?

\item You learned something crucial in this project which you won't discover for a while. Make a guess now about what that might be.

\end{itemize}

The individual project is worth 25 percent of your grade.

\subsubsection{Assessment Criteria for the Individual Project}

Projects for this course are focused on authentic and creative inquiry. Accordingly, the following criteria -- which are based on the philosophy of creative inquiry  --  are used to evaluate engagement and commitment to the individual project:

\begin{itemize}
\item Willingness to take appropriate risks and to challenge oneself.
\item Willingness to try new things, especially when doing so provokes discomfort.
\item Openness to personal and interpersonal process.
\item Willingness to collaborate with others.
\item Consideration of and responsiveness to others.
\item Willingness to examine personal values, beliefs, and judgments.
\item Ability to take personal responsibility and initiative for learning.
\item Willingness to approach academic inquiry as a skill with discrete steps and standards.
\item Commitment to improvement in writing and creative work.
\item Ability to be open and responsive to appropriate feedback.
\end{itemize}

\subsection{Group Project}

Each student will be a member of several different peer groups; each peer group will present at least one mini-presentation (roughly thirty minutes) on a topic of choice. Each class session after the second will involve presentations, with one presentation from each group. Class time will be given for preparing the presentations. The structure and content of the presentations will be discussed in class.

\subsubsection{Presentation Methods and Goals}

The central idea of the presentations for this course is to give you opportunities to practice interdisciplinary thinking and expression.  As such, the presentation should be interdisciplinary. Essentially, this means that you should try to use multiple presentation strategies and modalities. These might include (but are certainly not limited to) any of the following:

 \begin{itemize}
 \item Storytelling
 \item Poetry
 \item Music (playing)
 \item Drumming
 \item Singing
 \item Dance
 \item Movement
 \item Sport
 \item Ritual
 \item Film (showing)
 \item Film making
 \item Photography
 \item Web content
 \item Craft work
 \item Art making
 \item Individual reflection
 \item Meditation
 \item Health practices
 \item Creative process (any type)
 \item Group communication
 \item Cultural practices
 \item Nature experiences
 \end{itemize}

 Whenever possible (and workable), try to mix together multiple modalities into a single presentation. For example, you might ask the group to do some individual reflection using the modality of poetry,
 then create a series of movements based on the poetry, then work in small groups to talk about and share the process. Many configurations are possible. The trick is to choose an activity that you enjoy, then find a way to apply it to the content (suggested presentation topics are listed below). Please do not create your presentations using only written and/or spoken materials. In other words, don't just stand up at the front of the class and talk about the presentation topic. Utilize the energy of the group. Remember that in interdisciplinary work divergences are valued as unique opportunities. So, feel free to experiment with activities and modalities that may not seem, on the surface, to be related to the topic at hand but which might, upon experiment, yield surprising connections and results. Be playful.
 Allow yourself to laugh at yourself, to be embarrassed, to engage with the process in novel and interesting ways.

In interdisciplinary work, riddles and puzzles are highly prized. Accordingly, the presentations should (ideally) not be complete explanations or presentations of material. Feel free to play with challenging exercises, with impossible scenarios, and other conundra. One way to think about this is to consider insoluble riddles, such as the one in \textit{Alice in Wonderland}: Why is a raven like a
writing desk?

\begin{quote}
  ``Have you guessed the riddle yet?'' the Hatter said, turning to Alice again.\\
``No, I give it up,'' Alice replied. ``What's the answer?''\\
``I haven't the slightest idea,'' said the Hatter.\\
``Nor I,'' said the March Hare.\\
Alice sighed wearily. ``I think you might do something better with the time,'' she said, ``than wasting it in asking riddles that have no answers.''
\end{quote}

The best interdisciplinary topics offer more questions than answers. They, are essentially, gateways into the mysterious--which, as Einstein will tell you, is an important place to be:

\begin{quote}
  The most beautiful thing we can experience is the mysterious. It is
  the source of all true art and science.
\end{quote}

\subsubsection{Suggested Topics for Interdisciplinary Presentations}

\begin{itemize}
\item Akhenaten and the invention of monotheism
\item Albrecht Durer and alchemy
\item Aristotle's book of comedy
\item Bill Evans and the Peace Piece
\item Buckminster Fuller and the geodesic
\item Chenrizi and the politics of China
\item Chuang Tzu and the butterfly
\item Coleridge and the person from Porlock
\item Csikszentmihalyi and the flow experience
\item David Bohm's Implicate Order
\item Darwin, the bassoon, and the sundew
\item Eugen Herrigel and the practice of archery
\item Francis Yates and the \textit{Art of Memory}
\item Freud, Jung, and the ``bosh'' incident
\item Fulcanelli and \textit{Mysteries of the Cathedrals}
\item Giordano Bruno and the Hermetic tradition
\item Godel's uncertainty principle
\item Hanna Arendt at Nuremberg
\item Henri Rousseau in the jungle
\item Howard Carter and ``wonderful things''
\item Jacob Bronowski at Auschwitz
\item Jacob Bronowski, Nagasaki, and \textit{Science and Human Values}
\item Jan Tschichold and the Nazis
\item John Cage on the subway with the \textit{\textit{I Ching}}
\item Kepler's \textit{Somnium}
\item Mary Shelley and the genesis of \textit{Frankenstein}
\item Newton's \textit{Principia}
\item Nikola Tesla and universal energy
\item Philip K. Dick, VALIS, and 2-3-74
\item Picasso, Guernica, and Expo 1937
\item R.D. Laing and madness as reality
\item Ramanujan's notebooks
\item Richard Feynman and the invention of quantum mechanics
\item Schwaller de Lubicz at Karnak
\item Simone Weil and leading from desire
\item St. Exupery flying into the desert
\item The Reimann Hypothesis
\item The Voynich Manuscript
\item The visions of Hildegard of Bingen
\item Thoth's legacy
\item Walter Benajmin and the Angel of History
\item Wendell Berry going \textit{Into the Woods}
\item Wilhelm Reich's Cloudbuster
\item William Blake's \textit{Marriage of Heaven and Hell}

\end{itemize}

\subsubsection{Assessment Criteria for Group Presentations and Overall Engagement}

This course utilizes experiential learning approaches, which depend upon student involvement and active participation. Accordingly, the following criteria are used to evaluate overall participation and engagement in the group presentations and the class:

\begin{itemize}
\item Willingness to take appropriate risks and to challenge oneself.
\item Willingness to speak up and to lead.
\item Openness to interpersonal process.
\item Willingness to collaborate with others.
\item Consideration of and responsiveness to others.
\item Commitment to enhancing the interpersonal experience of everyone in the group.
\item Willingness to examine personal values, beliefs, and judgments.
\item Ability to take personal responsibility for learning.
\item Willingness to deal with conflicts appropriately if and when they arise.
\item Ability to be open and responsive to appropriate feedback.
 
\end{itemize}

The group presentations and overall course engagement are worth a total of 25 per cent of your grade.

\subsection{Online Community Project}

We will create an online component of the course for us to share ideas, resources, and conversations. Your contribution to this online community will represent your online community project, and will include weekly contributions (at least one comment or contribution per week). The assessment criteria for this project are as follows:

\begin{itemize}
\item Extent and appropriateness of online contributions
\item Willingness to take appropriate risks and to challenge oneself.
\item Willingness to lead and contribute positively to online conversations.
\item Openness to interpersonal process in the online sphere.
\item Willingness to collaborate with others online.
\item Consideration of and responsiveness to others online.
\item Commitment to enhancing the interpersonal experience of everyone in the online group.
\item Willingness to examine personal values, beliefs, and judgments in the context of online conversation.
\item Ability to take personal responsibility for learning and learning technologies.
\item Willingness to deal with online conflicts appropriately if and when they arise.
\item Ability to give and receive appropriate online feedback.
 
\end{itemize}

The online community project is worth a total of 25 per cent of your grade.

\subsubsection{Portfolio}

This course explores a wide range of modalities, disciplines, and approaches. By the end of the semester you will have accumulated a wealth of material: not only from your own projects but also from your group work, your online contributions, and your classroom experiences. This collected material is your portfolio, and it should become a foundational piece of your academic work this semester. The final project involves the synthesis of this material into a digital portfolio that you can carry forward (and build upon) as your studies evolve. The digital portfolio can be developed within a hosted content management system (such as Posterous, Evernote, or WordPress) or on a website you develop yourself (if you know how to do that). We will discuss the digital portfolio at length in class.

The portfolio will include a self-assessment of at least 500 words. The self-assessment (which can be part of the portfolio itself or submitted separately) will include answers to the following questions (answers need not be itemized): 

 \begin{itemize}

\item Of the research and readings you undertook this semester, what impressed you as being most interesting or relevant?

\item What went well, where did you struggle, and how do you feel about the process you undertook during this course?

\item What were the best and worst moments of this semester? What did you learn from these moments?

\item Are you proud of your work this semester? Is it your best work? If not, why not?

\item What grade would you give yourself, and why?

\item How might you have improved your experience this semester?

\item What did you learn about interdisciplinarity while working on your projects for this course?

\item What did you learn about yourself while working in this course?

\item What will you remember about this course in five years?

\item How does what you learned apply to your studies at Kwantlen and to your sense of your future direction?

\item What advice would you give to others who might be undertaking a similar path?

\item What's next for you?

\item You learned something crucial in this course which you won't discover for a while. Make a guess now about what that might be.

\end{itemize}

The portfolio is worth a total of 25 per cent of your grade.

\subsection{Attendance and Participation}
The expectation is that you will attend all sessions and involve yourself in the class process. Your willingness to engage creatively with the learning process, to take appropriate personal risks, and to participate in group activities are all central to your involvement in this class. Because developing a style of authentic and interdisciplinary inquiry is very much a process of blending your own personal awareness with skills and practical techniques, your own emotional involvement in the class is as important as your academic knowledge of the material.

Interdisciplinary inquiry is a unique process. Unlike many other fields, in which competence and skill may be measured objectively, using replicable and consistent means (tests of factual knowledge, for example), authentic interdisciplinary inquiry depends greatly on the interpersonal skills of the practitioner. Computer programmers can be assessed by their ability to write code; chiropractors can be evaluated based on their skill at manipulating the human skeleton; race car drivers can be clocked around a track. But for interdisciplinary scholars there are no such fixed measures. The interpersonal skills upon which authentic interdisciplinary inquiry so much depends are subtle, difficult to quantify, and complex beyond any measurement scheme.

And yet we can identify those who possess exemplary interpersonal and interdisciplinary skills. They are relaxed, open, responsive, kind (Jacob Bronowski is probably the best modern example). Often they exhibit skills that we tend to assign to the social sphere: personal warmth, consideration of others, hesitancy to judge, sensitivity to emotions. In our class we focus on these interpersonal factors as a foundation for our experiences with one another. And we itemize them as features along the continuum of self-awareness:

\begin{itemize}
\item Commitment to the development of self-awareness.
\item Openness to interpersonal process.
\item Ability to participate in appropriate self-disclosure.
\item Consideration of and responsiveness to others.
\item Commitment to enhancing the interpersonal experience of everyone in the class.
\item Willingness to examine personal values, beliefs, and judgments.
\item Ability to take personal responsibility for learning.
\item Willingness to deal with conflicts appropriately if and when they arise.
\item Ability to be open and responsive to appropriate feedback.
\end{itemize}

Each item on the above list is an aspect of the first item: self-awareness. The most foundational skill in creativity is self-awareness. Those who develop this skill consistently query their own responses, thoughts, and feelings. They ask themselves:

\begin{itemize}
\item What am I feeling right now?
\item What am I thinking right now?
\item Why am I reacting in this particular way?
\item What do my thoughts, feelings, and reactions tell me about myself?
\item Is there anything about my current behavior that suggests unresolved themes in my life?
\item Is my perception of myself consistent with what other people tell me about the kind of person I am?
\item When and how do I get stuck, and what am I doing to work on this?
\item In what ways do I get overwhelmed, or shut down, or avoid?
 
\end{itemize}

These questions, and many others, require the capacity for self-reflection and self-awareness. As we continue in the course, you may wish to consider these questions as they apply to you. At the very least, you might wish to consider what you are currently working on in your life, in which direction your attention is drawn, into which of the innumerable themes of human nature you are now called to delve.

In my role as your instructor, I will be paying attention to how thoughtful you are in examining and responding to questions like those in the lists above. I will not be analyzing you, but rather noticing what kinds of things you do, what your reactions are to various situations. My goal in observing your behaviors and interacting with you is to assist you in developing greater self-awareness (and, by extension, greater creativity).

I will use the lists above, as well as the assessment criteria listed for each assignment, to assess your overall participation in the course. I will not be evaluating your level of self-awareness but rather your openness to the process of developing your self-awareness.

\subsection{Grade Inflation}
Almost every semester there are students who do well on the assignments, complete all the associated learning goals of the course, participate well, and wonder why they do not receive a grade of one hundred percent (or 98, anyway). Here is the reason: almost every semester there are students who demonstrates a level of commitment that goes beyond the course requirement. Such students complete extra work, or hand in exemplary assignments, or undertake a significant amount of personal development in addition to the course expectations. Such students typically receive the highest grades.

If you do reasonably well in the course you will receive a reasonable grade. Very high grades are intended for extra or exemplary work. Unfortunately, over the past thirty years the post-secondary educational system in North America has participated in a process of grade inflation. Since the 1980's, the average grade for typical course work has been increasing by about 25 per cent each decade.
Elevated assessments do not accurately reflect the work of most students. Even worse, grade inflation has caused many students to expect high grades for average work. I am not a overly stringent assessor; but I will not inflate grades artificially.

In this course a small number of students will (likely) receive high grades; most students will receive grades in the middle range; and a few students will struggle with lower grades. If you are uncertain about your assessment for a given assignment, or if you wish to know where, roughly, you are along the distribution curve of the class, or if you would like suggestions for how to improve your grade, please ask me for clarification.

If you wish to achieve a good grade, please do the following:

\begin{itemize}
\item Show up for class -- every class. This course depends on student engagement. (This becomes especially important during the final weeks of the semester.)
\item Be attentive and mindful to the various criteria listed for each of the projects and the course overall.
\item Take the initiative to plan and develop your projects and presentations. This course is (very likely) more fluid and spontaneous than you are used to. Your ability to manage your time, commitment, and energy is crucial.
\item Speak up in every class (review the criteria for group engagement and presentations).
\item Don't look for the right answer to a question or challenge. Instead, find the answer that is meaningful to you.
\item Ask for help if you need it.
\item Commit to your projects in a substantial way. Good projects take time. Rushed projects are obviously rushed.
\end{itemize}

Finally, please be attentive to the Kwantlen policies on academic honesty and plagiarism, which can be found at the following URLs:

\noindent
Academic Honesty: \url{http://www.kwantlen.ca/__shared/assets/Honesty1432.pdf}\\
Plagiarism and Cheating: \url{http://www.kwantlen.ca/policies/C-LearnerSupport/c08.pdf}
\clearpage
\section{Due Dates}

Group presentation dates will be assigned in class \\(and will fall between weeks three and thirteen).
\noindent\\
The Individual project is due at the end of week four \\(by midnight on Sunday of that week).
\noindent\\
The Online Community project runs all semester; however, your contribution to the online community will be evaluated after week eight.
\noindent\\
The Portfolio project is due at the end of week twelve \\(by midnight on Sunday of that week).

\section{Thematic Schedule}
The class sessions will be balanced between presentations (by the instructor and students) academic material, group collaboration, and composition. The content for each session will evolve as the semester progresses. We will cover the following themes (though, perhaps not in the order listed below):
\\
\begin{compactdesc}
\item[Contexts and Properties of Interdisciplinarity]
Evolution and history of disciplinary and interdisciplinary methods and models. Hubs, nodes, and interdisciplinary connectivity. Properties of interdisciplinary networks and pathways.
\\
\item[Constructs and Principles of Interdisciplinarity]
Definitions, limits, and language. Linearity, nonlinearity, translinearity. Interdisciplinarity, transdisciplinarity, multidisciplinarity, and related incomplete models. Fractals, chaos, and dynamical systems. Incomplete knowledge and the problem of modern inquiry. Cultural approaches to incomplete knowledge. Different, divergent, and integrative perspectives. Wholes, parts, and the elusive sum.
\\
\item[Interdisciplinary Approaches to Domains]
Moving between and across domains. The domain of the spaces between domains. Curriculum and the evolution of domain systems in education. Cultural approaches to domains and interdisciplinarity. Bodies of knowledge, cross-linkages, and dead zones. Academic and social structures that reinforce or refract disciplinarity and interdisciplinarity.
\\
\item[Interdisciplinary Integration]
Shared characteristics of nodes and hubs. Multiplicities and unities among sources and experiences. Integrating diverse and contradictory points of view. Uses of fusion and fragmentation as interdisciplinary approaches. Contextuality and interdisciplinary integration. Holographic patterns and interdisciplinary integration.
\\
\item[Interdisciplinary Approaches to Academic, Personal, and Professional Development]
Creativity and purposeful play in interdisciplinary approaches. Mind, mindfulness, and heartfulness in interdisciplinary approaches. The trickster archetype in myth, creativity, and culture. Self-awareness as an interdisciplinary path. Leading and not leading in interdisciplinary approaches. Synchronicity in interdisciplinarity.
\\
\item[Interdisciplinary Values and Philosophies]
Hermeneutics and impulses toward meaning. Alchemy, Hermeticism, and related interdisciplinary philosophies. Cultural approaches to domains, reality, and interpretation (eg. Dzogchen, Duende, Bon, Taoism). Cultural concepts of primordial illumination and integration. Interpretations, values, and decisions in interdisciplinary approaches. Paths and reinforcements in interdisciplinary approaches. The viewer, the subject, and the object. Quantum theory and implications for interdisciplinarity. Assumptions, presuppositions (and the skin of the onion) in interdisciplinary approaches. Beyond ideas of domains and disciplines.
\\
\item[Optionality as an Interdisciplinary Method]
Boundary-crossing, transgression, and rule-breaking as interdisciplinary strategies. Auxiliary functions, back doors, trap doors, and black holes of domains and knowledge. Bricolage and wandering as interdisciplinary strategies. Unexpected, unknown, and unknowable paths in interdisciplinary approaches. Uncertainty and flow in interdisciplinary work. Transferability of interdisciplinary approaches to disciplines and domains. Maximizing exposure to forking avenues, labyrinths, recursive paths, and infinity in interdisciplinary work. 
\\
\item[Network Effects of Interdisciplinarity]
Robustness and fragility in disciplinary and interdisciplinary approaches (eg. the fourth quadrant and black swan events). Amplification and exponential effects in disciplinary and interdisciplinary approaches. Cascades and recursive links in disciplinary and interdisciplinary approaches. Interdependence and percolation in disciplinary and interdisciplinary approaches. Complexity and unity in interdisciplinary approaches.
\\
\item[Online Portfolios and Interdisciplinarity]
The World Wide Web and its applications as interdisciplinary methods and cultures. Online content creation for interdisciplinary approaches. Online content curation for interdisciplinary approaches.
\\
\item[Other Stuff (that we'll think up as we go along)]
Much of the course content will be generated by students.
\end{compactdesc}
\clearpage

\section{Resources}
\subsection{How to Start Writing}

Stop whatever else you are doing. Close your email application and Facebook, turn off the background music, silence your cell phone. Put it all away. Do it now. I'll wait.

Sit in silence, without distraction, and read this post. Silence the part of you that makes false claims about the utility of background music or the necessity of leaving your cell phone turned on. Silence the part of you that wants to argue with me, right now, about my unreasonableness, the part of you that makes claims for this or that distraction. Still the monkey mind that never shuts up, never stops talking, never ceases inventing new ways to jostle, cajole, argue. Stop arguing and listen: the voice of a writer can only be found within silence.

Silence.

Start with that. Stay within it. Allow it to grow around you, to blossom, to disclose the images and words that inhabit the landscape of your inner life. Don't control it, or direct the flow of that nascent energy. Sit, and read, and watch yourself.

Forget that you are trying to write. This fact is irrelevant to the creative process. It is a curiosity. A writer finds and follows the creative voice. The means by which this happens, the structure in which it unfolds, the particulars of the path: these are secondary and inconsequential. A writer follows the path, whenever it appears and wherever it leads.

A writer does not invent or create the writing. Instead, the act of authentic writing leads the writer. Accordingly, the task of the writer is to find -- within -- the stream, thread, and path of creative energy. Writing inhabits its own life, is its own animal, is a being struggling to be free of the cages we build around it. Don't take my word for it. Find the cage, find the animal.

Listen.

Stop arguing. Your arguments, like mine, only serve to strengthen the cage. The animal of the creative is not swayed by our smartness, our wit, our experiences. It does not care how many books we have read or how many fancy words we know. It is not interested in our expertise and the many ways in which we layer our insecurities one over the other.

The animal of the creative wanders the landscape of gods and heroes. The animal has seen things we no longer remember. The animal is what we once were but have chosen to cage as a means of protecting ourselves from the vastness of what we cannot grasp, the depths into which we no longer dare to gaze.

The creative animal is primordial, eternal, wise beyond our knowing. It has been waiting for us, all this time. Listen to what it has to say.

Write.

Allow the creative animal to write for you one good word, or sentence, or paragraph. Don't mess up the writing. It is difficult to say what this means, this messing up. Perhaps you are cool, or smart, or erudite. Forget all that crap. It is meaningless. Write honestly. Let the creative animal speak through you.

If, as you write, you start to worry about what people might think of your writing, you may as well not start. Give it up now, before you waste any more time. Or tell the part of you that wants to be a rabbit rather than a wolf to shut the hell up.

Write something. Don't worry about what genre it is. Genres have no meaning. Writing — all writing — is, at heart, an extended negotiation with the creative animal. That animal is partly you, yes; but is also not you, is wholly an emissary of that mystery we run from and slide toward.

And the animal is — for the most part — silent. Do not forget this. Words are not the creative, cannot be the creative, will never be the creative. They are echoes. Treat them as such. Find the source of those echoes.

Find the cage. Find the animal.

\subsection{Steps to Better Writing}

For most of its history literature has taken the form of epic poetry. This history is long: five thousand years, perhaps much longer. And within the genre of epic poetry -- from the Egyptian Pyramid Texts to Homer to the Kalevala -- every word counts. The rhythm counts. The resonance and fluidity count. No slack exists in these texts, no lazy meanderings of phrase or structure. These ancients texts are spare, clean, and tight. We could learn a great deal from these archaic authors. There are reasons for the enduring quality of their texts.


So, to be an epic poet:
\begin{itemize}
\item Write one sentence at a time.
\item Review the sentence before moving on. Make it as perfect as you can. Spend all day on one sentence if required (but don't spend too long...)
\item Make sure your sentence contains the best words for what you are trying to say.
\item Examine the phrase order. Look for a tighter order, more spare or visceral or elegant.
\item Speak the sentence aloud. Find its rhythm and sonority. Tweak as required. Don't rush.
\item Take out all extra words and lazy phrasings, especially those that are habitual. Excise adverbs, gerunds, and verb phrases (``there is...'', ``I've done...'', ``We're going...'') whenever possible.
\item Shorten the sentence if you can (without diminishing its meaning).
\item Take a short break, gaze out the window, return to the sentence, and review it once more.
\item Leave it alone. Build your next sentence.
\end{itemize}

Good writing builds sentence upon sentence. Each new contribution adds to the structure and the framework of clarity. Why go farther (Quick Tip: "farther" refers to distance or extent; "further" denotes an action in service of) -- why go farther down your creative track when the foundation is not yet established? I know, you have probably been told to just write, to get words on the page, to come back to them later and try to make sense of your scratchings. No, I am not a fan of this approach. I prefer to approach writing as a Zen-like activity, an action of the razor-sharp mind and open heart working together. Writing, for me, is not catharsis but clarity.

Let's take a practical example. Here's a possible sentence:

\vspace{1em}

\textsf{Down on Granville Street, where my grandfather's jewelry store used \\
to be, there are now a bunch of old, boarded-up buildings waiting quietly \\
to be renovated.}
\vspace{1em}

Alright, this is a start. I'm trying to say something in this sentence: about change, nostalgia, perhaps about renewal. It's not yet clear. So, let's start with the beginning:

\vspace{1em}
\textsf{Down on Granville Street}
\vspace{1em}

\textsf{Down} and \textsf{on} are both prepositions, only one of which is required. Therefore we can make this first phrase more succinct:

\vspace{1em}
\textsf{On Granville Street}
\vspace{1em}

Next up, the second phrase:
\vspace{1em}

\textsf{where my grandfather's jewelry store used to be}
\vspace{1em}

This phrase is the heart of the sentence; it needs to be clear and strong. \textsf{Used to be} is an awkward verb phrase. It tries to articulate, in three words, the nostalgia and ambivalence of the sentence. And yet, \textsf{used to be} is almost devoid of meaning here. It is a marker and nothing more. Let's try something more robust and imaginal:
\vspace{1em}

\textsf{where my grandfather's jewelry store once stood}
\vspace{1em}

By using \textsf{once stood} in this way, we're indicating the past in more resonant terms. We are also implying a fall -- what once stood, then fell. Also, we're implying a steadfastness of the old place, a sense of presence that was previously lacking. So far so good. Now, onto the tricky part:
\vspace{1em}

\textsf{there are now a bunch of old, boarded-up buildings \\
waiting quietly to be renovated}
\vspace{1em}

Well, this is a tidy mess. Too many things going on, too many overt indications when subtlety is called for. Not to mention the awkward phrase \textsf{a bunch of}. Yikes. Where to begin? How about with some editing:
\vspace{1em}

\textsf{...old, boarded-up buildings waiting quietly to be renovated}
\vspace{1em}

OK, this makes things a bit easier. Now we have the rudiments of a decent clause, something about old buildings:
\vspace{1em}

\textsf{boarded-up buildings waiting quietly to be renovated}
\vspace{1em}

We know that the gerunds and adverbs are typically (except right here!) to be avoided, so we can clean up the phrase:
\vspace{1em}

\textsf{boarded-up buildings wait} to be renovated
\vspace{1em}

Now, \textsf{tumbledown} is a better word than \textsf{boarded-up} (\textsf{ramshackle} would be good here, too). And \textsf{wait to be renovated} is awkward and anthropomorphic in a way that doesn't seem to suit the imagery of the sentence. And we might spruce up the language a bit with some alliteration (use sparingly!):
\vspace{1em}

\textsf{tumbledown buildings lie in lethargy}
\vspace{1em}

Better. But I keep thinking about \textsf{ramshackle} and \textsf{tumbledown}. Could I use both? Let's see:
\vspace{1em}

\textsf{ramshackle buildings lie in lethargy upon the tumbledown street.}
\vspace{1em}

I like this. But it will require that I abandon my theme of renewal. The sentence will be more sad without it, yet probably more authentic too. And less self-conscious. Let's try the whole thing out:
\vspace{1em}

\textsf{On Granville Street, where my grandfather's jewelry store once stood, \\
ramshackle buildings lie in lethargy upon the tumbledown street.}
\vspace{1em}

Not bad. The sentence embodies nostalgia, sadness, personal and social loss, and something else -- but we don't know what yet. It's something about what happens next, or later, the contrast between the past and the present. The sentence itself leads me on, as its writer, to the next stage. It provokes me to think about contrasts, about words such as \textsf{glittering} and \textsf{forlorn}, and about what we preserve and discard. I cannot write the next sentence without first the polished catalyst of the first.

\subsection{Practical Tips}

\begin{description}
\item [Use Concrete Imagery] The writer is a ruminative animal (like a cow, in fact). We chew over our lives and histories, we digest and express the stories to which we are drawn. This process requires a good deal of thought. Writers think -- cogitate speculate, perambulate -- and we write down our thoughts. Yet we forget, often, that the reader does not have access to our mind, does not perceive the interconnections and contexts which lead us to our conclusions. We must show the reader that context, the web of threads and images by which we derive our narratives. And this requires that we replace our inner ruminations with concrete imagery.

On its own, unaccompanied by imagery, \textsf{I am sad} is poor writing. The reader is offered no means by which to understand the cause or nature of the sadness. The writer must provide the path to understanding by way of imagery. Typically, the rule for this situation involves four or five concrete images for each internal rumination. Like so:
\vspace{1em}

\textsf{
Sheets of rain strike the window. I gaze into the dark, searching for  the glimmer of proud trees across the field. Too dark to see anything. The house, empty now, twists and groans in the sidelong wind. Every whisper of its movement reminds me of the long, sad night ahead.
}
\vspace{1em}

In the above passage, sad is embedded within imagery, layered, integrated. It is not a rumination but a concrete indication of feeling. And it comes in the last of five sentences, thus proving the rule. This rule also has another name: \textit{Show, Don't Tell}.

\item [Use the present tense] Even when writing about the past, the present tense is usually the best route to clarity. Other tenses and moods require more words, encourage abstraction, and introduce a barrier between the immediacy of the reader and the distance of the text. These problems are most evident when writers use the subjunctive mood (or tense), which is defined as follows:

\textit{Subjunctive: a mood that represent an act or state (not as a fact but) as contingent or possible.
}

Strictly speaking, the subjunctive mood uses words such as \textsf{if, that, though, lest, unless, except, until} and so on. But in a more general way, the subjunctive mood, or tense, might be described as any writing that removes the reader from the concrete and present tense:
\vspace{1em}


\textsf{I wish my dog were here.}
\vspace{1em}

\textsf{If Dave were stronger, he would have been able to lift the tree stump.}
\vspace{1em}

\textsf{If I had paid more attention to my instructor's tips about creative writing, I would be a better writer.}
\vspace{1em}

\textsf{I requested that John be present at the wedding.}
\vspace{1em}


The above examples illustrate various ways in which subjunctive approaches to writing introduce extra words and awkward phrases. Replacing these with concrete phrases in the present tense improves the writing:
\vspace{1em}

\textsf{My dog is here.}
\vspace{1em}

\textsf{Dave is not strong enough the lift the stump.}
\vspace{1em}

\textsf{I listen to my creative writing instructor. My writing improves.}
\vspace{1em}

\textsf{John comes to the wedding.}
\vspace{1em}

Here is a short list of words and phrases that often accompany writing that meanders away from the present tense and from concrete imagery. Avoid these words and phrases whenever possible:
\vspace{1em}

\textsf{there, would, could, were/was, has/had, even, those/these, that/this}
\vspace{1em}

\item [Avoid adverbs and gerunds] A gerund is a noun formed from a verb: walking, for example. Gerunds usually end in \textsf{ing}, and their erasure is one of the single most positive changes a writer can make. Gerunds are problematic for two reasons: they encourage adverbs and they drift toward abstraction (see tips one and two).

For example, a writer may describe a scene of walking as follows:
\vspace{1em}

\textsf{I was walking down the street, quickly, not paying too much attention to the hurtling traffic and the shop-keepers shouting stridently from their stalls.}
\vspace{1em}

This vignette is neither in the present tense (and is therefore less concrete than it might be) nor is it crisp and clear enough. As soon as the writer begins with the gerund walking, all the other adverbs and gerunds in the sentence follow naturally; awkwardly and naturally. Gerunds, therefore, imply adverbs and abstraction. Also, gerunds require more words and tenses: \textsf{was walking} requires two words to state a simple verb. Here's how to reduce the confusion:
\vspace{1em}

\textsf{I walk down the street. My steps are quick, my attention wanders. I do not notice the hurtling traffic, nor the shouts of strident shop-keepers as I pass their stalls.}
\vspace{1em}

The revised passage is more immediate, possesses an air of expectation (of suspense, almost), is constructed with precision, and contains no extra words. The revision begins with a present tense action: I walk down the street. The remainder of the passage builds naturally, preserves the original clarity. Better. Stronger.

Like gerunds, adverbs -- typically words that end in \textsf{ly} --  introduce abstraction and encourage awkward phrasings. An adverb qualifies a verb, limits or augments the verbal meaning. As such, an adverb serves the same function as a bank of fog laid across a shoreline. The fog obscures, mediates, filters. Almost always, adverbs may be replaced with more accurate verbs and fewer words; as in the example above, with the adverb quickly replaced by the verb quick and a concrete image of the steps of the walker.

Gerunds and adverbs are not always avoidable. They are necessary at times, and need not be shunned entirely. But often they are lazy solutions. Avoid them whenever feasible.


\item [Tune your vocabulary] Writing is not speech. And yet the habits of speech tend to infiltrate writing. Resist this whenever possible: look for these infiltrations, replace them with narrative structures, preserve the clarity of your vision. For example, \textsf{I am not a farmer, and I do not eat broccoli} is better than \textsf{I'm not a farmer, and I don't eat broccoli}. Apostrophes derive from speech, and are usually to be avoided in writing (except in the case of possessives, obviously). In similar fashion, it is best to limit the use, in writing, of the following common and conversational words:
\vspace{1em}

\textsf{really, very, it, quite, real, some, somehow, someone, something, everything, thing, anyone, maybe,little, been, your, never, always, only, just, deep, people, about, being, get, guess
}
\vspace{1em}

The vocabulary of a writer need not be coruscating or arcane or hermetic. Small and common words will do, and in most cases should do. Small words encourage precision and clarity. The order of such words, the manner in which they are structured as narrative and not as speech, determines much of the quality of writing.

\item [Know your habits] In addition to the various strategies outlined above, writers also need to be aware of their individual proclivities: their habits of laziness, their favorite and over-used words, the ways in which they sacrifice clarity for muddiness. The following are the most common secondary errors:

\begin{itemize}\item Awkward phrase order in compound sentences
\item Awkward shifts of scene or tense
\item Awkward use of vocabulary
\item Lack of clear narrative direction
\item Stating the obvious rather than showing through imagery
\end{itemize}

These habits and their kin -- the many pitfalls and hurdles that lie along the path of writing -- are avoided only through practice. Go slow, review each sentence as you write, make a list of your common errors and search for each error as you compose. Remember the primary goal: clarity.

\end{description}


\end{document}

