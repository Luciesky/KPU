\documentclass[letterpaper,oneside,11pt,draft]{memoir}
\settypeblocksize{*}{26pc}{1.6813}
%packages
\usepackage[latin1]{inputenc}
\usepackage{amsmath}
\usepackage{amsfonts}
\usepackage{amssymb}
\usepackage{lettrine}
\usepackage{bbding}
\usepackage{waterstitling}
\usepackage[oldstyle]{sabonlt}
%headers
\makepagestyle{idea}
	\makeevenhead{idea}{\textsf{\begin{small}\textsc{IDEA 3100}\end{small}}}{}{\textsf{\begin{small}\textsc{Presentations}\end{small}}}
	\makeoddhead{idea}{\textsf{\begin{small}\textsc{IDEA 3100}\end{small}}}{}{\textsf{\begin{small}\textsc{Presentations}\end{small}}}
	\makeevenfoot{idea}{}{\textsf{\thepage}}{}{\sffamily}
	\makeoddfoot{idea}{}{\textit{\thepage}}{}{\sffamily}
	\makeheadrule{idea}{26pc}{\normalrulethickness}
	\makerunningwidth{idea}{26pc}
%fonts and text formatting
\linespread{1.1}
\renewcommand{\sfdefault}{pwt}
\clubpenalty=2000
\widowpenalty=2000
\renewcommand{\chapnamefont}{\sffamily\MakeUppercase}
\renewcommand{\chapnumfont}{\textsf}
\renewcommand{\chaptitlefont}{\sffamily}
\renewcommand{\chapterstyle}{\sffamily}
\renewcommand{\secheadstyle}{\Large\textsf}
\renewcommand{\setsecindent}{0pt}
\renewcommand{\printchaptername}{\huge\sffamily\MakeUppercase}
\renewcommand{\printchaptertitle}{\huge\MakeUppercase}
\renewcommand{\LettrineFont}{\Huge\sffamily}
\renewcommand{\LettrineFontHook}{\Huge\sffamily}
%\renewcommand{\LettrineTextFont}{\Large\sffamily}
\pagestyle{idea}
\setcounter{secnumdepth}{-1}
%toc customization
%\renewcommand*{\cftchapterpagefont}{\textbf}
%\renewcommand*{\cftchapterfont}{\textbf}
\setpnumwidth{15em}
\renewcommand{\cftdot}{}
%includes
%\includeonly{files_listed_here_no_extension,file,file,etc} 
\usepackage{microtype} %must be loaded after all font corrections
%Titling
\title{\sffamily\Huge{Presentations for IDEA 3100}}
\author{Ross A. Laird}
\begin{document}
\pagestyle{idea}
\section{Presentation Methods and Goals}

The central idea of the presentations for this course is to give you
 opportunities to practice interdisciplinary thinking and expression.
 As such, the presentation should be interdisciplinary. Essentially,
 this means that you should try to use multiple presentation
 strategies and modalities. These might include (but are certainly not
 limited to) any of the following:

 \begin{itemize}
 \item Storytelling
 \item Poetry
 \item Music (playing)
 \item Drumming
 \item Singing
 \item Dance
 \item Movement
 \item Sport
 \item Ritual
 \item Film (showing)
 \item Film making
 \item Photography
 \item Web content
 \item Craft work
 \item Art making
 \item Individual reflection
 \item Meditation
 \item Health practices
 \item Creative process (any type)
 \item Group communication
 \item Cultural practices
 \item Nature experiences
 \end{itemize}

 Whenever possible (and workable), try to mix together multiple
 modalities into a single presentation. For example, you might ask the
 group to do some individual reflection using the modality of poetry,
 then create a series of movements based on the poetry, then work in
 small groups to talk about and share the process. Many configurations
 are possible. The trick is to choose an activity that you enjoy, then
 find a way to apply it to the content (suggested presentation topics
 are listed below). Please do not create your presentations using only
 written and/or spoken materials. In other words, don't just stand up
 at the front of the class and talk about the presentation topic.
 Utilize the energy of the group. Remember that in interdisciplinary
 work divergences are valued as unique opportunities. So, feel free to
 experiment with activities and modalities that may not seem, on the
 surface, to be related to the topic at hand but which might, upon
 experiment, yield surprising connections and results. Be playful.
 Allow yourself to laugh at yourself, to be embarrassed, to engage
 with the process in novel and interesting ways.

In interdisciplinary work, riddles and puzzles are highly prized.
Accordingly, the presentations should (ideally) not be complete
explanations or presentations of material. Feel free to play with
challenging exercises, with impossible scenarios, and other conundra.
One way to think about this is to consider insoluble riddles, such as
the one in \textit{Alice in Wonderland}: Why is a raven like a
writing desk?

\begin{quote}
  ``Have you guessed the riddle yet?'' the Hatter said, turning to Alice again.\\
``No, I give it up,'' Alice replied. ``What's the answer?''\\
``I haven't the slightest idea,'' said the Hatter.\\
``Nor I,'' said the March Hare.\\
Alice sighed wearily. ``I think you might do something better with the time,'' she said, ``than wasting it in asking riddles that have no answers.''
\end{quote}

The best interdisciplinary topics offer more questions than answers.
They, are essentially, gateways into the mysterious--which, as
Einstein will tell you, is an important place to be:

\begin{quote}
  The most beautiful thing we can experience is the mysterious. It is
  the source of all true art and science.
\end{quote}

\subsection{Suggested Topics for Interdisciplinary Presentations}

\begin{itemize}
\item Akhenaten and the invention of monotheism
\item Albrecht Durer and alchemy
\item Aristotle's book of comedy
\item Bill Evans and the Peace Piece
\item Buckminster Fuller and the geodesic
\item Chenrizi and the politics of China
\item Chuang Tzu and the butterfly
\item Coleridge and the person from Porlock
\item Csikszentmihalyi and the flow experience
\item David Bohm's Implicate Order
\item Darwin, the bassoon, and the sundew
\item Eugen Herrigel and the practice of archery
\item Francis Yates and the \textit{Art of Memory}
\item Freud, Jung, and the ``bosh'' incident
\item Fulcanelli and \textit{Mysteries of the Cathedrals}
\item Giordano Bruno and the Hermetic tradition
\item Godel's uncertainty principle
\item Hanna Arendt at Nuremberg
\item Henri Rousseau in the jungle
\item Howard Carter and ``wonderful things''
\item Jacob Bronowski at Auschwitz
\item Jacob Bronowski, Nagasaki, and \textit{Science and Human Values}
\item Jan Tschichold and the Nazis
\item John Cage on the subway with the \textit{I Ching}
\item Kepler's \textit{Somnium}
\item Mary Shelley and the genesis of \textit{Frankenstein}
\item Newton's \textit{Principia}
\item Nikola Tesla and universal energy
\item Philip K. Dick, VALIS, and 2-3-74
\item Picasso, Guernica, and Expo 1937
\item R.D. Laing and madness as reality
\item Ramanujan's notebooks
\item Richard Feynman and the invention of quantum mechanics
\item Schwaller de Lubicz at Karnak
\item Simone Weil and leading from desire
\item St. Exupery flying into the desert
\item The Reimann Hypothesis
\item The Voynich Manuscript
\item The visions of Hildegard of Bingen
\item Thoth's legacy
\item Walter Benajmin and the Angel of History
\item Wendell Berry going \textit{Into the Woods}
\item Wilhelm Reich's Cloudbuster
\item William Blake's \textit{Marriage of Heaven and Hell}

\end{itemize}

\end{document}

%%% Local Variables: 
%%% mode: latex
%%% TeX-master: t
%%% End: 
