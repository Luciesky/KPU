    %!TEX TS-program = xelatex
    %!TEX encoding = UTF-8 Unicode
    \documentclass[12pt, letterpaper]{article}
    \usepackage{fontspec}
    \usepackage{placeins}
    \usepackage{hyperref}
    \usepackage{multibbl}
    \usepackage{graphicx}
    \usepackage{txfonts}
    \usepackage{geometry}
    \geometry{letterpaper, textwidth=5.5in, textheight=8.5in, marginparsep=7pt, marginparwidth=.6in}
    %\setlength\parindent{0in}
    \defaultfontfeatures{Mapping=tex-text}
    \setromanfont [Ligatures={Common}, SmallCapsFont={ITC Officina Serif Std}, BoldFont={ITC Officina Serif Std Bold}, ItalicFont={ITC Officina Serif Std Book Italic}]{ITC Officina Serif Std Book}
    \setmonofont[Scale=0.8]{Lucida Sans Typewriter Std}
    \setsansfont [Ligatures={Common}, SmallCapsFont={ITC Officina Sans Std}, BoldFont={ITC Officina Sans Std Bold}, ItalicFont={ITC Officina Sans Std Book Italic}]{ITC Officina Sans Std}
    % ---- CUSTOM AMPERSAND
    \newcommand{\amper}{{\fontspec[Scale=.95]{StoneSansStd-MediumItalic}\selectfont\itshape\&}}
    % ---- MARGIN TASK (year, task, etc.)
    \newcommand{\task}[1]{\marginpar{\small #1}}
    \usepackage{sectsty}
    \usepackage[normalem]{ulem}
    \sectionfont{\sffamily\mdseries\upshape\Large}
    \subsectionfont{\sffamily\mdseries\scshape\normalsize}
    \subsubsectionfont{\sffamily\mdseries\upshape\normalsize}
    \setlength{\parindent}{0cm}
    \setcounter{secnumdepth}{0}
    \begin{document}
    \thispagestyle{empty}
    \reversemarginpar
    \noindent
    \includegraphics[scale=0.50]{/home/rosslaird/Dropbox/professional/ibis}\\[2em]




\section{IDEA 2100 \\A collaboration between Arts and Sciences}

Prepared by:

Lee Beavington (Faculty of Science)\\
Ross Laird (Faculty of Arts)

\section{Calendar Description}

Students will learn about the interconnected themes of ecology,
sustainability, nature experience, creativity, and personal and academic
development. They will contextualize their creative and academic
inquiries by way of experiences in nature and will broaden their
understanding of the relationship between humanity and the natural
world.

First offered: Summer 2014 or Fall 2014 (Weather too cold and uncertain
in the spring.)

\section{Learning Objectives and Outcomes}

\begin{itemize}
\item
  Read selected cultural, literary, and historical texts and discuss
  their origin, development, and contemporary relevance
\item
  Interpret interdisciplinary natural, creative, and spiritual
  traditions within the context of ecology and sustainability
\item
  Describe the role of natural experience as both an ancient and a
  current mode of knowledge inquiry
\item
  Evaluate diverse perspectives on the relationship between natural
  experience and creativity
\item
  Create interdisciplinary expressive arts projects using strategies
  developed through experiences in nature
\end{itemize}

\section{Content}

\begin{itemize}
\item
  Nature and creativity
\item
  Nature as aesthetic experience
\item
  Creative and spiritual inspiration in nature
\item
  Nature as mentor
\item
  Our relationship and lost connection with nature
\item
  Impermanence and cycles (astronomical, biological, climatic, cultural,
  geological, mythological)
\item
  Learning to listen in the forest
\item
  Embodiment and spirituality in relationship with nature
\item
  Ancestral and modern meditation and mindfulness practices in nature
\item
  Deep and social ecology
\item
  Community engagement and natural experience
\item
  Sustainable food
\end{itemize}

\section{Learning Activities}

\begin{itemize}
\item
  Participating in an integrative learning environment that entails a
  high level of communication, engagement, motivation, and collaboration
\item
  Experiencing nature in multiple locales in the Vancouver area
\item
  Reading materials from a variety of genres and historical periods
\item
  Researching and developing interdisciplinary expressive arts projects
\item
  Collaborating with other students on group projects and presentations
\end{itemize}

\section{Assessment}

\begin{itemize}
\item
  Engagement, collaboration, and student facilitation (25 percent).
  (Facilitation activities might include bushcraft, storytelling, sit
  spot, nature's grocery store, nature's medicine cabinet, the art of
  animal tracking, mapping, bird language, five-minute fire, nature
  photography, nutrient cycles, food webs, etc).
\item
  Written reflections (25 percent).
\item
  Semester-long creative project inspired by nature, ecology, or
  personal journey (two parts, 25 each).
\end{itemize}

\section{Proposed Field Sites}

\subsection{Ecology Around the Campus}

Explore the webs of connection and consider human impacts (positive and
negative).

Sciences perspective: food webs, invasive species, nutrient cycling, plant ID.

Arts perspective: patterns and symmetry, mindfulness, design systems.

\subsection{Boundary Bay (Tides and Transitions)}

Explore the in-between places, the estuary and intertidal zones. How do
organisms such as the screw snail, lugworm and ghost shrimp survive in
this chaotic environment? How do we cope during times of change?
Consider the influence of the moon.

Sciences perspective: intertidal zone, biodiversity, tides and lunar cycles.

Arts perspective: process and product, transitions, eco-psychology, dynamical systems.

\subsection{Watershed Park (A Walk/Run in the Woods)}

Explore the rich tapestry of the forest. Listen for birdsong. Every tree
has a story to tell in its rings, leaves and symbolism. What do they
tell us? Stand still, meditate, walk slowly, go for a run. Experience
yourself in ecological immersion.

Sciences perspective: dendrochronology, bird language, forest ecology (abiotic
and biotic factors, indicator species, nurse logs, adaptations,
disturbances, etc).

Arts perspective: folklore, mythology, aesthetic
experience, creativity.

\subsection{Cougar Creek (The River of Time)}

Follow the river and be mindful of the Journey. Water is central to
life, a simple compound of hydrogen and oxygen and yet amazingly unique
in its properties to sustain life.

Sciences perspective: limnology (biological, geological, chemical, and physical
characteristics of the river), water cycle, water quality (EPT) index.

Arts perspective: artistic inspiration, reflection, finding purpose.

\subsection{Delta Nature Reserve (Bogged Down)}

Burns Bog is a globally unique ecosystem, and one of the largest
undeveloped urban wilderness areas in North America. An immense amount
of carbon dioxide is trapped within this raised peat bog. What happens
when we get stuck? How can we shift things?

Sciences perspective: unique flora and fauna, carbon cycle, climate change,
sustainability.

Arts perspective: creativity, inspiration, personal growth.

\subsection{Tynehead Fish Hatchery (Life and Death)}

Understand how a hatchery works. Tynehead is run entirely by volunteers.
Perhaps have the opportunity to participate in an egg take.

Sciences perspective: life cycle (fish), fish biology, sustainability.

Arts perspective: life cycle (human), death and birth in art, connection to food, myth, and culture.

\subsection{Vancouver Aquarium}

Explore the fantastic marine environments at the aquarium.

Sciences perspective: marine biology, oceanography, evolution, sustainability.

Arts perspective: creative inspiration, colour and texture, movement, transformation.

\subsection{Other sites to consider}

Reifel Bird Sanctuary, Serpentine Fen, Bird
Banding at Colony Farm (Saturdays only), Mountain View Conservation
Centre (tours only, have unique animals both native and exotic),
Campbell Valley Park, Derby Reach Park, Tynehead Park, farmer's market,
the power-line trail near Kwantlen. We will also work with students to support their own suggestions about sites to visit.

\subsection{Backup activities (in case of extreme weather)}

Visit the biology lab.

In-class activities (dialogues, reflections, preparing for group presentations, ecology games, meditation, embodiment).

Indoor activities around the campus.

\section{Readings}

Beavington, Lee. \emph{Common Plants of Greater Vancouver.}\\
Butala, Sharon. \emph{The Perfection of the Morning: an Apprenticesehip in
Nature.}\\
 Emerson, Ralph Waldo. \emph{Nature.}\\ Krakauer, Jon. \emph{Into
the Wild.}\\ Kroodsma, Donald. \emph{The Singing Life of Birds: The Art
and Science of Listening to Bird Song.}\\ Laird, Ross. \emph{A Stone's
Throw: The Enduring Nature of Myth.}\\ Louv, Richard. \emph{Last Child in
the Woods: Saving Our Children from Nature-Deficit Disorder.}\\ Pojar, Jim
\& Mackinnon, Andy. \emph{Plants of Coastal British Columbia.}\\ Thoreau,
Henry David. \emph{Walden}.\\ Young, Jon, Ellen Haas, \& Evan McGown.
\emph{Coyote's Guide to Connecting with Nature.}

\end{document}
