    %!TEX TS-program = xelatex
    %!TEX encoding = UTF-8 Unicode
    \documentclass[11pt, letterpaper]{article}
    \usepackage{fontspec}
    \usepackage{placeins}
    \usepackage{hyperref}
    \usepackage{multibbl}
    \usepackage{graphicx}
    \usepackage{txfonts}
    \usepackage{geometry}
    \geometry{letterpaper, textwidth=5.5in, textheight=8.5in, marginparsep=7pt, marginparwidth=.6in}
    %\setlength\parindent{0in}
    \defaultfontfeatures{Mapping=tex-text}
    \setromanfont [Ligatures={Common}, SmallCapsFont={ITC Officina Serif Std}, BoldFont={ITC Officina Serif Std Bold}, ItalicFont={ITC Officina Serif Std Book Italic}]{ITC Officina Serif Std Book}
    \setmonofont[Scale=0.8]{Lucida Sans Typewriter Std}
    \setsansfont [Ligatures={Common}, SmallCapsFont={ITC Officina Sans Std}, BoldFont={ITC Officina Sans Std Bold}, ItalicFont={ITC Officina Sans Std Book Italic}]{ITC Officina Sans Std}
    % ---- CUSTOM AMPERSAND
    \newcommand{\amper}{{\fontspec[Scale=.95]{StoneSansStd-MediumItalic}\selectfont\itshape\&}}
    % ---- MARGIN TASK (year, task, etc.)
    \newcommand{\task}[1]{\marginpar{\small #1}}
    \usepackage{sectsty}
    \usepackage[normalem]{ulem}
    \sectionfont{\sffamily\mdseries\upshape\Large}
    \subsectionfont{\sffamily\mdseries\scshape\normalsize}
    \subsubsectionfont{\sffamily\mdseries\upshape\normalsize}
    \begin{document}
    \thispagestyle{empty}
    \reversemarginpar
    \noindent
    \includegraphics[scale=0.5]{/home/rosslaird/Dropbox/kwantlen/arts_logo}\\[2em]
      \\[2em]

{\LARGE Application for Arts Special Purpose Funds}

October 15, 2013

Interdisciplinary Expressive Arts (IDEA)

\section{Applicant Information}

Ross Laird KPU \\
Faculty, IDEA and Creative Writing\\
(604) 916-1675\\
ross@rosslaird.com\\[1em]
\noindent
Robert (Bob) Walker\\
IDEA Work/Study Student\\
KPU Anthropology Major\\
(778) 829-2621\\
rgwkpu@gmail.com or robert.walker1@kwantlen.net

\section{Department or Program Affiliation(s)}

Faculty of Arts, Interdisciplinary Expressive Arts (IDEA)\\[1em]
\noindent
The IDEA program also collaborates with the Faculty of Design (the
Amazon Field School), the Faculty of Science (the Ecology course), the
Department of English (the Performance and Theatre course), and with
Student Life (the Aboriginal IDEA 100 course). IDEA affiliations span many areas at KPU.

\subsection{Who are we?}

\subsubsection{Ross Laird, KPU Faculty, IDEA Program and Creative Writing}

Ross Laird is the founder and lead developer of IDEA. The scope of his
participation in this project will be to liaise with KPU colleagues and
students, to work with internal and external partners, and to help guide
the overall development of IDEA.

\subsubsection{Robert (Bob) Walker - IDEA Work/Study Student and KPU
Anthropology Major}

Bob returned to Kwantlen after many years in the workforce. He attended
his first IDEA course through the Amazon Field School in summer 2013 and
immediately realized the benefits IDEA learning had on himself as a
person and his academic pursuits. Bob was hired in September 2013 as a
single semester work/study student to assist Ross Laird in developing
the IDEA program and credentials at KPU and to embark on a research
study documenting first year students experience in IDEA courses.

\section{Funding Categories}

\subsection{D. Program Innovation and Expansion,\\
IDEA Program and Credential Development}

We seek funding to assist the development of a Minor and a Post-baccalaureate Certificate in Mentorship and Community Engagement.
Coupled with this initiative is the proposed development of an IDEA
Centre or Institute at KPU. Research and program development are
required to further this initiative, as the IDEA program currently
receives no developmental or coordination funding and does not operate
within a departmental budget. Funds such as the ASPF are imperative to
help IDEA do the necessary work to provide and improve the program
offerings and research opportunities for IDEA learners at KPU.

\subsubsection{Budget for Category D}


\begin{tabular}{|l|c|r|}
\hline
Item & Amount & Expense \\
\hline
Mileage & 1058 km & 497.27\\
Ferry & 3 car passengers & 195.50\\
Accommodations & 3 adults, Seattle & 450.00\\
Incidentals & 3 adults, 4 days & 60\\
Meals & 3 adults, 4 days & 480\\
\hline
Total Expenses & Consultations & 1682.77\\
\hline
\end{tabular}
\\[1em]
Note: all expense amounts have been calculated based on allowable amounts in the Collective Agreement and the KPU Expense Claim Regulations. All travel will be completed in accordance with these regulations.

These travel expenses will be allocated toward an extensive set of
external consultations with the four post-secondary institutions in
the Pacific Northwest that offer programming most similar to IDEA:
Antioch University in Seattle, Evergreen State College in Olympia,
Royal Roads University in Victoria, and Quest University in Squamish.

These visits will be accomplished by road, with three IDEA faculty
members and/or steering committee members. The purpose of the visits
is to learn, from peers and colleagues who have successfully
implemented programming similar to IDEA, how it was done and how it is
sustained. These visits will also help to build local relationships
between IDEA and other, similar programs.

The trip to visit Antioch University and Evergreen State College will be most efficient if accomplished over the course of two days (one day in Seattle, one day in Olympia) with a single overnight stay in Seattle.

\subsection{B. Student Research and Scholarship,\\
Experiences and Perspectives on the IDEA Learning Model}

As a complimentary and co-occurring project to the proposal laid out
above, this section consists of a research study conducted by Bob
Walker, IDEA Work/Study Student. The methodology will consist of 20
hours of interviews with IDEA learners which will serve to assess and
document their personal perspectives during and after their experiences
with the IDEA model of learning. The data collected will be used to
assist Ross Laird and the IDEA steering committee to continue to develop
IDEA courses and credentials at KPU.

The work/study will conduct the interviews, compile data, and compose a
written report of the findings from the study. As this research falls
within the category of internal quality evaluation, REB approval will
not be required.

\subsubsection{Budget for Category B}
\$615.00\\[1em]
\noindent
This budget provides for approximately 60 hours at a rate of \$10.25 per
hour, with a maximum of 15 hours per week (as laid out in the work/study
student guidelines). These funds will enable an IDEA work/study student
to continue beyond the currently-funded period which ends in December
2013. Ongoing engagement and support are required to assist in program
development and research.

\subsection{Total Budget Requested from Arts SPF}

This application requests the amount of \$2297.77.\\[1em]
\noindent
This application encompasses the IDEA program initiatives as outlined
below; however the IDEA program collaborates with other KPU Faculties
(such as the Faculties of Design and of Science) and therefore the IDEA
program and course offerings are able to benefit KPU learners from many
faculties and programs. The use of these funds will directly benefit
learners in the IDEA courses, but as IDEA learners come from all KPU
faculties and programs, the benefits are holistic to KPU.

These funds will be used for the development of IDEA initiatives
throughout the coming year.

\subsection{Total Budget from other sources}

No other funding for further or new IDEA program development has been
received or applied for as of the date of this application.

\subsection{Dates of the initiative}

As the IDEA program at KPU is growing, and building on the success of
the first Amazon Field School in the Summer semester of 2013, the
funding provided from this proposal will be used to further initiatives
in program development in the following year commencing on the fund
award date. A full report as to the allocation of funds and project
successes will be provided after one year.

\section{Project Description}

IDEA is founded on the basic idea that post-secondary learning can be
both holistic and integrative --- that learners can build their
self-awareness, empathy, and character at the same time that they
construct their disciplinary knowledge.

Since its inception in 2008, IDEA offerings at KPU have thrived as a
result of high learner demand and enrolment (typically 100\%) and the
good testimony of returning learners. With this encouragement it has
become apparent that the IDEA model of learning and personal development
is viewed as important and useful to learners at KPU. In turn, this has
sparked the need to embark upon the development of IDEA in a more robust
and formalized way within the Faculty of Arts and to provide learners
with the opportunity to be conferred a credential within this framework.
These are the primary goals of this funding application: to develop a
pathway for developing a set of credentials as a viable path for
learners wishing to continue their studies in Interdisciplinary
Expressive Arts, and to work toward IDEA becoming a recognized unit at
KPU. A Minor and a Post-baccalaureate Certificate program in Mentorship
and Community Engagement (the latter offered on a cost-recovery basis),
sponsored by a Centre of Institute, will provide an integrative and
self-directed learning environment for learners to develop personal and
professional mentorship and creativity skills they can use in the
workplace, in their communities, and in their families. These programs
are intended to be structured as learning communities in which
individual learners will build their own pathways based on how much they
already know, how they want to apply their skills, and how intensive
they wish their programs to be.

However, there is no current framework at KPU (beyond the normal process
for developing a Minor) to support and facilitate many of the kinds of
things we wish to do in IDEA (for example, moving toward
credential-based transcripts, moving toward a learning community model
without set class times, and developing learner-centered credentials
that do not prescribe set pathways but rather offer gap analyses and
personalized programming). IDEA currently works within the traditional
framework at KPU: grades, set classes, semester-based courses. We
believe that learning innovations are required to move beyond some, or
perhaps all, of these structures. Such innovations are currently
sweeping across educational institutions in the United States and are
now finding their way into Canadian universities (for example, the block
system and learning community model at Quest). There is nothing other
than an absence of precedent preventing us from exploring similar kinds
of innovations at KPU, and IDEA is well-poised to take the steps
required to make new precedents. We have already done so in several
areas (for example, IDEA developed the first cross-faculty course at
KPU). But we'd like to see how much farther we can go, what we can
accomplish, and what these innovations will entail in terms of our
credential offerings. We do not wish to offer traditional credentials.
We believe that the greatest strength of IDEA is its unique character,
and the credentials we offer should reflect the fact that IDEA is unique
and different. And we believe that KPU has made a firm commitment
(reflected in documents such as the Strategic and Academic Plans) to
precisely the kinds of innovations we wish to undertake.

For example, IDEA focuses on specific modalities such as inquiry-based
curriculum, reflective scholarship, intensive interaction, and
interdisciplinarity. These approaches are all listed in the KPU Academic
Plan as being foundational priorities. Additionally, the Academic Plan
highlights the importance of values such as global citizenship,
inspiration, compassion, imagination, and personal growth. These values
are core elements of the IDEA philosophy in general and are articulated
specifically in the IDEA core values list. The Academic Plan further
emphasizes skills such as effective communication, openness to diversity
and inclusion, and creative and critical thinking. Again, these are all
integral to the IDEA process and are specifically articulated as core
goals in our program philosophy documentation.

We find similar correlations when we examine the KPU Strategic Plan,
which describes the importance of distinctive programming, innovative
teaching and learning, experiential learning, enriched student
experience, and purposeful community engagement. Each of these core
goals for KPU matches a core value for IDEA.

The Arts Academic Plan is aligned with the KPU Academic and Strategic
Plans, and we therefore see similar correlations in values, approaches,
and goals. For example, the Arts Academic Plan specifically emphasizes
goals such as the enhancement of experiential learning, the wider
adoption of interdisciplinary approaches, the importance of preparing
learners for global citizenship, and the crucial role of mentorship. All
of these broad goals are specifically articulated in IDEA philosophies
and practices.

There is a high degree of symmetry between the stated goals of KPU, on
many levels, and the articulated goals of IDEA. We believe that IDEA
embodies, in a very fundamental way, the kind of institution that KPU
has said it wants to be. But we don't see a clear pathway forward. We don't have answers to some of our basic questions about what kind of structure IDEA can be, what types of innovations we can reasonably expect to achieve, and what parts of our vision are possible at KPU. In order to find answers to these pressing questions we will need to consult extensively with internal partners (on questions about structure and innovation) and with external partners (on questions about vision, growth, and sustainability).   

We believe strongly that KPU and IDEA are well-matched, and we'd like to help KPU achieve its goals. We just need a bit of help getting started.

\end{document}
