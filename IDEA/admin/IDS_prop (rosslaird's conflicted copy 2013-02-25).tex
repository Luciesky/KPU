   \TeXXeTstate=1
\documentclass[12pt,DIV11,letterpaper,oneside,abstractoff,headsepline]{scrreprt}
   \usepackage{fontspec,xunicode}
   \defaultfontfeatures{Numbers=OldStyle,Scale=MatchLowercase,Mapping=tex-text}
   \setmainfont{Sabon LT Std}
   \setromanfont[Mapping=tex-text]{Sabon LT Std}
   \setsansfont[Scale=MatchUppercase]{Myriad Pro}
   \usepackage{xltxtra}
\pagestyle{headings}
\usepackage{paralist}
\usepackage{datetime}
\usepackage{titling}
\usepackage[sf]{titlesec}
\usepackage{url}
\setcounter{secnumdepth}{-1}
\setkomafont{pagehead}{%
\normalfont\normalcolor\upshape\sffamily\small}
\author{}
\date{November 2010}
\pretitle{\begin{center}\LARGE\sffamily Proposal for a\\ Department of Interdisciplinary Studies \end{center}\begin{center}}

\posttitle{\end{center}\begin{center}\LARGE\sffamily Faculty of Humanities\end{center}}

\title{\huge\sffamily Kwantlen Polytechnic University}


\begin{document}
\maketitle
\tableofcontents
\chapter[Proposal: Department of Interdisciplinary Studies]{Department of Interdisciplinary Studies}
\section{Context and Rationale}
\label{sec-1}

Kwantlen is British Columbia's Polytechnic University. The integration of the terms Polytechnic and University in our title reflects our unique mandate, mission, and character as a post-secondary institution. The first principle articulated in our new mission and mandate is to be ``a leader in innovative and interdisciplinary education.''

The proposed new Department of Interdisciplinary Studies will embody the core values and vision of Kwantlen’s emerging identity. By working within and across disciplines, programming within the Department of Interdisciplinary Studies will encourage innovation and collaboration among faculty, will create opportunities for scholarship and teaching, and will prepare learners to be competitive in a rapidly changing world. The department will provide learning experiences that facilitate creativity, critical awareness, cultural sensitivity, social responsibility, civic engagement, and global citizenship.

In Kwantlen's Strategic Action Plan (2010), the first principle is to ensure that ``Kwantlen's learning environment inspires inquiry, collaboration, creativity, and application.'' The first priority is to ``develop and implement new programs, especially undergraduate degree programs'' that ``support initiatives to increase scholarly and research activity within and across a range of Faculties.'' The proposed Department of Interdisciplinary Studies will embody these aims as foundational to its practices of scholarship and teaching.

The first and primary principle of Kwantlen's Senate Subcommittee on Academic Planning and Priorities is to ``implement and support new programs, especially those that reflect community needs, labour market and broad societal education needs, and which are in keeping with Kwantlen’s values and mandate as a Polytechnic University.'' The proposed Department of Interdisciplinary Studies is precisely aligned with this interdisciplinary priority. 

The second principle of the Senate Subcommittee on Academic Planning and Priorities is to ``provide opportunities that encourage faculty to develop new teaching interests and methodologies in keeping with the institution’s mandate.'' The proposed Department of Interdisciplinary Studies will forge strategic partnerships between various stakeholders at Kwantlen (Humanities, Social Sciences, The School of Business, The Faculty of Academic and Career Advancement, The Centre for Academic Growth, among others) to encourage innovation, to support diverse teaching interests, and to broaden and integrate course offerings from various fields. The instructional philosophy of the department will recognize the pivotal role that the Kwantlen community fulfills in supporting learners to develop their values, direction, and fundamental character. Emerging and experimental teaching methods and research-driven pedagogy will be embraced. The Department will provide a dynamic community of inquiry that is learner-focused, innovative, and socially and culturally responsive. Personal, academic, and professional development will be integrated. Faculty will demonstrate an authentic spirit of interdisciplinary inquiry and will model this for learners. In turn, learners will be supported in diverse modes of inquiry by a teaching environment that is collaborative, innovative, creative and respectful.

The Department of Interdisciplinary Studies will embody Kwantlen’s vision to be an innovative and outstanding Polytechnic University that provides a balance of pure, practical and applied educational experiences to learners from diverse backgrounds. The department will serve regional needs, reflect community values, and provide a desirable destination for those seeking an accessible and relevant learning environment. The Department of Interdisciplinary Studies will be an archetypal example of Kwantlen’s diversity, inclusivity, and continued evolution.
\newpage

\section{A Definition of Interdisciplinary Studies}
\label{sec-4}

Interdisciplinary Studies refers to a specific set of educational activities, goals and strategies. Based on innovative pedagogy and integrative approaches to learning, Interdisciplinary Studies involve the synthesis and synergy of various disciplines toward a cohesive, unified educational experience. Interdisciplinarity is much more than enrollment in courses from more than a single discipline. Authentic interdisciplinarity emphasizes the linkages between disciplines by focusing on contrasting and complementary aspects of diverse educational domains. Interdisciplinary studies encourage students to develop broader intellectual skills, greater facility for critical thinking, and greater awareness of the social relevance of their education. 

Interdisciplinary students have the opportunity to develop exemplary skills in problem solving, insight, team-building, lateral thinking, and multi-modal learning styles. Interdisciplinary strategies involve approaching an issue or problem from various perspectives. This typically entails intellectual inquiries that range beyond the borders of any single discipline or domain.  Global warming, economic turbulence, and the \textsc{AIDS} pandemic are all examples of contemporary issues that require interdisciplinary approaches. 

The interdisciplinary perspective is one in which the network of disciplines is the domain. If traditional disciplines are described as a network of nodes, the following structure outlines the interdisciplinary approach:
\begin{itemize}

  \item Each traditional discipline is a node.
  \item Each node is connected to every other node.
  \item The nodes are connected by numerous (theoretically infinite) pathways.
  \item Interdisciplinary learners study the pathways.
\end{itemize}

As with other disciplines, the study of the pathways (Interdisciplinary Studies) has its own set of guiding principles, which might be outlined as follows:

\begin{description}
  \item[Translinearity:] focus on pathways and different perspectives (between and across fields, domains, curriculum, bodies of knowledge, networks and actor-networks, academic and social structures).
  \item[Integration:] focus on the shared characteristics of nodes, multiple sources and experiences, various settings, diverse and contradictory points of view, contextuality.
  \item[Hermeneutics:] focus on interpretation and meaning, analysis of verbal and nonverbal forms of communication, exploration and analysis of assumptions, presuppositions, pre-understandings, attention to the meaning of language. (The term \textit{Hermeneutic} is derived from the mythological Hermes/Thoth: boundary-crosser, keeper of keys, god of trespass and wisdom, inventor of the arts and sciences.)
  \item[Optionality:] focus on auxiliary functions of domains and knowledge, transferability, redundancy, bricolage, wandering, accidents, boundary-crossing, maximal exposure to forking avenues.
  \item[Network Effects:] focus on nonlinearity, robustness, cascades, recursive links, black swans, interdependence, percolation, complexity.
\end{description}

Interdisciplinarity is a lens that moves across all domains. That lens might be called the node of Interdisciplinarity.

Current well-known scholars working in the Interdisciplinary framework are David Edwards, Lewis Hyde, and Nassim Nicholas Taleb.

\newpage
\section{Current Status of Interdisciplinary Initiatives}
\label{sec-2}

Representatives from various departments within Humanities have been working toward a greater role for Interdisciplinary collaborations for several years. The departments of Fine Arts, Music, Modern Languages, and Philosophy have been particularly active in developing and promoting interdisciplinary projects. Additionally, various other groups have become involved in such projects since Kwantlen's designation as a university. Those groups include the School of Business, the Learning Centre, The Centre for Academic Growth, the Faculty of Academic and Career Advancement, and the Faculty of Social Sciences. Here are some examples of current and planned interdisciplinary projects:
\subsection{Current Projects}
\label{sec-2.1}

\begin{itemize}
\item Interdisciplinary Expressive Arts (IDEA) courses.
\item New media and social media teaching project (collaboration between
    Humanities, the Department of Instructional and Educational
    Technology, and the Centre for Academic Growth).
\item Humanities collaboration with Phoenix Society.
\item Arts, Culture, and Media course development (collaboration
    between Humanities and the School of Business).
\item Sustainability course developments (collaboration between
    Humanities, the Faculty of Social Sciences, and the Faculty of
    Academic and Career Advancement).
\item Modern Languages new media and social media teaching project
    (collaboration between Modern Languages and IDEA).
\item Interdisciplinary research project in creativity and mental
    health (collaboration between Humanities and Social Sciences).
  \item Gateway Theatre project (collaboration between Humanities, Social Sciences, The Centre for Interdisciplinary Research, the Gateway Theatre, and the City of Richmond).
  \item Annual Interdisciplinary Illumination events.
  \item Aboriginal program development project (collaboration between Humanities, The Centre for Interdisciplinary Research, and Anishnawbe Health).
\end{itemize}
\subsection{Planned Projects}
\label{sec-2.2}

\begin{itemize}
\item Interdisciplinary Expressive Arts course developments (underway).
\item Interdisciplinary course developments in Music, English (in
     discussion), Fine Arts (in discussion), Philosophy, Sociology (in discussion), Educational Studies (in discussion).
\item Course and project developments through expressions of
     interest, by faculty members, in innovative and
     interdisciplinary initiatives.
\item Cross-listings and specialized sections of existing courses (by
     invitation and/or application) from various academic units.
\item Co-op, work study, practicum, and work placements within Humanities.
\end{itemize}

The Department of Interdisciplinary Studies will provide a means of streamlining many of the interdisciplinary projects already underway, will provide support for those projects, and will offer faculty members avenues for developing new and innovative initiatives. The Department will promote cross-appointments to encourage interested faculty from other departments to extend their work into the Interdisciplinary area. Such cross-appointments will develop through invitation, application, and expressions of interest.

The Department of Interdisciplinary Studies will be an example of an increasingly popular mode of collaboration within the university system (The Humanities, Arts, Science, and Technology Advanced Collaboratory and the Townsend Humanities Lab are two other examples).

Soon after inception, the Department of Interdisciplinary Studies plans to offer first a Minor, then a Major, in Interdisciplinary Studies.
\newpage

\section{Goals and Timelines}
\label{sec-3}

Plans for the proposed Department of Interdisciplinary Studies include the following:
\subsection{Short Term (2010 -- 2011)}
\label{sec-3.1}

\begin{enumerate}
\item Governance and policy consultation and approval.
\item Assignment of institutional resources necessary to support the
     bootstrapping of the Department.
\item Development of Interdisciplinary Studies foundational courses (1100
      and 2100).
\item Cross-listing and/or development of specialized sections of
     interdisciplinary courses from various departments. Four courses
     are currently under discussion and/or development: \textsc{CRWR3301},
     \textsc{SUST1100}, \textsc{LCOM1100}, and \textsc{LCOM4330}.
\item Continued offering and promotion of IDEA 3100 and IDEA 4100.
\end{enumerate}
\subsection{Middle Term (2011 -- 2012)}
\label{sec-3.2}

\begin{enumerate}
\item Development and promotion of a Minor in Interdisciplinary Studies.
\item Continued development of new courses and collaborations.
\item Growth in numbers of faculty whose primary allocation resides in
     the Department of Interdisciplinary Studies.
\item Development of an interdisciplinary digital community (using open
     source content and collaboration tools such as Drupal).
   \item Collaboration and program development with Kwantlen's new Division of Continuing Studies.
 \end{enumerate}
\subsection{Long Term (2012 -- 2013)}
\label{sec-3.3}

\begin{enumerate}
\item Development of a Major in Interdisciplinary Studies (a version of
       the Major concept proposal has already been approved by the
       Humanities Curriculum Committee).
\item Ongoing growth and development to support the Kwantlen community.
\end{enumerate}
\end{document}
