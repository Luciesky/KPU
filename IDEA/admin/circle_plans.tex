   \TeXXeTstate=1
\documentclass[12pt,DIV11,letterpaper,oneside,abstractoff,headsepline]{scrreprt}
   \usepackage{fontspec,xunicode}
   \defaultfontfeatures{Numbers=OldStyle,Scale=MatchLowercase,Mapping=tex-text}
   \setmainfont{Sabon LT Std}
   \setromanfont[Mapping=tex-text]{Sabon LT Std}
   \setsansfont[Scale=MatchUppercase]{Myriad Pro}
   \usepackage{xltxtra}
\pagestyle{headings}
\usepackage{paralist}
\usepackage{datetime}
\usepackage{titling}
\usepackage[sf]{titlesec}
\usepackage{url}
\setcounter{secnumdepth}{-1}
\setkomafont{pagehead}{%
\normalfont\normalcolor\upshape\sffamily\small}
\author{Prepared by: Steve Dooley and Ross A. Laird}
\date{September 2010}
\pretitle{\begin{center}\LARGE\sffamily CIR:CLE\\The Centre for Interdisciplinary Research: Community Learning and Engagement \end{center}\begin{center}}

  \title{\huge\sffamily Purpose, Plans, and Direction}


\begin{document}
\maketitle
\tableofcontents
\chapter[CIR:CLE]{CIR:CLE}
\includegraphics[scale=0.25]{/home/ross/Dropbox/docs/kwantlen/enso}\\[2em]
\section{Our Context}
\label{sec-1}

Kwantlen is British Columbia's Polytechnic University. The integration of the terms Polytechnic and University in our title reflects our unique mandate, mission, and character as a post-secondary institution. The first principle articulated in our new mission and mandate is to be ``a leader in innovative and interdisciplinary education.''

\textsc{CIR:CLE} embodies the core values and vision of Kwantlen’s emerging identity. By working within and across disciplines, CIR:CLE encourages innovation and collaboration among faculty, creates opportunities for scholarship and teaching, and prepares learners to be competitive in a rapidly changing world. Faculty working within CIR:CLE provide learning experiences that facilitate creativity, critical awareness, cultural sensitivity, social responsibility, civic engagement, and global citizenship.

In Kwantlen's Strategic Action Plan (2010), the first principle is to ensure that ``Kwantlen's learning environment inspires inquiry, collaboration, creativity, and application.'' The first priority is to ``develop and implement new programs, especially undergraduate degree programs'' that ``support initiatives to increase scholarly and research activity within and across a range of Faculties.'' CIR:CLE embodies these aims as foundational to its practices of scholarship and teaching.

The first and primary principle of Kwantlen's Senate Subcommittee on Academic Planning and Priorities is to ``implement and support new programs, especially those that reflect community needs, labour market and broad societal education needs, and which are in keeping with Kwantlen’s values and mandate as a Polytechnic University.'' CIR:CLE is precisely aligned with this interdisciplinary priority. 

The second principle of the Senate Subcommittee on Academic Planning and Priorities is to ``provide opportunities that encourage faculty to develop new teaching interests and methodologies in keeping with the institution’s mandate.'' Plans for CIR:CLE involve the forging of strategic partnerships between various stakeholders at Kwantlen (Humanities, Social Sciences, The School of Business, The Faculty of Academic and Career Advancement, The Centre for Academic Growth, among others) to encourage innovation, to support diverse teaching interests, and to broaden and integrate educational opportunities from various fields. CIR:CLES recognizes the pivotal role that the Kwantlen community, and our surrounding communities, fulfill in supporting learners to develop their values, direction, and fundamental character. CIR:CLE provides a dynamic community of inquiry that is learner-focused, innovative, and socially and culturally responsive. At CIR:CLE, personal, academic, and professional development are integrated. Faculty demonstrate an authentic spirit of interdisciplinary inquiry and model this for learners. In turn, learners are supported in diverse modes of inquiry by a research and learning environment that is collaborative, innovative, creative and respectful.

CIR:CLE embodies Kwantlen’s vision to be an innovative and outstanding Polytechnic University that provides a balance of pure, practical and applied educational experiences to learners from diverse backgrounds. CIR:CLE serves regional needs, reflects community values, and provide a desirable destination for those seeking an accessible and relevant learning environment. CIR:CLE is an archetypal example of Kwantlen’s diversity, inclusivity, and continued evolution.
\newpage
\section{Our Compass and Landscape}
\label{sec-2}

The goal of CIR:CLE is to create productive interdisciplinary partnerships between Kwantlen faculty, students, staff and members of local organizations and other community groups. We are focused on community development as a research process in the context of needs assessments, program development and agency program evaluations. 

Our aim is to create meaningful opportunities for dialogue through research, speaker series, and other community events, to establish strong, mutually beneficial relationships with community partners, and to maintain a positive impact for members of a community through long-term engagement.

CIR:CLE works in the community, for the community, and with the community. We conduct research about the community. 

In May 2010, faculty members from both Social Sciences and Humanities attended an Institute visioning workshop, and the NIRSCD was re-branded as The Centre for Interdisciplinary Research: Community Learning and Engagement. CIR:CLE continues to create productive partnerships between Kwantlen faculty and students and members of local organizations and other community groups.

CIR:CLE has cultivated a wide range of relationships with community partners. Through meetings to discuss project opportunities, sitting on various community community boards, hosting community meetings, answering research questions, conducting site visits, and attending related conferences, CIR:CLE has established and maintained long-term relationships with community partners throughout British Columbia.

CIR:CLE directly participates in research activities involving community groups, school districts, municipal governments and police agencies to develop and evaluate the effectiveness of various programs. We also develop and deliver community-based conferences and meetings and organize and participate in conferences at Kwantlen. Examples include the Youth Homelessness Conferenc, the Youth Gang Violence workshop, and the ucoming Larry Scanlan event focused on philanthropy. 

CIR:CLE financially supports conferences developed and delivered by many departments and faculty members. For example, CIR:CLE is sponsoring and facilitating a regional roundtable discussion of challenges and opportunities facing Threat Assessment Training in British Columbia. Participating in this event will be school districts superintendents, RCMP school liaison officers, and threat assessment experts from different regions across the province.

CIR:CLE has collaborated with Safe Schools Division, Surrey School District in the evaluation of school programs that have a direct impact on the educational experience of local students. For example, the Intervention, Rethink, Refocus, Reintegrate (iR3) Surrey School District program is an Alternative to Suspension program that was launched in April, 2007. iR3 is an option for school administration for students from grades 6-8 who have been temporarily suspended from school. The program provides the tools necessary to assist young students in increasing their personal growth, leadership abilities, motivation and self-worth. This unique and innovative two-day strategy to the traditional at-home suspension offers at-risk youth various preventative workshops. The goal is to successfully reintegrate students back into their schools better connected to their school and community than when they left.

CIR:CLE has also been involved with the First Step, Surrey School District program, which is designed to enhance communications between parents, children and the community around guns, gangs and violence. This unique student-led project involves opportunities for youth to present to and engage with their parents about these issues, while developing a range of skills that build resilience to gang recruitment and connectedness to school and family.

Another CIR:CLE project is Bridges, a Surrey School District collaboration that assists at-risk students in making the crucial transition from elementary to secondary school. Bridges is a 3-year pilot project funded by Health Canada and operating at two Surrey high schools. The program focuses on reducing the anxiety associated with the transition from elementary to secondary school via a peer mentoring program. The program is offered in a structured and supportive environment to encourage independence and self-confidence. Mentors gain leadership experience and acquire valuable reference benefits for resume and post-secondary opportunities.

CIR:CLE is actively involved in many community-based committees, among them the 
Pacific Community Resources Society, Vibrant Surrey, Umoja of Surrey, HOPE Bridge Services, Community Vision for Children and Families, and the City of Surrey Social Planning Advisory Committee. CIR:CLE members maintain strong ties to community partners and peers.

\newpage
\section{Where to From Here?}
\label{sec-3}
CIR:CLE will continue to focus on making a positive impact to the community. We will apply for new projects with equal enthusiasm regardless of the profile level and monetary value. We will create the framework for new programs that can easily be transferred to other communities throughout the province and the country. We will set the benchmark by providing an innovative and heuristic approach to all of our activities.

CIR:CLE's immediate plans involve several interlocking stages and developmental tasks. Essentially, the core initiative is a movement toward greater interdisciplinary collaboration, enhanced community involvement, fresh research projects, and the bootstrapping of a learning environment based on contemporary best practices and tools (particularly new media and social media tools). This shift requires collaborative, consultative work at every stage, and will best be accomplished by enlarging the internal staff of CIR:CLE. By January of 2010, we plan to bring a new member onboard, who will assist CIR:CLE with the following initiatives:

\begin{itemize}
\item Rebranding CIR:CLE and coordinating an extended new launch of the organization.
\item Integrating and further developing ongoing research projects.
\item Developing (including securing funding for) the new community research project in addictions and mental health.
\item Designing, implementing, promoting, and training CIR:CLE staff and peers in the development of a digital collaboratory (digital collaboratories are now standard and essential tools for community-based research projects). The digital collaboratory will require ground-up design, programming, customization, maintenance, content tracking, and many other associated tasks. The collaboratory will be the first of its kind at Kwantlen, will anchor CIR:CLE's presence in the digital ecosystem, and will significantly boost the profile of the organization (and will boost Kwantlen's profile as well).
\item Training CIR:CLE faculty in the principles and practices foundational to new media and social media research, promotion, collaboration, creativity, and teaching.
\item Development of CIR:CLE initiatives in blended learning as a pedagogical approach to creativity, community immersion, and scholarship.
\item Assisting CIR:CLE faculty in developing new and unique community-based curriculum based on, and delivered through, new and social media.
\end{itemize}

The above initiatives will require consistent and long-term involvement from an individual who possesses the particular interdisciplinary blend of skills, experience, and community-based values that will best serve the needs and goals of CIR:CLE. Essentially, the new CIR:CLE member must embody the principles of CIR:CLE's new acronym: working with and for the Centre, leveraging an extensive Interdisciplinary background, motivated by and engaged with Research, involved daily in Community projects, dedicated to meaningful Learning, and committed to purposeful Engagement with many different stakeholders.

\end{document}
