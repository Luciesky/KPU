
## Katalyst Grant Application
### Ross A. Laird, PhD


1) Summary of proposed research

Provide a one-page summary of your research proposal. The proposal should be written in clear,
plain language that can be understood by all members of the Kwantlen committee. The summary
should clearly indicate the problem or issue to be addressed, the methods you will use to address it,
and the outcome, deliverables and/or product that should be available by the end of the funding
period.

2) Detailed description

Supplement your summary by providing specific information on your project. The detailed
description should include the following sections:

a. Research objective – You should locate the proposed research in the context of previous work
by yourself(ves) and the existing research literature. Consider the extent to which the research
will advance knowledge or creative efforts. Discuss the full ‘problematic’ underlying the
investigation.

b. Methodology (including student-training opportunities): The methodological or technical
approach should be clearly stated including (where relevant) sample size and methods,
documents or archives to be used, and why they are considered relevant, etc. Please note that
all activities need to be justified and explained clearly, demonstrating why they are appropriate.

c. Anticipated outcome: Katalyst funding is primarily for new scholars wishing to make a
significant start on their research career, and for experienced scholars changing research
direction into new fields. You should address how the proposed work will accomplish this.
Where appropriate, you should also indicate how your research is relevant to Knowledge
Mobilization, including the transfer and implementation of knowledge

3) List of works cited

Provide bibliographic entries sufficient to locate your work in the most appropriate empirical and
conceptual traditions.

4) Explanation of interdisciplinary and/or collaborative nature of project

Where applicable, applicants must clearly indicate the extent to which their proposed project is
collaborative, in both intellectual content and research roles, and to identify the roles and
contributions of each member of the research team. For projects that are interdisciplinary,
clearly articulate the actual nature of that interdisciplinary contribution. In doing so, applicants
should clearly indicate the manner and extent of the relevant and unique contribution of each
member of the research team and the contribution that each ‘discipline’ will make to the
research orientation and research tasks of the project. Applicants from the same disciplinary
background but in different departments and programs are not normally considered an
interdisciplinary team.

5) Plans for communication of research results

Outline your plans for communication of results both within and outside the academic community.
Reference to the information on Knowledge Mobilization on the SSHRC website http://www.sshrc-
crsh.gc.ca/home-accueil-eng.aspx (search ‘Knowledge Mobilization’) might assist you in deciding
how to frame this aspect of your research.

6) Budget justification

The budget justification should not be just a detailed itemization of the budget. For example, the
committee should be given an account of the duties of research assistants and these duties should
be reflected in the number of hours of their time requested. Similarly, equipment should be
justified – why do you need it, why is it not available to you as part of the research infrastructure,
how did you determine the anticipated cost, etc.

7) Research grant application history for applicants and co-applicants

Applicants should attach their research grant application history for the past six years. The history
should include all research grant applications whose status is still pending (i.e. competition results
are not yet known)

a.
The Katalyst Adjudication Committee will not consider an application that appears to be, in all
or in part, directly related to a proposed project currently under review elsewhere (e.g.
SSHRC). The decision to not consider a proposal for this reason will be made by the
committee during the adjudication process.

Applicants should attach the research abstracts from all research grant applications that have been
funded within the past six years, internal and external funding agencies. Applicants should also
include the abstracts from research grant application(s) whose status is still pending (competition
results are not yet known).

b.
If the abstracts are not available, applicants should provide a brief descriptive paragraph for
each of the research grant applications that have been funded within the past six years as well
as for research grant application(s) whose competition results are not yet known.




#### forms a new direction in the research career of the investigator

Intimidated by the form.
Research grant application history? None. Research abstracts? Nope. Research team? Just me. Knowledge Mobilization? Never heard of it (but the capitalization implies that I should have). But then, as I read more closely, I glimpse some signals of encouragement: _creative efforts_, _direction into new fields_, _interdisciplinary_.


My first impulse upon reading the Katalyst guidelines was to ...
Items: nope, nada, never.
But then, as I read more closely, I saw other terms -- here, here, here -- and I thought maybe, just maybe, this initiative would be perfect for me.


The form seems to focus on certain types of research that are different from my career experience.
What I don't have (all the stuff on the form), and what I do have: extensive career experience.
What I've done, and why it's research (but a different kind of research).
Coming to Kwantlen recently, wanting now to develop more formalized academic research.


I have met with excellent success in my previous academic research in creativity and culture. My doctoral research project won the Union Institute's Sussman Award for Academic Excellence (the university's highest academic award), was adapted into a best-selling book, and was short-listed for the Governor General's Award (the highest arts award in Canada). The sequel to the creativity project was a heuristic investigation of the creativity of mythology, which in turn led to my involvement with the Canada Council as jurist for the Governor General's Award, jurist for Canada Council Literary grants, member of the National Council of the Writer's Union of Canada, and my ongoing involvement with the Writers' Union of Canada in the capacity of professional development consultant. Overall, my research projects in creativity have turned out well and have led to many positive outcomes (including my arrival at Kwantlen).

I would like now to pursue research into a related but different area. For the past several years I have been providing consulting services to schools, universities, and other educational institutions on topics related to educational renewal. This work involves the interwoven themes of pedagogy, creativity, technology, culture, change management, and childhood development. I entered this area of work as a consultant building upon my 25 years of experience in social services (as a counsellor and clinical supervisor) and education (as a facilitator and professional development consultant). Until recently, I pursued this work outside the sphere of my academic interests; but in the last two years, as I have facilitated an increasing number of workshops for teachers and professors in an educational landscape increasingly fraught by questions of what to do about changes in technology and culture, I find myself in the timely position of knowing quite a bit already about a topic of great academic and public interest.

My plan is to conduct a heuristic research project into the lived experience of educators as they navigate the transformative changes underway in education. The emphasis of the project will be on the tension between traditional educational approaches and new educational opportunities: what to keep and what to discard, what works and what doesn't, what is lost and what is gained, and how best to make it all work. This project is [fourth quadrant][1] research which lends itself well to heuristic approaches.

Heuristic research involves the "search for the discovery of meaning and essence in significant human experience. It requires a subjective process of reflecting, exploring, sifting, and elucidating the nature of the phenomenon under investigation (Douglass and Moustakas, 1985, p.40). This collaborative research project will examine the inner lives and transformations of the participants (who will be our co-researchers) as well as the inner lives and transformations that the researchers encounter in themselves.

Part of the project will involve building and developing a digital collaboratory (such as has been done at [Hastac](http://hastac.org "Hastac") and Berkeley's [Townsend Humanities Lab](http://townsendlab.berkeley.edu/)). Co-researchers (educators and their students) will contribute to the collaboratory on a regular basis, will post insights and challenges, and will allow website visitors to track the development of the project in real time.

The project will culminate in open publishing (on the web), conference presentations, and a literary book for a general audience -- as my other books have been. I am very interested in open access to information and the use of scholarship to promote the public good. Accordingly, the results of this research will be published with the intent of providing useful information and insight to general readers, parents, educators, and students.

#### The grant especially encourages research that is interdisciplinary

I find myself in an unusual and interesting position at Kwantlen. I am a qualified faculty member in four academic areas (Interdisciplinary Expressive Arts, Creative Writing, Educational Studies, and Communications), a faculty professional development consultant (in Learning Technologies), and an administrator (in the Centre for Interdisciplinary Research). The accumulated momentum of these various positions, and the ways in which my various roles interact, has led to a situation in which I do not have a fixed home at Kwantlen but rather float across, between, and within many different disciplines. I am a homeless wanderer at Kwantlen. 

One of the many things I have discovered in my travels across the hinterland of Kwantlen is that many people (faculty, students, administrators) seek a more engaged, transformative, responsive, and digital future for their academic pursuits. In my role as a consultant for Learning Technologies I have facilitated a number of workshops on new possibilities in education, the evolution of digital collaboratories, the migration of knowledge toward digital formats, and the rise of digital publishing. At each of these workshops I have witnessed the strong motivation from faculty members, students, and administrators to reinvent what it means to teach and to learn, to leverage their knowledge into the online sphere, and to find serious, purposeful playfulness in the digital landscape.

The research project would involve faculty members, students, and administrators in a project to achieve their aims. Those members would come from various disciplines. I have already received strong indications of support from representatives of Criminology, Anthropology, and Sociology (in the Faculty of Social Sciences); from Modern Languages and Interdisciplinary Expressive Arts (in the Faculty of Humanities); from Marketing, Human Resources, and Communications (in the School of Business); from the Faculty of Design, from the Faculty of Trades and Technology, and from several departments from within the Faculty of Academic and Career Advancement. I have also received similar indications of interest and support from Kwantlen's Library, from the Office of Research and Scholarship, from Learning Technologies, from the bookstore, from the Counselling department, from the Kwantlen Student Association, and from various other non-academic units. The entire educational community -- not just at Kwantlen, but everywhere -- is grappling with the challenges and opportunities that lie before us. This project would involve participants from all of the above areas, and would be an interdisciplinary, community-based, non-aligned, non-partisan approach to addressing challenges and maximizing opportunities.


### The grants are meant for research projects that will enable the investigator(s) to establish a research record in a new area, enabling future external funding possibilities.

As indicated above, I have a strong track record in the research areas of creativity and culture. I have not published formally in the area of educational development (though I have been working professionally in this area for some years, and it was my main job at Kwantlen in 2010 and the first half of 2011). This project will enable me to expand my research profile into the educational area, with a view to further developing those research interests at the national level.

### Thus, proposals should detail how the Katalyst funded project will form the basis for future applications to provincial, national, and international funding agencies.

### Participation by undergraduate students in the proposed research is strongly encouraged, but their research responsibilities should be clearly explained.


the intent is to provide funding to enhance researchers’ competitive position when applying for external funding, and not to replace external funding. Thus, proposals that involve the development of new areas of research for the applicant or which prepare them to demonstrate a research competence when applying for external funding are favored by the Adjudication Committee over those which continue existing research endeavors.

GUIDELINES

1. Grants are awarded for projects that have clearly defined objectives and methods meeting the criteria established for the Katalyst Fund.

2. Awards are for a maximum period of two years.

3. Individual awards have a project minimum of $10,000 to a maximum of $40,000. Interdisciplinary collaborative projects are especially encouraged. However, it is emphasized that the Katalyst and HSS Grants do not cover ANY research expenses of research collaborators from institutions other than Kwantlen nor does it cover research expenses (including travel and accommodation) incurred by community partners. Katalyst and HSS Grant funds cannot be transferred to other institutions or community partners. 

When collaborative projects are being proposed, applicants are advised to clearly indicate in their project descriptions, the extent to which their proposed project is collaborative in both intellectual content and research roles, and to identify the roles and contributions of each member of the research team.

While interdisciplinary projects are especially encouraged, the actual nature of that interdisciplinary contribution must be clearly articulated. Applicants are advised to clearly indicate the manner and extent to which the project is interdisciplinary and the contribution that each ‘discipline’ will make to the research orientation and research tasks of the projects.

4. The Katalyst Grant award permits a maximum of two sections of Release Time, with a value of $12,500 for each. No Release Time is permitted for the HSS Grant. 
Other allowable expenses are detailed in the document entitled ‘Use of Grant Funds’,  located at http://www.kwantlen.ca/research/Frequently_Asked_Questions.html. This document is not intended to be all-inclusive and eligibility questions should be directed to the ORS for clarification. 

5. A Progress Report indicating what has been accomplished with the funding must be submitted to ORS within six months of the end of the term of the grant. This form is available on the ORS site http://www.kwantlen.bc.ca/research/forms_resources.html


DEADLINE

Completed applications must be submitted via the ORS website, on-line form, by 4:00 pm PST on June 30, 2011. If this date falls on a weekend or a statuary holiday the deadline will be extended to the next working day.  

COMMITTEE MEMBERS

Jason Dyer, Chair, Executive Director, Office of Research and Scholarship
Andrew Bartlett, English
Chris Burns, Library
Colin Green, History
Amandah Hoogbruin, Nursing
Kyle Matsuba, Psychology
Paul Richard, Environmental Projection Technology
Robin Russell, English Language Studies

[1]: [The fourth quadrant](http://www.edge.org/3rd_culture/taleb08/taleb08_index.html) describes an area of research in which sudden, unpredictable, and large-scale fluctuations tend to happen (but happen unpredictably). The best known example of this phenomenon is the market crash of 2007. Fourth quadrant (or _black swan_) situations typically involve a known and steady history of events (such as 200 years of industrial education), the sudden emergence of turbulent change (such as that brought to education by technology), and groups of people biased toward (or against) particular outcomes (such as educators and students and their divided views about technology in education). The proposed research project meets the criteria for fourth quadrant research, for which traditional statistical models are not well-suited.
