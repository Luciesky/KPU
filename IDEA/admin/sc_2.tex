%!TEX TS-program = xelatex
    %!TEX encoding = UTF-8 Unicode
%    \documentclass[10pt, letterpaper]{article}
\documentclass[letterpaper,10pt,headsepline]{scrreprt}
    \usepackage{fontspec} 
    \usepackage{placeins}
    \usepackage{multibbl}
    \usepackage{graphicx}
    \usepackage{hieroglf}
    \usepackage{txfonts}
    \usepackage{url}
    \usepackage{titling}
    \usepackage{geometry} 
    \geometry{letterpaper, textwidth=5.5in, textheight=8.5in, marginparsep=7pt, marginparwidth=.6in}
    %\setlength\parindent{0in}
    \defaultfontfeatures{Mapping=tex-text}
    \setromanfont [Ligatures={Common}, SmallCapsFont={ITC Officina Serif Std}, BoldFont={ITC Officina Serif Std Bold}, ItalicFont={ITC Officina Serif Std Book Italic}]{ITC Officina Serif Std Book}
    \setmonofont[Scale=0.8]{Lucida Sans Typewriter Std} 
    \setsansfont [Ligatures={Common}, SmallCapsFont={ITC Officina Sans Std}, BoldFont={ITC Officina Sans Std Bold}, ItalicFont={ITC Officina Sans Std Book Italic}]{ITC Officina Sans Std} 
\usepackage[ngerman,english]{babel}
\usepackage{scrpage2}
\usepackage{paralist}
\clubpenalty=6000
\widowpenalty=6000
\author{IDEA Steering Committee}
\title{Notes and Reflections}
\date{October 21, 2013}
\ohead{IDEA}
\chead{Notes and Updates}
\pagestyle{scrheadings}
\setcounter{secnumdepth}{-1}
%\contentsname{Contents}
\begin{document}
\begin{titlingpage}
\begin{center}
\maketitle
\end{center}
\end{titlingpage}
\tableofcontents
\newpage
\chapter{IDEA Steering Committee October 21, 2013}

This document is published under the MIT open source license and lives online at \\ \url{github.com/rosslaird/KPU/tree/master/IDEA/admin/sc_2.tex}.

\section{Updates}

\begin{itemize}
\itemsep1pt\parskip0pt\parsep0pt
\item
  Achievements and Recognition
\item
  Github
\item
  Notes from April 22, 2013
\item
  Work Study
\item
  The IDEA Way
\item
  Amazon
\item
  Funding opportunities
\item
  Credential development
\item
  Faculty development
\item
  Science course
\item
  Theatre and Performance Course
\item
  Course development process
\item
  Spring semester
\item
  Search committee
\item
  Institute Development
\item
  Wellness
\end{itemize}

\section{Achievements and Recognition}

\begin{itemize}
\itemsep1pt\parskip0pt\parsep0pt
\item
  Post-baccalaureate student experiences
\item Open source program development
\item
  SFU mentorship keynote
\item
  Canadian Federation of University Women presentation
\item
  BC Teacher Librarians keynote
\item
  IDEA session for Family Services of Greater Vancouver
\item
  IDEA session for Vancouver Coastal Health
\item
  IDEA in local secondary schools
\item
  Day of Drawing
\item
  Gateway Theatre collaboration
\end{itemize}

\section{Github}

All IDEA documentation now lives online at Github (\url{http://github.com/rosslaird/KPU}), the world's largest open source community. Transparency, openness, and community engagement are foundational to IDEA, and Github provides us with a way to fulfill these values in digital space. 

\section{Notes from April 22, 2013}

This was our first meeting. We began with sharing the words and objects
that each member had been asked to bring to the meeting as emblems of
what IDEA is and what it stands for. We then spent some time getting to
know one another, talking about our understanding of IDEA and its goals,
and contributing to the Wall: a collection of large-format pages, taped
to the four walls of the room and organized by theme or question. Each
physical wall of the room was assigned a thematic quadrant based on
where IDEA comes from, who we are now, what we are doing now, and where
we are going (north, east, south, west). Much of the meeting was spent
in small groups or individually, with discussions clustered around a
given page or quadrant. Committee members wrote questions and additions
on the pages based on the discussions.

We then created a collage based on what we envision as the IDEA
experience for learners.

We concluded with a general discussion about next steps, with an
emphasis on credential development.

\subsection{Our Words for IDEA}

organic, kinetic, community, integrative, journey, unique, active,
serious, fun, expressive, gestalt, prayerful, connective, playful,
experimental, story, culture, discover/y, exploration, open,
non-judgemental, spunk, longing, serious, whole, simple/complex,
understanding, introspection, progress, darkness, inhale, truth,
adaptive, generous, improvisational, hitch-hiking, audacity, attention,
connections, relationships, change

\subsection{Our Wall}

\subsubsection{The North: Where We Come From (Origins and Values)}

\paragraph{Origins}

\begin{itemize}
\itemsep1pt\parskip0pt\parsep0pt
\item
  Integrative Education (John Dewey, Antioch University, The Union
  Institute)
\item
  Contemplative Education (Mindfulness, Flow)
\item
  Creativity (Creative process philosophy, craft philosophy)
\item
  Inquiry Based Learning (McMaster, Faculty of Health Sciences)
\item
  Performance and Arts Inquiry (orchestra, improvisation, clown,
  autoethnogrophy, theatre)
\item
  Contemplative Education (mindfulness, embodiment)
\end{itemize}

\paragraph{Core Values}

Self-awareness, deep skill, empathy, community, character, mentorship

\paragraph{Expressions of Values}

\begin{itemize}
\itemsep1pt\parskip0pt\parsep0pt
\item
  Creative, integrative, and engaged teaching and learning for the whole
  person
\item
  Built by learners for learners
\item
  Mentorship, collaboration, and personal development
\item
  Humanistic
\item
  Learner initiative and passion
\item
  Practices that reach across and between disciplines
\item
  All arts and sciences as pathways of scholarship
\end{itemize}

\paragraph{Core of Mentorship}

\begin{itemize}
\itemsep1pt\parskip0pt\parsep0pt
\item
  Facilitating groups and collaborative learning
\item
  Resolving conflict and working in a group
\item
  Giving and receiving feedback
\item
  Asking and refining a question
\item
  Evaluating, finding and synthesizing resources/knowledge to answer a
  question
\item
  Self-regulation
\item
  Emotional intelligence
\item
  Time management
\item
  Mindfulness
\item
  Following through on a process
\item
  Embracing uncertainty, mystery and the unknown
\item
  Praxis: translating knowledge into action/practice and vice versa
\end{itemize}

\subsubsection{The East: Who We Are, Here and Now}

\paragraph{Who Else is Doing This?}

\begin{itemize}
\itemsep1pt\parskip0pt\parsep0pt
\item
  Quest (Motto: ``a university centered on you'')
\item
  Stanford d. School (Motto: ``innovation that combines creative and
  analytical approaches'')
\item
  Duke (``clusters of courses designed around an interdisciplinary
  theme'')
\item
  Harvard Artscience Lab (Motto: ``places of experience: we enter to
  explore'')
\item
  McMaster BHSc Program (4 year inquiry-based, student-centered program
  emphasizing personal development and collaboration)
\end{itemize}

\paragraph{Us}

Our Motto: Your life. Your learning. Your way.

\paragraph{Our Approaches}

\begin{itemize}
\itemsep1pt\parskip0pt\parsep0pt
\item
  The classroom as facilitation space
\item
  Instructors as colleagues and mentors
\item
  Student-built content within and beyond the classroom
\item
  Instructor role focused on contextualization, mentorship and unique
  wisdom
\end{itemize}

\paragraph{What Does it Look Like?}

\begin{itemize}
\itemsep1pt\parskip0pt\parsep0pt
\item
  Circle format, small groups, and excursions outside
\item
  Improvised maker spaces and creative spaces
\item
  Most class time facilitated by students (learners)
\item
  Emphasis on group work and self-reflection
\item
  Instructor facilitates process of learners facilitating one another
\item
  Instructor content delivery limited (10 minutes or less)
\item
  Instructor only delivers content not available in readings
  (e.g.~contextual knowledge)
\item
  Learners create self-directed projects
\item
  Evaluation includes self-assessment, peer assessment, interviews
\item
  No exams, no small assignments, no multiple choice tests
\item
  Group work represents 25-50\% of the grades
\item
  Individual, self-directed projects represent 50-75\% of grades
\end{itemize}

\paragraph{How Well Does it Work?}

\begin{itemize}
\itemsep1pt\parskip0pt\parsep0pt
\item
  Learners report feeling highly engaged and self-responsible
\item
  Learners report an overall increase in academic performance (across
  all subject categories)
\item
  Increased sense of community and belonging at Kwantlen
\item
  Greater ability to deal with inner conflict and group conflict and
  collaborate with diverse peers
\end{itemize}

\paragraph{Performance}

\begin{itemize}
\itemsep1pt\parskip0pt\parsep0pt
\item
  IDEA represents .004 of courses in Arts (\textless{}5\%)
\item
  IDEA represented, in 2012, .19 (\textasciitilde{}20\%) of the students
  on the Dean's Honour Roll (including the winner of the dean's medal:
  Tierney!)
\end{itemize}

\paragraph{Our Courses}

\begin{itemize}
\itemsep1pt\parskip0pt\parsep0pt
\item
  IDEA 1100: Foundations
\item
  IDEA 2100: Ecology, Creativity, and Nature Experience (in pipeline,
  collaboration with Sciences)
\item
  IDEA 3301: Myth, Culture, Creativity
\item
  IDEA 3302: Creativity and leadership
\item
  IDEA 3100: Creativity and personal development
\item
  IDEA 4100: Creativity and professional development
\end{itemize}

18 Credits total

\paragraph{In Development}

\begin{itemize}
\itemsep1pt\parskip0pt\parsep0pt
\item
  IDEA 3100 (Amazon): Amazon field school (collaboration with Faculty of
  Design)
\item
  IDEA 1100 (Aboriginal): aboriginal context focus (collaboration with
  Student Life and Princess Margaret secondary)
\end{itemize}

\paragraph{Our Service to KPU}

\begin{itemize}
\itemsep1pt\parskip0pt\parsep0pt
\item
  Foundations of Excellence
\item
  Scenario planning
\item
  Strategic planning
\item
  Student life orientations
\item
  Student life aboriginal initiatives
\item
  Mentorship programs
\item
  4th year students with IDEA experience
\end{itemize}

\paragraph{IDEA Scholarship}

\$500 per year (thanks to Christine Kwan)

This could be increased without too much sales-personship\ldots{}

\subsubsection{The South: Our Journey, Hurdles, and Challenges}

Central goal: credential plan

\paragraph{Questions and Considerations}

\begin{itemize}
\itemsep1pt\parskip0pt\parsep0pt
\item
  Does a minor exclude people?
\item
  Diploma, certificate, or both?
\item
  How many credits?
\item
  What to call it?
\item
  Course linkages with other areas?
\item
  Community partnerships?
\item
  Crossing borders ( with local schools, organizations, also with other
  provinces, countries -- including US)
\item
  Field work in our communities e.g.~digital storytelling, working with
  narratives of elders (cross-generational interviews)
\end{itemize}

\paragraph{The Minor: What to Call It?}

\begin{itemize}
\itemsep1pt\parskip0pt\parsep0pt
\item
  Creativity and Mentorship?
\item
  Interdisciplinary Arts?
\item
  Integrative Arts?
\item
  Integrated Arts?
\item
  Creativity and Integral Arts?
\item
  Interdisciplinary Creativity?
\item
  Integral Creativity?
\item
  Creativity, Mentorship, and Community?
\item
  Creative Studies?
\item
  Peace Studies?
\end{itemize}

\paragraph{New Course Proposals}

\begin{itemize}
\itemsep1pt\parskip0pt\parsep0pt
\item
  Web development
\item
  Meditation and mindfulness (consciousness studies)
\item
  Improvisation
\item
  Scaffold courses (on mentorship)
\item
  Embodiment
\item
  Peer Mentorship
\item
  Performative Inquiry
\item
  Sport and Play
\item
  Meaning
\end{itemize}

\paragraph{Issues, Ideas, Suggestions}

\begin{itemize}
\itemsep1pt\parskip0pt\parsep0pt
\item
  What should we think more about?
\item
  What hurdles require more attention?
\item
  Is there anything we are missing?
\item
  What about new faculty?
\end{itemize}

\subsubsection{The West: Our Destination}

\begin{itemize}
\itemsep1pt\parskip0pt\parsep0pt
\item
  A major
\item
  An institute
\item
  Cross institutional partners (beyond Kwantlen)
\item
  Field schools (more)
\item
  Publishing
\item
  Books, journals
\item
  Collective and individual initiatives
\item
  Old and new, digital media
\item
  Niche-casting
\item
  Continuing studies at KPU
\end{itemize}

\subsection{Our collage}

We created a collage based on ideas about the learner experience of IDEA. The image file can be downloaded from here:\\
\url{https://github.com/rosslaird/KPU/raw/master/IDEA/dev//IDEA_montage.jpg} 

\chapter{The IDEA Way}

\section{Purpose}

Learning for the Whole Person.\\
Your life. Your learning. Your way.

\section{Foundation}

\begin{itemize}
\itemsep1pt\parskip0pt\parsep0pt
\item
  Self-awareness (what we know about ourselves)
\item
  Empathy (how we work with and learn from others)
\item
  Character (how we use our self-awareness and empathy in the world)
\end{itemize}

\section{Structure}

\begin{itemize}
\itemsep1pt\parskip0pt\parsep0pt
\item
  The learning environment is built by the community of learners
  (including the instructor).
\item
  Self-awareness is built through creativity, play, and personal
  development.
\item
  Empathy is built through community engagement.
\item
  Character is built through collaboration and mentorship.
\end{itemize}

\section{Craft}

\begin{itemize}
\itemsep1pt\parskip0pt\parsep0pt
\item
  Creative, integrative, interdisciplinary and engaged teaching and
  learning for the whole person.
\end{itemize}

\section{Design}

\begin{itemize}
\itemsep1pt\parskip0pt\parsep0pt
\item
  Anything added dilutes everything else. (The program should be clear
  and simple.)
\item
  Practicality beats purity. (Skills are more important than theories.)
\item
  Half measures are as bad as nothing at all. (We should reach for what
  we want.)
\item
  Simple is better than complex. (Pathways and processes should be
  straightforward.)
\item
  Complex is better than complicated. (When complexities are required,
  they should be simple.)
\item
  Sparse is better than dense. (We prefer plain language over
  academic-speak.)
\item
  Flow is important. (Flow and meandering create purposeful paths.)
\item
  Humans make this. (Our emphasis is on relationship, interaction, and
  adaptability.)
\end{itemize}

\section{Skill}

\subsection{Self-awareness}

\begin{itemize}
\itemsep1pt\parskip0pt\parsep0pt
\item
  Developing a path of self-assessment, self-regulation, and
  self-reflection.
\item
  Following the search for knowledge and meaningful answers.
\item
  Finding useful solutions to complex problems.
\item
  Developing creativity through play and imagination.
\item
  Thinking for oneself.
\end{itemize}

\subsection{Empathy}

\begin{itemize}
\itemsep1pt\parskip0pt\parsep0pt
\item
  Opening oneself to empathy and compassion.
\item
  Communicating effectively in speaking, listening, writing and
  performing.
\item
  Building trust, emotional safety, and a culture of collaboration.
\item
  Being open to giving and receiving feedback and accepting others.
\item
  Embracing and resolving conflicts.
\end{itemize}

\subsection{Character}

\begin{itemize}
\itemsep1pt\parskip0pt\parsep0pt
\item
  Engaging in the reciprocal process of mentorship.
\item
  Joining and contributing to communities.
\item
  Modeling and teaching ethical practices.
\item
  Opening to, and learning from, other cultures.
\item
  Being a lifelong learner.
\end{itemize}

\chapter{Reflections from the Amazon Field School}

\begin{center}§\end{center}

This was a life-changing experience for me. I believe that Kwantlen
should have more of these classes which offer different learning
opportunities for students. Participating in this course was even more
interesting while being on location in the Amazon. This further enhanced
the experience by allowing students to get out of their comfort zones
and be exposed to new environments. As a result students were able to
tap into themselves and develop personal skills that may never be
developed in a regular classroom setting. I felt like I was working on
my development as a person and not just focusing on hard skills. I
believe that in traditional classroom settings students rarely get the
chance to focus on themselves and are not fully engaged. IDEA courses
allow for freedom to learn without too many restrictions. I would
recommend this course to anyone looking to experience something
life-changing and with much personal growth. Being a part of this course
while being in the Amazon rain forest is something I will never forget.

\begin{center}§\end{center}

Before making the decision to travel to Colombia as a representative of
Kwantlen Polytechnic University (KPU) I carried a burden in my heart
that prevented me from feeling completely satisfied with my experience
as a student. The routine I developed, reading books and attending
lectures, was no longer fulfilling so I shifted my focus toward finding
a way to be a part of something more, something great. The Amazon field
school was just that. Being a member of the Amazon field school program
not only provided an opportunity for me to be a part of a life altering
journey, it also changed how I view education. Furthermore, being a
member of the program offered more than an opportunity to learn about
Colombia and its culture, it also provided me a safe outlet to break
personal boundaries and increase my self-awareness. By participating in
the Amazon field school I acquired knowledge not only about Colombia but
also about myself and recommend that students who are searching for
direction (in their personal lives or in their studies) partake in the
Amazon adventure.

\begin{center}§\end{center}

The Amazon experience allowed me to flow like the river itself, in such
a form where each day was lived fully and intensely. It inspired me to
create more, and use the living spirit of the forest as my vessel of
expression. Getting long hours of sleep, waking up early and rested,
ready to encounter a new day. The air was constantly electrified by
sounds from nearby nocturnal wildlife. It was interesting to notice the
Yin and Yang in the forest, for which by day and night are were
completely different. I am both humbled and awed by all the generous,
intelligent and creative Colombians that we met along this journey. It
was a heart-guiding experience for most of us. The opportunity that
Kwantlen gave me to experience this is something I will be eternally
grateful for, and I encourage further Colombian and jungle field
schools. For me it was an extremely rich experience. I gained knowledge
on the biodiversity of Colombia, experienced the wild, thriving nature
for myself, and gained creative inspiration that will direct me on
multiple levels of my life and academics. I indulged in various local
Amazonian artistic media, while also learning about the politics and
social issues of Colombia as well as the ecology of the land. I came
back with a new drive and broadened perspectives. I came back with a new
layer of skin and soul that share the same storytelling nature as the
layered trees.

\begin{center}§\end{center}

The IDEA 3100 Amazon field school was a life experience that I will
never forget and one I will always brag about! IDEA has such a unique
approach to student learning, and I am so proud to be studying at a
University that has program like this. I have never, in my 6 years of
university, felt so involved in my learning. I thoroughly enjoyed every
aspect of this IDEA course. In IDEA you touch on multiple topics, and
experience multiple views and reactions. The course was very
interactive. Every student in IDEA 3100 wanted to be there, making the
learning process that much more interesting and rewarding. I feel like a
better-rounded student and person after just one IDEA course. I have
personally seen what a Science concentrated degree can do to you, and I
wish Simon Fraser had something like IDEA to keep me balanced when I was
studying for my B.Sc. Not all students are traditional learners and I
feel IDEA has an approach to education and learning that the world needs
to adopt; IDEA's teaching methods seem to fit more learning types. This
program is something to be proud of. Thank you to Ross and Lucie for
being the such passionate instructors. It's refreshing to have
instructors that are so dedicated and involved with student education
and learning.

\begin{center}§\end{center}

The Interdisciplinary Expressive Arts Amazon Field School was a great
experience. The instruction of this course embodies most learning
styles. This allows true learning to take place, each student at their
own speed. Personal reflection allowed me to garner a deeper
understanding of myself and the information I learned on this trip.
Current university structure puts all emphasis on memorization and
regurgitating information onto standardized tests. This fails to create
true intellectual development. The IDEA course structure of hands on
experience and self-reflection allows students to gain a deeper
understanding of the material ultimately accessing higher intellectual
thought. Everyone I told about my trip to Colombia was surprised to hear
that such an exotic trip was being offered at Kwantlen rather than UBC
or SFU. I assume this is because they still believe Kwantlen to be a
lesser institution. However, everyone I have told about the Amazon has
been really excited to experience the field school vicariously through
my photos and account of the trip. One thing that stood out to me after
a conversation with a friend was, ``Jeez, if my school had trips to
Colombia I woulda stayed in school''. I really believe this sums up the
benefit of having this Amazon Field School, it puts Kwantlen on the map.

\begin{center}§\end{center}

The Amazon field study was an ambitious endeavor to give students a
transformative experience, and Kwantlen's IDEA program was truly
successful at doing so. This adventure to Colombia is something that I
will never forget. When I started at Kwantlen I had a goal to further
educate myself, I never thought that my time at university would also
lead to such an amazing experience such as this. I thought University
would be a place and time for me to grow in a sense of education,
however this IDEA field study helped me to grow as a person, which I
believe in combination with my education I have received will help me to
be successful in my future after I graduate. It is because of the IDEA
program that I have become a more rounded student, with not just in
classroom experience but a more international perspective. Which is what
I believe will help me to go further in both studies and career goals.
Sometimes it takes straying from routines and pushing yourself past
where you thought your limits were to help further develop yourself.
This is what the IDEA program does; it gives you that push past the
routine and into the unknown. It is a program that allows for one to
further develop themselves in ways most traditional education doesn't.
It is the non-traditional ways of IDEA and the fact I got to go to the
Amazon Rainforest as a field trip that have made it one of my favorite
classes at Kwantlen.

\begin{center}§\end{center}

Kwantlen Polytechnic University was the first school in Canada to embark
on an adventure to the Colombian Amazon, and I was fortunate enough to
be one of its first participants. We hiked through the Amazon rainforest
with shaman guides, sometimes during rainstorms; we had an evening boat
ride down the Amazon with nothing but the stars and galaxies to light
our way; we met with professionals and locals to hear their tales of
social responsibility, global issues, and magic; and we witnessed
countless rainbows, sunsets, and thunderstorms, showcasing the diversity
and complexity that is the Amazon. I remember on the boat ride to
Calanoa there was an intense thunderstorm -- lightning bolts illuminated
the night sky as fireflies danced between the trees -- and all I was
thinking was: the adventure has officially begun. I am eternally
indebted to Ross, Lucie, Marlene, Diego, Carolina, Daniela, everyone at
Calanoa and Kwantlen Polytechnic University for providing this
transformative experience. I learned so much about myself, about
Colombia, and about the world around me. Ultimately, this was an
experience I will never, never forget. It was expensive, it was
terrifying, but it was exciting, necessary, and infinitely rewarding. I
will recommend this field school to whomever I can for as long as I can.

\begin{center}§\end{center}

The IDEA Amazon field school has reminded me how much I enjoy learning.
I have always found that I have had problems connecting to the material
in a traditional classroom setting. I have difficulty focusing during
lectures, as I know that my style of learning is a kinesthetic one. I
learn best when I am actively engaged, and generally rely on
trial-and-error to solve problems and understand material, (which is not
necessarily conducive to receiving high grades). While I do remember
material easily when participating in discussion, I am often one of only
a few members who are interested in speaking. Despite my relatively high
grades, I do not consider myself to be a good student. I procrastinate
often, rarely read assigned materials or texts, and draw pictures
instead of take notes. I calculate grades and percentages to determine
the minimum amount of work I can get away with doing and still maintain
my GPA. I do not blame the school for my general lack of motivation, as
the traditional system is something I simply do not fit into. IDEA does
not only encourage participation; it demands it. It demands 110\% in
everything that is done, and I was more than willing to accommodate that
demand. The student-directed approach to learning is one that really
appeals to me as well; I am much more willing to put effort into a
project that I devised, or devised within a group, than I am with an
assignment that is imposed upon me, or that I do not understand. I feel
as though my input is welcomed, and the learning environment is one that
is receptive, friendly, and encourages personal growth and development.
I love that every student gets a chance to speak. The concept of open
discussion is embraced, and IDEA is one of only a few classes where I
felt challenged to honestly and openly reflect on the material presented
to me. I will continue to take IDEA classes during my time at Kwantlen,
as I believe the skills I have developed this semester as an IDEA
student are skills that I will keep with me, and continue to use, for
the rest of my life.

\begin{center}§\end{center}

In Colombia, I went on an amazing journey -- and not just within the
trip itself but also an emotional roller coaster. I experienced so many
eye-opening things that I have difficulty putting it all into words. One
of things that I really wanted to do when I was on this trip was to go
into it without letting fear stop me. There was no doubt that I had
fears and anxieties about what I was getting into but I realized that
this was an opportunity of a lifetime and a chance to make as many
memories as possible. I wanted to do everything and experience
everything because I knew I would regret it if I didn't. I tackled my
fears while I was there and it has given me a new sense of courage now.
It was frightening enough just to decide to go to the Amazon in the
first place knowing that I would be in an truly unfamiliar environment.
I didn't know how I would feel while I was there. I felt that I pushed
myself to be brave and not worry about the outcome as much and I always
ended up realizing that it wasn't as bad as I thought it would be. I
surprised myself with how I reacted emotionally to the situations I was
put in. Even though overall I had anxieties there was something internal
pushing me to be brave. It was a feeling that I had never experienced
before.

\begin{center}§\end{center}

During this field school I learned that I don't know very much about the
world, but I know I really want to. Colombia made me realize that
university alone does not shape you into knowledgeable person.
Experiences such as this are essential to being a better rounded person.
IDEA exposed me to so many avenues of study that I was not aware of, and
many that I am genuinely interested in. This is the first course in my 6
years of university where the grade was not the focus of my study. I
believe that this is the reason why I fully engaged in the process and
the journey. Our group reflections helped me break out of my shell. They
were an important part of this process for me. I was shocked at the way
I was able to speak to the group. I was much more open than I am used to
being. It was not just about facing fear, it was about trusting others.
Something I feel I had lost. Now, I want to only associate myself with
things I believe in and things I am proud of. I want to see the world.
If Colombia alone has taught all this, I can't imagine what seeing the
world would do. Life is a journey, and I think that mine is finally
taking off. I want to have more of these experiences; they are priceless
lessons; an entirely different type of education. Thank you Ross and
Lucie for pushing for this field school.

\begin{center}§\end{center}

I now know a calm peace within myself like I have never known before.
Spending two weeks walking side by side the group, everyday has been the
catalyst for this. I realize now that I cannot move forward alone, I
must engage and be part of the greater whole in order to realize any
true and lasting meaning. Our group and the people of Colombia, the
Amazon and certainly the villages have shown me that it is possible to
be happier with less. Watching the children of the village on the
football field proved this to me. We are all connected. I can now feel
the connection after living it, it came from the Amazon and the people
-- it was not there before we left Canada.

\chapter{Credential Development}

\section{Minor in Mentorship and Community Engagement}

The IDEA minor is intended to embody the core values and practices of
KPU's emerging identity as articulated in the 2013 Academic Plan. The
plan asserts that KPU is

\begin{quote}
committed to providing enriched learning experiences that help prepare
students for global citizenship and success in life through fundamental
academic skills, inquiry-based curriculum, reflective scholarship and
student-faculty interaction, and the critical role of a supportive
academic community. Our goal is to attract students from across the
region and beyond because of the quality and relevance of the programs
we offer, and to develop the reputation for supporting them as lifelong
learners. This requires offering a broad suite of programs and
credentials (with interdisciplinary options) and flexible program
delivery models, ensuring ease of transfer across disciplines
(particularly in the 1st and 2nd years), and providing an appropriate
blend of content, tools and inspiration to enhance personal
capabilities. This approach is consistent with Nussbaum's (2006)
proposition that an education based on the idea of an inclusive global
citizenship, and on the possibilities of the compassionate imagination,
has the potential to transcend divisions created by distance, cultural
difference, and mistrust, and is one of the most exciting tasks we can
undertake as educators and citizens.
\end{quote}

\begin{quote}
A major characteristic of KPU's emerging hybrid polytechnic university
is reflected in how we are linking support for students' academic
success with opportunities for personal growth, enabling them to develop
skills to communicate effectively, to work well with people from
different backgrounds, and to think critically and creatively. KPU is
uniquely positioned and fully committed to guide and support students as
they develop the applied and academic learning that citizens in the 21st
century global society need to succeed.
\end{quote}

Approaches such as inquiry-based curriculum, reflective scholarship,
intensive interaction, interdisciplinarity, and flexibility are all
foundational to IDEA. Values such as the importance of global
citizenship and lifelong learning, the importance of inspiration, the
crucial role of compassion and imagination, and the underlying purpose
of personal growth are all core elements of the IDEA philosophy. Skills
such as effective communication, openness to diversity and inclusion,
and creative and critical thinking are all integral to the IDEA process.
The goals articulated for KPU are embodied by IDEA and are reflected in
our core values, skills, and curriculum.

Each course within the curriculum will focus on at least two skills from
each of the three IDEA core value areas (for a minimum of six core
skills for each course):

\subsection{Self-awareness}

\begin{itemize}
\itemsep1pt\parskip0pt\parsep0pt
\item
  Developing a path of self-assessment, self-regulation, and
  self-reflection.
\item
  Following the search for knowledge and meaningful answers.
\item
  Finding useful solutions to complex problems.
\item
  Developing creativity through play and imagination.
\item
  Thinking for oneself.
\end{itemize}

\subsection{Empathy}

\begin{itemize}
\itemsep1pt\parskip0pt\parsep0pt
\item
  Opening oneself to empathy and compassion.
\item
  Communicating effectively in speaking, listening, writing and
  performing.
\item
  Building trust, emotional safety, and a culture of collaboration.
\item
  Being open to giving and receiving feedback and accepting others.
\item
  Embracing and resolving conflicts.
\end{itemize}

\subsection{Character}

\begin{itemize}
\itemsep1pt\parskip0pt\parsep0pt
\item
  Engaging in the reciprocal process of mentorship.
\item
  Joining and contributing to communities.
\item
  Modeling and teaching ethical practices.
\item
  Opening to, and learning from, other cultures.
\item
  Being a lifelong learner.
\end{itemize}

\section{Core skill listing per course}

\subsection{IDEA 1100, Interdisciplinary Foundations}

\begin{itemize}
\itemsep1pt\parskip0pt\parsep0pt
\item
  Developing a path of self-assessment, self-regulation, and
  self-reflection.
\item
  Following the search for knowledge and meaningful answers.
\item
  Finding useful solutions to complex problems.
\item
  Thinking for oneself.
\item
  Communicating effectively in speaking, listening, writing and
  performing.
\item
  Building trust, emotional safety, and a culture of collaboration.
\end{itemize}

\subsection{IDEA 1200, Community Engagement (new course)}

\begin{itemize}
\itemsep1pt\parskip0pt\parsep0pt
\item
  Developing creativity through play and imagination.
\item
  Communicating effectively in speaking, listening, writing and
  performing.
\item
  Opening oneself to empathy and compassion.
\item
  Building trust, emotional safety, and a culture of collaboration.
\item
  Joining and contributing to communities.
\item
  Opening to, and learning from, other cultures.
\end{itemize}

\subsection{IDEA 2100, Creativity, Ecology, and Nature Experience}

\begin{itemize}
\itemsep1pt\parskip0pt\parsep0pt
\item
  Developing a path of self-assessment, self-regulation, and
  self-reflection.
\item
  Finding useful solutions to complex problems.
\item
  Developing creativity through play and imagination.
\item
  Opening oneself to empathy and compassion.
\item
  Joining and contributing to communities.
\item
  Being a lifelong learner.
\end{itemize}

\subsection{IDEA 1400, Community Performance and Theatre Exploration
(new course)}

\begin{itemize}
\itemsep1pt\parskip0pt\parsep0pt
\item
  Developing creativity through play and imagination.
\item
  Communicating effectively in speaking, listening, writing and
  performing.
\item
  Building trust, emotional safety, and a culture of collaboration.
\item
  Being open to giving and receiving feedback and accepting others.
\item
  Engaging in the reciprocal process of mentorship.
\item
  Joining and contributing to communities.
\end{itemize}

\subsection{IDEA 2300, Core Mentorship Skills (new course)}

\begin{itemize}
\itemsep1pt\parskip0pt\parsep0pt
\item
  Developing a path of self-assessment, self-regulation, and
  self-reflection.
\item
  Opening oneself to empathy and compassion.
\item
  Communicating effectively in speaking, listening, writing and
  performing.
\item
  Building trust, emotional safety, and a culture of collaboration.
\item
  Embracing and resolving conflicts.
\item
  Engaging in the reciprocal process of mentorship.
\item
  Modeling and teaching ethical practices.
\end{itemize}

\subsection{IDEA 3100, Interdisciplinary Creative Expression}

\begin{itemize}
\itemsep1pt\parskip0pt\parsep0pt
\item
  Developing a path of self-assessment, self-regulation, and
  self-reflection.
\item
  Developing creativity through play and imagination.
\item
  Communicating effectively in speaking, listening, writing and
  performing.
\item
  Building trust, emotional safety, and a culture of collaboration.
\item
  Joining and contributing to communities.
\item
  Being a lifelong learner.
\end{itemize}

\subsection{IDEA 3301, Myth, Culture, and Creativity}

\begin{itemize}
\itemsep1pt\parskip0pt\parsep0pt
\item
  Developing a path of self-assessment, self-regulation, and
  self-reflection.
\item
  Following the search for knowledge and meaningful answers.
\item
  Developing creativity through play and imagination.
\item
  Communicating effectively in speaking, listening, writing and
  performing.
\item
  Opening to, and learning from, other cultures.
\item
  Being a lifelong learner.
\end{itemize}

\subsection{IDEA 3302, Creativity and Mentorship in Groups}

\begin{itemize}
\itemsep1pt\parskip0pt\parsep0pt
\item
  Developing creativity through play and imagination.
\item
  Building trust, emotional safety, and a culture of collaboration.
\item
  Being open to giving and receiving feedback and accepting others.
\item
  Embracing and resolving conflicts.
\item
  Engaging in the reciprocal process of mentorship.
\item
  Modeling and teaching ethical practices.
\end{itemize}

\subsection{IDEA 3303, Interdisciplinary Field School (new course)}

\begin{itemize}
\itemsep1pt\parskip0pt\parsep0pt
\item
  Developing a path of self-assessment, self-regulation, and
  self-reflection.
\item
  Developing creativity through play and imagination.
\item
  Communicating effectively in speaking, listening, writing and
  performing.
\item
  Building trust, emotional safety, and a culture of collaboration.
\item
  Joining and contributing to communities.
\item
  Being a lifelong learner.
\end{itemize}

\subsection{IDEA 4100, Mentorship through Community Engagement}

\begin{itemize}
\itemsep1pt\parskip0pt\parsep0pt
\item
  Building trust, emotional safety, and a culture of collaboration.
\item
  Engaging in the reciprocal process of mentorship.
\item
  Joining and contributing to communities.
\item
  Modeling and teaching ethical practices.
\item
  Opening to, and learning from, other cultures.
\item
  Being a lifelong learner.
\end{itemize}

\section{Articulating the Core Skills}

Each course outline will either use the core skill statements directly
or will adapt them, as required, to the specific context of the course.
For example, in IDEA 1100 the core skill of \emph{developing a path of
self-assessment, self-regulation, and self-reflection} will be
articulated as the following three individual skills:

\begin{itemize}
\itemsep1pt\parskip0pt\parsep0pt
\item
  Practice mindfulness and self-reflection
\item
  Apply self-regulation skills in interpersonal dynamics
\item
  Apply self-assessment skills in the context of class projects and
  activities
\end{itemize}

\section{Developments and Adaptations}

\subsection{Unchanged Courses}

\begin{itemize}
\itemsep1pt\parskip0pt\parsep0pt
\item
  IDEA 1100, \emph{Interdisciplinary Foundations} (with mentorship
  program)
\item
  IDEA 2100, \emph{Creativity, Ecology, and Nature Experience}
\item
  IDEA 3301, \emph{Myth, Culture, and Creativity} (emphasis on
  community, culture, and the mentorship of myth)
\item
  IDEA 2100, \emph{Creativity, Ecology, and Nature Experience} (emphasis
  on mentorship of nature)
\end{itemize}

\subsection{Adapted and Developed Courses}

\begin{itemize}
\itemsep1pt\parskip0pt\parsep0pt
\item
  IDEA 1100 (mentors provided by practicum learners)
\item
  IDEA 3302 (emphasize mentorship language and change title to
  \emph{Creativity and Mentorship in Groups})
\item
  IDEA 3100 (emphasize mentorship language and change title to
  \emph{Interdisciplinary Creative Expression})
\item
  IDEA 4100 (emphasize mentorship and practicum language, and change
  title to \emph{Mentorship through Community Engagement})
\end{itemize}

\subsection{New courses}

\begin{itemize}
\itemsep1pt\parskip0pt\parsep0pt
\item
  IDEA 1200, \emph{Community Engagement} (student-built)
\item
  IDEA 2300, \emph{Core Mentorship Skills}
\item
  IDEA 1400, \emph{Community Performance and Theatre Exploration}
  (in-process)
\item
  IDEA 3303, \emph{Interdisciplinary Field School in the Amazon}
  (collaboration with Design)
\end{itemize}

\subsubsection{Total Courses: 10}

1100, 1200, 1400, 2100, 2300, 3100, 3301, 3302, 3303, 4100

\section{Post-baccalaureate Certificate Program}

\subsubsection{Mentorship and Community Engagement}

This program will provide an integrative and self-directed learning
environment for learners to develop mentorship skills they can use in
the workplace, in their communities, and in their families. The program
will be structured as a learning community in which individual learners
will build their own pathways based on how much they already know, how
they want to apply their skills, and how intensive they wish their
program to be.

There will be no set classes in this program. Instead, learners will
join a cohort during their participation in the initial one week
intensive. With the help of a program mentor, the cohort will plan
monthly meetings and activities based on their shared interests and
challenges. In consultation with a program mentor, each learner will
also develop an individualized plan that addresses how much skill
development the learner needs to accomplish during the program. There
will be no predetermined credit requirements, prerequisites, or learning
activities; the program will be different for each learner. Projects
will be based on skill demonstration and assessed using competency
criteria. There will be no grades. Instead, each learner who
successfully completes the program will receive a skills-based
transcript that shows the mentorship skills they have achieved and what
the learner can do with those skills.

The program will have four elements:

\begin{itemize}
\itemsep1pt\parskip0pt\parsep0pt
\item
  Initial one week intensive
\item
  Cohort activities (approximately monthly)
\item
  Individual projects (some of which will include other members of the
  cohort, courses from other programs, and similar customizations)
\item
  A practicum to demonstrate the skills learned
\end{itemize}

The program will emphasize the following characteristics:

\begin{itemize}
\itemsep1pt\parskip0pt\parsep0pt
\item
  Online component with digital portfolio and badges
\item
  Prior learning assessment and gap analysis
\item
  No fixed course requirements (depends on gap analysis)
\item
  Personalized pathways built through collaboration with faculty mentors
\item
  Competency-based transcript (competencies mapped to actual learning)
\item
  Competencies as core (as opposed to credits)
\item
  Courses defined by amount of learning (not hours)
\item
  No set course schedule or defined points for courses (individualized)
\item
  Self-paced program
\item
  Experiential learning
\item
  Broad acceptance of learning activities (independent study, learning
  communities, courses from elsewhere, etc.)
\end{itemize}

\chapter{Course Development Process}

\section{IDEA 1400, Community Performance and Theatre Exploration}

\subsection{Brief Description of Course}

Learners will explore performance and theatre within an expressive arts
context. They will develop performance skills and will present theatre
productions to various communities within and beyond KPU.

\subsection{Program Fit}

This course fulfills a long-standing initiative within IDEA to provide
curriculum specifically focused on expressive arts performance. The IDEA
steering committee has identified performance curriculum as a priority,
IDEA developers have been working toward performance curriculum for some
time, and learners have been requesting this kind of course since the
inception of IDEA.

There is no similar course at KPU.

No prerequisites will be required.

\subsection{Faculty and Institutional Fit}

This course utilizes specific modalities such as inquiry-based
curriculum, reflective scholarship, intensive interaction, and
interdisciplinarity. These approaches are all listed in the KPU Academic
Plan as being foundational priorities. Additionally, the Academic Plan
highlights the importance of values such as global citizenship,
inspiration, compassion, imagination, and personal growth. These values
are core elements of this course and of the IDEA philosophy in general.
The Academic Plan further emphasizes skills such as effective
communication, openness to diversity and inclusion, and creative and
critical thinking. Again, these are all integral to the course and to
the IDEA process. The broad goals articulated for KPU in the Academic
Plan are embodied by this course and are reflected in the core processes
and outcomes of the curriculum.

To illustrate the correlation between the course outcomes and the KPU
Academic Plan, we have emphasized (below) the specific areas in which
the learning outcomes for this course are aligned precisely with the KPU
Academic Plan.

\begin{itemize}
\item
  \emph{Explore and reflect upon} performance and theatre experiences as
  opportunities for \emph{developing the creative imagination, a
  critical sensibility, citizenship, and community engagement}
\item
  Utilize the context of \emph{mentorship} and the activities of
  expressive arts performance to achieve \emph{personal growth}
\item
  Interpret and apply \emph{diverse traditions} of performance and
  theatre
\item
  Develop the \emph{creative imagination} through performance,
  \emph{mentorship}, and \emph{community engagement}
\item
  Improve skills of \emph{communicating effectively} in speaking,
  listening, writing, and performing
\item
  \emph{Join and contribute to} an expressive arts \emph{community}
\item
  Create \emph{interdisciplinary creative projects} using strategies
  developed through exposure to a performance-based learning environment
\end{itemize}

We find similar correlations when we examine the KPU Strategic Plan. The
core vision of the Strategic Plan is as follows, with emphasis added
where the language matches the outcomes of the course under review:

\begin{itemize}
\item
  \emph{Inspiring} educators
\item
  All \emph{learners engaging in campus and community life}
\item
  \emph{Open and creative learning environments}
\item
  \emph{Relevant scholarship} and research
\item
  Authentic \emph{external and internal relationships}
\end{itemize}

The Strategic Plan goes on to describe the importance of distinctive
programming, innovative teaching and learning, experiential learning,
enriched student experience, and purposeful community engagement. We
have illustrated (below) how each of these core goals for KPU matches a
core outcome for this course.

\begin{itemize}
\item
  \emph{Distinctive programming} corresponds to the unique
  performance-based curriculum of the course.
\item
  \emph{Innovative teaching and learning} corresponds to the context of
  mentorship, creativity, and community engagement in the course.
\item
  \emph{Experiential learning} corresponds to the learning environment
  of IDEA, which is entirely experiential.
\item
  \emph{Enriched student experience} corresponds to the experiential
  learning environment focused on creativity, mentorship, engagement,
  and personal growth.
\item
  \emph{Purposeful community engagement} corresponds to performances in
  the community (by the class) and a community of learning constructed
  by the learners (in the class).
\end{itemize}

The Arts Academic Plan is aligned with the KPU Academic and Strategic
Plans, and we therefore see similar correlations in values, approaches,
and goals. For example, the Arts Academic Plan specifically emphasizes
goals such as the enhancement of experiential learning, the wider
adoption of interdisciplinary approaches, the importance of preparing
learners for global citizenship, and the crucial role of mentorship. All
of these broad goals are specifically articulated in the course outline.

There is a high degree of symmetry between the stated goals of KPU, on
many levels, and the articulated goals of this course. We believe that
this course embodies, in a very fundamental way, the kind of institution
that KPU has said it wants to be.

\subsection{Evidence of Demand}

Research about the demand for post-secondary theatre courses among
recent high school graduates has been conducted by one of our partners
in the Surrey School District. This data, which includes information
from all areas served by KPU (Surrey, Richmond, Delta, and Langley),
shows that there is strong interest in post-secondary theatre
curriculum. The majority of high school graduates who have taken theatre
courses in high school seek experiential theatre courses in their
post-secondary programs. However, at the moment, KPU does not provide
such curriculum. Accordingly, there is strong demand for this course
from incoming learners.

There is also strong demand from current learners at KPU. Every
semester, the IDEA program engages in consultation with learners about
program development (as IDEA is built by learners as well as faculty).
In each of the last three iterations of this consultation (over the last
three semesters), performance and theatre curriculum has been at the top
of the list of requested new courses. In addition, students in past
iterations of upper-level drama courses offered by KPU's English
Department have expressed a strong desire to take theatre courses at
KPU.

The IDEA steering committee has also highlighted performance and theatre
curriculum as priorities. Such curriculum fills a current gap in our
offering. The discipline of Expressive Arts utilizes a suite of
modalities that typically includes art, music, movement, writing,
storytelling, craft, nature experience, drama improvisation, and theatre
performance. At the moment, our curricular offerings include all of
these modalities except theatre performance.

\subsection{Consultation}

We have consulted with KPU's President, with the Dean of Arts, with our
internal IDEA faculty cohort, with the Surrey School District, with
colleagues among various departments at KPU, with colleagues teaching
high school theatre curriculum, and with learners at KPU. All of these
consultations have resulted in unequivocal support for the course.

Once the course is approved, we will meet further with high school
theatre teachers to forge a working relationship that will allow us to
perform for high schools. This initiative will be facilitated by the
fact that many of the KPU students enrolled in the course will have
taken theatre at local high schools. In addition, once we have developed
dramatic material appropriate for elementary school students, we will
approach elementary schools with the goal of performing for them.

We have begun consultation with the Gateway theatre about a
collaboration involving theatre curriculum (with very positive results),
and after the course is approved we will also consult with other
community theatres. Working relationships will inevitably follow, as
there is strong interest from local community theatres to collaborate
with post-secondary institutions.

\subsection{Cross- or Interdisciplinary Potential}

IDEA is, by definition and core purpose, an interdisciplinary
initiative. (The ``I'' in IDEA stands for Interdisciplinary.)
Accordingly, this course will be interdisciplinary. Moreover, the
instructor for this course will be developing relationships with faculty
in other departments with the goal of offering performances to enhance
the student experience in other fields. For example, History students
might benefit from performances involving historical events or dialogues
from history. English students might benefit from performances of scenes
from plays, stories, or entire plays they are studying in English
classes. Sociology students might benefit from performances involving
sociological situations. Anthropology students might benefit from
performances involving cultural contexts or situations. Business
students might benefit from performances and simulations of various
situations (interviews, meetings, presentations, informal business
situations, customer service, etc.) that can be dramatized effectively
and thus analyzed and understood more effectively by Business students.
In fact, a group of IDEA students could dramatize a given business
situation in alternative ways to illustrate the pros and cons of various
approaches to business.

This cross-disciplinary pollination can be extended to any field of
inquiry at KPU. In each case, the IDEA instructor will consult with
colleagues in such departments to maximize the integration of
performances with the curriculum of the other course (including the
alignment of assignments with performances, if desired).

Finally, the study of performance and theatre is inherently
interdisciplinary. Learners will work on and study plays, rituals,
theatres, and performances from various cultures and time periods, often
integrating different disciplinary approaches (e.g.~dramaturgical,
psychological, sociological, cultural, historical) as appropriate to the
demands of the given text to be performed and the anticipated
performance context (e.g.~location of performance and audience make-up).
The core approach --- in texts, activities, and performances --- is one
that addresses theatre from across the world and throughout history. A
global perspective, in other words

\subsection{Additional Considerations}

This is a bare-bones theatre and performance course. All we need is a
classroom to perform in and permission to perform in KPU common areas
(indoors and outdoors). Ideally we would also like to utilize the
Conference Centres (in Richmond and Surrey) as well as the Langley
theatre for special performances, but even this is not absolutely
required.

Some performances will be held off campus. These will be similar to
regular field trips and do not require special consultations in risk
management.

\end{document}
