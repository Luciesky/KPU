%!TEX TS-program = xelatex
    %!TEX encoding = UTF-8 Unicode
%    \documentclass[10pt, letterpaper]{article}
\documentclass[letterpaper,10pt,headsepline]{scrreprt}
    \usepackage{fontspec} 
    \usepackage{placeins}
    \usepackage{multibbl}
    \usepackage{graphicx}
    \usepackage{hieroglf}
    \usepackage{txfonts}
    \usepackage{url}
    \usepackage{titling}
    \usepackage{geometry} 
    \geometry{letterpaper, textwidth=5.5in, textheight=8.5in, marginparsep=7pt, marginparwidth=.6in}
    %\setlength\parindent{0in}
    \defaultfontfeatures{Mapping=tex-text}
    \setromanfont [Ligatures={Common}, SmallCapsFont={ITC Officina Serif Std}, BoldFont={ITC Officina Serif Std Bold}, ItalicFont={ITC Officina Serif Std Book Italic}]{ITC Officina Serif Std Book}
    \setmonofont[Scale=0.8]{Lucida Sans Typewriter Std} 
    \setsansfont [Ligatures={Common}, SmallCapsFont={ITC Officina Sans Std}, BoldFont={ITC Officina Sans Std Bold}, ItalicFont={ITC Officina Sans Std Book Italic}]{ITC Officina Sans Std} 
\usepackage[ngerman,english]{babel}
\usepackage{scrpage2}
\usepackage{paralist}
\clubpenalty=6000
\widowpenalty=6000
\author{Ross A. Laird, PhD}
\title{Interdisciplinary Expressive Arts 3100}
\date{\today}
\ohead{Ross Laird}
\chead{Interdisciplinary Expressive Arts 3100}
\pagestyle{scrheadings}
\setcounter{secnumdepth}{-1}

\begin{document}

\pagestyle{empty}
\vspace*{7em} 
\begin{center}
\huge{Interdisciplinary Expressive Arts 3100}\\
\vspace*{1em} 
\large{Course Outline}
\end{center}
\clearpage
\pagestyle{scrheadings}
\tableofcontents
\chapter{Interdisciplinary Expressive Arts 3100}
Instructor: Ross Laird, Ph.D.\\ 
Email: \url{ross@rosslaird.com}\\
Website: \url{www.rosslaird.com}\\
Office: D308, Surrey campus (by appointment)\\

\section{Course Overview}

Interdisciplinary Expressive Arts refers to a specific set of educational activities, goals and strategies. Based on innovative pedagogy and integrative approaches to learning, interdisciplinary studies involve the synthesis and synergy of various disciplines toward a cohesive, unified educational experience. Interdisciplinarity is much more than enrollment in courses from more than a single discipline. Authentic interdisciplinarity emphasizes the linkages between disciplines by focusing on contrasting and complementary aspects of diverse educational domains.

Interdisciplinary studies encourage students to develop broader intellectual skills, greater facility for critical thinking, and greater awareness of the social relevance of their education. Interdisciplinary students have the opportunity to develop exemplary skills in problem solving, insight, team-building, lateral thinking, and multi-modal learning styles. Interdisciplinary strategies involve approaching an issue or problem from various perspectives. This typically entails intellectual inquiries that range beyond the borders of any single discipline or domain. While still respecting the function of the boundaries between domains, interdisciplinary approaches recognize that those boundaries are essentially arbitrary and do not always serve the goals of learning. Global warming and the AIDS pandemic are two examples of contemporary issues that require interdisciplinary approaches. 

This course is about creativity, about making a claim for the fundamental right of intentional creative action. Within that context, we will explore the ancient and modern practices of creative endeavor (particularly as regards family and culture), the hurdles of creativity (as they involve craft and precision and clarity) and the great gifts we might receive from others of our creative kin (that is to say, the long tradition of writers, poets, sculptors, dancers, craftspeople of all stripes, musicians, myth-makers, and so on). Throughout this process, our guiding archetype will be that of the trickster.

In this course we stake out the territory of the creative, inspecting the geology of its forms and ideals, finding our own individual places to homestead. Creativity involves the search for truth, yet also an awareness that truth and fact are often provisional, and mythological; they are shapeshifters on the wide-open plain of creativity. We will explore what this means, and what to do about it.

And, finally, the goal of the course (from my point of view, at least), is to have fun: to preserve and nurture the creative and imaginative spirit that is the foundation of all the arts and sciences.

\section{Learning Goals}
\begin{itemize}

\item Read selected interdisciplinary texts and discuss their origin, development, and contemporary relevance
\item Interpret interdisciplinary literary traditions within the context of contemporary academic and professional inquiry
\item Articulate (verbally, in writing, and through creative endeavour) knowledge of interdisciplinarity as both an ancient and a current mode of inquiry
\item Describe the ways in which interdisciplinary expressive arts are more than a simple aggregation of various disciplines
\item Evaluate diverse interdisciplinary perspectives and approaches using heuristic modalities
\item Write a research-based essay on a topic related to interdisciplinary expressive arts
\item Create expressive projects and presentations using interdisciplinary principles

\end{itemize}
\clearpage

\section{Learning Experiences}

We use a rhizomatic approach in this course (and in all IDEA courses). In rhizomatic learning, the community is the curriculum. Dave Cormier, the leader in rhizomatic learning in Canada, describes the process as follows:

\begin{quote}
In the rhizomatic model of learning, curriculum is not driven by predefined inputs from experts; it is constructed and negotiated in real time by the contributions of those engaged in the learning process. This community acts as the curriculum, spontaneously shaping, constructing, and reconstructing itself and the subject of its learning... learners come from different contexts, they need different things, and presuming you know what those things are is like believing in magic... Organizing a conversation, a course, a meeting or anything else to be rhizomatic involves creating a context, maybe some boundaries, within which a conversation can grow... (\url{http://davecormier.com/})
\end{quote}

In rhizomatic learning -- which draws upon research into complex adaptive systems theory, cognitive science, anthropology, and hermeneutics, ``the knowledge lives in the community. You engage with it by probing into the community, sensing the response and then adjusting... It is a learning approach that is full of uncertainty, not least for the educator. But it's one that allows for the development of the literacies that will allow us to sharpen our ability to participate in complex decision making. Dealing with the uncertainty is what the learning is all about"  (\url{http://davecormier.com/}).

The course will include a variety of learning experiences contingent upon regular attendance and dedicated participation. Because creativity is an interactive process, much of the class time will be devoted to group experiential exercises, individual reflective tasks, collaborative endeavors, and practical assignments.

We will create a collaborative environment in this class. We are not going to cobble together the type of group one often hears about in the arts: competitive, cut-throat, critical. Repeat: we are not creating such a group. Instead, we will direct our efforts toward building upon the individual strengths of each participant, finding ways for each of us to be self-reflective in terms of assessing our creative work, discovering a means of protecting the quality and integrity of our writing. The creative spirit is remarkably persistent, yet it is also fragile, especially at its inception, and
we must be conscious of this fragility. Think about it: did you not experience, as a child, the strangulation of your creativity in school, by way of a culture of insensitive peers or teachers? Why do you think hardly anyone feels comfortable singing in public, or dancing, or drawing, or reading their written work to others? We have, most of us, been the victims of inappropriate feedback and judgment. We have to be careful about this, in our course, so that we do not harm one another.

\section{Readings}
\subsection{Required Course Texts}

\begin{description}
\item [Lehrer, Jonah.] \textit{Imagine: How Creativity Works}. Any edition.
\item [Hyde, Lewis.] \textit{The Gift: Imagination and the Erotic Life of Property}. Any edition.
\item [Pirsig, Robert.] \textit{Zen and the Art of Motorcycle Maintenance}. Any edition.
\end{description}

\subsection{Suggested Books}
\begin{description}
\item [Allen, Pat.] \textit{Art is a Way of Knowing.} \\Shambhala, 1995.
\item [Barron, F., Montouri, A., and Barron, A., eds.] \textit{Creators on Creating: Awakening and Cultivating the Imaginative Mind.} 
\\New York: Putnam, 1997.
\item [Butala, Sharon.] \textit{Wild Stone Heart}. \\HarperFestival,
  2000. \textsc{ISBN 000255397X}.
\item [Calvo, C\'esar.] \textit{The Three Halves of Ino Moxo}.
  \\Translated by Kenneth Symington. \\Inner Traditions, 1995.
  \textsc{ISBN 0892815191}.
\item [Campbell, Joseph.] \textit{The Mythic Image}.
  \\Princeton UP, 1974.
\item [Ellis, Normandi.] \textit{Dreams of Isis: A Woman's Spiritual
    Sojourn}.
  \\Quest, 1995.
\item [Hancock, Graham.] \textit{Heaven's Mirror: Quest for the Lost
    Civilization.}.
  \\Crown, 1998.
\item [Hedges, Chris.] \textit{War Is a Force that Gives Us Meaning}.
  \\Anchor, 2003. \textsc{ISBN 1400034639}.
\item [Kingston, Maxine Hong.] \textit{The Woman Warrior: Memoirs of a
    Girlhood Among Ghosts}. \\Vintage, 1989. \textsc{ISBN
    0072435194}.
\item [Kwan, Michael David.] \textit{Things that Must Not be
    Forgotten: A Childhood in Wartime China}. \\Soho Press
  \textsc{ISBN 1569472823}
\item [Laird, Ross A.] \textit{Grain of Truth: The Ancient Lessons of Craft}. \\MWR, 2000.
\item [Langewiesche, William.] \textit{American Ground: Unbuilding
    \\the World Trade Center}. \\North Point Press, 2002.
  \textsc{ISBN 0865475822}. (Also see \textit{Inside the Sky}.)
\item [London, Peter.] \textit{No More Secondhand Art.} 
\\Boston: Shambhala, 1989.
\item [Lopate, Phillip.] \textit{The Art of the Personal Essay: An
    Anthology from the Classical Era to the Present}. \\Anchor, 1997.
  \textsc{ISBN 038542339X}.
\item [Macfarlane, David.] \textit{The Danger Tree: Memory, War and
    the Search for a Family's Past}. \\Walker, 2001. \textsc{ISBN
    0802776167}.
\item[McNiff, S.] \textit{Art Heals: How Creativity Cures the Soul.} \\Boston: Shambhala Publications, 2004.
\item [Merwin, W.S.] \textit{The Mays of Ventadorn}. \\National
  Geographic Directions, 2002. \textsc{ISBN 0792265386}.
\item [Ondaatje, Michael.] \textit{Running in the Family}. \\Vintage,
  1993. \textsc{ISBN 0679746692}.
\item [Pirsig, Robert.] \textit{Zen and the Art of Motorcycle
    Maintenance}. \\HarperTorch, 2006 (reprint). \textsc{ISBN
    0060589469}.
\item [Saint-Exup\'ery, A.] \textit{Wind, Sand and Stars}. \\Harvest,
  2002. \textsc{ISBN 0156027496}.
\item [Sanders, Scott Russell.] \textit{Writing from the Center}.
  \\Indiana UP, 1997. \textsc{ISBN 0253211433}.
\item [Sullivan, William.] \textit{The Secret of the Incas: Myth,
    Astronomy, and the War Against Time.} \\Crown, 1996.
\item[Wilson, Frank.] \textit{The Hand: How Its Use Shapes the Brain, Language and Human Culture.}
\\New York: Vintage, 1998.

\end{description}

\subsection{Books on Creativity and Associated Philosophies}
\begin{description}
\item [Achebe, Chinua] \textit{Hopes and Impediments}. New York:
  Doubleday, 1989.
\item [Barron, F., ed] \textit{Creators on Creating: Awakening and
    Cultivating the Imaginative Mind}. New York: Putnam, 1997.
\item [Benjamin, Walter] \textit{Theses on the Philosophy of History}.
\item [Borges, Jorge Luis.] \textit{Collected Fictions}. \\Penguin,
  1999. \textsc{ISBN 0140286802}.
\item [Bohm, David] \textit{Wholeness and the Implicate Order}.
  London: Ark, 1980.
\item [Bohm, David] \textit{Unfolding Meaning}. New York: Routledge,
  1985.
\item [Bohm, David] \textit{On Creativity}. New York: Routledge, 1998.
\item [Bronowski, Jacob] \textit{Science and Human Values}. New York:
  Harper, 1956.
\item [---------] \textit{The Face of Violence}. London: Turnstile
  Press, 1964.
\item [---------] \textit{A Sense of the Future: Essays in Natural
    Philosophy}. Cambridge, MIT Press, 1977.
\item [Degler, Teri] \textit{The Fiery Muse: Creativity and the
    Spiritual Quest}. Toronto: Random House, 1996.
\item [Demos, John.] \textit{The Unredeemed Captive: A Family Story
    from Early America}. \\Vintage, 1995. \textsc{ISBN 0679759611}.
\item [Flack, Audrey] \textit{Art and Soul: Notes on Creating}. New
  York: Penguin, 1986.
\item [Franklin, Ursula] \textit{The Real World of Technology}.
  Toronto: Anansi, 1999.
\item [Fulford, Robert] \textit{The Triumph of Narrative: Storytelling
    in an Age of Mass Culture}. Toronto: Anansi, 1999.
\item [Goldberg, Natalie] \textit{Writing Down the Bones}. Boston:
  Shambhala, 1986.
\item [Herrigel, Eugen] \textit{Zen in the Art of Archery}. New York:
  Random House, 1977.
\item [Hildegard of Bingen] \textit{Secrets of God: Writings of
    Hildegard of Bingen}.\\ Boston: Shambhala, 1996.
\item [Hyde, Lewis] \textit{The Gift: Imagination and the Erotic Life
    of Property}. New York: Vintage, 1983.
\item [---------] \textit{Trickster Makes This World: Mischief, Myth,
    and Art}. New York: North Point Press, 1998.
\item [Jim\'enez, Juan Ramon] \textit{The Complete Perfectionist: A
    Poetics of Work}. Edited and translated by Christopher Maurer. New
  York: Doubleday, 1997.
\item [Jung, C.G] \textit{The Spirit in Man, Art and Literature}.
  Translated by R.F.C. Hull. Princeton: Princeton University Press,
  1998.
\item [London, Peter] \textit{No More Secondhand Art}. Boston:
  Shambhala, 1989.
\item [Lorca, Federico] \textit{In Search of Duende}. Translated by
  Christopher Maurer. New York: New Directions, 1998.
\item [Lyndon, Susan] \textit{The Knitting Sutra: Craft as a Spiritual
    Practice}. San Francisco: Harper, 1997.
\item [Needleman, Carla] \textit{The Work of Craft: An Inquiry Into
    the Nature of Crafts and Craftsmanship}. New York: Kodansha, 1979.
\item [Pye, David] \textit{The Nature and Art of Workmanship}.
  Cambridge: Cambridge UP, 1968.
\item [Richards, Mary] \textit{Centering in Pottery, Poetry and the
    Person}. Middletwon, CT: Wesleyan UP.
\item [Sarton, May] \textit{Journal of a Solitude}. New York: Norton,
  1973.
\item [Sennett, Richard] \textit{The Corrosion of Character: The
    Personal Consequences of Work in the New Capitalism}. New York:
  Norton, 1998.
\item [Thoreau, Henry David] \textit{Walden}. New York: Norton, 1985.
\item [Wilson, Frank] \textit{The Hand: How Its Use Shapes the Brain,
    Language and Human Culture}. New York: Vintage, 1998.
\end{description}
\newpage
\subsection{Mythological Fiction}
\begin{description}
\item [Joseph Conrad] \textit{Lord Jim, Heart of Darkness\/}
\item [Thomas Wharton] \textit{Salamander\/}
\item [Milan Kundera] \textit{Life is Elsewhere\/}
\item [Carlos Fuentes] \textit{The Orange Tree\/}
\item [Gabriel Garcia Marquez] \textit{One Hundred Years of Solitude\/}
\item [Jorge Luis Borges] \textit{Labyrinths\/}
\item [Alberto Manguel (editor)] \textit{Black Water: The Book of
    Fantastic Literature\/}
\item [Don DeLillo] \textit{Underworld\/}
\item [Somerset Maugham] \textit{The Razor's Edge\/}
\item [Philip K. Dick] \textit{The Man in the High Castle\/}
\item [Keri Hulme] \textit{The Bone People\/}
\item [Salman Rushdie] \textit{Midnight's Children\/}
\item [John Fowles] \textit{The Magus\/}
\item [Stephen King] \textit{The Stand\/}
\item [Philip Roth] \textit{Operation Shylock\/}
\item [Walter Miller] \textit{A Canticle for Leibowitz\/}
\end{description}

\clearpage

\section{Demonstration of Learning}

\subsection{Assignments}
Three projects and several presentations (see below) are required for this course. These assignments may be comprised of any type of art expression (writing, music, imagery, dance, movement, photography, etc.). The central idea is for you to choose a specific theme or thread and explore it in some depth. We will discuss these projects at length in class. They are opportunities for you to discover and explore creativity in your own life.

For philosophical reasons, I do not prescribe a particular length or structure for the projects. There is no upper limit on the length or complexity of the projects.

The three projects can be completed as separate projects or as one large project with three segments or stages. Either way, each project (or each segment) will include a written self-evaluation of at least 1,000 words. The self-evaluation will include answers to the following questions (answers need not be itemized): 

 \begin{itemize}
\item What research did you do to prepare for this project? Research might include readings, investigative interviews, online searches, self-reflections, ruminations, and many other modalities. 

\item What learning resources did you use? These might include books, articles, online resources, and so on.

\item Of the research and readings you undertook, what impressed you as being most interesting or relevant?

\item What kinds of experiments did you undertake with this project? What did you build, write, craft, or try? How did you spend your time, and how did it go? (Look at the criteria on the next page.)

\item What went well, where did you struggle, and how do you feel about the process you undertook during this project?

\item What were the best and worst moments of this project? What did you learn from these moments?

\item Are you proud of this project? Is it your best work? What grade would you give yourself?

\item How might you have improved this project, of your experience of it?

\item What did you learn about interdisciplinarity while working on this project?

\item What did you learn about yourself while working on this project?

\item What will you remember about this project in five years?

\item How does what you learned apply to your studies at Kwantlen and to your sense of your future direction?

\item What advice would you give to others who might be undertaking a similar project?

\item What did this project mean to you? What might it mean for others?

\item Do you plan to continue this project further, or to work on similar projects in the future?

\item You learned something crucial in this project which you won't discover for a while. Make a guess now about what that might be.

\end{itemize}

The three individual projects are worth 25 percent each.

\subsubsection{Assessment Criteria for Creative Projects}

Projects for this course are focused on creativity. Accordingly, the following criteria -- which are based on the philosophy of creativity  --  are used to evaluate engagement and commitment to the projects:

\begin{itemize}
\item Willingness to take appropriate risks and to challenge oneself creatively.
\item Willingness to try new things, especially when doing so provokes creative discomfort.
\item Openness to personal and interpersonal process.
\item Willingness to collaborate with others.
\item Consideration of and responsiveness to others.
\item Willingness to examine personal values, beliefs, and judgments.
\item Ability to take personal responsibility and initiative for learning.
\item Willingness to approach creativity as a skill with discrete steps and standards.
\item Commitment to improvement in writing and other creative projects.
\item Ability to be open and responsive to appropriate feedback.
\end{itemize}


\subsection{Group Presentations}

Each student will be a member of several different peer groups; each
peer group will present at least one mini-presentation (roughly thirty minutes each) on various topics. Each class session after the second will involve presentations, with one presentation from each group. Class time will be given for preparing the presentations. The structure and content of the presentations will be discussed in class.

\subsubsection{Presentation Methods and Goals}

The central idea of the presentations for this course is to give you
 opportunities to practice interdisciplinary thinking and expression.
 As such, the presentation should be interdisciplinary. Essentially,
 this means that you should try to use multiple presentation
 strategies and modalities. These might include (but are certainly not
 limited to) any of the following:

 \begin{itemize}
 \item Storytelling
 \item Poetry
 \item Music (playing)
 \item Drumming
 \item Singing
 \item Dance
 \item Movement
 \item Sport
 \item Ritual
 \item Film (showing)
 \item Film making
 \item Photography
 \item Web content
 \item Craft work
 \item Art making
 \item Individual reflection
 \item Meditation
 \item Health practices
 \item Creative process (any type)
 \item Group communication
 \item Cultural practices
 \item Nature experiences
 \end{itemize}

 Whenever possible (and workable), try to mix together multiple
 modalities into a single presentation. For example, you might ask the
 group to do some individual reflection using the modality of poetry,
 then create a series of movements based on the poetry, then work in
 small groups to talk about and share the process. Many configurations
 are possible. The trick is to choose an activity that you enjoy, then
 find a way to apply it to the content (suggested presentation topics
 are listed below). Please do not create your presentations using only
 written and/or spoken materials. In other words, don't just stand up
 at the front of the class and talk about the presentation topic.
 Utilize the energy of the group. Remember that in interdisciplinary
 work divergences are valued as unique opportunities. So, feel free to
 experiment with activities and modalities that may not seem, on the
 surface, to be related to the topic at hand but which might, upon
 experiment, yield surprising connections and results. Be playful.
 Allow yourself to laugh at yourself, to be embarrassed, to engage
 with the process in novel and interesting ways.

In interdisciplinary work, riddles and puzzles are highly prized.
Accordingly, the presentations should (ideally) not be complete
explanations or presentations of material. Feel free to play with
challenging exercises, with impossible scenarios, and other conundra.
One way to think about this is to consider insoluble riddles, such as
the one in \textit{Alice in Wonderland}: Why is a raven like a
writing desk?

\begin{quote}
  ``Have you guessed the riddle yet?'' the Hatter said, turning to Alice again.\\
``No, I give it up,'' Alice replied. ``What's the answer?''\\
``I haven't the slightest idea,'' said the Hatter.\\
``Nor I,'' said the March Hare.\\
Alice sighed wearily. ``I think you might do something better with the time,'' she said, ``than wasting it in asking riddles that have no answers.''
\end{quote}

The best interdisciplinary topics offer more questions than answers.
They, are essentially, gateways into the mysterious--which, as
Einstein will tell you, is an important place to be:

\begin{quote}
  The most beautiful thing we can experience is the mysterious. It is
  the source of all true art and science.
\end{quote}

\subsubsection{Suggested Topics for Interdisciplinary Presentations}

\begin{itemize}
\item Akhenaten and the invention of monotheism
\item Albrecht Durer and alchemy
\item Aristotle's book of comedy
\item Bill Evans and the Peace Piece
\item Buckminster Fuller and the geodesic
\item Chenrizi and the politics of China
\item Chuang Tzu and the butterfly
\item Coleridge and the person from Porlock
\item Csikszentmihalyi and the flow experience
\item David Bohm's Implicate Order
\item Darwin, the bassoon, and the sundew
\item Eugen Herrigel and the practice of archery
\item Francis Yates and the \textit{Art of Memory}
\item Freud, Jung, and the ``bosh'' incident
\item Fulcanelli and \textit{Mysteries of the Cathedrals}
\item Giordano Bruno and the Hermetic tradition
\item Godel's uncertainty principle
\item Hanna Arendt at Nuremberg
\item Henri Rousseau in the jungle
\item Howard Carter and ``wonderful things''
\item Jacob Bronowski at Auschwitz
\item Jacob Bronowski, Nagasaki, and \textit{Science and Human Values}
\item Jan Tschichold and the Nazis
\item John Cage on the subway with the \textit{\textit{I Ching}}
\item Kepler's \textit{Somnium}
\item Mary Shelley and the genesis of \textit{Frankenstein}
\item Newton's \textit{Principia}
\item Nikola Tesla and universal energy
\item Philip K. Dick, VALIS, and 2-3-74
\item Picasso, Guernica, and Expo 1937
\item R.D. Laing and madness as reality
\item Ramanujan's notebooks
\item Richard Feynman and the invention of quantum mechanics
\item Schwaller de Lubicz at Karnak
\item Simone Weil and leading from desire
\item St. Exupery flying into the desert
\item The Reimann Hypothesis
\item The Voynich Manuscript
\item The visions of Hildegard of Bingen
\item Thoth's legacy
\item Walter Benajmin and the Angel of History
\item Wendell Berry going \textit{Into the Woods}
\item Wilhelm Reich's Cloudbuster
\item William Blake's \textit{Marriage of Heaven and Hell}

\end{itemize}


\subsubsection{Assessment Criteria for Group Presentations and Overall Engagement}

This course utilizes experiential learning approaches, which depend upon student involvement and active participation. Accordingly, the following criteria are used to evaluate overall participation and engagement in the group presentations and the class:

\begin{itemize}
\item Willingness to take appropriate risks and to challenge oneself.
\item Willingness to speak up and to lead.
\item Openness to interpersonal process.
\item Willingness to collaborate with others.
\item Consideration of and responsiveness to others.
\item Commitment to enhancing the interpersonal experience of everyone in the group.
\item Willingness to examine personal values, beliefs, and judgments.
\item Ability to take personal responsibility for learning.
\item Willingness to deal with conflicts appropriately if and when they arise.
\item Ability to be open and responsive to appropriate feedback.
 
\end{itemize}


The group presentations and overall course engagement are worth a total of 25 per cent of your grade.

\subsection{Attendance and Participation}
The expectation is that you will attend all sessions and involve yourself in the class process. Your willingness to engage creatively with the learning process, to take appropriate personal risks, and to participate in group activities are all central to your involvement in this class. Because developing a style of creativity is very much a process of blending your own personal awareness with skills and practical techniques, your own emotional involvement in the class is as important as your academic knowledge of the material.

Creativity is a unique process. Unlike many other fields, in which competence and skill may be measured objectively, using replicable and consistent means (tests of factual knowledge, for example), authentic creativity depends greatly on the interpersonal skills of the practitioner. Computer programmers can be assessed by their ability to write code; chiropractors can be evaluated based on their skill at manipulating the human skeleton; race car drivers can be clocked around a track. But for writers and artists there are no such fixed measures. The interpersonal skills upon which creativity so much depends are subtle, difficult to quantify, and complex beyond any measurement scheme.

And yet we can identify those who possess exemplary personal and creative skills. They are relaxed, open, responsive, kind. Often they exhibit skills that we tend to assign to the social sphere: personal warmth, consideration of others, hesitancy to judge, sensitivity to emotions. In our class we focus on these interpersonal factors as a foundation for our experiences with one another. And we itemize them as features along the continuum of self-awareness:

\begin{itemize}
\item Commitment to the development of self-awareness.
\item Openness to interpersonal process.
\item Ability to participate in appropriate self-disclosure.
\item Consideration of and responsiveness to others.
\item Commitment to enhancing the interpersonal experience of everyone in the class.
\item Willingness to examine personal values, beliefs, and judgments.
\item Ability to take personal responsibility for learning.
\item Willingness to deal with conflicts appropriately if and when they arise.
\item Ability to be open and responsive to appropriate feedback.
\end{itemize}

Each item on the above list is an aspect of the first item: self-awareness. The most foundational skill in creativity is self-awareness. Those who develop this skill consistently query their own responses, thoughts, and feelings. They ask themselves:

\begin{itemize}
\item What am I feeling right now?
\item What am I thinking right now?
\item Why am I reacting in this particular way?
\item What do my thoughts, feelings, and reactions tell me about myself?
\item Is there anything about my current behavior that suggests unresolved themes in my life?
\item Is my perception of myself consistent with what other people tell me about the kind of person I am?
\item When and how do I get stuck, and what am I doing to work on this?
\item In what ways do I get overwhelmed, or shut down, or avoid?
 
\end{itemize}

These questions, and many others, require the capacity for self-reflection and self-awareness. As we continue in the course, you may wish to consider these questions as they apply to you. At the very least, you might wish to consider what you are currently working on in your life, in which direction your attention is drawn, into which of the innumerable themes of human nature you are now called to delve.

In my role as your instructor, I will be paying attention to how thoughtful you are in examining and responding to questions like those in the lists above. I will not be analyzing you, but rather noticing what kinds of things you do, what your reactions are to various situations. My goal in observing your behaviors and interacting with you is to assist you in developing greater self-awareness (and, by extension, greater creativity).

I will use the lists above, as well as the assessment criteria listed for each assignment, to assess your overall participation in the course. I will not be evaluating your level of self-awareness but rather your openness to the process of developing your self-awareness.

\subsection{Grade Inflation}
Almost every semester there are students who do well on the assignments, complete all the associated learning goals of the course,
participate well, and wonder why they do not receive a grade of one hundred percent (or 98, anyway). Here is the reason: almost every
semester there are students who demonstrates a level of commitment that goes beyond the course requirement. Such students complete extra
work, or hand in exemplary assignments, or undertake a significant amount of personal development in addition to the course expectations.
Such students typically receive the highest grades.

If you do reasonably well in the course you will receive a reasonable grade. Very high grades are intended for extra or exemplary work.
Unfortunately, over the past thirty years the post-secondary educational system in North America has participated in a process of
grade inflation. Since the 1980's, the average grade for typical course work has been increasing by about 25 per cent each decade.
Elevated assessments do not accurately reflect the work of most students. Even worse, grade inflation has caused many students to
expect high grades for average work. I am not a overly stringent assessor; but I will not inflate grades artificially.

In this course a small number of students will (likely) receive high grades; most students will receive grades in the middle range; and a few students will struggle with lower grades. If you are uncertain about your assessment for a given assignment, or if you wish to know where, roughly, you are along the distribution curve of the class, or if you would like suggestions for how to improve your grade, please ask me for clarification.

If you wish to achieve a good grade, please do the following:

\begin{itemize}
\item Show up for class -- every class. This course depends on student engagement. (This becomes especially important during the final weeks of the semester.)
\item Be attentive and mindful to the various criteria listed for each of the projects and the course overall.
\item Take the initiative to plan and develop your projects and presentations. This course is (very likely) more fluid and spontaneous than you are used to. Your ability to manage your time, commitment, and energy is crucial.
\item Speak up in every class (review the criteria for group engagement and presentations).
\item Don't look for the right answer to a question or challenge. Instead, find the answer that is meaningful to you.
\item Ask for help if you need it.
\item Commit to your projects in a substantial way. Good projects take time. Rushed projects are obviously rushed.
\end{itemize}

Finally, please be attentive to the Kwantlen policies on academic honesty and plagiarism, which can be found at the following URLs:

\noindent
Academic Honesty: \url{http://www.kwantlen.ca/__shared/assets/Honesty1432.pdf}\\
Plagiarism and Cheating: \url{http://www.kwantlen.ca/policies/C-LearnerSupport/c08.pdf}

\section{Due Dates}

Group presentation dates will be assigned in class \\(and will fall between weeks three and thirteen).
\noindent\\
Project one (or segment one) is due at the end of week four \\(by midnight on Sunday of that week).
\noindent\\
Project two (or segment two) is due at the end of week eight \\(by midnight on Sunday of that week).
\noindent\\
Project three (or segment three) is due at the end of week twelve \\(by midnight on Sunday of that week).

\clearpage

\section{Class Schedule}
The class sessions will be balanced between presentations (by the instructor and students) academic material, group collaboration, and composition. The content for each session will evolve as the semester progresses. We will cover the following themes (though, perhaps not in the order listed below):
\\
\begin{compactdesc}

\item[The Nature of Interdisciplinarity]
A vignette: Bronowski at Nagasaki.
Definitions and principles.
Clarifications of common misunderstandings about interdisciplinary approaches and practices.
An introduction to interdisciplinarity as a mode of inquiry in the arts and sciences.
\\
\item[Traditions and Practices]
Consideration of interdisciplinary practices as effective vehicles for the transmission of  sacred, social, political, artistic, and scientific information.
Examination of interdisciplinarity as a fundamental and necessary function of human nature and inquiry.
\\
\item[Developments and Milestones]
Consideration of the development of interdisciplinarity and its role in the contemporary world.
Examination of the relationship between interdisciplinary practices and traditional, faculty-based inquiry in the academic environment.
\\	
\item[Themes and Philosophies]
Introduction to the historical background of interdisciplinary philosophy and practice in science and literature.
Explication of ancient world views, with particular emphasis on spirituality, science, mythological concepts, and approaches to the imagination.
\\
\item[Hermetic Threads]
Reading of excerpts from core Egyptian, alchemical, and early scientific texts, with particular emphasis on foundational interdisciplinary ideas.
Examination of the ways in which the sciences and the arts were entwined in the practices and perspectives of all peoples until the twentieth century.
Exploration of the transmission of interdisciplinary ideas into the contemporary world.
\\
\item[Interdisciplinarity in the Expressive Arts]
Exploration of the transmission of interdisciplinary philosophies and practices by way of the expressive arts.
Examination of practices and structures within the interdisciplinary expressive arts (writing, storytelling, dance, movement, ritual, religious practice, music, philosophy, etc.), with emphasis on the traditions of psychology and mythology.
Reading of selected texts within the interdisciplinary expressive arts.
Examination of the ways in which texts within the interdisciplinary expressive arts have influenced the development of the arts and sciences.
\\
\item[Multicultural Principles]
Introduction to the interdisciplinarit of Asian mythology, literature, and science.
Explication of ancient Asian world views, with particular emphasis on interdisciplinary spirituality, mythological concepts, and approaches to knowledge.
Reading of excerpts from Asian texts, with particular emphasis on the synthesis of interdisciplinary expressive arts within those traditions.
\\
\item[Other Stuff (that we'll think up as we go along)]
Much of the course content will be generated by students.

\end{compactdesc}

\end{document}

