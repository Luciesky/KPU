\documentclass[letterpaper,10pt,headsepline]{scrreprt}
\usepackage[latin1]{inputenc}
\usepackage[ngerman]{babel}
\usepackage[T1]{fontenc}
\usepackage[dvips]{graphicx}
\usepackage{mathpazo}
\usepackage{microtype}
\usepackage{scrpage2}
\usepackage{paralist}
\clubpenalty=6000
\widowpenalty=6000
\author{Ross Laird}
\title{Interdisciplinary Arts 3100}
\date{09/09/08}
\ohead{Interdisciplinary Expressive Arts 3100}
\pagestyle{scrheadings}
\setcounter{secnumdepth}{-1}
\begin{document}
\section{Interdisciplinary Arts 3100}
Instructor: Ross Laird, Ph.D.\\ 
Telephone: 604-916-1675\\
Email: ross@rosslaird.info\\
Website: www.rosslaird.info\\
Location: Kwantlen University College, Surrey Campus\\
Schedule: Thursdays, 11:00-12:50\\
September 4 to December 8, 2008\\

\subsection{Basic Philosophy of the Course}

Interdisciplinary Expressive Arts refers to a specific set of educational activities, goals and strategies. Based on innovative pedagogy and integrative approaches to learning, interdisciplinary studies involve the synthesis and synergy of various disciplines toward a cohesive, unified educational experience. Interdisciplinarity is much more than enrollment in courses from more than a single discipline. Authentic interdisciplinarity emphasizes the linkages between disciplines by focusing on contrasting and complementary aspects of diverse educational domains.

Interdisciplinary studies encourage students to develop broader intellectual skills, greater facility for critical thinking, and greater awareness of the social relevance of their education. Interdisciplinary students have the opportunity to develop exemplary skills in problem solving, insight, team-building, lateral thinking, and multi-modal learning styles. Interdisciplinary strategies involve approaching an issue or problem from various perspectives. This typically entails intellectual inquiries that range beyond the borders of any single discipline or domain. While still respecting the function of the boundaries between domains, interdisciplinary approaches recognize that those boundaries are essentially arbitrary and do not always serve the goals of learning. Global warming and the AIDS pandemic are two examples of contemporary issues that require interdisciplinary approaches. 

This course is about creativity, about making a claim for the fundamental right of intentional creative action. Within that context, we will explore the ancient and modern practices of creative endeavor (particularly as regards family and culture), the hurdles of creativity (as they involve craft and precision and clarity) and the great gifts we might receive from others of our creative kin (that is to say, the long tradition of writers, poets, sculptors, dancers, craftspeople of all stripes, musicians, myth-makers, and so on). Throughout this process, our guiding archetype will be that of the trickster.

In this course we stake out the territory of the creative, inspecting the geology of its forms and ideals, finding our own individual places to homestead. Creativity involves the search for truth, yet also an awareness that truth and fact are often provisional, and mythological; they are shapeshifters on the wide-open plain of creativity. We will explore what this means, and what to do about it.

And, finally, the goal of the course (from my point of view, at least), is to have fun: to preserve and nurture the creative and imaginative spirit that is the foundation of all the arts and sciences.

\subsection{Learning Goals}
\begin{itemize}

\item Read selected interdisciplinary texts and discuss their origin, development, and contemporary relevance
\item Interpret interdisciplinary literary traditions within the context of contemporary academic and professional inquiry
\item Articulate (verbally, in writing, and through creative endeavour) knowledge of interdisciplinarity as both an ancient and a current mode of inquiry
\item Describe the ways in which interdisciplinary expressive arts are more than a simple aggregation of various disciplines
\item Evaluate diverse interdisciplinary perspectives and approaches using heuristic modalities
\item Write a research-based essay on a topic related to interdisciplinary expressive arts
\item Create expressive projects and presentations using interdisciplinary principles

\end{itemize}
\clearpage

\subsection{Learning Experiences}
The course will include a variety of learning experiences contingent upon regular attendance and dedicated participation. Because creativity is an interactive process, much of the class time will be devoted to group experiential exercises, individual reflective tasks, collaborative endeavors, and practical assignments.

We will create a collaborative environment in this class. We are not going to cobble together the type of group one often hears about in the arts: competitive, cut-throat, critical. Repeat: we are not creating such a group. Instead, we will direct our efforts toward building upon the individual strengths of each participant, finding ways for each of us to be self-reflective in terms of assessing our creative work, discovering a means of protecting the quality and integrity of our writing. The creative spirit is remarkably persistent, yet it is also fragile, especially at its inception, and
we must be conscious of this fragility. Think about it: did you not experience, as a child, the strangulation of your creativity in school, by way of a culture of insensitive peers or teachers? Why do you think hardly anyone feels comfortable singing in public, or dancing, or drawing, or reading their written work to others? We have, most of us, been the victims of inappropriate feedback and judgment. We have to be careful about this, in our course, so that we do not harm one another.

\subsection{Interlude: (A Few) Great Moments in Interdisciplinarity}

\begin{itemize}
\item Akhenaten and the invention of monotheism
\item Albrecht Durer and alchemy
\item Aristotle's book of comedy
\item Bill Evans and the Peace Piece
\item Buckminster Fuller and the geodesic
\item Chenrizi and the politics of China
\item Chuang Tzu and the butterfly
\item Coleridge and the person from Porlock
\item Cs�kszentmih�lyi and the flow experience
\item David Bohm's Implicate Order
\item Darwin, the bassoon, and the sundew
\item Eugen Herrigel and the practice of archery
\item Francis Yates and the Art of Memory
\item Freud, Jung, and the "bosh" incident
\item Fulcanelli and Mysteries of the Cathedrals
\item Giordano Bruno and the Hermetic tradition
\item Godel's uncertainty principle
\item Hanna Arendt at Nuremberg
\item Henri Rousseau in the jungle
\item Howard Carter and "wonderful things"
\item Jacob Bronowski at Auschwitz
\item Jacob Bronowski, Nagasaki, and Science and Human Values
\item Jan Tschichold and the Nazis
\item John Cage on the subway with the I Ching
\item Kepler's Somnium
\item Mary Shelley and the genesis of Frankenstein
\item Newton's Principia
\item Nikola Tesla and universal energy
\item Philip K. Dick, VALIS, and 2-3-74
\item Picasso, Guernica, and Expo 1937
\item R.D. Laing and madness as reality
\item Ramanujan's notebooks
\item Richard Feynman and the invention of quantum mechanics
\item Schwaller de Lubicz at Karnak
\item Simone Weil and leading from desire
\item St. Exupery flying into the desert
\item The Reimann Hypothesis
\item The Voynich Manuscript
\item The visions of Hildegard of Bingen
\item Thoth's legacy
\item Walter Benajmin and the Angel of History
\item Wendell Berry going Into the Woods
\item Wilhelm Reich's Cloudbuster
\item William Blake's Marriage of Heaven and Hell

\end{itemize}

\section{Readings}
\subsection{Required Course Texts}

\begin{description}
\item [David Edwards.] \textit{ArtScience: Creativity in the Post-Google Generation}
\item [Maxine Greene.] \textit{Releasing the Imagination: Essays on Education, the Arts, and Social Change}
\item [Eric Booth.] \textit{The Everyday Work of Art: Awakening the Extraordinary in Your Daily Life.}
\end{description}

\subsection{Suggested Books}
\begin{description}
\item [Allen, Pat.] \textit{Art is a Way of Knowing.} \\Shambhala, 1995.
\item [Barron, F., Montouri, A., and Barron, A., eds.] \textit{Creators on Creating: Awakening and Cultivating the Imaginative Mind.} 
\\New York: Putnam, 1997.
\item [Hyde, Lewis.] \textit{The Gift: Imagination and the Erotic Life of Property.} 
\\New York: Vintage, 1983.
\item [Butala, Sharon.] \textit{Wild Stone Heart}. \\HarperFestival,
  2000. \textsc{ISBN 000255397X}.
\item [Calvo, C\'esar.] \textit{The Three Halves of Ino Moxo}.
  \\Translated by Kenneth Symington. \\Inner Traditions, 1995.
  \textsc{ISBN 0892815191}.
\item [Campbell, Joseph.] \textit{The Mythic Image}.
  \\Princeton UP, 1974.
\item [Ellis, Normandi.] \textit{Dreams of Isis: A Woman's Spiritual
    Sojourn}.
  \\Quest, 1995.
\item [Hancock, Graham.] \textit{Heaven's Mirror: Quest for the Lost
    Civilization.}.
  \\Crown, 1998.
\item [Hedges, Chris.] \textit{War Is a Force that Gives Us Meaning}.
  \\Anchor, 2003. \textsc{ISBN 1400034639}.
\item [Kingston, Maxine Hong.] \textit{The Woman Warrior: Memoirs of a
    Girlhood Among Ghosts}. \\Vintage, 1989. \textsc{ISBN
    0072435194}.
\item [Kwan, Michael David.] \textit{Things that Must Not be
    Forgotten: A Childhood in Wartime China}. \\Soho Press
  \textsc{ISBN 1569472823}
\item [Laird, Ross A.] \textit{Grain of Truth: The Ancient Lessons of Craft}. \\MWR, 2000.
\item [Langewiesche, William.] \textit{American Ground: Unbuilding
    \\the World Trade Center}. \\North Point Press, 2002.
  \textsc{ISBN 0865475822}. (Also see \textit{Inside the Sky}.)
\item [London, Peter.] \textit{No More Secondhand Art.} 
\\Boston: Shambhala, 1989.
\item [Lopate, Phillip.] \textit{The Art of the Personal Essay: An
    Anthology from the Classical Era to the Present}. \\Anchor, 1997.
  \textsc{ISBN 038542339X}.
\item [Macfarlane, David.] \textit{The Danger Tree: Memory, War and
    the Search for a Family's Past}. \\Walker, 2001. \textsc{ISBN
    0802776167}.
\item[McNiff, S.] \textit{Art Heals: How Creativity Cures the Soul.} \\Boston: Shambhala Publications, 2004.
\item [Merwin, W.S.] \textit{The Mays of Ventadorn}. \\National
  Geographic Directions, 2002. \textsc{ISBN 0792265386}.
\item [Ondaatje, Michael.] \textit{Running in the Family}. \\Vintage,
  1993. \textsc{ISBN 0679746692}.
\item [Pirsig, Robert.] \textit{Zen and the Art of Motorcycle
    Maintenance}. \\HarperTorch, 2006 (reprint). \textsc{ISBN
    0060589469}.
\item [Saint-Exup\'ery, A.] \textit{Wind, Sand and Stars}. \\Harvest,
  2002. \textsc{ISBN 0156027496}.
\item [Sanders, Scott Russell.] \textit{Writing from the Center}.
  \\Indiana UP, 1997. \textsc{ISBN 0253211433}.
\item [Sullivan, William.] \textit{The Secret of the Incas: Myth,
    Astronomy, and the War Against Time.} \\Crown, 1996.
\item[Wilson, Frank.] \textit{The Hand: How Its Use Shapes the Brain, Language and Human Culture.}
\\New York: Vintage, 1998.

\end{description}

\subsection{Books on Creativity and Associated Philosophies}
\begin{description}
\item [Achebe, Chinua] \textit{Hopes and Impediments}. New York:
  Doubleday, 1989.
\item [Barron, F., ed] \textit{Creators on Creating: Awakening and
    Cultivating the Imaginative Mind}. New York: Putnam, 1997.
\item [Benjamin, Walter] \textit{Theses on the Philosophy of History}.
\item [Borges, Jorge Luis.] \textit{Collected Fictions}. \\Penguin,
  1999. \textsc{ISBN 0140286802}.
\item [Bohm, David] \textit{Wholeness and the Implicate Order}.
  London: Ark, 1980.
\item [Bohm, David] \textit{Unfolding Meaning}. New York: Routledge,
  1985.
\item [Bohm, David] \textit{On Creativity}. New York: Routledge, 1998.
\item [Bronowski, Jacob] \textit{Science and Human Values}. New York:
  Harper, 1956.
\item [---------] \textit{The Face of Violence}. London: Turnstile
  Press, 1964.
\item [---------] \textit{A Sense of the Future: Essays in Natural
    Philosophy}. Cambridge, MIT Press, 1977.
\item [Degler, Teri] \textit{The Fiery Muse: Creativity and the
    Spiritual Quest}. Toronto: Random House, 1996.
\item [Demos, John.] \textit{The Unredeemed Captive: A Family Story
    from Early America}. \\Vintage, 1995. \textsc{ISBN 0679759611}.
\item [Flack, Audrey] \textit{Art and Soul: Notes on Creating}. New
  York: Penguin, 1986.
\item [Franklin, Ursula] \textit{The Real World of Technology}.
  Toronto: Anansi, 1999.
\item [Fulford, Robert] \textit{The Triumph of Narrative: Storytelling
    in an Age of Mass Culture}. Toronto: Anansi, 1999.
\item [Goldberg, Natalie] \textit{Writing Down the Bones}. Boston:
  Shambhala, 1986.
\item [Herrigel, Eugen] \textit{Zen in the Art of Archery}. New York:
  Random House, 1977.
\item [Hildegard of Bingen] \textit{Secrets of God: Writings of
    Hildegard of Bingen}.\\ Boston: Shambhala, 1996.
\item [Hyde, Lewis] \textit{The Gift: Imagination and the Erotic Life
    of Property}. New York: Vintage, 1983.
\item [---------] \textit{Trickster Makes This World: Mischief, Myth,
    and Art}. New York: North Point Press, 1998.
\item [Jim\'enez, Juan Ramon] \textit{The Complete Perfectionist: A
    Poetics of Work}. Edited and translated by Christopher Maurer. New
  York: Doubleday, 1997.
\item [Jung, C.G] \textit{The Spirit in Man, Art and Literature}.
  Translated by R.F.C. Hull. Princeton: Princeton University Press,
  1998.
\item [London, Peter] \textit{No More Secondhand Art}. Boston:
  Shambhala, 1989.
\item [Lorca, Federico] \textit{In Search of Duende}. Translated by
  Christopher Maurer. New York: New Directions, 1998.
\item [Lyndon, Susan] \textit{The Knitting Sutra: Craft as a Spiritual
    Practice}. San Francisco: Harper, 1997.
\item [Needleman, Carla] \textit{The Work of Craft: An Inquiry Into
    the Nature of Crafts and Craftsmanship}. New York: Kodansha, 1979.
\item [Pye, David] \textit{The Nature and Art of Workmanship}.
  Cambridge: Cambridge UP, 1968.
\item [Richards, Mary] \textit{Centering in Pottery, Poetry and the
    Person}. Middletwon, CT: Wesleyan UP.
\item [Sarton, May] \textit{Journal of a Solitude}. New York: Norton,
  1973.
\item [Sennett, Richard] \textit{The Corrosion of Character: The
    Personal Consequences of Work in the New Capitalism}. New York:
  Norton, 1998.
\item [Thoreau, Henry David] \textit{Walden}. New York: Norton, 1985.
\item [Wilson, Frank] \textit{The Hand: How Its Use Shapes the Brain,
    Language and Human Culture}. New York: Vintage, 1998.
\end{description}
\newpage
\subsection{Mythological Fiction}
\begin{description}
\item [Joseph Conrad] \textit{Lord Jim, Heart of Darkness\/}
\item [Thomas Wharton] \textit{Salamander\/}
\item [Milan Kundera] \textit{Life is Elsewhere\/}
\item [Carlos Fuentes] \textit{The Orange Tree\/}
\item [Gabriel Garcia Marquez] \textit{One Hundred Years of Solitude\/}
\item [Jorge Luis Borges] \textit{Labyrinths\/}
\item [Alberto Manguel (editor)] \textit{Black Water: The Book of
    Fantastic Literature\/}
\item [Don DeLillo] \textit{Underworld\/}
\item [Somerset Maugham] \textit{The Razor's Edge\/}
\item [Philip K. Dick] \textit{The Man in the High Castle\/}
\item [Keri Hulme] \textit{The Bone People\/}
\item [Salman Rushdie] \textit{Midnight's Children\/}
\item [John Fowles] \textit{The Magus\/}
\item [Stephen King] \textit{The Stand\/}
\item [Philip Roth] \textit{Operation Shylock\/}
\item [Walter Miller] \textit{A Canticle for Leibowitz\/}
\end{description}

\clearpage

\section{Demonstration of Learning}

\subsection{Written Assignments}
Three individual assignments are required for this course: a research
essay and two creative compositions (projects). The research essay involves you choosing a specific theme or thread and exploring it in
some depth. We will discuss this project at length in class. The
creative compositions are opportunities for you to discover
and explore creativity in your own life. Again, we will
discuss these projects in class.

For philosophical reasons, I do not prescribe a particular length or structure for the projects: a great essay can be a few pages long (as we'll see). Yet it is difficult to craft a good essay in less than a few thousand words, and difficult to craft a good creative artifact in less than many hours of work. So, I offer two recommendations: make the projects as long as they need to be; and make the projects longer than you think they need to be. There is no upper limit on the length or complexity of the projects.

In terms of the research essay, I'm not interested in how much you can write but rather in the quality of your writing. Perhaps you write like Hemingway, perhaps like Melville or Tolstoy. I don't know, and maybe you don't know either. But I can tell you this: writing a shorter piece of great precision is more difficult than writing a longer, more relaxed and wandering work. In the context of smaller projects every word is on display and under scrutiny, whereas in longer works the sheer bulk of the material tends to hide various flaws. Melville, in fact, is a good example of this.

You may write short narratives in this course, but please do not write
short form as a means of avoiding work. You will know, I will notice,
and neither of us will be happy. Instead, make your work as long as it
needs to be. If you compose a lovely, resonant, short piece, you will
receive an excellent evaluation. But as I said, writing shorter pieces
is actually more difficult.

The research essay is worth 30 per cent of your grade. The two creative compositions (projects) are worth 20 per cent each.

\newpage

\subsection{Group Presentations}

Each student will be a member of several different peer groups; each
peer group will present at least one mini-presentation (roughly fifteen minutes each) on various topics. Each class session after the first will involve presentations, with one presentation
from each group. Class time will be given for preparing the presentations. The structure and content of the presentations will be discussed in class.

The group presentations are worth a total of 30 per cent of your grade.

\subsection{Evaluation of Assignments and Presentations}
My primary focus, as an instructor, is to assist you in developing
your creativity. Grades are quite far down on the list of priorities
for me. I am focused on your engagement with the process, your
commitment to your own work, the extent to which you show up,
metaphorically, to be as present as you can be. These are evaluation
criteria for me.

\subsection{Attendance and Participation}
The expectation is that you will attend all sessions and involve
yourself in the class process. Your willingness to engage creatively
with the learning process, to take appropriate personal risks, and to
participate in group activities are all central to your involvement in
this class. Your emotional involvement in the class is
as important as your academic knowledge of the material.

\subsection{Grade Inflation}
Every semester there are students who do well on the
assignments, complete all the associated learning goals of the course,
participate well, and wonder why they do not receive a grade of one
hundred percent (or 98, anyway). Here is the reason: almost every
semester there are students who demonstrates a level of commitment
that goes beyond the course requirement. Such students complete extra
work, or hand in exemplary assignments, or undertake a significant
amount of personal development in addition to the course expectations.
Such students typically receive the highest grades.

If you do reasonably well in the course you will receive a reasonable
grade. Very high grades are intended for extra or exemplary work.
Unfortunately, over the past thirty years the post-secondary
educational system in North America has participated in a process of
grade inflation. Since the 1980's, the average grade for typical
course work has been increasing by about 25 per cent each decade.
Elevated assessments do not accurately reflect the work of most
students. Even worse, grade inflation has caused many students to
expect high grades for average work. I am not a particularly stringent
assessor; but I will not inflate grades artificially.

The grades for the course will be distributed along a curve, with a
small number of students (likely) receiving high grades, most students
receiving grades in the middle range, and a few students struggling
with lower grades. If you are uncertain about your assessment for a
given assignment, or if you wish to know where, roughly, you are along
the distribution curve of the class, or if you would like suggestions
for how to improve your grade, please ask me for clarification.
\clearpage

\section{Class Schedule}
The class sessions will be balanced between presentations (by the instructor and students) academic material, group collaboration, and composition. The content for each session will evolve as the semester progresses. We will cover the following themes (though, perhaps not in the order listed below):
\\
\begin{compactdesc}

\item[The Nature of Interdisciplinarity]
A vignette: Bronowski at Nagasaki.
Definitions and principles.
Clarifications of common misunderstandings about interdisciplinary approaches and practices.
An introduction to interdisciplinarity as a mode of inquiry in the arts and sciences.
\\
\item[Traditions and Practices]
Consideration of interdisciplinary practices as effective vehicles for the transmission of  sacred, social, political, artistic, and scientific information.
Examination of interdisciplinarity as a fundamental and necessary function of human nature and inquiry.
\\
\item[Developments and Milestones]
Consideration of the development of interdisciplinarity and its role in the contemporary world.
Examination of the relationship between interdisciplinary practices and traditional, faculty-based inquiry in the academic environment.
\\	
\item[Themes and Philosophies]
Introduction to the historical background of interdisciplinary philosophy and practice in science and literature.
Explication of ancient world views, with particular emphasis on spirituality, science, mythological concepts, and approaches to the imagination.
\\
\item[Hermetic Threads]
Reading of excerpts from core Egyptian, alchemical, and early scientific texts, with particular emphasis on foundational interdisciplinary ideas.
Examination of the ways in which the sciences and the arts were entwined in the practices and perspectives of all peoples until the twentieth century.
Exploration of the transmission of interdisciplinary ideas into the contemporary world.
\\
\item[Interdisciplinarity in the Expressive Arts]
Exploration of the transmission of interdisciplinary philosophies and practices by way of the expressive arts.
Examination of practices and structures within the interdisciplinary expressive arts (writing, storytelling, dance, movement, ritual, religious practice, music, philosophy, etc.), with emphasis on the traditions of psychology and mythology.
Reading of selected texts within the interdisciplinary expressive arts.
Examination of the ways in which texts within the interdisciplinary expressive arts have influenced the development of the arts and sciences.
\\
\item[Multicultural Principles]
Introduction to the interdisciplinary contributions made by Asian mythology, literature, and science.
Explication of ancient Asian world views, with particular emphasis on interdisciplinary spirituality, mythological concepts, and approaches to knowledge.
Reading of excerpts from Asian texts, with particular emphasis on the synthesis of interdisciplinary expressive arts within those traditions.
\\
\item[Other Stuff (that we'll think up as we go along)]
Much of the course content will be generated by students.

\end{compactdesc}



\end{document}

