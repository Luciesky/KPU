%!TEX TS-program = xelatex
    %!TEX encoding = UTF-8 Unicode
%    \documentclass[10pt, letterpaper]{article}
\documentclass[letterpaper,10pt,headsepline]{scrreprt}
    \usepackage{fontspec} 
    \usepackage{placeins}
    \usepackage{multibbl}
    \usepackage{graphicx}
    \usepackage{hieroglf}
    \usepackage{txfonts}
    \usepackage{url}
    \usepackage{titling}
    \usepackage{geometry} 
    \geometry{letterpaper, textwidth=5.5in, textheight=8.5in, marginparsep=7pt, marginparwidth=.6in}
    %\setlength\parindent{0in}
    \defaultfontfeatures{Mapping=tex-text}
    \setromanfont [Ligatures={Common}, SmallCapsFont={ITC Officina Serif Std}, BoldFont={ITC Officina Serif Std Bold}, ItalicFont={ITC Officina Serif Std Book Italic}]{ITC Officina Serif Std Book}
    \setmonofont[Scale=0.8]{Lucida Sans Typewriter Std} 
    \setsansfont [Ligatures={Common}, SmallCapsFont={ITC Officina Sans Std}, BoldFont={ITC Officina Sans Std Bold}, ItalicFont={ITC Officina Sans Std Book Italic}]{ITC Officina Sans Std} 
\usepackage[ngerman,english]{babel}
\usepackage{scrpage2}
\usepackage{paralist}
\clubpenalty=6000
\widowpenalty=6000
\author{Ross A. Laird, PhD}
\title{Interdisciplinary Expressive Arts 3302}
\date{\today}
\ohead{Ross Laird}
\chead{IDEA 3302}
\pagestyle{scrheadings}
\setcounter{secnumdepth}{-1}
\begin{document}
\pagestyle{empty}
\vspace*{7em} 
\begin{center}
\huge{Creativity and Leadership in Groups}\\
\vspace*{1em} 
\large{Course Outline}
\end{center}
\clearpage
\pagestyle{scrheadings}
\tableofcontents
\chapter{Creativity and Leadership in Groups}
Instructor: Ross Laird, Ph.D.\\ 
Email: \url{ross@rosslaird.com}\\
Website: \url{www.rosslaird.com}\\
Office: D308, Surrey campus (by appointment)\\

\section{Course Intention}
Working creatively with groups is immensely rewarding and uniquely challenging, requiring of the facilitator a particular blend of professional skills and self-awareness that develops only through practice and experiment. This course is designed to offer learners a wide spectrum of both theoretical and experiential approaches to group work, focusing especially on core facilitation skills and creativity. We will examine group dynamics and evolution, interpersonal relationships (e.g. conflicts, alliances and other structures), leadership styles, curriculum development, and the role of health and healing practices. Particular attention will be given to developing effective facilitation strategies through achieving greater self-awareness and active sensing. Participants will also learn about the emerging models of group work which focus on collaboration, community-building and creativity.

\section{Learning Goals}
\begin{itemize}
\item To experience and experiment with the group format as a context for personal growth and change.
\item To become familiar with current applications and theories of group work, with particular emphasis on the group as a dynamic and creative system.
\item To explore multicultural and alternative approaches to group work, with an emphasis on understanding the role of culture within the group format.
\item To examine the relationship between facilitation skills and self-awareness.   
\end{itemize}

\section{Learning Experiences}
The course will include a variety of learning experiences contingent upon regular attendance and dedicated participation. Because group work is an interactive process, much of the class time will be devoted to group experiential exercises, individual reflective tasks and practical assignments. An integral aspect of the class process will be the formation of a \emph{class group} which will meet each session. The development of this group will be discussed in class. The intent is to give participants an opportunity to experience a modified form of group work within the class structure. This serves as a complement to the theoretical material offered.
Completion of the group presentations, readings, and final projects (see Demonstration, below) form the balance of the class.

\section{Readings}

\subsection{Required Course Texts}
\noindent
Kottler, Jeffrey. \textit{Learning Group Leadership: An Experiential Approach}. Sage, 2009.\\ Latest edition.\\
London, Peter. \textit{No More Secondhand Art: Awakening the Artist Within}. Shambhala, 1989.\\ Any edition.

\subsection{Suggested Reading}
\noindent
Mindell, A. \textit{The Leader as Martial Artist}. San Francisco: HarperCollins, 1993.\\
Corey, G. and Corey, M. \textit{Groups: Process and Practice}. Pacific Grove: Brooks Cole, 2006. \\
Sun Tzu. \textit{The Art of War} (T. Cleary, Trans.). \\Boston: Shambhala, 1996.  (Original work published BCE).\\
Sennett, R. \textit{The Corrosion of Character: The Personal Consequences of Work in the New Capitalism}. New York: Norton, 1998.\\
Storr, A. \textit{Feet of Clay: Saints, Sinners, and Madmen}.\\ New York: Free Press, 1996.\\

\section{Demonstration of Learning}
\subsection{Attendance and Participation}
The expectation is that you will attend all sessions and involve
yourself in the class process. Your willingness to engage creatively with the learning process, to take appropriate personal risks, and to participate in group activities are all central to your involvement in this class. Because developing a facilitation style is very much a process of blending your own personal awareness with skills and practical techniques, your own emotional involvement in the class is as
important as your academic knowledge of the material.

\subsubsection{Assessment Criteria for Attendance and Participation}

\begin{itemize}
\item Demonstration of commitment to the development of self-awareness.
\item Openness to interpersonal process.
\item Ability to participate in appropriate self-disclosure.
\item Consideration of and responsiveness to others.
\item Willingness to take appropriate risks and to challenge oneself.
\item Commitment to enhancing the interpersonal experience of everyone in the class.
\item Ability to take personal responsibility for learning.
\item Willingness to deal with conflicts appropriately if and when they arise.
\item Ability to be open and responsive to appropriate feedback.
\item Willingness to speak up, to join conversations, and to contribute.
\end{itemize}

The above criteria represent 30 per cent of your course grade. 

\subsection{Design Project}
Group effectiveness can be greatly augmented by careful planning and attention to detail. This assignment is designed to have you consider some of the core considerations important to making a group run smoothly. Using the discussions and material from class, selections from the text and your own research (citing at least 3 books or articles from the field), design a group for a specific population. Use as much of your own authenticity and creativity as you can, bearing in mind that groups run best when the leader stays true to his or her own needs. This project has three main sections, as shown below (suggested content is given for each area).

\subsubsection{Section 1: Design and Rationale}
Why are you running this group? What does it mean to you? Why is there a need for it? Specify design details such as how long your group will run, how many members it will have, where it will be held, how you will get clients and so on. How will you facilitate? Will you get help? If so, why? What population-specific concerns do you have? How will you meet those concerns? What general guidelines will you provide to participants to help build safety and trust?

\subsubsection{Section 2: Self Awareness}
What specific goals and expectations will you have for yourself (and for group members)? What will you need to be most aware of so that you avoid your common character traps? What will be your main personal challenge in running the group? What kinds of clients particularly stir you up? What kinds of situations? How will you handle these as they arise? What is your greatest strength or asset in running the group? How will you ensure you don't lose sight of it?

\subsubsection{Section 3: Sessional Outline}
Provide an outline for each session which includes content or themes, exercises and scheduling. If you plan to adapt exercises from our class, simply mention the names of them. If you design or use your own exercises, provide a description of each (and citation, if appropriate). Provide supporting handouts or materials as required.

\subsubsection{Assessment Criteria for the Design Project}
\begin{itemize}
\item Commitment to the development of professional self-awareness.
\item Awareness of personal strengths and challenges in facilitation.
\item Ability to apply themes of personal development to the professional setting.
\item Originality and creativity.
\item Comprehensiveness of the group design.
\item Application of new learning from class.
\item Consideration of and responsiveness to potential problems in the proposed group.
\item Quality of composition.
\item Clarity and organization.
\item References to similar materials or programs (at least three are required). 
\end{itemize}

The design project need not be limited to a compositional essay, though it must include the compositional elements above. It may be augmented by various works of creative expression.

The purpose of the design project is to give you an opportunity to practice the skills required to develop a professional group proposal. These skills are essential in the field of facilitation and are best acquired in an educational setting.

The design project is worth 40 per cent of the course grade and is due in the second-to-last session.

\subsection{Group Presentation}
Each participant will help develop, with two or three other class members, a practice facilitation exercise to be completed in one of the last 6 sessions of the course. The schedule will be determined in class. Selecting a modality, exercise or approach, and with prior instructor approval, each group will practice facilitation (with the class as a practice group) for between 45 and 90 minutes. The practice facilitation exercises must provide an experiential component, a group discussion component, and an informational component. Additionally, each practice facilitation group must provide approximately equal practice facilitation time to each of its members. Each group will also prepare a one page summary of their modality or exercise (with short reference bibliography, as required) to present to each member of the group as a professional resource.

\subsubsection{Assessment Criteria for the Group Presentation}
\begin{itemize}
\item Willingness to take appropriate risks and to challenge oneself.
\item Willingness to speak up and to lead.
\item Openness to interpersonal process.
\item Willingness to collaborate with other practice co-facilitators.
\item Consideration of and responsiveness to others.
\item Commitment to enhancing the interpersonal experience of everyone in the group.
\item Willingness to examine personal values, beliefs, and judgments.
\item Ability to take personal responsibility for learning.
\item Willingness to deal with conflicts appropriately if and when they arise.
\item Ability to be open and responsive to appropriate feedback.
 
\end{itemize}

The group presentation is worth 30 per cent of the course grade.

\section{Assessment of Learning}
Assessment will be based on group participation, the group
presentation, and completion of the design project. Please review the
information on the previous pages for details about each assignment.
Also please consult the assessment forms, which appear in the next
section.

\subsection{Assessment Philosophy}
Unlike many other fields, in which competence and skill may be measured objectively, using replicable and consistent means (tests of factual knowledge, for example), group faciulitation depends almost entirely on the interpersonal skills of the practitioner. Computer programmers can be assessed by their ability to write code; chiropractors can be evaluated based on their skill at manipulating the human skeleton; race car drivers can be clocked around a track. But for group facilitators there are no such fixed measures. Interpersonal skills are subtle, difficult to quantify, and complex beyond any measurement scheme.

And yet we can identify those who possess exemplary personal skills. They are relaxed, open, responsive, kind. Often they exhibit skills that we tend to assign to the social sphere: personal warmth, consideration of others, hesitancy to judge, sensitivity to emotions. To some extent, these features -- which are aspects of temperament more than they are learned skills -- can be evaluated using rating scales based on observation. Empathy rating scales are often used for this purpose in counselling training programs. Such scales, or other, similar assessment measures, are useful as baselines, or starting points; but they cannot replace the interpretations of peers and colleagues -- of regular people, in other words -- in assessing the interpersonal skills of a facilitator. There are simply far too many factors in interpersonal communication for any standardized evaluation procedure to measure.

Practicing facilitators are assessed by their clients and to a lesser extent by their colleagues. In all large-scale studies that have examined satisfaction and success in group work, clients consistently report that their trust of the facilitator and their feelings of good will in the relationship were the most important factors in contributing to growth and change. The actual approach of the facilitator appears to be irrelevant, essentially, with regard to the progress clients make.

Facilitators in training have few (if any) opportunities to be assessed by actual clients. Instead, they must be assessed by their student peers and by their instructors. The process of this assessment works best when it takes into account the subtle interpersonal factors described above. These include (but are not limited to):

\begin{itemize}
\item Commitment to the development of self-awareness.
\item Openness to interpersonal process.
\item Ability to participate in appropriate self-disclosure.
\item Consideration of and responsiveness to others.
\item Commitment to enhancing the interpersonal experience of everyone in the class.
\item Willingness to examine personal values, beliefs, and judgments.
\item Ability to take personal responsibility for learning.
\item Willingness to deal with conflicts appropriately if and when they arise.
\item Ability to be open and responsive to appropriate feedback.
\end{itemize}

Each item on this list is an aspect of the first item: self-awareness. The most proficient counsellors are those who demonstrate commitment to self-awareness. They consistently query their own responses, thoughts, and feelings. They ask themselves:

\begin{itemize}
\item What am I feeling right now?
\item What am I thinking right now?
\item Why am I reacting in this particular way?
\item What do my thoughts, feelings, and reactions tell me about myself?
\item Is there anything about my current behavior that suggests unresolved themes in my life?
\item Is my perception of myself consistent with what other people tell me about the kind of person I am?
\item When and how do I get stuck, and what am I doing to work on this?
\item In what ways do I get overwhelmed, or shut down, or avoid?
 
\end{itemize}

These questions, and many others, require the capacity for self-reflection and self-awareness. As we continue in the course, you may wish to consider these questions as they apply to you. At the very least, you might wish to consider what you are currently working on in your life, in which direction your attention is drawn, into which of the innumerable themes of human nature you are now called to delve.

In my role as your instructor, I will be paying attention to how thoughtful you are in examining and responding to questions like those in the first list above. I will not be analyzing you, but rather noticing what kinds of things you do, what your reactions are to various situations. This is a basic facilitation skill and one which I will demonstrate repeatedly throughout the course. My goal in observing your behaviors and interacting with you is to assist you in developing greater self-awareness. Self-awareness is the most foundational skill of all, and is therefore an aspect of assessment in this course.

I will use the self-awareness list above, as well as the assessment criteria listed for each assignment, to assess your overall participation in the course. I will not be evaluating your level of self-awareness but rather your openness to the process of developing your self-awareness.

\subsection{Grade Inflation}
Almost every semester there are students who do well on the assignments, complete all the associated learning goals of the course, participate well, and wonder why they do not receive a grade of one hundred percent (or 98, anyway). Here is the reason: almost every semester there are students who demonstrates a level of commitment that goes beyond the course requirement. Such students complete extra work, or hand in exemplary assignments, or undertake a significant amount of personal development in addition to the course expectations. Such students typically receive the highest grades.

If you do reasonably well in the course you will receive a reasonable grade. Very high grades are intended for extra or exemplary work. Unfortunately, over the past thirty years the post-secondary educational system in North America has participated in a process of grade inflation. Since the 1980's, the average grade for typical course work has been increasing by about 25 per cent each decade. Elevated assessments do not accurately reflect the work of most students. Even worse, grade inflation has caused many students to expect high grades for average work. I am not a particularly stringent assessor; but I will not inflate grades artificially.

If you are uncertain about your assessment for a given assignment, or if you wish to know where, roughly, you are along the distribution curve of the class, or if you would like suggestions for how to improve your grade, please ask me for clarification.

If you wish to achieve a good grade, please do the following:

\begin{itemize}
\item Show up for class -- every class. This course depends on student engagement. (This becomes especially important during the final weeks of the semester.)
\item Be attentive and mindful to the various criteria listed for each of the projects and the course overall.
\item Take the initiative to plan and develop your projects and presentations. This course is (very likely) more fluid and spontaneous than you are used to. Your ability to manage your time, commitment, and energy is crucial.
\item Speak up in every class (review the criteria for group engagement and presentations).
\item Don't look for the right answer to a question or challenge. Instead, find the answer that is meaningful to you.
\item Ask for help if you need it.
\item Commit to your projects in a substantial way. Good projects take time. Rushed projects are obviously rushed.
\end{itemize}

Finally, please be attentive to the Kwantlen policies on academic honesty and plagiarism, which can be found at the following URLs:

\noindent
Academic Honesty: \url{http://www.kwantlen.ca/__shared/assets/Honesty1432.pdf}\\
Plagiarism and Cheating: \url{http://www.kwantlen.ca/policies/C-LearnerSupport/c08.pdf}

\section{Assessment Forms}
These forms are used by me to assess your involvement in the class. The rankings, from 0 to 3, represent the following:\\
0: Failed to complete the given criteria\\
1: Approaches course goals for the given criteria; some work still needed\\
2: Meets expected course goals\\
3: Exceeds course goals\\
\subsection{Attendance and Participation}
\begin{tabular}{|l|r|}
\hline
Assessment Criteria & Level\\
\hline
Attendance &  /10\\
Commitment to the development of self-awareness & \ 1 \ 2 \ 3 \\
Openness to interpersonal process & \ 1 \ 2 \ 3 \\
Ability to participate in appropriate self-disclosure & \ 1 \ 2 \ 3 \\ 
Consideration of and responsiveness to others & \ 1 \ 2 \ 3 \\
Willingness to take appropriate risks and to challenge oneself & \ 1 \ 2 \ 3 \\ 
Commitment to enhancing the interpersonal experience of all class members & \ 1 \ 2 \ 3 \\ 
Ability to take personal responsibility for learning & \ 1 \ 2 \ 3 \\
Willingness to deal with conflicts appropriately if and when they arise & \ 1 \ 2 \ 3 \\ 
Ability to be open and responsive to appropriate feedback & \ 1 \ 2 \ 3 \\
Willingness to speak up, to join conversations, and to contribute & \ 1 \ 2 \ 3 \\
\hline
\end{tabular}
\subsection{Group Design Paper}
\begin{tabular}{|l|r|}
\hline
Assessment Criteria & Level\\
\hline
Commitment to the development of professional self-awareness & /10 \\
Awareness of personal strengths and challenges in facilitation & \ 1 \ 2 \ 3 \\
Ability to apply themes of personal development to the professional setting & \ 1 \ 2 \ 3 \\
Originality and creativity & \ 1 \ 2 \ 3 \\
Comprehensiveness of the group design & \ 1 \ 2 \ 3 \\
Application of new learning from class & \ 1 \ 2 \ 3 \\
Consideration of and responsiveness to potential problems & \ 1 \ 2 \ 3 \\
Quality of composition & \ 1 \ 2 \ 3 \\
Clarity and organization & \ 1 \ 2 \ 3 \\
References to similar materials or programs & \ 1 \ 2 \ 3 \\
Considerations of future development & \ 1 \ 2 \ 3 \\
\hline
\end{tabular}
\subsection{Group Presentation}
\begin{tabular}{|l|r|}
\hline
Assessment Criteria & Level\\
\hline
Willingness to take appropriate risks and to challenge oneself & \ 1 \ 2 \ 3 \\
Willingness to speak up and to lead & \ 1 \ 2 \ 3 \\
Openness to interpersonal process & \ 1 \ 2 \ 3 \\
Willingness to collaborate with other practice co-facilitators & \ 1 \ 2 \ 3 \\
Consideration of and responsiveness to others & \ 1 \ 2 \ 3 \\
Commitment to enhancing the interpersonal experience of all class members & \ 1 \ 2 \ 3 \\
Willingness to examine personal values, beliefs, and judgments & \ 1 \ 2 \ 3 \\
Ability to take personal responsibility for learning & \ 1 \ 2 \ 3 \\
Willingness to deal with conflicts appropriately if and when they arise & \ 1 \ 2 \ 3 \\
Ability to be open and responsive to appropriate feedback & \ 1 \ 2 \ 3 \\
\hline
\end{tabular}

\section{General Course Guidelines}
Perhaps the best way to learn the intricacies of facilitation is to involve yourself in groups in as many ways as possible. This means practicing not only as facilitator but also as participant. If you have not been in faciliated groups before, now is an excellent time to consider the option. If you want to know what group clients need -- become one.

The classroom is an artificial setting. As such, personal issues brought by students for exploration in the process sessions should carry a smaller emotional charge than what you might bring to a genuine session with a professional facilitator or counsellor. For example, talking about a small conflict at work is typically a good practice theme, whereas your experience as the only survivor of a plane crash is not.

Remember to honor confidentiality as regards your practice sessions. It is a serious ethical breach to discuss session content outside of the session with others who were not involved. Although session content may come up in class, this material will remain confidential to our group. It is very difficult to maintain confidentiality, especially when you hear juicy gossip. It is best to start practicing confidentiality as a skill early in your career, because it requires constant reinforcement. If you are serious about it, others are likely to be as well.

Trust the wisdom of your own resources as you move forward. If you know how to follow your own centre, you will have no trouble. However, none of us is able to be on track all the time, so if you get overwhelmed talk to fellow students or to the instructor. Speak up if something is not working for you. I am especially interested in you bringing forward views, techniques and opinions that are contrary to those presented in class.

And, finally: have fun.

\section{Thematic Schedule}
The class structure involves 14 sessions. These sessions will be
balanced between presentations (by the instructor and students)
academic material, group collaboration, and other activities. The content for each session will evolve as the semester progresses. We will cover the following themes (though, perhaps not in the order listed below):
\\

\begin{compactdesc}

\item[Getting Started] Introductions and student goals.\\
Course objectives and assignments.\\
Teaching as modeling.\\
Group chairs exercise; group discussion.
\item[Core Skills] Three core skills: grounding, centering, boundaries. How to develop core skills.\\ 
Elements of effective facilitation and leadership.
\item[Development] Group evolution and development (the mandala).\\
Frameworks and the limitations of frameworks.\\ Dynamical systems.
\item[Evolution] Stages in self-awareness and group development.\\
The labyrinth, the path, and the crucible.\\ Culture and myth. 
\item[Depth] Therapeutic contexts and statements.\\ Cautions and dangers.\\
Effective and ineffective statements.\\
The continuum of practice.
\item[Interactions] Sensing and working with behaviors and conflicts.\\
Mask, shadow, and authentic self.
\item[Common Themes] Working with addictions, depression, and related themes.
\item[More Themes] Spirituality and creativity in the group context.
\item[Out in the World] Working with specific populations: support groups, corporate groups, self-development groups.
\item[Closure] Completing group closure.\\ Closure and completion as foundational stages.\\ The spiral path.
\end{compactdesc}

\end{document}
