\documentclass[12pt,DIV7,oneside,tocindent,headsepline]{scrbook}
\addtocounter{tocdepth}{1}
\usepackage[oldstyle]{sabonlt}
\typearea[current]{last}
\usepackage{lettrine}
\usepackage{ragged2e}
\usepackage{epigraph}
\usepackage{titling}
%\usepackage{waterstitling}
%\usepackage{color}
\hyphenation{sl-ow think-ing wheth-er every-thing quick-ly ex-per-ien-ce seek-er quan-dary schoo-ls thr-ead-ed ren-der-ed tr-au-ma}
\setkomafont{chapter}{\fontfamily{ps8}\Large}
\setkomafont{section}{\fontfamily{ps8}\large}
\setkomafont{subsection}{\fontfamily{ps8}\normalsize}
\setkomafont{pagehead}{\fontfamily{pwt}\normalsize}\setkomafont{pagefoot}{\fontfamily{pwt}\normalsize}
\setkomafont{title}{\fontfamily{pwt}\Huge}
%For toc page numbers
\makeatletter
\renewcommand*{\@pnumwidth}{2.2em}
\renewcommand*{\@tocrmarg}{12.55em}
\makeatother
%For lettrines and epigraphs
\renewcommand{\LettrineFontHook}{\fontfamily{pwt}\fontseries{mn}}
\setlength{\epigraphwidth}{.4\textwidth}
%Text formatting
\usepackage{microtype} %must be loaded after all font corrections
\usepackage{everysel}
\clubpenalty=6000
\widowpenalty=6000
\sloppy
%Skip two lines for lettrines at page bottoms
\newcommand\shortpagel[1][2]{\enlargethispage{-#1\baselineskip}}
%Skip one line for widows and orphans
\newcommand\shortpagewo[1][1]{\enlargethispage{-#1\baselineskip}}
%Document details
\pretitle{}
\title{\fontfamily{pwt}\Huge{Group Counselling\\}}
\posttitle{\small{Sessional Outline}\vspace{.5in}}
\author{Ross A. Laird}
\begin{document}
\begin{titlingpage}
\begin{center}
\maketitle
\end{center}
\end{titlingpage}
\frontmatter
\begin{FlushLeft}
\tableofcontents  
\end{FlushLeft}
\clearpage
\mainmatter
\pagestyle{headings}
\chapter{Sessional Outline}
\section{Overview and Structure}  

This course takes place over the course of ten weeks, with classes one night per week (for ten weeks) and one Saturday session. With the exceptions of the first evening session and the Saturday session (which is devoted to the creation of an authentic group experience), the first half of each evening session is designated for presentations by the instructor. The second half of each session is intended for group development.  Breaks are at the discretion of the group and instructor. Typically, a break takes place after about 90 minutes.

\subsection{General Recommendations}
Within the context of the Vancouver Community College counselling programs, various cultural and academic norms prevail. It is recommended that the instructor of Group Counselling make an effort to become aware of and respond to these norms, among which include the following:

\begin{itemize}
\item The students typically have personal experience either as counsellors themselves or as clients in counselling (or both). The accumulated knowledge of any given class, therefore, is substantial. Instructors are wise to draw upon the wisdom of the group.

\item Students often demonstrate a preference for experiential modes of learning as opposed to lecture format.

\item In general, the classroom environment at VCC works best when the instructor takes time to balance presentations with student-led discussions.

\item Often VCC counselling students demonstrate a preference for concrete examples as opposed to abstract theories and models. They will be working, or already are working, in environments where practical skills are essential, and they appreciate learning these in the classroom. Of particular utility are sheets such as the \textit{Ten Statements} handout in the reader for this course.

\end{itemize}

 

\section{Evening One: Beginnings}  
The first session emphasizes safety, gentleness, and a smooth transition toward group development. In this sense, the first session is foundational. The regular division of the subsequent sessions (instructor presentations first half, group process second half) is here reversed, to allow the participants time to acquaint themselves to one another.  

\subsection{Introductions}  
An introductory exercise may be used here, at the discretion of the facilitator. Examples include: 
\begin{itemize} 
\item Participants meet in small groups (about three) and introduce themselves, then gather in the large group for more formal introductions. 
\item Participants circulate freely within a social environment for a few minutes, choose a different place in which to sit, and discuss the process. This may be repeated several times, with different emphases (e.g. suggestions to choose a favorite location, or a challenging location, or to choose by instinct alone; these choices may later be discussed in the context of group formation). 
\item Participants make a posture, or a sound, or an image, that represents their state of mind at the moment. Later, these can be explored as aspects of group formation. 
\end{itemize}  
The traditional method of formal introductions moving in a circle around the group is not recommended due to its general unsuitability to the VCC context. A more creative and fluid approach is called for, which respects both the diversity and informality of the group. 

\subsection{Student Goals}  
A conversation about what each student wishes to learn, or to accomplish, from this group experience. The facilitator may make notes from this discussion, and these notes might be revisited in the last session.  

\subsection{Course Outline and Assignments}  
The instructor reviews the course outline, with emphasis on the blend between personal and professional development in the course. Course outlines are distributed and discussed. Assignments are described and discussed. (Assignments include one group presentation and one major paper). Evaluation is also discussed, particularly in terms of how evaluation of participation is derived (from personal commitment and involvement, not by how much a given participant speaks). 

\subsection{Instructor Presentation}  
The instructor describes the delicate balance between facilitation and teaching, and the manner in which these roles are unavoidably blended in this course. The instructor also discusses behavior modeling as a central feature of both teaching and facilitation and indicates how he/she intends to convey that modeling.  

\subsection{Process Group Discussion}  
The group discusses the manner in which the class will also, at times, be a process group. The instructor describes the commitment asked of students, the importance of student personal development in terms of eventual professional practice, and the challenge of developing safety for this type of blended group. Questions are asked and answered.

\section{Evening Two: Foundational Skills}  
\subsection{Core Competencies}  
The instructor discusses core competencies, which may be framed in an interdisciplinary and cross-cultural framework of grounding, centering, and boundaries.  
\begin{description} 
\item[Grounding]
The ability to be present, and engaged, and accountable for one's own emotional and professional presence. This may be modeled as relaxation, clarity, self-monitoring, responsiveness, etc. It may also be described in multicultural terms as connection to the earth, awareness of others, and embodiment. \end{description} 

\begin{description} 
\item[Centering] 
The ability to be aware of one's own behavior, reactions, impressions, feelings, and thoughts. Also the ability to respond to these appropriately, and with neutrality. \end{description} 

\begin{description} 
\item[Boundaries]  
The awareness of where we end and others begin. This includes confidentiality, limits on time and attention, awareness of the importance of self-care, dangers of dual roles, etc. Boundaries may also be contextualized in a multidisciplinary and multicultural manner by talking about the energy boundary. \end{description} 

\subsection{Leading from Desire}  
The instructor leads a conversation about the \textit{Leading from Desire} portion of the group reader. 

\subsection{Group Process}  
The class meets in small groups of three or four and discusses the elements that comprise different group experiences. One suggestion is to ask each member of each small group to discuss one good group experience, one challenging group experience, and their orientation to groups overall. A scribe may take notes for each small group, with emphasis on writing down the various things that represent good facilitation and poor facilitation. This is followed by a large group discussion to share results and to articulate the elements of effective facilitation.  

\subsection{Eight Pieces of Silk}  The session closes with \textit{Eight Pieces of Silk} or a similar movement exercise from yoga, Tai Chi, Chi Kung, Aikido, etc.

\section{Evening Three: Group Evolution}  

\subsection{The Mandala}  Using the mandala page from the reader, The instructor explains how groups form, and what considerations occupy the mind of the facilitator who is planning to run a group. The emphasis is on working from broad concepts to particular moments, and on using the flow and energy of the group as a guide. This occupies the first half of the session.  It is recommended that the instructor spend time of the general question of how to design and facilitate a group exercise (as the groups will be doing this themselves).  \subsection{Group Process}  Using the principle that each group exercise is an opportunity for personal challenge and development, the large group finds ways of arranging itself into small groups of three or four. The instructor is recommended to leave this to the groups themselves, rather than assigning membership. This exercise may be introduced by explaining that in any group, the members of the group are responsible for their own experience and for the formation and health of the group.  Once the small groups are formed, their assignment is to discuss various group activities they have done, or have heard about, that seem fun and useful. They will eventually design a group exercise to be facilitated by them in class; for now, the assignment is simply to talk about the kinds of group experiences that appeal to them.  The instructor's role, during this and each subsequent presentation group discussion, is to circulate and provide assistance.  The session ends with the facilitator explaining that the presentation groups will be given more opportunities in class to design their exercises, but they may also need to meet outside of class.

\section{Evening Four: Stages and Milestones} 

\subsection{Stages in Self-Awareness and Group Development}  The instructor discusses the various stages of group development and the manner in which these are precisely the same as the stages of personal development (and the stages of health and illness). The instructor uses the relevant chart in the course reader. The emphasis here is on the group as an organic process, and a reflection of human nature, rather than the group as a mechanism easily-deconstructed by simple models. This discussion occupies the first half of the session. 

\subsection{Group Process}  The instructor leads the group in an exercise which models the complex interactions in groups, using foam balls, yarn, or other creative means to concretize the flow of interpersonal energy within any group. The instructor also describes the challenge of being aware of as much of this dynamic chaos as possible, and how easy it is to miss small items that develop into large challenges.  The presentation groups are also given time, near the end of the session, to further discuss and plan their group exercise. In their second meeting, the presentation groups should decide on a topic and a basic approach. 

\section{Evening Five: Facilitator Statements}  
\subsection{What to Say, and What Not...}
The instructor discusses effective therapeutic statements, as outlined in the reader, and how such statements must both follow the stages and themes of the group and also must be consistent with the temperament of the facilitator. Also, the instructor explains how much of group work is an opportunity to re-imprint unfinished development themes for participants, and that effective statements are therefore a type of re-parenting.  Using the \textit{Ten Statements} page from the reader, the instructor also describes basic strategies for general verbal interaction.  The instructor also describes how different styles of facilitation apply to different contexts and groups, and indicates some of the considerations for planning divergent groups (such as the blend between structure and fluidity, for example, in different groups, and the importance of being attentive to group cultural norms).  These discussions occupy the first half of the session. 

\subsection{Group Process}  
The instructor leads the group in exercises involving relaxation, meditation, and movement. The precise design of these exercises is left to the facilitator.  The presentation groups are also given time, near the end of the session, to further discuss and plan their group exercise. In their third meeting, the presentation groups should decide on roles for individual members and a structure for their exercise. It is recommended that the instructor learn exactly what each group is proposing, and approve or modify each exercise as required. 

\section{Saturday Session: Group Workshop}  
The Saturday session is an opportunity for the group to experience an authentic workshop. It is recommended that the facilitator not teach on this day, but instead run the group as he/she would do in his/her professional work as a facilitator. As such, no specific guidelines as to how to facilitate the group are given here. The structure and form of the day are left to the facilitator's discretion.  Depending on scheduling requirements for a given semester, the Saturday session may fall near the middle of the course, or may be later. It is not recommended to offer the Saturday session before the group has met at least twice.

The schedule for the Saturday session normally runs from 9:30 am to 4:30 pm (or somewhat before 4:30, depending on the group energy).

The Saturday session normally takes the form of a self-development workshop for students, a forum for them to participate as if they were members of an authentic therapeutic group directed toward self-discovery. The extent to which the group challenges its participants, and the extent to which the participants challenge themselves, will depend entirely on the interpersonal chemistry of the group. However, it is recommended that the facilitator model safety, comfort, fun, and a non-challenging style that allows participants to discover their own levels of safety and, by extension, their own insights.

\section{Evening Six: Responding to Behaviors}  

\subsection{The Three Phases}
The instructor describes how to sense and work with various group behaviors, with particular emphasis on alliances, cliques, isolation, intensity, and conflicts. It is recommended that the instructor use the \textit{Three Phase} pages in the reader as a guide in this presentation.  

\subsection{Student Presentations}  
The first of the student-led presentations/exercises is held in session six or seven (depending on the size of the class and the overall scheduling of the course).  Before the first presentation, it is recommended that the instructor indicate to the class the importance of appropriate and helpful participation and responses on the part of students.  Each presentation occupies between 30 and 90 minutes.  After the first presentation, the class discusses how things went, with emphasis on positive feedback.

\section{Evening Seven: Emerging Models}  

\subsection{Innovation and Creativity}

The instructor discusses various trends in group work (creativity, open space, collaboration, etc.) and how these models work in the professional world of facilitation. Much of this session is left to the discretion of the facilitator; but it is recommended that the basic emphasis be on new models of group work that get people out of a circle, out of chairs, and into activities and the community.  Also, in this session the instructor may wish to discuss emerging models for specific types of groups, such as substance abuse, trauma, or self-discovery.  

\subsection{Student Presentations}  
The second of the student-led presentations/exercises.

\section{Evening Eight: Facilitator Development}  

\subsection{The Foundation of Self-Awareness}


The instructor discusses the importance of facilitator self-care and personal development. For this discussion, the instructor may wish to use the team development pages in the reader, the charts of personal development discussed in session four, the mask and shadow material from the reader, or other personal resources (as facilitator development is a highly personalized matter).

The central goal of this session is to describe and demonstrate that facilitation is less about skills, after all, and more about self-development.

\subsection{Student Presentations}  
The third of the student-led presentations/exercises.  

\section{Evening Nine: Trauma and Addictions}  

\subsection{A Personal and Professional View}


Many of the VCC counselling students have personal experience with trauma and addictions. Additionally, many will be working in these areas in their professional practice. Therefore, it is important for the instructor to cover these topics, with particular emphasis on the balance between resource and liability that having gone through a particular experience presents for the facilitator.  The experiences and insights of the group may easily be called upon in this session.

For this session, the instructor may wish to use the trauma chart in the reader.

\subsection{Student Presentations}  
The fourth of the student-led presentations/exercises.

\section{Evening Ten: Closure}  
This last session is for loose ends and closure. It is recommended that the instructor design a closure exercise involving sharing from the group about their experiences and insights. The list of goals from the first session may be reviewed, as well as any outstanding items of course content that require further attention.  Students often appreciate a potluck during this last session. Course evaluations complete the class.  
\end{document} 
%%% Local Variables:  %%% mode: latex %%% TeX-master: t %%% End: 