%!TEX TS-program = xelatex
    %!TEX encoding = UTF-8 Unicode
%    \documentclass[10pt, letterpaper]{article}
\documentclass[letterpaper,10pt,headsepline]{scrreprt}
    \usepackage{fontspec} 
    \usepackage{placeins}
    \usepackage{multibbl}
    \usepackage{graphicx}
    \usepackage{hieroglf}
    \usepackage{txfonts}
    \usepackage{url}
    \usepackage{titling}
    \usepackage{geometry} 
    \geometry{letterpaper, textwidth=5.5in, textheight=8.5in, marginparsep=7pt, marginparwidth=.6in}
    %\setlength\parindent{0in}
    \defaultfontfeatures{Mapping=tex-text}
    \setromanfont [Ligatures={Common}, SmallCapsFont={ITC Officina Serif Std}, BoldFont={ITC Officina Serif Std Bold}, ItalicFont={ITC Officina Serif Std Book Italic}]{ITC Officina Serif Std Book}
    \setmonofont[Scale=0.8]{Lucida Sans Typewriter Std} 
    \setsansfont [Ligatures={Common}, SmallCapsFont={ITC Officina Sans Std}, BoldFont={ITC Officina Sans Std Bold}, ItalicFont={ITC Officina Sans Std Book Italic}]{ITC Officina Sans Std} 
\usepackage[ngerman,english]{babel}
\usepackage{scrpage2}
\usepackage{paralist}
\clubpenalty=6000
\widowpenalty=6000
\author{Ross A. Laird, PhD}
\title{Interdisciplinary Expressive Arts\\ at Kwantlen}
\date{\today}
\ohead{Ross Laird}
\chead{IDEA}
\pagestyle{scrheadings}
\setcounter{secnumdepth}{-1}
%\contentsname{Contents}
\begin{document}
\pagestyle{empty}
\vspace*{7em} 
\begin{center}
\huge{Interdisciplinary Expressive Arts}\\
\vspace*{1em} 
\large{Testimonials}
\end{center}
\clearpage
\pagestyle{scrheadings}
\tableofcontents
\chapter{IDEA Testimonials}

\section{Short Testimonials}

"Every week when I came to class I felt very encouraged and inspired by all of the different creative ideas... I am walking away from this semester after putting in a great effort, with new skills and experiences that I will be able to reflect on forever."
\vspace{2em}

"I have never taken a course like this before. I think there should be more courses that follow this criteria. You allowed me to take my creativity to a whole new level. I think that creativity should be a bigger part of education because it is so crucial to every individual. Creativity shapes people as individuals and gives them personality. It also allows them to experience things they never have experienced before."
\vspace{2em}

"This course forced me to let go of the constraints I place on myself for constant organization and planning, for which I am grateful. I have been able to explore my creative side in a way I had always wanted to but never thought I had time for. I have found time for the kinds of things I enjoy and feel I have accomplished some real personal growth."
\vspace{2em}

"My progress within the semester in this class was been great for me not only academically but personally. I have gained so much. A simple little project turned into something wonderful. I built new relationships that I didn’t know existed."
\vspace{2em}

"I realized that this class is very different from other classes that I have taken previously. Therefore, I was going to drop this class because I was afraid to be in this unfamiliar learning environment, especially with my poor English skill. Furthermore, I am an introvert person who feels stressful every time need to present our own thoughts in front of the class. However, class activities create a positive atmosphere which allows me to learn how to express my individuality in a group setting. After a period of time, the supporting and encouraging atmosphere have totally changed my mind, and I gradually enjoy to be in this class... I like this class because I get to know and understand people, and it helps me to develop a better interpersonal skills... All in all, I do learn tremendous knowledge and have fun while participating in this class."
\vspace{2em}

"The thing that this class brought to my life was permission to create
and be creative. Though my reasoning was often related to a final
grade, I think that creating is something that needs to be a bigger
part of my life — even when grades aren't a part of it. I thought
differently about various things throughout the semester and the 3
hours in IDEA were often the time of the week where I actually
reflected about myself, and my family and the path that we're on. I
listened in IDEA."
\vspace{2em}

"I loved participating in the class. The atmosphere was very welcoming to speak out. All the students were very nice to each other and supportive. I think generally a student wanting to participate is much better than a teacher picking on or forcing a student to participate. You have created a classroom setting where I felt the need to participate because of the thought provoking topics and discussions."
\vspace{2em}

"Our IDEA class demonstrates the benefits of a non-traditional classroom format. Everyone has to pay attention in circle; there is no hiding. Everyone is an equal, including the teacher."
\vspace{2em}

"This IDEA class has opened my eyes to a new kind of learning. I can confidently say that I have never enjoyed a university course as much as this one. That being said I must explain how I view things. I think it’s very unfortunate that to succeed in life I’m having to go to university and take courses that I don’t even enjoy in order to get a piece of paper after four or five years that says I have taken blank number of courses and have passed with adequate grades. That piece of paper is then somehow supposed to give me an advantage in obtaining a good-paying job so that I can then buy a bunch of stuff that I may or may not need. It makes me very depressed to think that I have spent so much of my time being miserable, just so that I can work for the rest of life. This is why I liked IDEA class so much."
\vspace{2em}

"The weekly sessions of IDEA 3100 was probably the only academic session during this semester which kept me thinking after the class and even for while after that! The course moved me to my core. I was made to go deep into analysis and simply think about my life in general, and get away from the simple small things. This does not happen a lot, especially in an academic context, and I appreciate the structure and the content of this course."
\vspace{2em}

"IDEA 3100 influenced me in a productive way. I was constantly reflecting on the ideas thrown in the class (by the instructor and the other students). The instructor was truly motivating, and mind-opening for me. Overall, I was continuously engaged with the message of the course through self-reflection, tasks done as a result of a generated motivation, developing a creative approach to a desired healthy lifestyle as a projects, and finally doing a project and learning from it."
\vspace{2em}

"Thank you for a wonderful semester. I have fundamentally renewed how I operate. I’ve sought to add and welcome creativity to my life... The culmination of applying these tools and projects is that I am happier and more efficient, and have maintained my motivation to keep moving through it all. Also, being in the circle re-conceptualized my ideas of instruction and the classroom. I plan on becoming a teacher, and reflecting on this semester, my consciousness is now awakened to the way that traditional education and its content is formatted, constrained and determined through the physical arrangement of a classroom. Further, my eyes are now incredibly open to how children interact with their traumas. I will carry these lessons forward."
\vspace{2em}
\clearpage
\section{Letters of Support}

Some students have taken the initiative to write letters supporting the continuance and growth of IDEA at Kwantlen. (I have received permission from the students to include their names along with these letters.)

\subsection{Lee Beavington}

 The Interdisciplinary Expressive Arts (IDEA) courses offered at Kwantlen Polytechnic University challenge students to be the captains of their education. I have observed students, including myself, come out of their shells, contribute with purpose and meaning to class discussion and activities, and have transformative learning experiences.

I have taken three IDEA courses at Kwantlen. While each course provided a different focus -- mythology, science and religion, and writing for new media, respectively -- they all helped in development of my academic, social, professional and personal lives. The IDEA courses have a way of linking these sometimes divergent paths into a clearer whole.

Within the contextual framework of each IDEA class, and under Ross Laird's inspired guidance, I was able to achieve a number of significant goals. I am more confident as a presenter. My writing, both formal and creative, has dramatically improved. I gained clarity in my journey toward graduate studies. I honed skills in communication and group facilitation that prove invaluable in my current job as an educator.

I have enrolled in a wide spectrum of courses at Kwantlen that span across the arts and sciences. The IDEA classes have, unequivocally, had the most impact on my life. They have fuelled my passion for learning and have been a tremendous influence in my pursuit of a career in education and creativity.

\subsection{Tierney Wisniewski}

 I am about to graduate from Kwantlen with a BA in Psychology and somewhere in the neighborhood of a 4.2 GPA. I'm currently applying to MA programs in counselling psychology. I've taken a number of courses with Ross -- 5 at this institution, another at VCC. Two of the courses at Kwantlen have been IDEA courses. I enjoy creative writing, so the other three were under that banner.

I came to this institution knowing that I wanted to be a counsellor, and knowing that my BA was a necessary grind. I knew that most of my courses would give me a good intellectual background in psychology, and that was important, but it wasn't sufficient. I wanted to use my education to develop as a person. Now, I've had a few courses in psychology that helped me do that, and I've heard good things about the counselling courses (I took mine at VCC, so I've never experienced one here). But these courses are primarily for students taking degrees in psychology or minors in counselling. We need a banner under which students from all disciplines can talk about and work on the things that matter and grow as people. We shouldn't split our students into intellectual and non-intellectual sides and then commit to growing only half of them. This is madness.

At Kwantlen, building student community is a particular challenge. It's no surprise; I've been in many a course in which I've sat in the front row and never gotten to know the classmates who sat behind me by face or by name. It's kind of embarrassing for me, but it should be more embarrassing for Kwantlen. You can't ignore community in the classroom and then expect it to magically appear in extracurriculars. IDEA courses are the only ones, besides small seminars, in which I've gotten to know all of my classmates. We sit in a circle. Everyone participates. We get to know each other at a level that is unusual in the Kwantlen experience, and our ties extend past the posting of final grades.

It's for this reason IDEA courses attract the kind of students who are prone to being engaged and involved citizens -- the kind of students Kwantlen struggles right now to develop. Out of a class of 20, I would say that at least 5 of us participated in the SGM to oust the old KSA council members who have been causing us so much trouble. I also recognized a number of other current and past IDEA students helping out or voting. You just don't get that kind of commitment from a 3-hour lecture.

\subsection{Marion Buan}

This class does not get the credit it deserves. I am in my final year of post-secondary education and I am graduating with a BA in General Studies. If a university is offering such a degree, then a class like IDEA is absolutely fitting for it. More specifically, it is a course with useful subject matter that I could apply to my future endeavours. Learning about expressiveness, creativity and awareness of self and others is extremely relevant to my future profession (an elementary school teacher) and my everyday life. This is not an easy class and should not be viewed as such. It is challenging and, I feel, it is neccessary for people to experience challenge in their lives. It's the only way to discover our potential.

\subsection{Cameron Doyle}

I have been attending Kwantlen Polytechnic University since 2008, and just
recently took IDEA 3301: Mythological Narratives with Ross Laird. I took
this course for two reasons: first, it was formerly a Creative Writing
course, which meant that I could broaden my artistry as a Creative Writing
major. Secondly, Ross Laird's vast knowledge, great personality, and unique
teaching style are just shy of infamous. Many, many people know of Ross and
his courses -- and this isn't an easy feat.

The students across every campus of Kwantlen are much more disassociated with each other than at other
universities. As there's less of a student social scene, difficulties can
arise when it comes to learning, through other students, about great
courses and great instructors. Ross doesn't have this problem. If anyone
takes an IDEA course with Ross, they talk about it --incessantly. This is
because no one leaves an IDEA classroom without feeling like they've had a
great time learning. Furthermore, I have never left a class without
something written down to punch into Google -- extra-curricular study stuff.
This wasn't a requirement, this was Ross sparking genuine interest in me,
which spurred me to continue learning in my own spare time.

Another reason
people talk about Ross and IDEA is because of how perfectly suited they are
for each other. Ross works well within IDEA, because IDEA has just as much
personality as Ross does. You couldn't give IDEA a rigid structure in the
same way you couldn't throw a pair of slacks and suspenders on Ross and
make him speak a monotone physics lecture. It doesn't work like that. IDEA
is Ross' home, and IDEA is where Ross flourishes as an instructor. IDEA is
Ross and Ross is IDEA. Remove one from the other, and you're not only
losing out on a brilliant instructor, you're also missing out on a
genuinely engaging string of courses that are refreshing, informative, and
geared toward the students of today. IDEA is the future of higher learning.
If you have any doubts about any of this, I suggest you refer to his survey
scores or, better yet, speak to a few of his students, who are always
speaking highly of Ross and IDEA.

\subsection{Erica Ryan}

During this semester in IDEA 3100, I had the opportunity to explore new things and use my experiences to deepen my creativity and personal awareness. I started the semester doubting that I had any inner creativity, and I am ending this semester with the confidence that I do. By actively exploring my own creativity outside of the classroom (during individual and group assignments) and participating in creative class activities (during other group activities), I have broadened my creative abilities. I have also enjoyed watching and listening to other peoples’ creative experiences.

As a student at Kwantlen for the past six years, I have mostly aimed my studies towards academic areas, such as English, psychology, mathematics, and history. Needless to say, in these areas of study, there has not been much room for personal creativity; structure and ridged course outlines are what I am used to. In the IDEA class, I have thoroughly enjoyed the opportunities for personal creativity and project flexibility. This class gave me opportunities to explore my inner life, not just my academic, rational life. As the instructor, Ross (2011) states, “the inner life (of creativity, of self-awareness, of character development) is the route to fulfilment in so many ways”. I feel fulfilled when I am given the chance to choose my own areas of study and exploration. Each day that I come to IDEA class, I come with an open mind, ready to learn something new, or engage in a new art form. I honestly have enjoyed every activity we have done in class. However, the activities that I have enjoyed the most were the photo scavenger hunt, airplane making, and dancing “the Macarena”. During these three activities I was able to have more fun at Kwantlen than I have ever had. I have never participated in anything at Kwantlen other than going to class, listening to a lecture or presentation, writing notes, and handing in assignments. Running around campus taking unusual pictures, flying paper airplanes down the hallways, and dancing in a classroom are activities out of the norm of my usual studies. These activities freed me from the sense of boredom and sterility that Kwantlen usually fills me with. This IDEA class entailed a whole lot more than just my usual classroom duties. In IDEA learning is not only self-directed, it is fun.

I also enjoyed leading the class in creative activities. During my two group assignments, the class did a pumpkin carving workshop and made Christmas crafts based on Christmas traditions. I enjoyed the class discussions the most. During the pumpkin carving workshop, I enjoyed looking at everyone’s pumpkin creations and listening to why they chose the carving they did. Pumpkin carving for me is a very relaxing, creative process. It was fun to experience other people taking pleasure in the same activity. Each pumpkin took on its own unique creative entity. The class also seemed to enjoy the Christmas craft activity. The discussion brought out many different people’s experiences and traditions around the holiday season. I feel that these two presentations will serve me well when I am a teacher because I can use them as learning tools and creative activities in my future classroom.  
I would like to be an Elementary school teacher when my degree is complete, and I truly believe this IDEA class has given me a different way to look at the structure of a classroom. As a teacher, I would like to make my future classrooms have as many creative outlets for my students as possible, so learning can be fun and involve self-exploration. I think children can experience great pride in themselves when they have the opportunity to let their interests and abilities shine, in a safe and encouraging environment. Personally, I feel that a creative classroom is more relaxed and invites an opportunities for discussion and learning. I feel like I learn more when I enjoy what I am doing, such as crafts, sewing, and dancing, than I do when I am listening to a dry lecture and taking notes. I wish I had taken more IDEA classes during the completion of my Bachelor’s degree. However, I will be competing my degree this semester with fresh, new ideas about ways to introduce (and keep) creativity in the classroom.

\subsection{Marija Prodanovic}

First and foremost I would like to thank you (Ross) and my classmates in IDEA 3100 for an amazing semester. I feel lucky to have had the opportunity to be a member of a hands-on student directed class. Initially, I was very intimidated by the “structure” of this IDEA class because I have never been exposed to such a student-driven classroom setting. My challenge at the beginning of the course was to let go of the traditional course structure that I have been accustomed to for the past four years as a psychology student at Kwantlen. I was shocked to find out that the whole 3 exams one essay and individual/group project course outline fiasco did not apply to this class whatsoever. I’ve heard the saying that “letting go is the hardest thing to do” and believe me that is definitely an understatement. I feel like I have changed quite a bit in terms of my understanding of what learning is and how we can teach information to others.

My classmates are all so creative, inventive, understanding, and supportive and I feel privileged to have been able to participate and actively engage in the facilitation groups and activities my classmates had allowed us to be a part of. I never thought that I would have learned about so many interesting topics. I found that this class forced me to think outside the box and not be so concerned about criteria or constraints in terms of time, page numbers, hours of work etc. Presenting my first individual project was nerve racking and I was afraid as to what my classmates would think of my art. The feedback I got was great, I was so happy to get pointers about the best way to preserve and protect my canvas and even where to get the supplies to do.

On a more personal note, taking this class has helped me reorganize my life. I have taken your advice on “de-cluttering” different areas of my life and so I am now an advocate of list making and regular e-mail checking and funny enough I sleep more soundly at night. In terms of the in class discussions we had on closure, I have caught myself telling my friends and family on the importance of finishing things and how it will be of benefit to them. This semester I have really focused on taking on tasks one day at a time since I tend to get overwhelmed easily and stress over the simplest things. Now I focus on accomplishing three major tasks a day and I am attempting to “let go” of negative feelings if everything is not accomplished. I also find that I have been more aware of my participation in the classroom setting and I feel that I have made an effort to speak up more frequently in class discussions even though at times I may not feel comfortable doing so. Public speaking is something I still have to work on but the acceptance I felt from my classmates really helped this semester. I really enjoyed the deep conversations we would get into and I felt really fortunate to have been able to hear about how passionate some of our classmates were about their interests and individual projects they presented.

I wish that I would have taken a class similar to this during my first year at Kwantlen and continued to do so until I graduated because it allowed me to explore a different aspect of myself, a more inventive and creative me. This class pushed me to relax and reflect on my true feelings about the personal goals I would like to accomplish and not just on the grades I want to get or which courses would look good on a potential resume. I wish I would have had the opportunity to register in more IDEA classes in the future but I will be graduating this semester and it is the time for a new adventure. I know that the experience I had in IDEA 3100 will help me think of creative ways to teach the new program at A.H.P Matthew Elementary school and I am thankful for being a part of such an amazing classroom community. Thank you.

\subsection{Gurp Chahal}

Just a few months ago, a friend had informed me about this course and I was quite surprised because I did not know these courses were even offered at Kwantlen. After reading a brief description of the course, I instantly wanted to sign up because I have never taken a course like this before, and in a way it would be a good way for me to end my year in Kwantlen as I will be graduating. When it comes to interacting with smaller groups, I think I am very social, however when it comes to speaking out loud in front of a larger group, typically a classroom, I become really shy and try to avoid speaking out loud. Therefore, I set a goal for myself to become more comfortable when speaking publicly and I thought this course would be the perfect opportunity for me to achieve this goal. I have to admit, when I first stepped into the classroom, I felt really anxious and was really hoping that Ross would not pick on us on the first day.

Many teachers do not realize that students have a fear of public speaking and often pick on students with questions. But what I really liked about Ross is that he addressed that issue on the first day of class by meditating and calming our nerves down. Although I did still feel a little anxious, it just made me feel better knowing that Ross is aware of this fear. Throughout the whole semester, he never forced an answer or an opinion on any of his students; instead he allowed us a choice as to whether we wanted to share something with the class. If I have to compare my presentation performance from my first and last project, not only did I see the difference but I felt it too. On our last day of class, I felt completely comfortable speaking to my classmates about my experience with the last project. I even left the class with a smile on my face because I felt as if I achieved a huge goal.

I would definitely recommend this course to other students attending Kwantlen because this course forces you to think about things that you normally would not have time for with our busy schedules. It also gives you an opportunity to explore your options and learn more about what you enjoy doing. In a way, this course took me away from reality and made me feel free and relieved from the stress we experience in our day to day lives. If these courses were offered when I first attended Kwantlen, it would definitely help me in terms of career paths. I know a number of students who take courses for the sake of taking it, and eventually come back to school because they felt as if they made the wrong choice. Instead of making appointments with educational advisors to seek advice, why not offer classes like Interdisciplinary Expressive Arts, so students feel a sense of achievement when graduating with a degree they feel proud of. Normally, I often forget the material I have learned in my academic courses, but my experience in this course is something that I would carry with me for the rest of my life. Considering that I am hoping to become an elementary school teacher, I know I will apply the things I learned from this course and from my classmates, into my lesson plans. Overall, I absolutely loved this course. 

\subsection{Adam Denson}

 When I first attended Kwantlen in the fall of 1999 I would sit near the back of the class and would rarely open my mouth unless called upon. When I worked my way up with Safeway going from a stock boy at 19, to a meat manager at 24, and finally a district merchandiser I was still pretty much the same in terms of keeping quiet and not really speaking up unless it was about something really important. I have slowly come out of my shell and a big reason for that is having a class like IDEA that has provided a place where I feel confident and comfortable expressing myself and sharing my ideas with other peers. I now coach senior boys basketball at North Surrey Secondary and have a job working for the VGH Thrift Store Society a non profit organization. I work with volunteers who are mainly young university students or grown-ups who have a mental disability.

I think that students who get though the rigours of university and who have some of the quiet shy tendencies like myself, will struggle or find themselves unhappy in the working world if they do not have the opportunity to take classes like IDEA that provide a space for students of all different disciplines to share and express themselves and their ideas on a variety of topics and feel comfortable and gain confidence while doing so. The experiences that I have had this semester in IDEA helped further remove me from my shell. I have been challenged by pushing my boundaries and rewarded with experience and new friendships with peers. I have been able to use this course to help counsel myself on who I am, where I am going, and how to get there. This has been one of the few courses that I have taken where I take it home with me, meaning I discuss with others outside of class about something we talked about or did or I will try something that was suggested in a class discussion. I was very happy with the semester and am looking forward to build on it in the spring, in IDEA 4100.

\subsection{Gurveen Kaur}

IDEA courses foster independence and offer a multidisciplinary alternative to traditional education. They allow us to expand beyond the ‘narrow-focus’ of any typical academic discipline. Through the various we completed in class we were able to explore our own creativity and expand our research on any given topic from a superficial level to something more substantial. The expressive arts projects that we worked on help develop our integrative thinking skills. They further our knowledge through the use of modalities such as creative writing, fine arts, photography and various arts and craft projects. I’ve enjoyed every single project that I worked on irrespective of how it turned out. Each one of them enriched me and taught me something meaningful, be it about myself or something other than that. I am looking forward to more learning.

\subsection{Sandy Patrola}

The positive, warm, encouraging and safe environment of this class really provided the perfect atmosphere for personal growth. I know this class was about letting our creativity flow, but honestly, it provided me with much more than that. I learned how to manage my emotions better when things weren’t going as planned. I learned how to come up with other strategies and methods to complete a task when my original strategy wasn’t working. I learned how to apply creativity to my work life, to make sure I actually had some fun at work. I learned how to be a kid again, and do arts and crafts and not feel too old or immature for doing so. I learned that having a creative outlet is healthy for our bodies, our minds, and our souls. Just overall this course taught me to not take life so seriously, and to actually enjoy school and enjoy the time we have with other people. We take such very little time out of our busy lives to spend on ourselves, and our loved ones. Creativity really seems to bring people together. For example, it was pretty amazing to see how out class interacted with one another during presentations and the hands-on components. Overall I really enjoyed myself in this course and look forward to applying the skills and lessons learned from this class into my life. 

\subsection{Meghan Kinnarny}

What are ideas?

Ideas are the innovations made by those who want change. Ideas are the inventions made by people who want to improve something. Ideas are thoughts made into tools to build a better future. Ideas are dreams made into a reality by those who care enough to see them through. Ideas are made from curiosity, creativity, and critical thinking. Ideas are what make life experiences beautiful and righteous. Ideas are what drive the human species to succeed.

An Idea is possibility made from impossibility. An Idea is a challenge. An Idea is change. An Idea is made from everyone. 

IDEA is where ideas are made from everyone who want change, who want to challenge the modern education system, to better our school and the people within it. Kwantlen is in need of ideas. We are in need of change, and we should challenge the education system that dictates the definition of “higher learning”. IDEA should not just be a course here at Kwantlen, it should be the idea that supports what we stand for. Kwantlen stands for everything that an idea is; so why are we getting rid of it? 

We are not Simon Fraser University or the University of British Columbia. We are Kwantlen Polytechnic University. We offer the most diverse set of courses and programs compared to other Universities, students don’t have to pay sixty thousand dollars a year at the Art Institute for a Certificate in Fashion, they can come here and get a Bachelor. We are the most racially diverse University in the country, supporting people who have different religions and cultures and making them feel welcome. We are one of the few schools that support a creative learning process that other Universities would “look down on”, because it’s not the norm of the modern classroom. We support the Foundations of Excellence, which states that the education system should be changed to a more hands on, engaging, interdisciplinary curriculum; where students are not listening for three hours to a professor who copies notes directly from the textbook instead of his or her own knowledge. Where students can sit away from the table and engage with one another, and learn interpersonally with each other’s experiences, help each other, encourage each other, learn from not only the professor but with everyone else. I want to learn at Kwantlen. And from what my experience is so far, of three years, I am not learning anything. I am sitting at a table for three to nine hours a day, listening to someone rant from a chapter in a textbook, and assigning a paper that I have to recite the exact words and opinions of my professor. I say opinions because the only ideas that spawn from the classroom nowadays are opinions; the rest of teaching is from books and not from someone’s personal knowledge from where they trained. A professor needs a Ph.D to teach a class, so why are they reading off of a textbook? Why are professors not being placed at a more equal level to their students if we are relying on experience-based knowledge? And if that’s what we learn from, why is this not a part of the classroom all the time? 

This is the interdisciplinary element that we are not supporting at Kwantlen, because we think we need to sit students in a strict environment, and subject them to learning system that fails. When I leave a lecture based classroom at the end of my semester, I don’t remember anything from that course. Why? Because I didn’t learn anything. I regurgitated notes from a professor and gave it back. I answered A, B, C, D on my final exam. I wasn’t encouraged to involve my ideas in the classroom and was instead ridiculed for them and told I was wrong. I was called flat out stupid by a teacher once, and she told me that I couldn’t write because my paper wasn’t properly formatted. I took one idea from that situation, and that was to question why the ability of students and their quality of work is considered inferior because of an outdated, strict education system that we call modern?

The future of education resides in the interdisciplinary creative process, where students are allowed to question, they are allowed to think outside of the classroom, and they are allowed to experience a class where learning is vital. And learning is vital through the physical engagement between students, and their professors. In IDEA, I sat in a circle with my other classmates and my professor, I remembered all of their names, I was allowed to talk to the professor like he was my equal, I wasn’t made to feel like I was another stupid student cashing out my hard earned money to a professor who would most likely forget who I was, and not care about what I had to say. We worked as a team to achieve what other classes think they’re achieving; an experience that changed our lives. We learned so many things that other classes can’t hope to teach their classes, because IDEA is interdisciplinary.

The difference between IDEA and the modern lecture-hall-classrooms is that I believed that I could learn something from this class. The questions were “how” instead of “why”, and “how?” is the question that all inventions and major changes start with, so my question is “how” can Kwantlen expect to become the first interdisciplinary school, and better our education system by introducing a more creative process, if we plan to veto the only set of classes that support our supposed change for the future?

\subsection{Abbey Ratcliff}

Firstly, I would really like to say that this is a much-needed course for any student. It teaches you to find creativity in everything you do, even if you have never considered yourself a creative person. We don’t realize just how much we (should) rely on our own thought production. Unfortunately, the tendency is for students’ opinions to become products of the amalgamated information we glean from textbooks. What I have found is that this is more prevalent in some disciplines than in others. 

I believe that this course, and especially one for 1st and 2nd year students, would help students ‘find themselves’, and allow them to make more confident choices in terms of ‘what they want to do with their education’. Because the projects are so self-driven, there are endless possibilities to explore what interests you and try things you might normally not try. The fact that it is such an informal class also helps those who are not confident in interpersonal situations, because it is an open and fairly judgment-free environment, where students are encouraged to talk to one another and engage in fun and interesting activities.

Although it may be unorthodox, this teaching style and course design is what is needed in order to create new ideas and promote creative thinking. Even though I am a very outgoing person, quite comfortable in many situations, I still find that I am inhibited in many of my formal, traditional classes. I find that I have a steadier flow of ideas and stronger problem solving skills in this class, perhaps even more than in my fine arts classes, believe it or not, potentially because we still have constraints and boundaries to curb our creativity.

I would love to take more IDEA courses in the future (as a matter of fact, I’m registered for Myths and Narratives for Spring 2011), because I feel that it has helped me in many aspects of my life, in and out of school. I plan on promoting the class to as many other students as possible so that they can try something other than ‘information regurgitation’ because I believe it could be as beneficial to them as it has been to me.

\subsection{Lisa Burgis}

IDEA 3100 has been one of the best classes I have ever taken at Kwantlen. I have never felt closer to a classroom of people either. I think the biggest thing that I personally got from this class is that creativity should be in my everyday life -- it isn't just for people who have lots of time, or who are artists. I need to bring creativity into everything I do. I also learned that creativity makes happiness -- after every class I was instantly happy and proud that I accomplished something really meaningful and cool. Learning new art forms and experimenting with my own skills was very rewarding because I realized I can do it -- I don't need to be an artist, I just have to be me!

The thing I will take away from this class is that creativity has weight and value and I need to bring it into my own life and those I care about. I have done several art projects I learned from IDEA and recreated them with my 7 year old niece because I want her to see her own skill and feel proud. I also have put a greater value on making things with my hands; the feeling I got from doing projects in this class with my own hands was something I had never really felt before but it made me proud and made me feel good about my own creativity. Our discussion on procrastination also really helped me; I tend to procrastinate a lot! I bought David Allen's book and am finding already through reading it half way through that it really is easy to get your life together and feel more free and weightless by organizing your tasks!

I will miss this class a lot; it was the highlight of my school week because every time I went I felt happy and satisfied and ready to take on the rest of my week. I am sad to leave Kwantlen, but also sad that this class has to end -- I really will miss all our activities, discussions, and laughter. I hope IDEA becomes a staple at Kwantlen and I feel blessed I was able to experience it.

\subsection{Punreet Nijjar}

Coming into the IDEA class I was extremely nervous about what the class and the experience would be like. I am usually shy and am not a big fan of speaking in front of large crowds. My biggest fear was that I would end up having to do a presentation all by myself and that I would mess it up big time. However, I went into this class with an open mind and I told myself that my goal by the end of this class was to learn to be more comfortable with myself and to help bring out my creative side. And I feel that is exactly what I am walking away with. I feel much more comfortable voicing my opinion in front of the class and I noticed myself speaking out more in my other classes as well. I have also learnt so many different art projects that I have been teaching my daycare kids, and have also found myself looking at ways in which I can create new art projects. This class has been amazing!

It was one of the most interesting classes I have ever taken, and it allowed me to try things that I normally wouldn’t have ever tried. I was able to show off my creative side that I didn’t even know I had, and I am also more willing to try new things now. I didn’t think of myself as being an artistic person, but I have come to realize being artistic isn’t just about how well you can draw or paint, but about how you express yourself and the different ways in which you can do so. I am extremely proud of myself for successfully completing this class and achieving the goal I had first set for myself upon entering the class. I feel like a different and more confident person now and I am ready to take on the next IDEA class.


\end{document}
