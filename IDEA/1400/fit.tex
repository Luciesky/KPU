%!TEX TS-program = xelatex
    %!TEX encoding = UTF-8 Unicode
%    \documentclass[10pt, letterpaper]{article}
\documentclass[letterpaper,10pt,headsepline]{scrreprt}
    \usepackage{fontspec} 
    \usepackage{placeins}
    \usepackage{multibbl}
    \usepackage{graphicx}
    \usepackage{hieroglf}
    \usepackage{txfonts}
    \usepackage{url}
    \usepackage{titling}
    \usepackage{geometry} 
    \geometry{letterpaper, textwidth=5.5in, textheight=8.5in, marginparsep=7pt, marginparwidth=.6in}
    %\setlength\parindent{0in}
    \defaultfontfeatures{Mapping=tex-text}
    \setromanfont [Ligatures={Common}, SmallCapsFont={ITC Officina Serif Std}, BoldFont={ITC Officina Serif Std Bold}, ItalicFont={ITC Officina Serif Std Book Italic}]{ITC Officina Serif Std Book}
    \setmonofont[Scale=0.8]{Lucida Sans Typewriter Std} 
    \setsansfont [Ligatures={Common}, SmallCapsFont={ITC Officina Sans Std}, BoldFont={ITC Officina Sans Std Bold}, ItalicFont={ITC Officina Sans Std Book Italic}]{ITC Officina Sans Std} 
\usepackage[english]{babel}
\usepackage{scrpage2}
\usepackage{paralist}
\clubpenalty=6000
\widowpenalty=6000
\author{Ross A. Laird, PhD}
\title{Interdisciplinary Expressive Arts\\ at KPU}
\date{\today}
\ohead{Purpose and Rationales}
\chead{IDEA 1400}
\pagestyle{scrheadings}
\setcounter{secnumdepth}{-1}
%\contentsname{Contents}
\begin{document}
\pagestyle{empty}
\vspace*{7em} 
\begin{center}
\huge{Interdisciplinary Expressive Arts 1400}\\
\vspace*{1em} 
\large{Purpose and  Rationales}
\end{center}
\clearpage
\pagestyle{scrheadings}

\section{IDEA 1400, Community Performance and Theatre Exploration}

\subsection{Brief Description of Course}

Learners will explore performance and theatre within an expressive arts
context. They will develop performance skills and will present theatre
productions to various communities within and beyond KPU.

\subsection{Program Fit}

This course fulfills a long-standing initiative within IDEA to provide
curriculum specifically focused on expressive arts performance. The IDEA
steering committee has identified performance curriculum as a priority,
IDEA developers have been working toward performance curriculum for some
time, and learners have been requesting this kind of course since the
inception of IDEA.

There is no similar course at KPU.

No prerequisites will be required.

\subsection{Faculty and Institutional Fit}

This course utilizes specific modalities such as inquiry-based
curriculum, reflective scholarship, intensive interaction, and
interdisciplinarity. These approaches are all listed in the KPU Academic
Plan as being foundational priorities. Additionally, the Academic Plan
highlights the importance of values such as global citizenship,
inspiration, compassion, imagination, and personal growth. These values
are core elements of this course and of the IDEA philosophy in general.
The Academic Plan further emphasizes skills such as effective
communication, openness to diversity and inclusion, and creative and
critical thinking. Again, these are all integral to the course and to
the IDEA process. The broad goals articulated for KPU in the Academic
Plan are embodied by this course and are reflected in the core processes
and outcomes of the curriculum.

To illustrate the correlation between the course outcomes and the KPU
Academic Plan, we have emphasized (below) the specific areas in which
the learning outcomes for this course are aligned precisely with the KPU
Academic Plan.

\begin{itemize}
\item
  \emph{Explore and reflect upon} performance and theatre experiences as
  opportunities for \emph{developing the creative imagination, a
  critical sensibility, citizenship, and community engagement}
\item
  Utilize the context of \emph{mentorship} and the activities of
  expressive arts performance to achieve \emph{personal growth}
\item
  Interpret and apply \emph{diverse traditions} of performance and
  theatre
\item
  Develop the \emph{creative imagination} through performance,
  \emph{mentorship}, and \emph{community engagement}
\item
  Improve skills of \emph{communicating effectively} in speaking,
  listening, writing, and performing
\item
  \emph{Join and contribute to} an expressive arts \emph{community}
\item
  Create \emph{interdisciplinary creative projects} using strategies
  developed through exposure to a performance-based learning environment
\end{itemize}

We find similar correlations when we examine the KPU Strategic Plan. The
core vision of the Strategic Plan is as follows, with emphasis added
where the language matches the outcomes of the course under review:

\begin{itemize}
\item
  \emph{Inspiring} educators
\item
  All \emph{learners engaging in campus and community life}
\item
  \emph{Open and creative learning environments}
\item
  \emph{Relevant scholarship} and research
\item
  Authentic \emph{external and internal relationships}
\end{itemize}

The Strategic Plan goes on to describe the importance of distinctive
programming, innovative teaching and learning, experiential learning,
enriched student experience, and purposeful community engagement. We
have illustrated (below) how each of these core goals for KPU matches a
core outcome for this course.

\begin{itemize}
\item
  \emph{Distinctive programming} corresponds to the unique
  performance-based curriculum of the course.
\item
  \emph{Innovative teaching and learning} corresponds to the context of
  mentorship, creativity, and community engagement in the course.
\item
  \emph{Experiential learning} corresponds to the learning environment
  of IDEA, which is entirely experiential.
\item
  \emph{Enriched student experience} corresponds to the experiential
  learning environment focused on creativity, mentorship, engagement,
  and personal growth.
\item
  \emph{Purposeful community engagement} corresponds to performances in
  the community (by the class) and a community of learning constructed
  by the learners (in the class).
\end{itemize}

The Arts Academic Plan is aligned with the KPU Academic and Strategic
Plans, and we therefore see similar correlations in values, approaches,
and goals. For example, the Arts Academic Plan specifically emphasizes
goals such as the enhancement of experiential learning, the wider
adoption of interdisciplinary approaches, the importance of preparing
learners for global citizenship, and the crucial role of mentorship. All
of these broad goals are specifically articulated in the course outline.

There is a high degree of symmetry between the stated goals of KPU, on
many levels, and the articulated goals of this course. We believe that
this course embodies, in a very fundamental way, the kind of institution
that KPU has said it wants to be.

\subsection{Evidence of Demand}

Research about the demand for post-secondary theatre courses among
recent high school graduates has been conducted by one of our partners
in the Surrey School District. This data, which includes information
from all areas served by KPU (Surrey, Richmond, Delta, and Langley),
shows that there is strong interest in post-secondary theatre
curriculum. The majority of high school graduates who have taken theatre
courses in high school seek experiential theatre courses in their
post-secondary programs. However, at the moment, KPU does not provide
such curriculum. Accordingly, there is strong demand for this course
from incoming learners.

There is also strong demand from current learners at KPU. Every
semester, the IDEA program engages in consultation with learners about
program development (as IDEA is built by learners as well as faculty).
In each of the last three iterations of this consultation (over the last
three semesters), performance and theatre curriculum has been at the top
of the list of requested new courses. In addition, students in past
iterations of upper-level drama courses offered by KPU's English
Department have expressed a strong desire to take theatre courses at
KPU.

The IDEA steering committee has also highlighted performance and theatre
curriculum as priorities. Such curriculum fills a current gap in our
offering. The discipline of Expressive Arts utilizes a suite of
modalities that typically includes art, music, movement, writing,
storytelling, craft, nature experience, drama improvisation, and theatre
performance. At the moment, our curricular offerings include all of
these modalities except theatre performance.

\subsection{Consultations}

We have consulted with KPU's President, with the Dean of Arts, with our
internal IDEA faculty cohort, with the Surrey School District, with
colleagues among various departments at KPU, with colleagues teaching
high school theatre curriculum, and with learners at KPU. All of these
consultations have resulted in unequivocal support for the course.

Several faculty colleagues have expressed interest in having students
from IDEA 1400 perform in their classes. These colleagues include Tracey
Kinney (History), Sam Migliore (Anthropology), Amir Mirfakhraie
(Sociology), Wendy Smith (English), and Jennifer Williams (English).
Additionally, Lucie Gagne (Design), Aaron Bushkowsky (Creative Writing),
and Nicola Harwood (Creative Writing) have expressed strong interest in
possible collaborations with this course.

Once the course is approved, we will meet further with high school
theatre teachers to forge a working relationship that will allow us to
perform for high schools. This initiative will be facilitated by the
fact that many of the KPU students enrolled in the course will have
taken theatre at local high schools. In addition, once we have developed
dramatic material appropriate for elementary school students, we will
approach elementary schools with the goal of performing for them.

We have begun consultation with the Gateway theatre about a
collaboration involving theatre curriculum (with very positive results),
and after the course is approved we will also consult with other
community theatres. Working relationships will inevitably follow, as
there is strong interest from local community theatres to collaborate
with post-secondary institutions.

\subsection{Interdisciplinary Potential}

IDEA is, by definition and core purpose, an interdisciplinary
initiative. (The ``I'' in IDEA stands for Interdisciplinary.)
Accordingly, this course will be interdisciplinary. Moreover, the
instructor for this course will be developing relationships with faculty
in other departments with the goal of offering performances to enhance
the student experience in other fields. For example, History students
might benefit from performances involving historical events or dialogues
from history. English students might benefit from performances of scenes
from plays, stories, or entire plays they are studying in English
classes. Sociology students might benefit from performances involving
sociological situations. Anthropology students might benefit from
performances involving cultural contexts or situations. Business
students might benefit from performances and simulations of various
situations (interviews, meetings, presentations, informal business
situations, customer service, etc.) that can be dramatized effectively
and thus analyzed and understood more effectively by Business students.
In fact, a group of IDEA students could dramatize a given business
situation in alternative ways to illustrate the pros and cons of various
approaches to business.

This cross-disciplinary pollination can be extended to any field of
inquiry at KPU. In each case, the IDEA instructor will consult with
colleagues in such departments to maximize the integration of
performances with the curriculum of the other course (including the
alignment of assignments with performances, if desired).

Finally, the study of performance and theatre is inherently
interdisciplinary. Learners will work on and study plays, rituals,
theatres, and performances from various cultures and time periods, often
integrating different disciplinary approaches (e.g.~dramaturgical,
psychological, sociological, cultural, historical) as appropriate to the
demands of the given text to be performed and the anticipated
performance context (e.g.~location of performance and audience make-up).
The core approach --- in texts, activities, and performances --- is one
that addresses theatre from across the world and throughout history. A
global perspective, in other words

\subsection{Additional Considerations}

This is a bare-bones theatre and performance course. All we need is a
classroom to perform in and permission to perform in KPU common areas
(indoors and outdoors). Ideally we would also like to utilize the
Conference Centres (in Richmond and Surrey) as well as the Langley
theatre for special performances, but even this is not absolutely
required.

Some performances will be held off campus. These will be similar to
regular field trips and have been approved in principle by risk management.

\end{document}
