    %!TEX TS-program = xelatex
    %!TEX encoding = UTF-8 Unicode
    \documentclass[10pt, letterpaper]{article}
    \usepackage{fontspec}
    \usepackage{placeins}
    \usepackage{hyperref}
    \usepackage{multibbl}
    \usepackage{graphicx}
    \usepackage{txfonts}
    \usepackage{geometry}
    \setcounter{secnumdepth}{1}
    \geometry{letterpaper, textwidth=5.5in, textheight=8.5in, marginparsep=7pt, marginparwidth=.6in}
    %\setlength\parindent{0in}
    \defaultfontfeatures{Mapping=tex-text}
    \setromanfont [Ligatures={Common}, SmallCapsFont={ITC Officina Serif Std}, BoldFont={ITC Officina Serif Std Bold}, ItalicFont={ITC Officina Serif Std Book Italic}]{ITC Officina Serif Std Book}
    \setmonofont[Scale=0.8]{Lucida Sans Typewriter Std}
    \setsansfont [Ligatures={Common}, SmallCapsFont={ITC Officina Sans Std}, BoldFont={ITC Officina Sans Std Bold}, ItalicFont={ITC Officina Sans Std Book Italic}]{ITC Officina Sans Std}
    % ---- CUSTOM AMPERSAND
    \newcommand{\amper}{{\fontspec[Scale=.95]{StoneSansStd-MediumItalic}\selectfont\itshape\&}}
    % ---- MARGIN TASK (year, task, etc.)
    \newcommand{\task}[1]{\marginpar{\small #1}}
    \usepackage{sectsty}
    \usepackage[normalem]{ulem}
    \sectionfont{\sffamily\mdseries\upshape\Large}
    \subsectionfont{\sffamily\mdseries\scshape\normalsize}
    \subsubsectionfont{\sffamily\mdseries\upshape\normalsize}
    \begin{document}
    \thispagestyle{empty}
    \reversemarginpar
    \noindent
    \includegraphics[scale=0.28]{/home/rosslaird/Dropbox/kwantlen/kwantlen_logo}\\[1em]
    {\LARGE First-Year Initiative}\\[2em]

In July 2012, the first phase of the 2011-12 Foundations of Excellence
(FoE)program produced reports and recommendations from each of the nine
FoE dimension committees. The resulting Executive Summary identified
specific themes within the recommendations. In February 2013, the FoE
steering committee reconvened under the revised heading of KPU First
Year Initiative (KPU -- FYI) to review this work. The committee
aggregated the recommendations according to themes (rather than
dimensions) and worked to identify correlations between and across themes.

Concurrent to the work of FYI, Kwantlen also began developing and implementing a formalized Academic Plan. This planning process resulted in many key insights (about international education, the needs of Aboriginal students, and so on) and also confirmed the critical importance of first-year experiences. The FYI results therefore complement and augment the themes of the academic planning initiative, which focus on a time horizon of five years. Accordingly we have included the following FYI recommendations within that five-year time frame.

We also make the general recommendation that the KPU academic plan incorporate, as much as possible, the FYI material. Both of these broad initiatives ask foundational questions about what what KPU will look like in five years, what kinds of experiences we want for our students, and what our learning environments should provide for our community.

We expect (and hope) that the positive momentum for support of first-year students at KPU will continue to increase. In order to facilitate this momentum, we advocate systematic research about the first year, ongoing governance and process for first-year experiences, specific first-year courses, and a robust set of services for first-year students. These and many other recommendations are grouped thematically below and form the core of our results.

\section{Leadership}

Senior leadership at Kwantlen must endorse a First Year Student
Experience Committee to envision, develop, and implement an integrative
approach to improving first-year experiences for our students.

\subsection{Recommendations}

\begin{itemize}
\item
  Create a Standing Committee responsible for oversight and coordination
  of first year initiatives.
\item
  Etablish a Standing Committee on the first-year student experience
  reporting to Kwantlen senior leadership via the AVP Students.
\item
  Implement an ongoing ``First Year Experience'' Council, reporting to
  the AVP Students, that should be developed to continue work on FoE
  projects and implementations.
\end{itemize}

\begin{itemize}
\item
  Dedicate resources to support the integration and coordination of
  efforts across departments and faculties (for example, a framework for
  supporting faculty with release time and staff from appropriate
  service areas).
\item
  Establish and support a coordinated and integrated unit that focuses
  on first-year student best practices for teaching and learning.
\item
  Provide a consistent budget for central functions as well as
  divisional budgets for first-year organizational structures (not
  necessarily centralized).
\end{itemize}

\begin{itemize}
\item
  Create a forum for engagement in discussions and collaborations for
  first-year initiatives.
\item
  Ensure that staff members are included in invitations for dialog and
  discussion regarding first-year student support, trends and
  transitions, and that sufficient resources are allocated to support
  this staff inclusion.
\item
  Establish intentional faculty-involvement programs and practices to
  encourage formal and informal engagement between faculty and students.
\end{itemize}

\begin{itemize}
\item
  Identify and share knowledge about issues specific to first-year
  students.
\end{itemize}

\begin{itemize}
\item
  Further develop initiatives focused on the first-year experience (such
  as curriculum specifically designed for the first year).
\end{itemize}

\begin{itemize}
\item
  Disseminate survey results to the university community and identify
  who has responsibility for further action.
\end{itemize}

\section{Philosophy}

Kwantlen must craft and embrace a philosophy statement for the
first-year experience.

\subsection{Recommendations}

\begin{itemize}
\item
  Adopt an explicit institution-wide philosophy for the first-year.
\item
  Establish a transition philosophy.
\item
  Develop and communicate a philosophy of intent (purpose) for
  first-year student experiences.
\end{itemize}

\begin{itemize}
\item
  Establish University-wide aligned learning and life experience goals
  for the first-year.
\item
  Establish University-wide learning goals for entry-level courses.
\item
  Craft a clear mission statement on diversity that articulates the
  underlying philosophy behind Kwantlen's approach to first-year
  experiences.
\end{itemize}

\section{Culture}

Kwantlen must develop a core identity and institutional culture based on
shared values and practices that inform our fundamental approach to
teaching and learning.

\subsection{Recommendations}

\begin{itemize}
\item
  Expand of first-year student activities (both in and out of the
  classroom) to encourage interactions with people from different
  cultures, orientations, and abilities.
\item
  Enhance disability awareness and mental health awareness on campus.
\item
  Identify and remove institutional barriers that prevent access and
  hamper student success, especially among under-represented groups.
\item
  Promote curricula and instruction that reflect diversity.
\item
  Educate the Kwantlen community (faculty, staff, community partners)
  about the importance of supporting and encouraging first-year
  students.
\item
  Invite students representing a specific group or set of interests to
  speak at faculty meetings.
\item
  Offer more training programs to help faculty and staff become better
  informed about how to effectively support diversity efforts.
\item
  Support research projects focused on first-year experiences.
\item
  Provide time release funding for faculty to engage in research
  projects focused on the first-year experience.
\item
  Provide various mechanisms for professional development for faculty
  who wish to deepen their knowledge about first-year experiences.
\item
  Engage in dialog about first-year experiences within departments and
  divisions (at formal meetings, for example) with the explicit goal of
  supporting first-year students both in and out of the classroom.
\item
  Develop branding to be used by instructors who are interested in the
  specific promotion of enhanced first-year experiences.
\item
  Develop a course or community series on diversity and global learning
  to help students, faculty, and staff to explore different cultures and
  world views.
\item
  Develop experiential approaches to teaching and learning.
\item
  Develop outcomes-based, engagement-focused curriculum for the
  first-year.
\item
  Prioritize resource allocations to support initiatives to recognize
  and support faculty and staff involvement with the first-year student
  experience.
\item
  Encourage the faculty and staff evaluation and reporting process (peer
  review reports, self-input, PD/AT plans) to include opportunities for
  comments on and reflection about engagement with first-year student
  experience and instruction.
\item
  During the orientation of new faculty and staff, ensure inclusion of
  information and discussion regarding first-year student demographics,
  populations, generational differences and transition issues specific
  to teaching and working at Kwantlen.
\item
  Create methods of acknowledging, recognizing and rewarding faculty and
  staff who facilitate exemplary experiences for first-year students.
\item
  Promote excellence in student engagement as integral to faculty
  culture.
\item
  Develop and distribute expectations for interactions with first-year
  students for all new faculty and staff.
\item
  Encourage search committees to include specific expectations about
  experience with and commitment to excellence in first-year instruction
  in job postings and interview questions.
\item
  Increase recognition of excellence.
\item
  Encourage greater faculty collaboration, recognition, and mutual
  support.
\item
  Expand peer mentorship and institutional mentorship.
\item
  Create a formal faculty-student mentorship framework.
\item
  Develop learning goals specific to first-year students.
\item
  Review and update current learning goals for all course outlines.
\item
  Revise the process by which curricular outcomes are developed.
\item
  Include students in defining, developing, and evaluating teaching
  excellence.
\item
  Assess the impact of special learning opportunities.
\item
  Align teaching/learning activities and assessment of student learning
  with intended outcomes (as articulated in course outlines).
\item
  Develop a clear strategy for effective use of technology in engaging
  students in learning.
\item
  Increase flexibility and variety of course offerings
  (partially-online, online, evening, and Saturday classes).
\end{itemize}

\section{Community}

Kwantlen needs a shared perspective on co-curricular and
community-focused service learning and work-integrated learning.

\subsection{Recommendations}

\begin{itemize}
\item
  Promote, encourage and support out of class (for credit) activities
  related to diversity and global learning including participating in
  field schools, student exchanges or service learning experiences.
\item
  Develop more service learning and experiential learning opportunities
  for students outside of the classroom.
\item
  Adopt a co-curricular transcript.
\end{itemize}

\section{Admissions}

Kwantlen must develop strategic admissions pathways for first-year
entrance that reflect our identity as a polytechnic university and
improve opportunities for student success.

\subsection{Recommendations}

\begin{itemize}
\item
  Develop a more robust and streamlined admissions framework.
\item
  Open a conversation with the institution about Open Access.
\item
  Develop a broad-based admissions process.
\item
  Develop a separate success pathway for students entering Kwantlen
  without an Academic GPA.
\item
  Adapt the learning environment to stream students for admissions,
  placement, and success.
\item
  Assess student preparedness more effectively and respond to student
  needs for greater preparation.
\item
  Create a more holistic and centralized assessment and placement
  process.
\end{itemize}

\section{Evaluation and Assessment}

Kwantlen must develop a definition of first-year for the purposes of
consistent data collection.

\subsection{Recommendations}

\begin{itemize}
\item
  Provide opportunities for training in assessment to faculty and staff.
\item
  Embed assessment into first-year experiences.
\item
  Adopt a widely-accepted assessment methodology to evaluate first-year
  student experiences (including retention and persistence).
\item
  Retain but refocus elements of the FoE survey in the new assessment
  instrument, and define key terms to clarify the questions for
  first-year students.
\item
  Share the results of assessments with the Kwantlen community on a
  regular basis.
\item
  Assess the status of awareness and involvement in the first-year
  concept among faculty and staff.
\item
  Evaluate the long-term impact of first-year activities.
\item
  Develop self-assessment tools for special learning opportunities.
\item
  Conduct regular reviews of improvements related to diversity.
\item
  Survey students' perspectives of the Early Alert Referral systems with
  the intent of increasing awareness and program effectiveness.
\item
  Define precisely what we mean by first-year students and transfer
  students, and treat the groups separately.
\end{itemize}

\section{Transitions}

Kwantlen must intentionally coordinate transition programs for all new
students.

\subsection{Recommendations}

\begin{itemize}
\item
  Provide advising and mentorship for students before they arrive at
  Kwantlen.
\item
  Provide advising and mentorship for all first-year students.
\item
  Require mandatory advising and/or mentorship for students on academic
  probation.
\item
  Develop peer mentorship and faculty mentorship programs.
\item
  Improve accessibility to front-line services for students by
  increasing hours of operations to include evening and weekend
  accessibility.
\item
  Encourage instructors to discuss advising and mentorship in classes.
\item
  Provide dedicated resources to support the integration, coordination
  and delivery of transitions efforts.
\item
  Develop methods of increasing student participation in transitions and
  orientation programming.
\item
  Provide a series of first-year courses focused on the integration of
  personal, academic, and professional development.
\item
  Increase the efficiency and timeliness of information distributed to
  first-year students regarding financial aid.
\item
  Develop a more robust web presence for first-year student services and
  curriculum.
\item
  Prioritize the allocation of space for social and recreational
  activities for students.
\item
  Offer a greater range of extracurricular and social activities for
  students.
\end{itemize}

\section{Integration and Coordination}

\subsection{Recommendations}

\begin{itemize}
\item
  Develop a communication strategy to inform employees, faculty and
  students about programs and services available to first-year students.
\item
  Establish a dedicated coordination position responsible for engaging
  across departments.
\item
  Provide a central calendar where departments can add events and
  activities that are taking place for first-year students.
\item
  Provide representation from the Registrar's office on all Faculty
  Councils.
\item
  Establish clear policy on diversity and religion to accurately reflect
  Kwantlen students, staff and faculty.
\item
  Establish a Centre for Diversity.
\item
  Allocate space for students to gather for specific purposes related to
  diversity.
\item
  Provide child care facilities on campus for students with young
  children.
\item
  Expand the mandate of the Centre for Interdisciplinary Research:
  Community Learning and Engagement (CIR:CLE) to focus on promoting
  diversity on campus, including outreach to develop relationships with
  diverse communities.
\item
  Support campaigns that promote, express and educate others about race,
  sexual orientation, cultures, traditions, disabilities, religion,
  socio-economic diversity, first generation university diversity, etc.
\item
  Implement a committee to work with diversity issues as they relate to
  first-year students.
\item
  Investigate best practices for first-year student initiatives at other
  post-secondary institutions.
\item
  Establish a year-long timetable to allow more effective student course
  planning and advising.
\end{itemize}

\end{document}
