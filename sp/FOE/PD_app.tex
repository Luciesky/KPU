%!TEX TS-program = xelatex
    %!TEX encoding = UTF-8 Unicode
%    \documentclass[10pt, letterpaper]{article}
\documentclass[letterpaper,10pt,headsepline]{scrreprt}
    \usepackage{fontspec}
    \usepackage{placeins}
    \usepackage{multibbl}
    \usepackage{graphicx}
    \usepackage{hieroglf}
    \usepackage{txfonts}
    \usepackage{url}
    \usepackage{listings}
    \lstset{language=HTML}
    \usepackage{titling}
    \usepackage{geometry}
    \geometry{letterpaper, textwidth=5.5in, textheight=8.5in, marginparsep=7pt, marginparwidth=.6in}
    %\setlength\parindent{0in}
    \defaultfontfeatures{Mapping=tex-text}
    \setromanfont [Ligatures={Common}, SmallCapsFont={Sabon LT Std}, BoldFont={Sabon LT Std Bold}, ItalicFont={Sabon LT Std Italic}]{Sabon LT Std}
    \setmonofont[Scale=0.8]{Lucida Sans Typewriter Std}
    \setsansfont [Ligatures={Common}, SmallCapsFont={Myriad Pro Regular}, BoldFont={Myriad Pro Bold}, ItalicFont={Myriad Pro Italic}]{Myriad Pro Italic}
\usepackage[ngerman,english]{babel}
\usepackage{scrpage2}
\usepackage{paralist}
\clubpenalty=6000
\widowpenalty=6000
\author{Ross A. Laird, PhD}
\title{Professional Development Application}
\date{\today}
\ohead{Ross Laird}
\chead{PD Application}
\pagestyle{scrheadings}
\setcounter{secnumdepth}{-1}
%\contentsname{Contents}
\begin{document}
\begin{titlingpage}
\begin{center}
\maketitle
\end{center}
\end{titlingpage}
\setcounter{tocdepth}{3}
\tableofcontents

\vspace*{1cm}
\section{Abstract}
\label{sec-1}
The Foundations of Excellence (FoE) is the largest initiative ever undertaken by Kwantlen to critically examine our approaches to teaching and learning in the first-year (and beyond). FoE is also the only (non-governance) initiative ever undertaken at Kwantlen in which faculty, administrators, staff, and students worked together across the disciplines on a shared organizational development project. However, When the initial stage of FoE ends, in summer 2012, the Kwantlen Faculty of Arts will not have a single member specifically tasked and funded to continue this work and to build upon the outstanding momentum that has been created across the institution.

My current arts-based research is a critical examination of teaching and learning, with primary focus on first-year transitions and the integration of personal, professional, and academic development. I am in the process of writing a new book (my fourth) on the challenges and opportunities faced by educational institutions, and I would like to combine my book research with ongoing work to support FoE.

I am asking for one-quarter time-release for two semesters.
\section{Proposal Description}
\label{sec-2}

For the past several years I have been working on various projects focused on innovation and renewal at Kwantlen (the Mission and Mandate Task Force, the Learning Technologies faculty development project, Scenario Planning, Foundations of Excellence, etc.). I have also worked om similar themes with many external organizations including Vancouver Coastal Health; the School Districts of Surrey, Langley, Richmond, Delta, Vancouver, and Coquitlam; SFU's Learning and Teaching with Technology Field School; UBC's Division of Aboriginal Peoples' Health; the BC Association for Supervision and Curriculum Development, and the BC Principals and Vice-Principals' Association. I have presented workshops to over 100 schools on themes of educational innovation and renewal. This consultative and collaborative work has led to plans for a new book (my fourth) on the future of education.

For the past year I have been involved in the Foundations of Excellence (FoE) initiative at Kwantlen, in which I have served on the Learning Committee (as Chair) and the Steering Committee. The alignment between FoE and my ongoing work in educational renewal has been remarkable and deeply rewarding. I have also been heartened to see much broad-based support for FoE: over 130 administrators, staff, students, and faculty members from across Kwantlen have been involved with FoE. At the same time, even though FoE is broadly supported within the Faculty of Arts, the initiative is not formally supported by the Faculty of Arts in any way. Indeed, no faculty member anywhere at Kwantlen has formal mandate or funding to work on themes or actions arising from FoE.

This is one of the challenges that I consistently see in my work with educational institutions: at a certain point it's obvious what needs to be done, but everyone is busy doing other things. In my experience, very few initiatives of innovation and renewal are able to leverage initial enthusiasm and excitement into transformative and lasting change. The more common scenario involves enthusiasm giving way to fatigue and excitement dwindling toward cynicism -- unless the momentum can be sustained.

FoE needs to keep and build momentum, and in order for this to happen, faculty members must be involved on an ongoing basis. And, as it happens, my own research activities are perfectly consistent with the aims and themes of FoE. It seems an obvious strategy to integrate my plans for a new book with the needs of FoE for ongoing involvement from a member of the Faculty of Arts.

My new book will be a heuristic research project into the lived experience of educators and learners (at many institutions) as they navigate the transformative changes underway in education. As for our own changes, the FoE Learning Committee has made 25 urgent recommendations -- which, if followed, will transform Kwantlen. But how do we undertake these transformations? How best can we make it all work? The book will explore these themes as I continue my involvement with FoE: promoting change, collaborating with like-minded faculty members, working across the institution.

My experiences, and the (de-identified) experiences of my colleagues and students, will be documented and utilized in the narrative. (Two examples of a similar approach are Parker Palmer's \emph{The Heart of Higher Education} and Mark Taylor's \emph{Crisis on Campus}).

I am asking for two time-releases of one-quarter each, spread across two semesters (Fall 2012 and Spring 2103), but I do not expect that this will be sufficient time to complete the book manuscript. It will be sufficient to write the first draft.

The phases of the project are as follows:

Phase 1, Fall 2012: Track and promote the progress of the 25 recommendations of the FoE Learning Committee. Document this progress (or lack of it) in heuristic, contextual narratives. Activities might include meetings, events, and other collaborations. (As the exact structure for the next phase of FoE has not yet been determined, I cannot provide exact examples of how the process will unfold.)

Phase 2, Spring 2013: Gather together the narratives, build the broad structure of the book (using heuristic, arts-based research modalities), and work toward a first draft. Also keep working with FoE in whatever capacity is most useful (again, the exact structure is not yet known).

Phase 3, Summer 2013: Completion of the book and progress report.
\section{Fit with Criteria and Institutional Priorities}
\label{sec-3}

\subsection{Career Plans of the Applicant}
\label{sec-3-1}
I am a teacher, writer, and presenter. These three spheres of may activity constantly interact and build upon one another. They also involve distinct but interconnected audiences. My teaching is in the classroom with students; my writing is targeted toward a general audience; and my presentations are to educators, parents, and social service providers. Over the past year I have worked extensively as a teacher (at Kwantlen) and a presenter (across Canada), but I have had difficulty finding time for writing (and arts-based research, which is the methodology of my writing). My professional profile and development are contingent upon my continued involvement in professional writing and publishing.

For career writers, ongoing professional development and performance consist of publishing a new book of high literary quality every couple of years. However, as a result of my deep commitment to supporting positive change at Kwantlen, I have been involved in many purposeful activities (task forces, committees, initiatives, etc.) that I have enjoyed but which have taken me farther from the important goal of professional publication. I have had to make difficult choices between supporting Kwantlen and advocating for my own plans, and for the past two years I have consistently chosen to support Kwantlen -- because now is the time for us to make change. As I've moved ever more deeply into my commitments to change management at Kwantlen (going well beyond my commitments as an instructor), I've also been aware that somewhere along this track I will need to get back to writing. So far, it has seemed to be a matter of choosing between my own professional aims and the aspirations of Kwantlen.

However, FoE has provided a potential opportunity to integrate my activities in a manner that has not seemed possible previously. By continuing to work with the FoE initiative while also writing about it (and about the more general themes of educational innovation and renewal), I will be able to meet my own needs as well as those of Kwantlen. My research and scholarly activity will match my commitment to Kwantlen.

Additionally, as the new book will focus on teaching and learning, my professional development project precisely matches one of Kwantlen's core institutional priorities: the promotion of teaching excellence. Indeed, the thematic core of the book will be an exploration of how best to facilitate learning experiences for the students of today, how best to provide purposeful mentorship, and how best to craft learning experiences that promote global citizenship.

The new book will allow me to continue building my profile as a scholar working toward educational innovation and renewal in Canada. My previous books have met with excellent success (I am a best-selling author, and I have been shortlisted for the Governor General's Award); my new book will continue my commitment to providing accessible, well-written books on important social and cultural themes.
\subsection{Relevance to Current Work at Kwantlen}
\label{sec-3-2}
Kwantlen has recently undertaken a series of initiatives to redefine who we are and what we do. I have been involved in these initiatives, and my general impression of them has been one of both hope and concern. We want to move forward, but many of us are moving at different speeds. We wish to honour the rich legacy of our past, yet many of us want to shape a different future. We seek an environment that is ``collaborative, innovative, creative, and respectful'' (as our new Mission statement indicates), yet we easily become disoriented and uncertain. In the midst of all this turbulence, it really shouldn't be surprising that 40 percent of our students leave before completing a single year.

Are we, as our Mission statement asserts, ``a dynamic educational community that embraces emerging and experimental teaching methods''? No, not yet. Are we ``a leader in innovative and interdisciplinary education''? No, we're not. These goals, along with many others we've set for ourselves, are still a long way off. And yet we are moving forward with purpose. And this, as I see it, is my core contribution to Kwantlen: to help with our organizational growth toward our vision of what we can be. My scholarly activity is ``a socially relevant obligation and opportunity'' (to quote the Mission statement again) which explores and examines how best to achieve the ambitious aim of providing an exemplary learning environment.

My plan for this professional development project is to use the context of FoE, at Kwantlen (the first Canadian university to undertake this very successful American initiative) to conduct a heuristic research study into the lived experience of Kwantlen's odyssey of renewal. In the work of Douglass and Moustakas (1985, p. 38), the heuristic methodology is described as a means to ``obtain qualitative depictions that are at the heart and depths of a person's experience -- depictions of situations, events, conversations, relationships, feelings, thoughts, values, and beliefs. A heuristic quest enables the investigator to collect \ldots{} the raw material of knowledge and experience from the empirical world.'' Heuristic research focuses on direct first-person accounts of ``individuals who have directly encountered the phenomenon in experience\ldots{} Whereas phenomenology encourages a kind of detachment from the phenomena being investigated, heuristics emphasizes connectedness and relationship.'' The results of my heuristic study will be published in the new book.

The Foundations of Excellence is the largest initiative ever undertaken at Kwantlen to promote teaching excellence (for the first-year and beyond). And, so far, all involvement from faculty has been on a volunteer basis. This is a great way to start, but volunteer contributions are almost always insufficient to continue the momentum of change. If FoE is to grow and thrive, some financial resources must be allocated so that faculty members can help shepherd the FoE process forward. Faculty members, after all, represent the core of Kwantlen's services to the community, and it is essential that faculty members be formally involved in this large, important initiative. Without such formal involvement, the future of FoE is questionable.

My ongoing involvement with FoE will also offer benefits to my own teaching and to the experiences of students. The courses that I teach at Kwantlen focus on the fusion of personal, professional, and academic development. My curriculum explores innovative and integrative teaching practices, interdisciplinary creativity, and the possibilities of transformational learning. These principles and approaches are entirely consistent with FoE and with my ongoing arts-based research in education. Accordingly, many reciprocal learning opportunities will present themselves if I continue my involvement with FoE. In turn, those opportunities will yield benefits to students (in the classroom), to my own research and writing (in the new book), and to Kwantlen as a whole, as we move ever closer to manifesting all we can be.

\subsection{Budget Justification}

I am requesting two sections of time release (Fall 2012 and Spring 2013), for a total of twenty five thousand dollars. Writing a book is a significant undertaking requiring more than a thousand hours of work. My experience with my previous books has been that it takes me about three hours per day, for at least a year, to write a first draft. The subsequent steps of editing, shaping, and copy-editing require another few hundred hours. I am committed to writing this book, but my responsibilities as a faculty member teaching full-time do not allow sufficient time to write. This proposal requests time release so that I can focus more directly on writing, in addition to my teaching and service at Kwantlen. I have committed the last two years exclusively to teaching and service at Kwantlen, with the unintended consequence that my writing has been put aside. I'd like to re-balance that equation so that the needs of Kwantlen dovetail more harmoniously with my own professional development as a writer and scholar.

Twenty five thousand dollars is the amount allocated to two sections of time release, but it is also the amount specified by the Canada Council for grants to writers in the category to which I belong: mid-career writers who have published two or more books. Canada Council grants for writers are intended to provide ``Canadian authors (emerging, mid-career and established) time to write new literary works." This is precisely what I will be doing, and the Canada Council's industry-standard is the best guide to budgetary projection for this type of work.


\end{document}
