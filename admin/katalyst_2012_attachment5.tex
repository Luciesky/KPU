%!TEX encoding = UTF-8 Unicode
%    \documentclass[10pt, letterpaper]{article}
\documentclass[letterpaper,10pt,headsepline]{scrreprt}
    \usepackage{fontspec}
    \usepackage{placeins}
    \usepackage{color}
    \usepackage[usenames,dvipsnames,svgnames,table]{xcolor}
    \usepackage{multibbl}
    \usepackage{graphicx}
    \usepackage{hieroglf}
    \usepackage{marginnote}
    \usepackage{mparhack}
    \usepackage{txfonts}
    \usepackage{url}
    \usepackage{listings}
    \lstset{language=HTML}
    \usepackage{titling}
    \usepackage{geometry}
    \geometry{letterpaper, textwidth=5.5in, textheight=8.5in, marginparsep=7pt, marginparwidth=.6in}
    %\setlength\parindent{0in}
    \defaultfontfeatures{Mapping=tex-text}
    \setromanfont [Ligatures={Common}, SmallCapsFont={Sabon LT Std}, BoldFont={Sabon LT Std Bold}, ItalicFont={Sabon LT Std Italic}]{Sabon LT Std}
    \setmonofont[Scale=0.8]{Lucida Sans Typewriter Std}
    \setsansfont [Ligatures={Common}, SmallCapsFont={Myriad Pro Regular}, BoldFont={Myriad Pro Bold}, ItalicFont={Myriad Pro Italic}]{Myriad Pro Italic}
\usepackage[ngerman,english]{babel}
\usepackage{scrpage2}
\usepackage{paralist}
\clubpenalty=6000
\widowpenalty=6000
\author{Ross Laird and Jody Gordon}
\title{Katalyst Grant Application Attachment 5}
\date{\today}
\ohead{Katalyst}
\chead{Research Grant Application}
\pagestyle{scrheadings}
\setcounter{secnumdepth}{-1}
%\contentsname{Contents}
\begin{document}
\begin{titlingpage}
\begin{center}
\maketitle
\end{center}
\end{titlingpage}
\setcounter{tocdepth}{3}
\tableofcontents

\section{Research Grant Application History}

\subsection{Ross Laird}


Ross Laird entered traditional academia only recently, after a 25-year career
as a consultant, educator, and author. His career has been mostly in the wild,
so to speak: bidding for and earning contracts, competing with other
consultants and educators, working creatively to spread ideas, collaborating
with a wide range of professionals. He has achieved a strong track record in
his career without ever having applied for a research grant in the sense that
this section of the application requests. It seems that the criteria for this
section are focused on applications for funding from organizations such as
SSHRC. Ross Laird's funding has not come from such sources. Instead, his
research has been undertaken with support from organizations such as The
Canada Council for the Arts, Heritage Canada, the BC Ministry of Health,
Corrections Canada, and many private corporations. These types of
organizations do not typically use the term ``research grant.'' Instead, they
use words such as competition, contract, and bid. This is the landscape in
which Ross Laird has built his track record of success.

We assume that this section of the application is primarily intended to
determine whether the researchers have a history of successful peer review
(and, by implication, how much research funding the researcher has raised). As
Ross Laird's research has taken place in an environment distinct from the
pathways of SSHRC and related agencies, it may be useful to illustrate his
success with some examples of the context of peer review and funding in which
he typically works.

\subsubsection{Peer Review in the Arts}

Ross Laird has written three books on interdisciplinary subjects (creativity,
myth, culture, addictions). His first book evolved out of his doctoral
research project, which was a heuristic study of creativity. The dissertation,
which won the Union Institute's Sussman Award for Academic Excellence (the
university's highest academic award), was adapted into a best-selling book and
was short-listed for the Governor General's Award. In the environment of
professional writing in Canada, a short-list for the Governor General's Award
is the highest peer review accolade one can receive.

The success of his first book led to a second, which was published by Canada's
premier literary publisher, McClelland and Stewart. Publication by McClelland
and Stewart is a type of peer endorsement for Canadian writers (an endorsement
from the gatekeepers of Canadian literature). Most of Canada's historical
literary figures have published with McClelland and Stewart.

His ongoing success as a writer has led Ross Laird to involvement with the
Canada Council for the Arts. He was asked to be a jury member for the Literary
Grants program and for the Governor General's Award. One cannot apply for such
positions; they are by invitation only, from the foremost arts funding agency
in the country. Essentially, his participation on the Governor General's Award
jury (in 2008) was an endorsement from the government of Canada that he is
competent to assess the literary merit of the best writers in Canada. This is
a very strong peer review endorsement.

One more example: last year Ross Laird responded to a call, from the Writers'
Union of Canada (for which he has served on the National Council), to compete
for a Heritage Canada contract to be one of two professional development
consultants for professional writers in Canada. This was a national call, open
to any professional writer. His successful bid is a strong endorsement of his
role within the writing community and a clear indication that writers value
what he offers. (This Heritage Canada project was intended to help writers
navigate the evolving landscape of publishing. In many ways, writers are
grappling with the same challenges as educators.)

From readers, writers, publishers, and from the government of Canada, Ross
Laird has consistently received strong endorsements. His peers --- at every
level --- have acknowledged that he makes an important contribution.

None of this work has involved research grants as described by this
application process nor pathways of publication in scholarly journals.
Instead, Ross Laird's research has been published in the books and articles he
has written based on his experiences as a researcher of creativity, education,
and personal development. His work has been published or reviewed in most
major publications in Canada (The Globe and Mail, Maclean's, The Toronto Star,
The Ottawa Citizen, The Hamilton Spectator, The Vancouver Sun, etc.), he has
appeared on television and radio many times, and his writing and research
continue to evolve. This research project will represent his fourth book.

The funding for these professional projects has included book advances, jury
fees, literary grant funding, and contract fees. The total amount over the
past few years is approximately \$80,000.

\subsubsection{Peer Review in Social Services}

Ross Laird has been a professional counsellor and clinical supervisor to
social service agencies for many years. He has worked with hundreds of
organization and thousands of individuals. He has presented at more than 100
conferences, has facilitated hundreds of professional development projects,
and has worked hard to establish a strong reputation for professionalism. In
2003 he earned the Communications Award of the BC Association of Clinical
Counsellors for the heuristic research of his first two books. This year he
will publish a third book, on addictions, that continues and deepens that
research.

\subsubsection{Peer Review in Education}

For the past several years Ross Laird has provided professional development
services to elementary schools, secondary schools, community colleges,
vocational schools, teacher training programs, teaching associations, school
districts, parent groups, student groups, and universities (in 2010-11 he
provided professional development consulting for faculty at Kwantlen, through
his role as consultant for Kwantlen's Learning Technology department). This
work continues to grow and expand in scope; during the past year he has worked
with more than 1,000 educators in many institutions (repeating the examples
from earlier: the UBC Teacher Education program; the SFU Advanced Professional
Studies Field program; the UFV Teacher Education program; Vancouver Coastal
Health; the School Districts of Vancouver, Richmond, Surrey, Delta, Langley,
Abbotsford, Coquitlam, and others). Overall, he has earned roughly \$100,000
over the past six years (the length of time requested by this application)
providing educational consulting services.

Finally, SSHRC defines a ``record of research achievement'' as ``any identifiable
contributions made by applicants to the advancement, development and
transmission of knowledge.'' By these criteria, Ross Laird possesses a strong
track record. He has made extensive contributions to several fields and has
helped to advance, develop, and transmit knowledge through his professional
activities and my writing.

SSHRC also defines research and creation as ``any research activity or approach
to research that forms an essential part of a creative process or artistic
discipline and that directly fosters the creation of literary/artistic works.''
By these criteria also, Ross Laird has a demonstrated record of outstanding,
peer-reviewed, consistently endorsed literary and artistic work. The Katalyst
program is designed not to mirror SSHRC but to offer applicants a chance to
build bridges between academic scholarship and professional pursuits that have
not yet made their way into channels such as SSHRC. This is exactly what Ross
Laird hopes to do.

Ross Laird applied for a Katalyst grant in 2011 but was not successful.

\subsection{Jody Gordon}

During her time as a graduate student, Jody Gordon held two research assistant
posts. From 1994-1996 she was a Research Assistant for an SFU faculty member
on contract with the Solicitor General of Canada; and from 1993-1994 she
conducted data analysis and acted as Research Coordinator for two SFU and UBC
faculty members. In this latter position she drafted a proposal for funding
and conducted data analysis. In 2008, Jody Gordon (along with two of her
direct reports) applied for and was awarded a \$14,000 research grant by the
BC Provincial Government under the auspices of the Default Prevention Grant
Program. This research entailed the delivery of the Academic Boost Camp
program in which students on academic warning and academic probation are
exposed to high-impact practices to improve their academic standing. Data from
this research demonstrated that students enrolled in the program succeeded at
a higher rate than those not enrolled. Further, the students who completed the
program persisted through their education and were less likely to be required
to withdraw from the university. Since receiving this grant, the Academic
Boost Camp continues to be offered at Kwantlen (senior administration provided
permanent funding to maintain the program). The results of the Academic Boost
Camp research were published in a journal article and have been presented at
numerous conferences around North America.

\end{document}
