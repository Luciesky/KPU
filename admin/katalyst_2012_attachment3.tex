%!TEX encoding = UTF-8 Unicode
%    \documentclass[10pt, letterpaper]{article}
\documentclass[letterpaper,10pt,headsepline]{scrreprt}
    \usepackage{fontspec}
    \usepackage{placeins}
    \usepackage{color}
    \usepackage[usenames,dvipsnames,svgnames,table]{xcolor}
    \usepackage{multibbl}
    \usepackage{graphicx}
    \usepackage{hieroglf}
    \usepackage{marginnote}
    \usepackage{mparhack}
    \usepackage{txfonts}
    \usepackage{url}
    \usepackage{listings}
    \lstset{language=HTML}
    \usepackage{titling}
    \usepackage{geometry}
    \geometry{letterpaper, textwidth=5.5in, textheight=8.5in, marginparsep=7pt, marginparwidth=.6in}
    %\setlength\parindent{0in}
    \defaultfontfeatures{Mapping=tex-text}
    \setromanfont [Ligatures={Common}, SmallCapsFont={Sabon LT Std}, BoldFont={Sabon LT Std Bold}, ItalicFont={Sabon LT Std Italic}]{Sabon LT Std}
    \setmonofont[Scale=0.8]{Lucida Sans Typewriter Std}
    \setsansfont [Ligatures={Common}, SmallCapsFont={Myriad Pro Regular}, BoldFont={Myriad Pro Bold}, ItalicFont={Myriad Pro Italic}]{Myriad Pro Italic}
\usepackage[ngerman,english]{babel}
\usepackage{scrpage2}
\usepackage{paralist}
\clubpenalty=6000
\widowpenalty=6000
\author{Ross Laird and Jody Gordon}
\title{Katalyst Grant Application Attachment 3}
\date{\today}
\ohead{Katalyst}
\chead{Research Grant Application}
\pagestyle{scrheadings}
\setcounter{secnumdepth}{-1}

\begin{document}
\vspace*{1cm}

\section{Explanation of Interdisciplinary and Collaborative Nature of Project}

Our research project is an interdisciplinary inquiry involving members of the
Kwantlen community from various departments and disciplines. Ross Laird is a
member of the Creative Writing department (composition of a literary book is
part of this project) as well as an instructor of Interdisciplinary Expressive
Arts (this project is an interdisciplinary inquiry). Jody Gordon is a senior
administrator in Student Life and Community as well as an instructor in
Criminology (one goal of this project is to improve the learning experiences
of students both in and out of the classroom). Kurt Penner is a faculty member
in Student Life and Community and Josh Mitchell is a senior administrator in
Student Life and Community (Kurt Penner is also an instructor in Psychology,
and this project involves assessment using instruments from Psychology). Our
student participants will come from various disciplines, and our collaborative
project will integrate these diverse perspectives into a single
interdisciplinary focus.

Each of these participants offers a unique contribution. Ross Laird brings his
extensive experience as an author and consultant in education and change
management. Jody Gordon brings her exemplary track record as a university
administrator, researcher in education, and leader in student life and
community. Kurt Penner and Josh Mitchell bring their outstanding skills as
student life administrators and developers of student-led initiatives. And,
in turn, the students bring their diversity, enthusiasm, and goodwill.

Each participant will participate in the ten-step framework outlined above and
will collaborate extensively with all the other participants. This is a truly
interdisciplinary project and is therefore highly unusual for Kwantlen.
Typically, the term ``interdisciplinary'' is used at Kwantlen to describe a
variety of activities that are, in fact, multidisciplinary. This distinction
is crucial. Interdisciplinary initiatives always seek integration, whereas
multidisciplinary approaches tend to keep things separate and domain-specific.
In this sense, true interdisciplinary collaboration is much more than people
from different disciplines working together. Authentic interdisciplinarity is
an integrated, holistic approach that goes far beyond the multidisciplinary
model. Perhaps a lighthearted metaphor from interdisciplinary scholar Moti
Nissani (1995) says it best: studies are smoothies, whereas multidisciplinary
studies are bowls of fruit. Our project is a smoothie.

\end{document}
