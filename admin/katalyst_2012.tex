%!TEX encoding = UTF-8 Unicode
%    \documentclass[10pt, letterpaper]{article}
\documentclass[letterpaper,10pt,headsepline]{scrreprt}
    \usepackage{fontspec}
    \usepackage{placeins}
    \usepackage{color}
    \usepackage[usenames,dvipsnames,svgnames,table]{xcolor}
    \usepackage{multibbl}
    \usepackage{graphicx}
    \usepackage{hieroglf}
    \usepackage{marginnote}
    \usepackage{mparhack}
    \usepackage{txfonts}
    \usepackage{url}
    \usepackage{listings}
    \lstset{language=HTML}
    \usepackage{titling}
    \usepackage{geometry}
    \geometry{letterpaper, textwidth=5.5in, textheight=8.5in, marginparsep=7pt, marginparwidth=.6in}
    %\setlength\parindent{0in}
    \defaultfontfeatures{Mapping=tex-text}
    \setromanfont [Ligatures={Common}, SmallCapsFont={Sabon LT Std}, BoldFont={Sabon LT Std Bold}, ItalicFont={Sabon LT Std Italic}]{Sabon LT Std}
    \setmonofont[Scale=0.8]{Lucida Sans Typewriter Std}
    \setsansfont [Ligatures={Common}, SmallCapsFont={Myriad Pro Regular}, BoldFont={Myriad Pro Bold}, ItalicFont={Myriad Pro Italic}]{Myriad Pro Italic}
\usepackage[ngerman,english]{babel}
\usepackage{scrpage2}
\usepackage{paralist}
\clubpenalty=6000
\widowpenalty=6000
\author{Ross Laird and Jody Gordon}
\title{Katalyst Grant Application}
\date{\today}
\ohead{Katalyst}
\chead{Research Grant Application}
\pagestyle{scrheadings}
\setcounter{secnumdepth}{-1}
%\contentsname{Contents}
\begin{document}
\begin{titlingpage}
\begin{center}
\maketitle
\end{center}
\end{titlingpage}
\setcounter{tocdepth}{3}
\tableofcontents

\vspace*{1cm}
\section{Summary of Proposed Research}
\label{sec-1}

The Foundations of Excellence (FoE) is the largest initiative ever undertaken
by Kwantlen to critically examine our approaches to teaching and learning in
the first-year (and beyond). When the initial stage of FoE ends, in summer
2012, Jody Gordon and her team in Student Life and Community will continue to
work on the FoE initiative; but Kwantlen will not have a single faculty member
specifically tasked and funded to continue this work and to build upon the
outstanding momentum that has been created across the institution.


However, Ross Laird's current research is also a critical examination of
teaching and learning, with primary focus on first-year transitions and the
integration of personal, professional, and academic development. He is in the
process of writing a new book (his fourth) on the challenges and opportunities
faced by educational institutions, and he would like to combine his book
research with ongoing work to support FoE.

Accordingly, this research proposal is for an interidsicplinary collaboration
involving Ross Laird, Jody Gordon, Josh Mitchell, Kurt Penner, and members of
the Kwantlen community (students, faculty members, administrators, staff). The
research undertaken will examine the lived experience of institutional change
in the context of a broad institutional self-study process (FoE) focused on
improving teaching and learning at Kwantlen.

We are asking for one-quarter time-release for two semesters (for Ross Laird) as well as funding to support student involvement in research.


\section{Detailed Description}

\subsection{Research Objective}

What factors contribute to a quality post-secondary education? What
experiences -- especially in the first-year -- contribute to academic
retention and success? What types of principles and practices should educators
employ in their delivery of quality educational services? These questions are
now more pressing now than ever before. New technologies challenge established
norms. Emerging practices promise new modes and methods. Cultural, economic,
and social changes encourage (and perhaps even demand) a comprehensive review
of what education is, and what it's for. We are --- to put it mildly ---
living through an age of educational destruction and renewal.

Kwantlen has responded to the urgent need for innovation and review by
undertaking a comprehensive self-study, the largest organization-wide
initiative ever undertaken by the institution. Over 120 participants have
worked, for the past year, on the Foundations of Excellence (FoE) guided
self-study process. The structure of the process was provided by the John N.
Gardner Institute for Excellence in Undergraduate Education, an organization
that has worked with over 300 educational institutions in the United States.
Kwantlen is the first institution in Canada to participate in FoE.

The FoE participants have concluded the first stage of this work, which
involved a comprehensive self-study across nine different dimensions of
inquiry (organization, philosophy, learning, campus culture, transitions, all
students, diversity, roles and purposes, and improvement). Each dimension was
examined by a committee, and each committee has now completed a report with
many recommendations. In total, the FoE process has generated hundreds of
recommendations for institutional renewal and innovation.

But now what? How do we carry this energy and momentum forward? How do we
develop, implement, and measure the next steps? What are the next steps?

Our objective is to learn about the lived experience of members of the
Kwantlen community as we (all) move through the current changes that lie
before us. The FoE recommendations precisely encapsulate and mirror these
changes, and therefore provide an excellent framework. For example, several
recommendations discuss the possibility of implementing high-impact, engaging
practices for first-year students. How do we work toward this? What happens
when we try to make it happen? When we make it happen, do high-impact
practices lead to greater academic success? How do we know? Many of these
types of questions arise naturally out of the FoE recommendations. Our
objective is not to answer each one but rather to explore the process in two
ways:

\begin{enumerate}
\item As a whole: to see which recommendations come to be implemented and
  which fall away (and why); to examine the organizational dynamics which
  promote or subvert the recommendations; to attempt to understand the
  Kwantlen experience more deeply by examining what we do at this important
  juncture. This aspect of the research is intended to be heuristic,
  narrative-based (through interviews and observations), and will employ a
  qualitative methodology.

\item By way of the specific challenge of implementing high-impact practices
  in the first year. This aspect of the research is intended to involve
  measurable outcomes of student success and will combine quantitative course
  completion rates; student persistence rates; student satisfaction rates) as
  well as qualitative methodologies (student interviews).

  We are interested in the experiences of our community and the meanings we
  derive from those experiences. Our work will build upon our existing
  backgrounds and will augment the large body of current research currently
  exploring similar themes. Our proposed investigation is interdisciplinary
  and hermeneutic, and also combines qualitative as well as quantitative
  approaches. Our collaborative research team has a strong background in
  working on projects such as these, and we hope to use the Katalyst
  initiative to take our research to the next level.

\end{enumerate}

\subsection{About the Co-researchers}

  \subsubsection{Jody Gordon}

  Jody Gordon is the co-author of ``Restoring Hope: Early Warning and Early
  Intervention Matters,'' published in the American Association of Collegiate
  Registrars and Admissions Officers SEM Source Journal, May 2012 (co-authored
  with Ray Darling and Julie Pong). She is also the co-author, with Devron
  Gaber and Yves Jodoin), of ``Beyond Transfer: The Mobile Student'' (2011),
  in \emph{SEM in Canada: Promoting Student and Institutional Success in
    Canadian Colleges and Universities}, Washington D.C. She is also the
  co-author, with Josh Mitchell (one of our research team members) of
  ``Partnering for Success: A Collaborative Approach To Student Persistence,''
  published in the American Association of Collegiate Registrars and
  Admissions Officers SEM Source Journal, March 2010.

  Jody Gordon presents regularly at conferences across North America focusing
  on the role of student affairs in student success. She was selected by her
  peers and awarded in 2009 the AACRAO Strategic Enrolment Management Award of
  Excellence recognizing her outstanding achievement and visionary leadership
  in strategic enrolment management in Canada. This year she was selected by
  the national organization of Collegiate Registrars (ARUCC) to lead an
  all-day workshop on Strategic Enrolment Management and Student Success in
  Ottawa. In addition to her work in Student Affairs, Jody Gordon also
  teaches first year Criminology at Kwantlen where she often incorporates
  high-impact student success practices into her classroom.

  \subsubsection{Ross Laird}

  Ross Laird works with teachers, professors, students, educational
  administrators, and policy makers on initiatives related to educational
  renewal and innovation. He has consulted with primary and secondary schools,
  teacher training programs, colleges and universities, Parent Advisory
  Councils, health authorities, government ministries, and many other groups
  (examples: the UBC Teacher Education program; the SFU Advanced Professional
  Studies Field program; the UFV Teacher Education program; Vancouver Coastal
  Health; the School Districts of Vancouver, Richmond, Surrey, Delta, Langley,
  Abbotsford, Coquitlam, and others). He has presented at hundreds of
  conferences on education and has facilitated workshops for parents,
  teachers, learners, counsellors, and government curriculum developers.

  Dr. Laird's educational work has focused on a single, pressing, and vexing
  theme: how to reinvent education to be more purposeful, relevant, and
  meaningful?

  Over the past 25 years Dr. Laird's scholarship has been nomadic and highly
  individualistic. His typical way of working has been to pursue an interest
  in a given area, compete for (and win) consulting contracts to work in that
  area, conduct heuristic research, and write books and articles. But
  recently, after coming to Kwantlen, Dr. Laird has become interested in
  establishing a base for his scholarship, a means of integrating various
  professional threads into a more streamlined approach to enable him to write
  the next several books based on his research. (He has written three books to
  date, one of which was a finalist for the Governor General's Award).

\subsection{Existing Literature}

As the challenges faced by educators accelerate rapidly (particularly with
regard to technology and youth culture), research activity into the state of
education is also accelerating. Many leading researchers have already written
books using collaborative approaches that combine qualitative and quantitative
strategies (to deal with big-picture as well as specific-application
questions).

Recent examples include Parker Palmer and Arthur Zajonic's ``The Heart of
Higher Education,'' Clayton Christensen and Henry Eyring's ``The Innovative
University,'' Richard Arum's, ``Academically Adrift,'' Tony Wagner's ``The Global
Achievement Gap,'' John Seely Brown's ``A New Culture of Learning,'' and David
Weinberger's ``Too Big to Know: Rethinking Knowledge Now That the Facts Aren't
the Facts, Experts Are Everywhere, and the Smartest Person in the Room Is the
Room.'' Weinberger, a Senior Researcher at Harvard University's Berkman Center
for the Internet and Society, is an exemplary example of a scholar who has
worked both collaboratively and across the disciplines to create educational
research that deals both with the practical hurdles of strategy implementation
(such as high-impact strategies) as well as broader questions about the
overall direction of education and society. This is the type of approach that
we wish to explore.


\section{Methodology}

We will employ an interdisciplinary research methodology as defined in Allen
Repko's foundational text ``Interdisciplinary Research: Process and Theory''
(Sage, 2011). That methodology includes 10 steps (Repko's terms in bold):

\begin{description}
\item[Step One: state the problem or focus question.] For our research, the problem
  involves educational innovation and renewal at Kwantlen. The focus question
  is as follows: how can innovative and high-impact educational practices be
  implemented in an educational environment unaccustomed to rapid change?

\item[Step Two: justify an interdisciplinary approach.] The
  justification for our interdisciplinary research flows from the complexity
  of the dynamics in play and the involvement of many parts of the Kwantlen
  community (including many different disciplines and organizational roles).
  The current situation of rapid change at Kwantlen is precisely the type of
  situation described by Klein and Newell's (Newell, 1998) classic definition
  of interdisciplinary application: ``Interdisciplinary studies may be defined
  as a process of answering a question, solving a problem or addressing a
  topic that is too broad or complex to be dealt with adequately by a single
  discipline or profession\ldots IDS draws on disciplinary perspectives and
  integrates their insights through construction of a more comprehensive
  perspective'' (p. 245). Kwantlen is currently undergoing sweeping changes in
  response to demographic trends, educational challenges, and social and
  economic changes. These are very broad considerations and are well beyond
  the ambit of any single discipline. Our approach, therefore is to integrate
  various disciplines into our research.

\item[Step 3: identify relevant disciplines.] Repko reinforces the fact that
  interdisciplinary studies are not confined to the traditional notion of
  departments and faculties. Indeed, ``The term \emph{discipline}, in the
  context of interdisciplinary research, encompasses subdisciplines, schools
  of thought, and interdisciplinary fields'' (Repko, 2008, pp. 52, 101).
  Accordingly, our research will involve several domains: education,
  creativity, and psychology.

\item[Step 4: Conduct the literature search.] As noted previously, recent
  literature on our topic of inquiry is expanding rapidly. We will explore
  this body of literature in depth, drawing upon examples from
  well-established scholars such as Parker Palmer, Arthur Zajonic, Clayton
  Christensen, Henry Eyring, Richard Arum, Thomas Carey, Tony Wagner, John
  Seely Brown, David Weinberger, and others.

\item[Step 5: Develop adequacy in each relevant discipline.] The contemporary
  consensus from interdisciplinary scholars is that ``interdisciplinary study
  should build explicitly and directly upon the work of the disciplines''
  (Newell, 1998, p. 542). Accordingly, our research team includes members from
  the core disciplines noted above. Ross Laird will represent the fields of
  Education and Creativity; Jody Gordon and Josh Mitchell will represent the
  field of Education; and Kurt Penner will represent the fields of Psychology
  and Education.

\item[Step 6: Analyze the problem and evaluate insights.] Each
  discipline will contribute a unique perspective to our research, and it will
  be important to explore and integrate these perspectives as we move forward.
  Accordingly, our research will include documentation about how each
  discipline views our situation and what each discipline recommends.
  Interdisciplinary understanding arises from the integration of points of
  view, not their simple aggregation. Our research and companion documentation
  will emphasize this. ``The interdisciplinarian needs to contextualize the
  contribution of each discipline within the overall complex system'' (Newell,
  2006, p. 250).

\item[Step 7: Identify conflicts between insights and their sources.] The
  diverse points of view offered by distinct disciplines can, and often are,
  at odds with one another. Assumptions, concepts, and theories can be widely
  divergent across the disciplines. For example, the recent trend in
  Psychology has been to focus on constituent parts (of the nervous system, of
  the brain, of personality) whereas the emphasis in Creative Process theory
  and Education is increasingly on the whole (whole systems, interdependency,
  rhizomatic systems). We will identify these divergences and will discuss and
  integrate them into our research. This integration is one of the reasons we
  are using a blended approach that includes both a narrative-based, creative
  and heuristic inquiry (whole systems thinking) as well as a quantitative
  assessment of high-impact practices (targeted, event-driven inquiry).



\item[Step 8: Create or discover common ground.] The core of interdisciplinary
  inquiry is integration. For our project, this will mean drawing together the
  diverse strands of insight from the various disciplines to discover and
  express commonalities. Newell (2006) articulates the challenge of finding
  ``latent'' or ``potential'' commonalities (p. 257), which is a process
  involving rational analysis as well as a more hermeneutic cultivation of
  intuitive and nonlinear approaches. Essentially, our goal is to find the
  common threads among the weave of our research. Interdisciplinary research
  is a broader, more holistic form of research than traditional methods. It
  requires analysis as well as interpretation, rationality as well as
  intuition.

\item[Step 9: Integrate insights.] Interdisciplinary insight is, by
  definition, integrative. ``Once the common ground has been constructed, the
  modified insights can be integrated into a more comprehensive understanding
  of the complex problem\ldots which should be responsive to each disciplinary
  perspective but beholden to none of them'' (Newell, 2006, pp. 257, 261). This
  is our goal: to offer an integrative understanding of how Kwantlen is
  navigating the current changes that lie before us. We intend to reflect upon
  the entire community: students, faculty members, administrators, staff, and
  community partners. Our approach will be broad (by way of the narratives
  created by Ross Laird) as well as focused (on the specific high-impact
  practices implemented by the whole team led by Jody Gordon).

\item[Step 10: Produce and test understanding.]
  Through our interdisciplinary understanding of both our broad narrative
  (change at Kwantlen) and focused strategy (high-impact practices in the
  classroom), we will develop guidelines and practices that we hope will
  inform the Kwantlen community about how best to navigate the changes with
  which we are grappling. These research-derived guidelines and practices will
  also become the basis for our publications and for further research funding.
  We will test our results by applying our research to the Kwantlen
  environment.

\end{description}

\subsection{Application of the Steps}

Repko reminds us that ``this model is by no means linear. The order of steps
will vary. Researchers are also encouraged to revisit steps if they feel it is
necessary'' (Repko, 2012, p 49). We view the steps as a fluid framework and a
set of guiding principles. With this in mind, we plan to proceed in the
following way:

\begin{enumerate}

\item Seek REB approval for each of the subsequent steps (as follows).

\item Engage ten students to work directly on this project, for which they
  will be paid. (Some students may also participate in this project by way of
  course-based research in the Interdisciplinary Expressive Arts courses.) The
  role of students is foundational to our approach. ``The heuristic method
  places the student in the role of the discoverer of knowledge'' (Repko,
  2012, p. 32).

\item Using a heuristic and hermeneutic approach and the FoE recommendations
  as a source, extract five core themes of change and five practical
  strategies for high-impact classroom experiences.

\item Consult the literature to compare our FoE-derived themes and strategies
  to those in the literature. Refine as required.

\item Begin the narrative process by way of interviews, observations, and
  heuristic data-gathering (conducted by Ross Laird). Heuristic research
  involves the ``search for the discovery of meaning and essence in
  significant human experience. It requires a subjective process of
  reflecting, exploring, sifting, and elucidating the nature of the phenomenon
  under investigation'' (Douglass and Moustakas, 1990, p. 40). The narrative
  aspect of our research will examine the inner lives and transformations of
  the participants (students, faculty, administrators, staff, community
  partners --- all of whom will be co-researchers) as well as the inner life
  and transformations of the co-researchers (Ross Laird, Jody Gordon, Kurt
  Penner, Josh Mitchell, and students). It is important to note that heuristic
  methods move beyond models of objective researcher and of subject and
  object. The heuristic perspective is one in which all participants are
  co-researchers, all phenomena are perceived in reciprocal relationship with
  the researchers (we influence what we study; what we study influences us),
  and all approaches seek integration and meaning. In this sense, heuristic
  research is a methodology in which all things are connected.

  In the work of Douglass and Moustakas (1990), the heuristic methodology is
  described as a means to ``obtain qualitative depictions that are at the heart
  and depths of a person's experience --- depictions of situations, events,
  conversations, relationships, feelings, thoughts, values, and beliefs. A
  heuristic quest enables the investigator to collect\ldots the raw material of
  knowledge and experience from the empirical world.'' Heuristic research
  focuses on direct first-person accounts of ``individuals who have directly
  encountered the phenomenon in experience\ldots Whereas phenomenology encourages
  a kind of detachment from the phenomena being investigated, heuristics
  emphasizes connectedness and relationship'' (p. 38).

\item Finalize high-impact practices and train volunteer faculty members in
  their application (see below).

\end{enumerate}

Steps one through six will take place in the Fall semester of 2012. The
subsequent steps (below) will take place in the following four semesters and
will end in the summer of 2014.

\begin{enumerate}
\setcounter{enumi}{6}
\item Begin the process of implementing high-impact practices in the
  classroom. This will involve the participation of three faculty members who
  will participate as co-researchers (their commitment has already been
  secured). These faculty members will introduce high-impact practices in the
  Spring semester of 2013. Each faculty member will utilize the high-impact
  practices in two sections of first-year courses, with 35 students each, for
  a total of 210 student participants. This aspect of the research will be
  coordinated by Jody Gordon.

\item Work with student co-researchers to continue gathering narratives and
  themes, with particular emphasis on the implementation of the high-impact
  strategies. In this stage of the project, students will help to gather
  narrative documentation (through interviews, for example) and will help to
  coordinate the development and implementation of the high-impact practices.
  Students will assist the volunteer faculty members who implement the
  practices by conducting tasks such as setting up classrooms, facilitating
  the collection of data, and liaising between the faculty members and the
  project team. We anticipate that the total number of hours of student
  participation for this stage of the project will be roughly 15 hours.

\item Work with the lead researchers and student co-researchers to integrate
  insights and develop an interdisciplinary understanding of the narrative
  research gathered.

\item Gather quantitative data from student participants in the high-impact
  classrooms. This will involve a standardized questionnaire about their
  experiences to be administered at the beginning of the semester and again at
  the end of the semester. We may use the College Student Experiences
  Questionnaire from the Center for Postsecondary Research and Planning at
  Indiana University. This questionnaire ``measures the quality of students'
  experiences inside and outside the classroom, perceptions of environment,
  satisfaction, and progress toward 25 desired learning and personal
  development outcomes.'' Students will assist with the data gathering. We
  anticipate that the total number of hours of student participation for this
  stage of the project will be roughly 15 hours. We further plan to follow the
  students who were enrolled in the high-impact classes through to the end
  of the Fall 2013 term to measure their rate of student persistence and
  success when compared to a control group of students enrolled in the same
  course but not exposed to the high-impact practices in their classes. We
  will also compare the grade distribution from the high-impact classes to
  those same courses offered over the previous 5 years.

\item Write the first draft of a book-length manuscript intended for
  publication in 2014. This part of the project will be led by Ross Laird.

\item Analyze the quantitative results of the classroom survey and publish
  those results in academic journals. This stage of the project will be led by
  Jody Gordon, with assistance from Ross Laird, Kurt Penner, and Josh
  Mitchell. We anticipate that the total number of hours of student
  participation for this stage of the project will be roughly 30 hours.

\item Seek further funding (from national agencies) to continue this work.

  \end{enumerate}

  The above steps are not intended to be a formalized, locked-in structure.
  ``Interdisciplinary research is a decision-making process that is heuristic,
  iterative, and reflexive'' (Repko, 2010, p. 24). There is a great deal more
  flexibility, recursiveness, self-reflection, and process-oriented thinking
  in interdisciplinary research that in many other methodologies. Each of the
  above steps will influence those that follow, in complex and dynamic ways.
  Interdisciplinary approaches require an attentiveness to this, and a
  willingness to consistently adapt and respond. As Repko (2012) affirms, the
  researcher should periodically ask questions such as the following:

  \begin{itemize}
    \item Have I defined the problem or the question too broadly or too
      narrowly?
    \item Have I correctly identified the parts of the problem?
    \item Have I gathered the most important insights concerning the problem?
    \item Am I privileging one discipline's literature or terminology over another's simply
      because I am more comfortable working in the discipline?
    \item Have I allowed my personal bias to shape the direction of the study?
      \end{itemize}

      These questions will be at the forefront of our inquiry as we move
      ahead. Indeed, this type of self-reflection and process-orientation is
      the hallmark of authentic interdisciplinary research. For
      interdisciplinary scholars, the ``definition of intellectuality shifts
      from absolute answers and solutions to tentativeness and reflexivity''
      (Klein, 1996, p. 214).

\section{Anticipated Outcome}

\subsection{Goals of the Research Partners}

\subsubsection{Ross Laird}

Ross Laird is a highly-experienced consultant, best-selling author, and
award-winning interdisciplinary scholar. He now seeks to establish a new
research direction into the field of (interdisciplinary) education. He also
wishes integrate his nomadic scholarship more fully with the Kwantlen
community.

For the past several years Ross Laird has been providing consulting services
to schools, universities, and other educational institutions on topics related
to educational renewal. This work involves the interwoven themes of pedagogy,
creativity, technology, culture, change management, and childhood development.
He entered this area of work as a consultant building upon his 25 years of
experience in social services (as a counsellor and clinical supervisor) and
education (as a facilitator, professional development consultant, and
instructor at many institutions). Until recently, he pursued this work outside
the sphere of his academic interests; but in the last few years, as he has
facilitated an increasing number of workshops for teachers and professors in
an educational landscape increasingly fraught by questions about how to manage
change, he finds himself in an excellent position to leverage his professional
expertise into a research path. This project will enable him to integrate his
academic and professional activities with his academic interests and will also
promote educational renewal at Kwantlen and in the community.


\subsubsection{Jody Gordon}

Jody Gordon is an award-winning student affairs professional. In addition to
the AACRAO Strategic Enrolment Management Award of Excellence she received,
she also received an achievement award in recognition of her contribution to
knowledge, leadership, innovation, and professional development in the
Registrar profession in BC. Jody Gordon was also nominated and received an
appointment by the provincial government to the BC Council on Admissions and
Transfer (BCCAT) as a council director. For the past three years, she has
chaired the Admissions subcommittee for BCCAT. She has also held elected or
appointed positions on the Association of Registrars of the Universities and
Colleges of Canada, the Pacific Association of Collegiate Registrars and
Admissions Officers, the Western Association of Registrars of the Universities
and Colleges of Canada and the BC Registrars Association. In addition to her
external leadership, Jody Gordon has been the co-liaison on the FoE process at
Kwantlen.

As an administrator who also teaches at Kwantlen, Jody Gordon finds that
teaching informs how she advises and leads her division in support of student
life, engagement and development. In turn, her administrative role informs how
she supports students in her class, seeing them as more than just students
taking an introduction to criminology. This unique experience has led Jody
Gordon to consider returning to graduate school to pursue a PhD in Student
Affairs and Higher Education, commencing in 2014. This research will greatly
inform both her application to graduate school and the thesis work she
pursues.

\subsubsection{About Knowledge Mobilization}

This project is entirely focused on Knowledge Mobilization as defined by the
Social Sciences and Humanities Research Council of Canada (SSHRC):
``connectivity, the spread of knowledge, community diffusion, implementation
of innovation, and social change'' (SSHRC Knowledge Mobilization definition).
One of the priority areas for SSHRC knowledge mobilization involves ``Innovation,
Leadership, and Prosperity.'' This priority is specifically focused on
supporting projects that explore complex, multi-faceted problems for which a
diverse number of approaches are required. In particular, SSHRC's Insight
program (a subset of the Innovation, Leadership, and Prosperity area) provides
funding for research that builds ``knowledge and understanding from
disciplinary, interdisciplinary and cross-sector perspectives through
support for the best researchers; support new approaches to research on
complex and important topics, including those that transcend the capacity of
any one scholar, institution or discipline; provide a high-quality research
training experience for students; fund research expertise that relates to
societal challenges and opportunities; and mobilize research knowledge, to and
from academic and non-academic audiences, with the potential to lead to
intellectual, cultural, social and economic influence, benefit and impact.''

Our project meets all of these criteria. Our intent is to apply for a SSHRC
Insight Development grant as the next stage of our research.

\section{List of works cited}

Arum, Richard. ``Academically Adrift: Limited Learning on College Campuses.'' Chicago: University of Chicago Press, 2011. Print.

Bilton. Nick. ``I Live in the Future and This is How It Works.'' New York: Crown, 2010. Print.

Borgman, Christine. ``Scholarship in the Digital Age.'' Cambridge, MA: MIT Press, 2010. Print.

Brown, John Seely. ``A New Culture of Learning.'' New York: CreateSpace, 2011. Print.

Carr, Nicholas. ``The Big Switch: Rewiring the World, from Edison to Google.'' New York: Norton, 2007. Print.

Carr, Nicholas. ``The Shallows.'' New York: Norton, 2011. Print.

Christensen, Clayton, and Eyring, Henry. ``The Innovative University.'' New York: Jossey-Bass, 2011. Print.

Davidson, Cathy. ``The Future of Learning Institutions in a Digital Age.'' Cambridge, MA: MIT Press, 2009. Print.

Edwards, David. ``ArtScience: Creativity in the Post-Google Generation.'' Cambridge, MA: Harvard UP, 2009. Print.

Edwards, David. ``The Lab: Creativity and Culture.'' Cambridge, MA: Harvard UP, 2010. Print.

Kamenetz, Anya. ``DIY U: Edupunks, Edupreneurs, and the Coming Transformation of Higher Education.'' White River Jct., VT: Chelsea Green, 2010. Print.

Lanier, Jaron. ``You are Not a Gadget.'' New York: Vintage, 2011. Print.

Palmer, Parker, and Arthur Zajonic. ``The Heart of Higher Education.'' New York: Jossey-Bass, 2010. Print.

Repko, Allen, and Newell, William. ``Case Studies in Interdisciplinary Research.'' New York: Sage, 2011. Print.

Repko, Allen. ``Interdisciplinary Research: Process and Theory.'' New York: Sage, 2012. Print.

Wagner, Tony. ``The Global Achievement Gap.'' New York: Basic Books, 2010. Print.

Weinberger, David. ``Too Big to Know: Rethinking Knowledge Now That the Facts Aren't
the Facts, Experts Are Everywhere, and the Smartest Person in the Room Is the Room.'' New York: Basic Books, 2012. Print.


\section{Explanation of Interdisciplinary and Collaborative Nature of Project}

Our research project is an interdisciplinary inquiry involving members of the
Kwantlen community from various departments and disciplines. Ross Laird is a
member of the Creative Writing department (composition of a literary book is
part of this project) as well as an instructor of Interdisciplinary Expressive
Arts (this project is an interdisciplinary inquiry). Jody Gordon is a senior
administrator in Student Life and Community as well as an instructor in
Criminology (one goal of this project is to improve the learning experiences
of students both in and out of the classroom). Kurt Penner is a faculty member
in Student Life and Community and Josh Mitchell is a senior administrator in
Student Life and Community (Kurt Penner is also an instructor in Psychology,
and this project involves assessment using instruments from Psychology). Our
student participants will come from various disciplines, and our collaborative
project will integrate these diverse perspectives into a single
interdisciplinary focus.

Each of these participants offers a unique contribution. Ross Laird brings his
extensive experience as an author and consultant in education and change
management. Jody Gordon brings her exemplary track record as a university
administrator, researcher in education, and leader in student life and
community. Kurt Penner and Josh Mitchell bring their outstanding skills as
student life administrators and developers of student-led initiatives. And,
in turn, the students bring their diversity, enthusiasm, and goodwill.

Each participant will participate in the ten-step framework outlined above and
will collaborate extensively with all the other participants. This is a truly
interdisciplinary project and is therefore highly unusual for Kwantlen.
Typically, the term ``interdisciplinary'' is used at Kwantlen to describe a
variety of activities that are, in fact, multidisciplinary. This distinction
is crucial. Interdisciplinary initiatives always seek integration, whereas
multidisciplinary approaches tend to keep things separate and domain-specific.
In this sense, true interdisciplinary collaboration is much more than people
from different disciplines working together. Authentic interdisciplinarity is
an integrated, holistic approach that goes far beyond the multidisciplinary
model. Perhaps a lighthearted metaphor from interdisciplinary scholar Moti
Nissani (1995) says it best: studies are smoothies, whereas multidisciplinary
studies are bowls of fruit. Our project is a smoothie.

\section{Plans for Communication of Research Results}


Communication of the processes and products of this project will begin on day
one, with the launch of our collaborative project. Members of the broad
community will be able to read about, and participate in, the conversations
and sub-projects of the co-researchers. An online community will also be
developed, with the research project at its core.

As the project reaches the stage of formal publication of results, several
articles and a book will be published. The book will be made available in
print, in several digital formats, and as a full text web document. We are
committed to open access to information and the use of scholarship to promote
the public good. Accordingly, the results of this research are intended to
provide useful information and insight not only to those in the academic
sphere but also to general readers, parents, educators, students, and the
entire Kwantlen community.

In conjunction with the publication of the articles and the book, members of
the research team will present at various conferences.

\section{Budget justification}

\subsection{Time Release (Ross Laird): \$ 25,000}

We are requesting two semesters of one-quarter time release for Ross Laird to
work on this project and to write the first draft of the book that will
represent the broad narrative component of our research results. This includes
planning, outlining, synthesizing and interpreting the heuristic data, writing
various drafts, fact-checking, editing, copy-editing, designing, typesetting,
digitizing (in multiple formats), and arranging for publication. Writing a
book is a substantial effort and requires a significant investment in time and
focus.

\$ 25,000 is the amount allocated to two sections of time release, but it is
also the amount specified by the Canada Council for grants to writers in the
category to which Ross Laird belongs: mid-career writers who have published
two or more books. Canada Council grants for writers are intended to provide
``Canadian authors (emerging, mid-career and established) time to write new
literary works.'' This is precisely what Ross Laird will be doing, and the Canada
Council's industry-standard is the best guide to budgetary justification for
this type of work.

\subsection{Funding for Student Researchers}

We are requesting a total of \$ 10,000 in funding to support the collaborative
work of ten students. These students will be partners in our research and will
be fully involved in all of our activities. We project that the activities of
the students will require roughly 65 hours in total, and we propose to pay the
students \$ 15 per hour.

\subsection{Funding for Attendance at Conferences}

We request a total of \$ 5,000 to support the travel of the two research leads
to the First-Year Experience Conference, which is the largest and most
multidisciplinary (as well as interdisciplinary) conference on the first year
experience in the United States. This conference will be held in February
2013, and the work presented will be directly relevant to our research at
Kwantlen. Attendance at this conference by the two research leads (with the
added possibility of presenting our work to date at this conference) will
greatly inform us about high-impact practices occurring at universities and
colleges across North America.

\subsection{Total Budget}

The total projected cost of the project is \$ 40,000.

\section{Timeline}

The timeline for this project is two years.


\section{Research Grant Application History}

\subsection{Ross Laird}


Ross Laird entered traditional academia only recently, after a 25-year career
as a consultant, educator, and author. His career has been mostly in the wild,
so to speak: bidding for and earning contracts, competing with other
consultants and educators, working creatively to spread ideas, collaborating
with a wide range of professionals. He has achieved a strong track record in
his career without ever having applied for a research grant in the sense that
this section of the application requests. It seems that the criteria for this
section are focused on applications for funding from organizations such as
SSHRC. Ross Laird's funding has not come from such sources. Instead, his
research has been undertaken with support from organizations such as The
Canada Council for the Arts, Heritage Canada, the BC Ministry of Health,
Corrections Canada, and many private corporations. These types of
organizations do not typically use the term ``research grant.'' Instead, they
use words such as competition, contract, and bid. This is the landscape in
which Ross Laird has built his track record of success.

We assume that this section of the application is primarily intended to
determine whether the researchers have a history of successful peer review
(and, by implication, how much research funding the researcher has raised). As
Ross Laird's research has taken place in an environment distinct from the
pathways of SSHRC and related agencies, it may be useful to illustrate his
success with some examples of the context of peer review and funding in which
he typically works.

\subsubsection{Peer Review in the Arts}

Ross Laird has written three books on interdisciplinary subjects (creativity,
myth, culture, addictions). His first book evolved out of his doctoral
research project, which was a heuristic study of creativity. The dissertation,
which won the Union Institute's Sussman Award for Academic Excellence (the
university's highest academic award), was adapted into a best-selling book and
was short-listed for the Governor General's Award. In the environment of
professional writing in Canada, a short-list for the Governor General's Award
is the highest peer review accolade one can receive.

The success of his first book led to a second, which was published by Canada's
premier literary publisher, McClelland and Stewart. Publication by McClelland
and Stewart is a type of peer endorsement for Canadian writers (an endorsement
from the gatekeepers of Canadian literature). Most of Canada's historical
literary figures have published with McClelland and Stewart.

His ongoing success as a writer has led Ross Laird to involvement with the
Canada Council for the Arts. He was asked to be a jury member for the Literary
Grants program and for the Governor General's Award. One cannot apply for such
positions; they are by invitation only, from the foremost arts funding agency
in the country. Essentially, his participation on the Governor General's Award
jury (in 2008) was an endorsement from the government of Canada that he is
competent to assess the literary merit of the best writers in Canada. This is
a very strong peer review endorsement.

One more example: last year Ross Laird responded to a call, from the Writers'
Union of Canada (for which he has served on the National Council), to compete
for a Heritage Canada contract to be one of two professional development
consultants for professional writers in Canada. This was a national call, open
to any professional writer. His successful bid is a strong endorsement of his
role within the writing community and a clear indication that writers value
what he offers. (This Heritage Canada project was intended to help writers
navigate the evolving landscape of publishing. In many ways, writers are
grappling with the same challenges as educators.)

From readers, writers, publishers, and from the government of Canada, Ross
Laird has consistently received strong endorsements. His peers --- at every
level --- have acknowledged that he makes an important contribution.

None of this work has involved research grants as described by this
application process nor pathways of publication in scholarly journals.
Instead, Ross Laird's research has been published in the books and articles he
has written based on his experiences as a researcher of creativity, education,
and personal development. His work has been published or reviewed in most
major publications in Canada (The Globe and Mail, Maclean's, The Toronto Star,
The Ottawa Citizen, The Hamilton Spectator, The Vancouver Sun, etc.), he has
appeared on television and radio many times, and his writing and research
continue to evolve. This research project will represent his fourth book.

The funding for these professional projects has included book advances, jury
fees, literary grant funding, and contract fees. The total amount over the
past few years is approximately \$ 80,000.

\subsubsection{Peer Review in Social Services}

Ross Laird has been a professional counsellor and clinical supervisor to
social service agencies for many years. He has worked with hundreds of
organization and thousands of individuals. He has presented at more than 100
conferences, has facilitated hundreds of professional development projects,
and has worked hard to establish a strong reputation for professionalism. In
2003 he earned the Communications Award of the BC Association of Clinical
Counsellors for the heuristic research of his first two books. This year he
will publish a third book, on addictions, that continues and deepens that
research.

\subsubsection{Peer Review in Education}

For the past several years Ross Laird has provided professional development
services to elementary schools, secondary schools, community colleges,
vocational schools, teacher training programs, teaching associations, school
districts, parent groups, student groups, and universities (in 2010-11 he
provided professional development consulting for faculty at Kwantlen, through
his role as consultant for Kwantlen's Learning Technology department). This
work continues to grow and expand in scope; during the past year he has worked
with more than 1,000 educators in many institutions (repeating the examples
from earlier: the UBC Teacher Education program; the SFU Advanced Professional
Studies Field program; the UFV Teacher Education program; Vancouver Coastal
Health; the School Districts of Vancouver, Richmond, Surrey, Delta, Langley,
Abbotsford, Coquitlam, and others). Overall, he has earned roughly \$ 100,000
over the past six years (the length of time requested by this application)
providing educational consulting services.

Finally, SSHRC defines a ``record of research achievement'' as ``any identifiable
contributions made by applicants to the advancement, development and
transmission of knowledge.'' By these criteria, Ross Laird possesses a strong
track record. He has made extensive contributions to several fields and has
helped to advance, develop, and transmit knowledge through his professional
activities and my writing.

SSHRC also defines research and creation as ``any research activity or approach
to research that forms an essential part of a creative process or artistic
discipline and that directly fosters the creation of literary/artistic works.''
By these criteria also, Ross Laird has a demonstrated record of outstanding,
peer-reviewed, consistently endorsed literary and artistic work. The Katalyst
program is designed not to mirror SSHRC but to offer applicants a chance to
build bridges between academic scholarship and professional pursuits that have
not yet made their way into channels such as SSHRC. This is exactly what Ross
Laird hopes to do.

\subsection{Jody Gordon}

During her time as a graduate student, Jody Gordon held two research assistant
posts. From 1994-1996 she was a Research Assistant for an SFU faculty member
on contract with the Solicitor General of Canada; and from 1993-1994 she
conducted data analysis and acted as Research Coordinator for two SFU and UBC
faculty members. In this latter position she drafted a proposal for funding
and conducted data analysis. In 2008, Jody Gordon (along with two of her
direct reports) applied for and was awarded a \$ 14,000 research grant by the
BC Provincial Government under the auspices of the Default Prevention Grant
Program. This research entailed the delivery of the Academic Boost Camp
program in which students on academic warning and academic probation are
exposed to high-impact practices to improve their academic standing. Data from
this research demonstrated that students enrolled in the program succeeded at
a higher rate than those not enrolled. Further, the students who completed the
program persisted through their education and were less likely to be required
to withdraw from the university. Since receiving this grant, the Academic
Boost Camp continues to be offered at Kwantlen (senior administration provided
permanent funding to maintain the program). The results of the Academic Boost
Camp research were published in a journal article and have been presented at
numerous conferences around North America.

\end{document}
