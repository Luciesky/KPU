%!TEX TS-program = xelatex
%!TEX encoding = UTF-8 Unicode

\documentclass[10pt, letterpaper]{article}
\usepackage{fontspec} 
\usepackage{placeins}
\usepackage{multibbl}
\usepackage{graphicx}
\usepackage{hieroglf}
\usepackage{txfonts}
% DOCUMENT LAYOUT
\usepackage{geometry} 
\geometry{letterpaper, textwidth=5.5in, textheight=8.5in, marginparsep=7pt, marginparwidth=.6in}
%\setlength\parindent{0in}

% FONTS
\defaultfontfeatures{Mapping=tex-text} % converts LaTeX specials (``quotes'' --- dashes etc.) to unicode
\setromanfont [Ligatures={Common}, SmallCapsFont={ITC Officina Serif Std}, BoldFont={ITC Officina Serif Std Bold}, ItalicFont={ITC Officina Serif Std Book Italic}]{ITC Officina Serif Std Book}
\setmonofont[Scale=0.8]{Lucida Sans Typewriter Std} 
\setsansfont [Ligatures={Common}, SmallCapsFont={ITC Officina Sans Std}, BoldFont={ITC Officina Sans Std Bold}, ItalicFont={ITC Officina Sans Std Book Italic}]{ITC Officina Sans Std} 
% ---- CUSTOM AMPERSAND
\newcommand{\amper}{{\fontspec[Scale=.95]{StoneSansStd-MediumItalic}\selectfont\itshape\&}}
% ---- MARGIN TASK
\newcommand{\task}[1]{\marginpar{\small #1}}

% HEADINGS
\usepackage{sectsty} 
\usepackage[normalem]{ulem} 
\sectionfont{\sffamily\mdseries\upshape\Large}
\subsectionfont{\sffamily\mdseries\scshape\normalsize} 
\subsubsectionfont{\sffamily\mdseries\upshape\normalsize} 

% PDF SETUP
% ---- FILL IN HERE THE DOC TITLE AND AUTHOR
\usepackage[dvipdfm, bookmarks, colorlinks, breaklinks, pdftitle={Ross A. Laird - Professional Development Fund Application},pdfauthor={Ross A. Laird}]{hyperref}  
\hypersetup{linkcolor=blue,citecolor=blue,filecolor=black,urlcolor=blue} 

% DOCUMENT
\begin{document}
\thispagestyle{empty}
\reversemarginpar
\noindent
\includegraphics[scale=0.25]{/home/ross/Dropbox/docs/kwantlen/logoprint}\\[2em]
{\LARGE .6 Faculty Professional Development Fund Application}\\[2em]
{\Large Ross A. Laird, PhD}\\ Faculty member: Creative Writing, Learning Communities,\\Interdisciplinary Arts (Humanities), Communications\\
Consultant: Kwantlen Educational Technologies, Kwantlen Educational Development\\
Phone: \texttt{604-916-1675}\\
Email: \href{mailto:ross@rosslaird.com}{ross@rosslaird.com}\\
\textsc{url}: \href{http://www.rosslaird.com}{http://www.rosslaird.com}\\
\textsc{PGP key}: \href{http://keyserver.ubuntu.com:11371/pks/lookup?op=get&search=0x623D9CC650BD6C0B}{\texttt{50BD6C0B}}
%%\vfill
\\[2em]
September, 2010
\section*{New and Social Media Project}

\section*{Abstract}
\label{sec-1}

This proposed project is a service initiative to Kwantlen. It involves work to create and promote a much-needed community of instructional innovation at the University. My aim is to bring together faculty members and students to build a community of inquiry using creative pedagogy and the tools of new and social media. Such tools are now foundational to the contemporary post-secondary landscape, and the Kwantlen faculty community has been very slow in adapting to the benefits and challenges of this new landscape. (Sorry, but we have.\dag)

My professional goals for this project involve collegiality, creativity, and new media utilization. Publications will result from this initiative, but my primary goal is to assist those members of the Kwantlen community who wish to explore innovative pedagogy, new and experimental teaching practices, and emerging models of what it means to teach and to learn. I am a service-minded person, and this project will allow me to leverage my sensibilities of service into practical benefits for the University. In this sense, the professional development for me will involve service to the community at large.\\ 

({\footnotesize \dag Our standard document forms, which you will notice I have not used in the application, are one example of our difficulties with technological innovation.})

\section*{Administrative Details}

The proposed start date (and funding start date) for the project will be January 2011. The project is intended to last for two semesters: Spring 2011 and Fall 2011.\\

This project does not involve research with human subjects.\\

This application does not involve tuition or funding requests for external initiatives.\\

I have not received other funding for this project. However, this proposed project is adjunct to, and will be made more effective by, my ongoing work for Kwantlen's Educational Technology and Educational Development areas. I am currently working in both of those areas to promote aspects of Kwantlen's development that are related to this project. In the Educational Development area I am working on innovation in new degree implementation; in Educational Technology I am working on technology infrastructure and training. The proposed project is intended to complete the circle, so to speak, so that innovation can be brought more directly (and in greater capacity) to the faculty communities.

\section*{Budget}

I am requesting two time releases for this project (Spring and Fall 2011), for a total of 25,000 dollars.

\section*{Signatures}

Applicant Name: Ross A. Laird\\

\includegraphics[scale=0.25]{/home/ross/Dropbox/docs/professional/signature}\\[1em]

September 24, 2010\\

\noindent Dean: Mazen Guirguis\\


September 24, 2010


\section*{Completed by Dean's Office}

Applicant's employment status: PT Regular 75 per cent\\

\subsection*{Applicant's prior contribution to Kwantlen}
\newpage
\section*{Proposal Description}

\subsection*{The Context}

Kwantlen is British Columbia’s Polytechnic University. The integration of the terms Polytechnic and University in our title reflects our unique mandate, mission, and character as a post-secondary institution. The first principle articulated in our new mission and mandate is to be ``a leader in innovative and interdisciplinary education.''

Kwantlen’s emerging identity must be founded upon collaborations within and across disciplines, activities that encourage innovation among faculty, and practices that prepare learners to be competitive in a rapidly changing world. Accordingly, the learning experiences we provide must facilitate creativity, critical awareness, cultural sensitivity, social responsibility, civic engagement, and global citizenship.The challenges of our new charter are substantial, and we are now undertaking them with growing confidence.

In Kwantlen's Strategic Action Plan (2010), the first principle is to ensure that ``Kwantlen’s learning environment inspires inquiry, collaboration, creativity, and application.'' A core principle of the Senate Subcommittee on Academic Planning and Priorities is to ``provide opportunities that encourage faculty to develop new teaching interests and methodologies in keeping with the institution's mandate.'' This broad initiative requires us to develop new methods and practices, to build communities of inquiry, and to utilize emerging technologies and tools. Essentially, we are tasked with revisioning our approach to teaching and learning.

\subsection*{The Challenge}

The principles and goals articulated in Kwantlen's new vision are indeed innovative, and their implementation will increase the distinctiveness and appeal of our many programs. However, most of the practices required to deliver on the promises of innovation, creativity, and outstanding education (particularly in new media) are not in place. Not even close. Further instructor professional development, program integration, curriculum development, and digital content development are all required to fulfill these promises.

We must increase the tech-savvy of instructors, promote instructional innovation, and create learning environments that leverage technology in service of creative and innovative pedagogy. The success of this plan depends upon the development of a culture of technology and creativity at Kwantlen -- a \textit{geek} culture, in fact -- in which students and faculty collaborate on a broad range of tasks and activities that promote the synergy of technology, learning, and academic development. This culture must grow organically, must be modeled by diverse practices and examples, and must be founded upon many creative contributions by many people. The delivery of the promises of our new mission and mandate depend upon our attentiveness to the nurturing of that culture of inquiry.

\subsection*{Making it Happen}

How might we create and promote a culture of creativity, innovation, technological integration, and collaborative inquiry? First, we must recognize that cultural and pedagogical shifts will take time, and will require considerable attention to the comfort and buy-in from faculty and students. New technologies must slowly be suffused throughout the learning environment. Traditional delivery methods must carefully and strategically be augmented with technological support. Blended courses must be designed and developed. Faculty  professional development activities must create, spread, and integrate these initiatives. Specific foundational projects might include: 

\begin{enumerate}
\item Faculty training in the principles and practices of new media and social media teaching and learning: not just \textit{how}, but \textit{why}.
\item Faculty training in the augmentation of current curriculum beyond (far beyond) Moodle and related tools (blogging, social media, integrative media, content management, and so on).
\item Faculty training in how to choose and use new media and social media.
\item Faculty training in creative and innovative approaches to the engaged classroom experience.
\item Faculty training in blended learning as a pedagogical approach to creativity and scholarship.
\end{enumerate}

In terms of faculty professional development for current best practices in the uses of technology, this will entail more than simple exposure to new and social media. The integration of new and social media into pedagogical design is a complex and iterative process that requires considerable time, effort, and persistence. Such initiatives are key to the cultural and educational plans of Kwantlen and must be introduced with sensitivity to interpersonal process. The implementation of new technologies and practices is best handled in a context that recognizes the challenges of substantive change. Faculty members must be comfortable enough -- while still challenged by -- new developments, and must be assisted in finding ways to utilize new methods that are consistent with their own preferences and skills. This shift will take time, and should (ideally) evolve as a collegial, collaborative conversation. Accordingly, faculty will require assistance and consultation for the creation of projects such as digital collaboratories (including student digital portfolios) to begin working toward pedagogies of blended learning using new and social media.

Moreover, if Kwantlen is to fulfill its promise of innovation, we must develop innovative course curriculum for new and social media. Technological immersion must be an aspect of almost all courses. Yet we must go farther in showing our commitment to fulfilling our core promises to students: faculty require assistance and collaborative support in developing innovative and unique curriculum based on, and delivered through, a blend of new media and classroom engagement.

\subsection*{Making it Happen}

The evolution of our new identity involves many interlocking stages and developmental tasks. Essentially, the core initiative is a movement toward a learning environment founded upon creativity, innovation, and learner engagement through the use of contemporary best practices and tools. This shift requires collaborative, consultative work at every stage, and will be best accomplished through consistent involvement of a member of the Kwantlen community with extensive experience in the overlapping domains of Kwantlen's degrees and departments.

A provisional timeline and structure for this consultation, over the next few semesters, might resemble the following:\\[1em]
\noindent
\textit{Spring 2011}
\begin{itemize}
\item Articulation of steps, goals, and plans by a group (to be formed) of like-minded faculty and students
\item Faculty development in the principles and practices foundational to new media and social media teaching and learning
\item Faculty development in creative and innovative approaches to the engaged classroom experience
\item Faculty assistance and consultation for the creation of digital collaboratories to begin working toward pedagogies of blended learning using new and social media
\item Faculty development training in new media and social media
\item Faculty development training in blended learning as a pedagogical approach to creativity and scholarship
\item Faculty assistance and consultation for the ongoing development of digital collaboratories (including student portfolios)
\end{itemize}
\noindent
\textit{Fall 2011}
\begin{itemize}
\item Faculty assistance and collaborative support in developing new and unique curriculum based on, and delivered through, new and social media
\item Faculty assistance and consultation for the further development of digital collaboratories
\item Faculty assistance and collaborative support in implementing and testing new and unique curriculum based on, and delivered through, new and social media
\item Faculty assistance and consultation for the further development of digital collaboratories
\end{itemize}

The above provisional timeline (which is, of course, subject to revision based on consultations) outlines how the process might start and how it might evolve over the short term. (As an aside, this sample timeline could be extended to roughly five years, which is approximately the length of time typically required to fully develop and implement the types of principles and practices outlined in our new mission, mandate, and organizational priorities.

\section*{Fit with Criteria and Institutional Priorities}

\subsection*{How will my professional development, performance, expertise and/or career plans be enhanced?}

My career plan involves greater immersion in the Kwantlen community. I enjoy the diversity of Kwantlen, the way in which it seems to contain so many complementary and sometimes contradictory aspects. Kwantlen is an interesting place, and over time my involvement with it has deepened. At the same time  my roles have also broadened, and I have found myself becoming increasingly involved in various departments and disciplines (five, as of this writing) that share my interest in creativity, interdisciplinarity, and innovation. But what has been missing from these interactions is a sense of synthesis or synergy. I hear a great deal about the desire of faculty members to invigorate (and modernize) their approaches to teaching and learning, yet I also hear much frustration about the lack of formalized strategies to address these desires. This project would provide that synergy, and accordingly would assist me with my goal of greater involvement at Kwantlen.

I view this project as one of mutual support: by helping the Kwantlen community, the project will help me to broaden and deepen my relationship with Kwantlen (which is my goal). Everybody wins.

\subsection*{How is my proposal related to my work at Kwantlen, and how will it benefit the institution and students?}

This project will formalize initiatives that have already begun but which are currently stalled because of absent facilitative support. Many faculty members and students wish to work on innovative pedagogy (especially with new and social media) but do not possess the skills to make any headway. I possess such skills, and I am motivated to help -- especially as innovation and creativity are foundational to Kwantlen's purpose moving forward (as mentioned earlier, with reference to planning and visioning documents), and equally foundational to my own values and practices as an educator.

Students are hungry for pedagogical innovation, and they already inhabit the worlds of new and social media. We should meet them there, bringing with us the best of ourselves as educators (which also entails our own new learning). This project will allow us to begin.

I believe in the promise of Kwantlen. I do not believe we have yet embraced that promise, nor have we come anywhere close to the kinds of educational contributions we are capable of making. We just need a little help, some small nudges here and there, to get us moving toward greater creativity and innovation. The desire for innovation is palpable; perhaps you feel it too. With this project we could jumpstart that innovation. We could move beyond talk, beyond speculation, and actually build something we could use.

%\vspace{1cm}
\vfill{}
\hrulefill

% FILL IN THE FULL URL
\begin{center}
{\footnotesize \href{http://www.rosslaird.com}{http://www.rosslaird.com}\pmglyph{\Hibw}}
\end{center}


\end{document}
